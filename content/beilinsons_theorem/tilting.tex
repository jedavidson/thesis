\section{Tilting in the derived category}

We are now justified in wondering if anything more interesting
happens in the derived category where this information ought to be more visible.
The logical conclusion of this inquiry is Beilinson's theorem.
The proof via \emph{tilting theory} for this result first emerged in
the literature with the publication of \cite{baer}, though in the
context of smooth \emph{weighted projective varieties}.
Rather than using this original work as our source, we instead adapt
the simpler exposition given in \cite{craw} for the details to sketch
how this result follows.

In the interest of keeping an already lengthy and multidisciplinary
thesis focused, we will not discuss tilting theory as a subject area,
and rather just give an overview of how its methods apply in this case.
This is an area best approached and motivated from the point of view
of representation theory, and we defer the interested reader to
\cite{handbook_of_tilting_theory}, the peerless source on the topic.

The key notion is that of a \emph{tilting object}.
We will present only the definition relevant to this geometric case.

\begin{definition}
  Let $\abcat{A}$ be an abelian category and $\triangcat{D} =
  \dercat[b]{\abcat{A}}$ its bounded derived category.
  Call a subcategory of $\triangcat{D}$ \emph{\'{e}paisse} if it is
  closed under isomorphisms and shifts, as well as taking cones of
  morphisms and direct summands of objects.
  An object $A \in \abcat{A}$ is said to \emph{classically generate}
  $\triangcat{D}$ if the \emph{\'{e}paisse envelope} of $A$, i.e. the
  smallest \'{e}paisse subcategory of $\triangcat{D}$ containing $A$
  (as a complex concentrated in degree 0, say), is $\triangcat{D}$ itself.
\end{definition}

\begin{definition}
  A coherent sheaf $\shf{T}$ is a \emph{tilting sheaf} on $\proj{1}$ if
  \begin{enumerate}[leftmargin=3.9em]
    \item[(T1)] $\Extalt{i}(\shf{T}, \shf{T}) = 0$ for all $i > 0$;
    \item[(T2)] the $k$-algebra $\End(\shf{T})$ has finite global
      dimension (c.f. \cref{sect_resolutions});
    \item[(T3)] $\shf{T}$ classically generates $\dercat[b]{\Coh(\proj{1})}$.
  \end{enumerate}
  We call a locally free tilting sheaf a \emph{tilting bundle} on $\proj{1}$.
\end{definition}

With the following series of lemmas, we check that $\tiltshf$
satisfies all three of the properties of a tilting bundle, showing it
deserves its name.

\begin{lemma}
  The sheaf $\tiltshf$ satisfies (T1).
\end{lemma}

\begin{proof}
  Since $\Extalt{1}$ is an additive functor, it follows that
  \begin{align*}
    \Extalt{1}(\tiltshf, \tiltshf)
    % &\cong \Extalt{1}(\shf{O}, \shf{O})
    %     \biprod \Extalt{1}(\shf{O}, \twist{1})
    %     \biprod \Extalt{1}(\twist{1}, \shf{O})
    %     \biprod \Extalt{1}(\twist{1}, \twist{1}) \\
    \cong H^1(\shf{O}) \biprod H^1(\twist{1}) \biprod H^1(\twist{-1})
    \biprod H^1(\shf{O}) = 0
  \end{align*}
  by \cref{prop_ext_from_twist} and \cref{lemma_h1_twist}.
\end{proof}

\begin{lemma}
  The sheaf $\tiltshf$ satisfies (T2).
\end{lemma}

\begin{proof}
  As we have seen, $\End(\tiltshf)$ is the algebra opposite to the
  path algebra $k \Delta$.
  But it is well known that any quiver without oriented cycles has a
  \emph{hereditary} path algebra, i.e. of global dimension 1.
  We will not prove this, and instead defer to
  \cite[Proposition~III.1.4]{auslander_reiten_smalo}.
\end{proof}

\begin{lemma}
  The sheaf $\tiltshf$ satisfies (T3).
\end{lemma}

\begin{proof}
  This is the most challenging property to establish.
  Let $\triangcat{E}$ be the \emph{\'{e}paisse envelope} of
  $\tiltshf$ in $\dercat[b]{\Coh(\proj{1})}$.
  Throughout the proof, we implicitly treat sheaves as complexes
  concentrated in degree 0 in $\dercat[b]{\Coh(\proj{1})}$ and write
  $\shf{F} \in \triangcat{E}$ if $\triangcat{E}$ contains $\shf{F}$
  in this manner.
  Since the cone of $0: \shf{F}[-1] \to \shf{G}$ is $\shf{F} \biprod
  \shf{G}$, it follows that $\triangcat{E}$ is also closed under
  taking direct sums.
  Thus to show that $\triangcat{E} = \dercat[b]{\Coh(\proj{1})}$, it
  is enough to show that we can build every coherent sheaf on
  $\proj{1}$ in degree 0, as taking direct sums of shifted complexes
  then recovers all other bounded complexes.
  Moreover, Grothendieck's splitting theorem further reduces this to
  obtaining the twisting sheaves and thickened skyscraper sheaves.
  To facilitate this, we establish one last closure property.
  Given coherent sheaves $\shf{F}, \shf{G} \in \triangcat{E}$, any
  short exact sequence
  \[
    \begin{tikzcd}
      0 & {\shf{F}} & {\shf{G}} & {\shf{H}} & 0.
      \arrow[from=1-1, to=1-2]
      \arrow["\Phi", from=1-2, to=1-3]
      \arrow["\Psi", from=1-3, to=1-4]
      \arrow[from=1-4, to=1-5]
    \end{tikzcd}
  \]
  induces an exact triangle in $\dercat[b]{\Coh(\proj{1})}$ by
  \cref{prop_dercat_ses_exact_triang} isomorphic to the strict
  triangle on $\Phi$.
  In particular, we have a quasi-isomorphism $\cone{u} \to \shf{H}$
  which becomes an isomorphism in $\dercat[b]{\Coh(\proj{1})}$, so
  closure under cones and isomorphisms gives $\shf{H} \in \triangcat{E}$.
  Using the rotation axiom (TR2) for triangulated categories, this
  argument shows by extension that given that any two of the three
  terms in the above short exact sequence belong to $\shf{E}$, the
  third also necessarily belongs to $\shf{E}$.

  Since $\shf{E}$ contains $\shf{O}$ and $\twist{1}$ by construction,
  we can obtain $\twist{2} \in \triangcat{E}$ by considering the
  degree 2 twist of the Euler exact sequence, i.e.
  \[
    \begin{tikzcd}[cramped]
      0 & {\shf{O}} & {\twist{1}^{\biprod 2}} & {\twist{2}} & 0.
      \arrow[from=1-1, to=1-2]
      \arrow[from=1-2, to=1-3]
      \arrow[from=1-3, to=1-4]
      \arrow[from=1-4, to=1-5]
    \end{tikzcd}
  \]
  Taking successive degree 1 twists of the above sequence
  inductively, we also obtain $\twist{n} \in \triangcat{E}$ for all $n > 0$.
  Similarly, to obtain $\twist{-1}$, we instead take the degree 1
  twist of the Euler exact sequence
  \[
    \begin{tikzcd}[cramped]
      0 & {\twist{-1}} & {\shf{O}^{\biprod 2}} & {\twist{1}} & 0
      \arrow[from=1-1, to=1-2]
      \arrow[from=1-2, to=1-3]
      \arrow[from=1-3, to=1-4]
      \arrow[from=1-4, to=1-5]
    \end{tikzcd}
  \]
  and then negatively twist this exact sequence inductively to obtain
  $\twist{n} \in \triangcat{E}$ for all $n < 0$ too.
  Finally, we now take the ideal sheaf sequence
  \[
    \begin{tikzcd}[cramped]
      0 & {\twist{-m}} & {\shf{O}} & {\skyscraper{mp}} & 0
      \arrow[from=1-1, to=1-2]
      \arrow[from=1-2, to=1-3]
      \arrow[from=1-3, to=1-4]
      \arrow[from=1-4, to=1-5]
    \end{tikzcd}
  \]
  shows that $\skyscraper{mp} \in \triangcat{E}$ for all $m \in
  \zpos$ and $p \in \proj{1}$.
  This completes the proof.
\end{proof}

At long last, we can now state and prove Beilinson's theorem reimagined by Baer.

\begin{theorem}[Beilinson, Baer]
  Let $\shf{T}$ be the tilting bundle of $\proj{1}$ as defined in
  \cref{def_tilting_bdl} and $\Lambda$ the corresponding tilted algebra.
  Then the total derived functor
  \[
    \rightderfunc{\Hom(\shf{T}, -)}: \dercat[b]{\Coh(\proj{1})} \to
    \dercat[b]{\mod{\op{\Lambda}}}
  \]
  induces an equivalences of bounded derived categories whose quasi-inverse is
  \[
    \dertensor{\Lambda}{-}{\shf{T}}: \dercat[b]{\mod{\op{\Lambda}}}
    \to \dercat[b]{\Coh(\proj{1})}.
  \]
  \vspace{-30pt}
\end{theorem}

We will stop short of giving the proof of this result, which is found
in \cite[Theorem~2.1]{craw} and instead briefly comment on the
necessity of the tilting bundle assumptions in its proof.
The requirement that the tilted algebra has finite global dimension
is necessary to ensure that the image of
$\dertensor{\Lambda}{-}{\shf{T}}$ produces bounded complexes.
The property (T2) shows that
\[
  \rightderfunc{\Hom(\shf{T}, \dertensor{\Lambda}{\Lambda}{\shf{T}})} = A,
\]
i.e. the composite is the identity on $\Lambda$.
Finally, the property (T3) shows that the image of
$\dertensor{\Lambda}{-}{\shf{T}}$, which is the triangulated
subcategory of $\dercat[b]{\Coh(\proj{1})}$ generated by $\shf{T}$.
This is then where (T3) applies to close out the proof.

The tilting theory literature expands more upon the significance of
this theorem.
In particular, a classical theorem of Happel is that any hereditary
abelian category with tilting object is, up to equivalence, the
category of coherent sheaves on some \emph{weighted projective line}
or the category of left modules over some finite dimensional algebra.
We defer to \cite{handbook_of_tilting_theory} for more details.
