\chapter{Abelian categories}
\label{chap_abelian_categories}

In this first chapter, we introduce \emph{abelian categories}, a
class that captures some essential properties of categories such as
$\Ab$ and $\Mod{R}$.
We will mostly follow
\cite{category_theory_for_the_working_mathematician, freyd_abelian_categories}.
The origin of abelian categories in contemporary form is
Grothendieck's influential \emph{T\={o}hoku paper} of 1957, though
this work is predated by Buchsbaum's similar notion of an \emph{exact
category} (which now means something else entirely) defined two years prior.

The main selling point of working with abelian categories is that
they are endowed with enough structure to make sense of exact
sequences of morphisms, and in a way that is sufficiently general
enough to describe many categories arising in practice.
Owing to this flexibility, they are now a foundational abstraction in
several branches of modern algebra, especially those areas which this
thesis deals in.

\section{Additive structures on categories}

Using $\Mod{R}$ as our prototype, we will identify and develop some
suitable analogues of these aforementioned essential properties.
We begin with an observation about its morphisms.
For any pair of $R$-modules $M$ and $N$, the set $\Hom_R(M, N)$ is
not only a set but also an abelian group, since we are able to add
module homomorphisms together pointwise.
This addition also interacts bilinearly with composition, i.e. we have
\[
  f(g + h) = fg + fh
  \mathand
  (f + g)h = fh + gh
\]
for any compatible module homomorphisms $f$, $g$ and $h$.
If $R$ is commutative, then each hom-set has even more structure,
becoming itself an $R$-module with the usual action.
Since similar structure is found on the hom-sets of other, more
exotic categories and makes manipulating morphisms far more pleasant,
this is a reasonable starting point.

\begin{definition}
  A category $\cat{A}$ is said to be \emph{preadditive} if every
  hom-set of $\cat{A}$ is an abelian group and the composition of
  morphisms is bilinear with respect to addition.
  A preadditive category is \emph{$R$-linear} if every hom-set is
  also a module over some ring $R$.
\end{definition}

It may also be possible to multiply and add objects themselves, which
for $R$-modules is achieved by taking direct products and sums.
Moreover, these two notions coincide if one only considers finite
collections of modules.
There is also a zero module $0$ which, among other things, satisfies
$M \oplus 0 \cong M$ for any $R$-module $M$.
Phrased in the language of category theory, the direct product and
sum constructions are the product and coproduct of the category
$\Mod{R}$ respectively, and the zero module is its zero object.

\begin{definition}
  \label{def_additive_cat}
  A preadditive category $\cat{A}$ is said to be \emph{additive} if
  has a zero object $0 \in \cat{A}$ and every pair of objects in
  $\cat{A}$ has a product in $\cat{A}$.
\end{definition}

\begin{proposition}[{\cite[Theorem~VIII.2.2]{category_theory_for_the_working_mathematician}}]
  In an additive category, the product of two objects is also a
  coproduct of those objects, and hence a biproduct.
\end{proposition}

If a category has all binary biproducts, then it must also have all
finite biproducts, and the existence of a zero object allows one to
make sense of empty biproducts in $\abcat{A}$ too.
This gives a slightly more precise working definition of additive categories.
We shall refer to finite biproducts in $\abcat{A}$ as direct sums and
use the same notation as for modules.

Many common functors one works with in practice interact with this
additive structure nicely.
The most important examples in module theory, the Hom and tensor
product functors, are group homomorphisms on hom-sets and preserve direct sums.
This motivates the following class of functors which appear
throughout \cref{chap_homological} and \cref{chap_derived_categories}.

\begin{definition}
  A functor $F: \cat{A} \to \cat{B}$ between additive categories is
  said to be \emph{additive} if it is a group homomorphism on hom-sets.
\end{definition}

\begin{lemma}[{\cite[Proposition~VIII.2.4]{category_theory_for_the_working_mathematician}}]
  \label{lemma_additive_func_commutes_with_dirsum}
  A functor $F: \abcat{A} \to \abcat{B}$ is additive if and only if
  it commutes with direct sums, i.e. $F(A \biprod B) \cong F(A)
  \biprod F(B)$ for any $A, \, B \in \cat{A}$.
\end{lemma}

The final properties left to look at are those which enable exact
sequences to be defined.
Recall that any module homomorphism $f: M \to N$ has a kernel and
image, as well as a cokernel and coimage defined by the quotient modules
\[
  \coker(f) := \quot{N}{\im(f)}
  \mathand
  \coim(f) := \quot{M}{\ker(f)}.
\]
By the first isomorphism theorem, there is a canonical isomorphism
$\coim(f) \cong \im(f)$, and hence a commutative diagram
\begin{equation*}
  \label{eq_factoring_of_module_hom}
  \begin{tikzcd}[cramped]
    {\ker(f)} & M & N & {\coker(f)}. \\
    & {\coim(f)} & {\im(f)}
    \arrow[hook, from=1-1, to=1-2]
    \arrow["f", from=1-2, to=1-3]
    \arrow[two heads, from=1-2, to=2-2]
    \arrow[two heads, from=1-3, to=1-4]
    \arrow["\cong", from=2-2, to=2-3]
    \arrow[hook, from=2-3, to=1-3]
  \end{tikzcd}
\end{equation*}
In particular, we observe that $f$ factors through its image and coimage.
We generalise this by way of the following pair of definitions.

\begin{definition}
  An additive category $\abcat{A}$ is said to be \emph{preabelian} if
  any morphism in $\abcat{A}$ has a kernel and a cokernel in $\abcat{A}$.
  We say that $\abcat{A}$ is \emph{abelian} if every monic in
  $\abcat{A}$ (resp. epic) is \emph{normal}, i.e. is the kernel
  (resp. cokernel) of a morphism in $\abcat{A}$.
\end{definition}

Using the universal properties of kernels and cokernels, the abelian
category condition may be refined to the more precise statement that
any monic is the kernel of its cokernel, and dually any epic is the
cokernel of its kernel.
Applying the categorical definitions rather than the usual
module-theoretic ones, one may verify that these statements are true
in $\Mod{R}$, which provides some a posteriori justification for
defining abelian categories in this way.

\iffalse
We can refine (AB2) to a more precise statement.
Given a monic $m$ in $\abcat{A}$, we have $m = \ker(f)$ for some morphism $f$.
Since $f m = 0$, $f$ factors through $\coker(m)$ by the universal
property of cokernels.
By definition $\coker(m) \circ m = 0$, and for any other morphism
$g$ satisfying $\coker(m) \circ g = 0$ it follows that $fg = 0$,
% \[
%     f g = f' \circ \coker(m) \circ g = 0,
% \]
so $g$ factors through $m$ by the universal property of kernels.
This shows that in fact $m = \ker(\coker(m))$, and by duality $e =
\coker(\ker(e))$ for any epic $e$ in $\abcat{A}$.
\fi

\iffalse
Given a monic $k: K \to A$ in $\abcat{A}$, let $(C, c)$ be a
cokernel of $k$ and suppose that $(K, k)$ is a kernel of the
morphism $f: A \to B$.
Then $fk = 0$ by construction, so $f$ factors through a unique
morphism $f': C \to B$ by the universal property of $(C, c)$.
We also have by construction that $ck = 0$, and for any other
morphism $g: D \to A$ satisfying $cg = 0$ it follows that $fg =
f'cg = 0$, so $g$ factors through a unique morphism $g': D \to K$
by the universal property of $(K, k)$.
Thus $(K, k)$ is also a kernel of $(C, c)$, i.e. any monic in
$\abcat{A}$ is a kernel of its cokernel.
This argument dualises to show also that any epic in $\abcat{A}$ is
a cokernel of its kernel.

As similar kernels and cokernels appear in relation to any morphism
in a (pre)abelian category, we give them familiar names.
\fi

Let us work towards developing a more convincing intuition.
In studying morphisms of preabelian categories, kernels of cokernels
and cokernels of kernels appear already, making for good analogues of
the image and coimage.
Once again, we leave it to the reader to check that these agree with
the module-theoretic concepts defined above.

\begin{definition}
  Let $f$ be a morphism in a preabelian category $\abcat{A}$.
  Then
  \[
    \im(f) := \ker(\coker(f))
    \mathand
    \coim(f) := \coker(\ker(f))
  \]
  are the categorical \emph{image} and \emph{coimage} of $f$ respectively.
\end{definition}

The upshot of this definition is that similar to the above diagram in
$\Mod{R}$, any morphism in a preabelian category canonically factors
through its image and coimage.

\begin{lemma}
  \label{lemma_exists_canonical_coim_to_im}
  Let $f$ be a morphism in a preabelian category $\abcat{A}$.
  Then there is a canonical morphism $\coim(f) \to \im(f)$ fitting
  into a commutative diagram
  \[
    \begin{tikzcd}[cramped]
      {\ker(f)} & A & B & {\coker(f)}. \\
      & {\coim(f)} & {\im(f)}
      \arrow[hook, from=1-1, to=1-2]
      \arrow["f", from=1-2, to=1-3]
      \arrow[two heads, from=1-2, to=2-2]
      \arrow[two heads, from=1-3, to=1-4]
      \arrow[dashed, from=2-2, to=2-3]
      \arrow[hook, from=2-3, to=1-3]
    \end{tikzcd}
  \]
  \vspace{-12pt}
\end{lemma}

\begin{proof}
  By the universal property of cokernels, $f$ must factor uniquely
  through a morphism $i: \coim(f) \to B$, so
  \[
    \coker(f) \circ i \circ \coim(f) = \coker(f) \circ f = 0.
  \]
  Recalling that all cokernels are epic, it follows that $i$ factors
  uniquely through a morphism $u: \coim(f) \to \im(f)$.
  Arguing dually, $f$ factors uniquely through a morphism $p: A \to
  \im(f)$ that in turn factors uniquely through a morphism $v:
  \coim(f) \to \im(f)$.
  By this uniqueness, $u$ and $v$ must coincide, so the morphism is
  indeed canonical.
  % Because $f \circ \ker(f) = 0$, the universal property of
  % cokernels implies that $f$ factors through a unique morphism $i:
  % \coim(f) \to B$, and
  % \[
  %     \coker(f) \circ i \circ \coim(f) = \coker(f) \circ f = 0.
  % \]
  % Since any cokernel is an epic, we must have $\coker(f) \circ i =
  % 0$, and by the universal property of kernels we conclude that $i$
  % (and thus $f$) factors through a unique morphism $u: \coim(f) \to \im(f)$.
  % This says that
  % \[
  %     f
  %     = i \circ \coim(f)
  %     = \im(f) \circ u \circ \coim(f).
  % \]
  % Arguing by duality starting from the fact that $\coker(f) \circ f
  % = 0$ shows that $f$ factors uniquely through a morphism $p: A \to
  % \im(f)$ that in turn factors through a unique morphism $v:
  % \coim(f) \to \im(f)$, so
  % \[
  %     f
  %     = \im(f) \circ p
  %     = \im(f) \circ v \circ \coim(f).
  % \]
  % The morphisms $u$ and $v$ coincide by uniqueness, so this
  % morphism is canonical.
\end{proof}

% Good resource:
% https://people.clas.ufl.edu/rcrew/files/homology-lect2.pdf

% Also, Freyd:
% http://www.tac.mta.ca/tac/reprints/articles/3/tr3.pdf

If $\abcat{A}$ is abelian, a tighter analogy with $\Mod{R}$ may be
drawn: they are preabelian categories where the first isomorphism theorem holds.
In many treatments, such as the original \cite{tohoku}, this
observation is actually taken as the definition.
It is instructive to show that these definitions are equivalent,
since doing so requires the following lemmas which give familiar
characterisations of morphisms.

\begin{lemma}[{\cite[Theorem~2.12]{freyd_abelian_categories}}]
  \label{lemma_abelian_category_isos}
  In an abelian category, any monic which is also an epic is an isomorphism.
\end{lemma}

\begin{lemma}[{\cite[Theorems~2.17 and 2.17*]{freyd_abelian_categories}}]
  \label{lemma_abelian_category_monos_and_epis}
  Let $f: A \to B$ be a morphism in an abelian category.
  Then $f$ is monic if and only if $\ker(f) = 0$ as an object, i.e.
  $\coim(f) = A$.
  Dually, $f$ is epic if and only if $\coker(f) = 0$, i.e. $\im(f) = B$.
\end{lemma}

\begin{proposition}
  Let $\abcat{A}$ be a preabelian category.
  Then $\abcat{A}$ is abelian if and only if the morphism $\coim(f)
  \to \im(f)$ of \cref{lemma_exists_canonical_coim_to_im} is an
  isomorphism for any $f \in \abcat{A}$.
\end{proposition}

\begin{proof}[Proof sketch]
  For the forward direction, one uses the abelian category condition
  and the universal properties of kernels and cokernels to show that
  in \cref{lemma_exists_canonical_coim_to_im}, $i$ is monic. By
  dualising this argument, we may also conclude that $p$ is an epic.
  This shows that $\coim(f) \to \im(f)$ is monic and epic, and hence
  an isomorphism by \cref{lemma_abelian_category_isos}.
  For the reverse direction, it is enough to prove that any monic $f:
  A \to B$ in $\abcat{A}$ is a kernel.
  To this end, one uses \cref{lemma_abelian_category_monos_and_epis}
  and the assumption to rewrite the commutative diagram in
  \cref{lemma_exists_canonical_coim_to_im} as
  \[
    \begin{tikzcd}[cramped]
      0 & A & B & {\coker(f)}, \\
      & A & {\im(f)}
      \arrow["0", hook, from=1-1, to=1-2]
      \arrow["f", hook, from=1-2, to=1-3]
      \arrow["\id"', two heads, from=1-2, to=2-2]
      \arrow[two heads, from=1-3, to=1-4]
      \arrow["\cong", from=2-2, to=2-3]
      \arrow[hook, from=2-3, to=1-3]
    \end{tikzcd}
  \]
  and then argues by universal properties that $f = \ker(\coker(f))$.
\end{proof}

\section{Working with abelian categories}

With the definition at hand, we now look at how an abelian category
can be reasoned about, once again with $\Mod{R}$ serving as inspiration.
Though several concepts that we will introduce here can be defined
just as well for less strictly defined categories, it is often the
case that they are most potent in the abelian setting.
As a first example, we may consider analogues of submodules and
quotient modules in any abelian category, which will be important for
defining \emph{(co)homology} in \cref{sect_complexes}.

\begin{definition}
  Let $\abcat{A}$ be an abelian category and fix an object $A \in \abcat{A}$.
  A \emph{subobject} of $A$ is the data of a monic $s: S \mono A$
  identified up to equivalence with all monics $t: T \mono A$ such
  that $s = t u$ for some isomorphism $u: S \cong T$.
  We use the suggestive notation $S \leq A$.
  \iffalse
  % We may define a partial order on the monics $* \mono A$ as
  % follows: given $s: S \mono A$ and $t: T \mono A$, we say that $s
  % \leq t$ if $s = t \phi$ for some monic $\phi: S \mono T$.
  % This induces an equivalence relation $s \sim t$ by $s \leq t$ and
  % $t \leq s$, and its equivalence classes are called \emph{subobjects} of $A$.
  \fi
  Moreover, we define $A/S := \coker(s)$ to be the \emph{quotient
  object} of $A$ by its subobject $S$.
\end{definition}

Exactly as promised at the beginning of this chapter, the language of
exact sequences can also be directly imported from $\Mod{R}$ to
abelian categories.

\begin{definition}
  An \emph{exact sequence} in an abelian category $\abcat{A}$ is a sequence
  \[
    \begin{tikzcd}[cramped]
      \cdots & {A_{i - 1}} & {A_{i}} & {A_{i + 1}} & \cdots
      \arrow[from=1-1, to=1-2]
      \arrow["{f_{i - 1}}", from=1-2, to=1-3]
      \arrow["{f_i}", from=1-3, to=1-4]
      \arrow[from=1-4, to=1-5]
    \end{tikzcd}
  \]
  such that $\im(f_{i - 1}) = \ker(f_i)$ for each $i$, and a
  \emph{short exact sequence} is one of the form
  \[
    \begin{tikzcd}[cramped]
      0 & A & B & C & 0.
      \arrow[from=1-1, to=1-2]
      \arrow["f", from=1-2, to=1-3]
      \arrow["g", from=1-3, to=1-4]
      \arrow[from=1-4, to=1-5]
    \end{tikzcd}
  \]
  In particular, $f$ is monic and $g$ is epic.
  \iffalse
  We say a short exact sequence \emph{splits} (or is \emph{split}) if
  there is a commutative diagram
  \[
    \begin{tikzcd}[cramped]
      0 & A & B & C & 0 \\
      0 & A & {A \oplus C} & C & 0
      \arrow[from=1-1, to=1-2]
      \arrow["f", from=1-2, to=1-3]
      \arrow["\id"', from=1-2, to=2-2]
      \arrow["g", from=1-3, to=1-4]
      \arrow["\cong"', from=1-3, to=2-3]
      \arrow[from=1-4, to=1-5]
      \arrow["\id"', from=1-4, to=2-4]
      \arrow[from=2-1, to=2-2]
      \arrow[hook, from=2-2, to=2-3]
      \arrow[two heads, from=2-3, to=2-4]
      \arrow[from=2-4, to=2-5]
    \end{tikzcd}
  \]
  in $\abcat{A}$ with exact rows.
  \fi
\end{definition}

\begin{definition}
  \label{def_exact_functors}
  An additive functor $F: \abcat{A} \to \abcat{B}$ between abelian
  categories is \emph{left exact} if it sends any short exact sequence
  \[
    \begin{tikzcd}[cramped]
      0 & A & B & C & 0
      \arrow[from=1-1, to=1-2]
      \arrow["f", from=1-2, to=1-3]
      \arrow["g", from=1-3, to=1-4]
      \arrow[from=1-4, to=1-5]
    \end{tikzcd}
  \]
  in $\abcat{A}$ to the left exact sequence
  \[
    \begin{tikzcd}[cramped]
      0 & F(A) & F(B) & F(C)
      \arrow[from=1-1, to=1-2]
      \arrow["F(f)", from=1-2, to=1-3]
      \arrow["F(g)", from=1-3, to=1-4]
    \end{tikzcd}
  \]
  in $\abcat{B}$.
  Similarly, $F$ is \emph{right exact} if it instead turns it into a
  right exact sequence
  \[
    \begin{tikzcd}[cramped]
      F(A) & F(B) & F(C) & 0.
      \arrow["F(f)", from=1-1, to=1-2]
      \arrow["F(g)", from=1-2, to=1-3]
      \arrow[from=1-3, to=1-4]
    \end{tikzcd}
  \]
  We say that $F$ is \emph{exact} if it is both left and right exact.
\end{definition}

\begin{example}
  \label{exmp_hom_is_left_exact}
  Module theory provides examples of each type of exactness.
  The functors $\Hom_R(M, -)$ and $- \tensor_R M$ are left exact and
  right exact respectively, which in the case where $R$ is
  commutative follows from the tensor-hom adjunction (since left
  adjoint functors are right exact and vice versa).
  Given a multiplicatively closed subset $S \subseteq R$, the
  localisation functor $\locln{(-)}{S}: \Mod{R} \to
  \Mod{\locln{R}{S}}$ is exact.
\end{example}

\begin{remark}
  \label{rem_exact_limit_colimit_pres}
  An equivalent way to define left and right exact functors (as in
    \cite[Section~VIII.3]{category_theory_for_the_working_mathematician},
  say) are as those which preserve all finite \emph{limits} and
  \emph{colimits} respectively.
  We will not define these categorical concepts here, so instead we
  simply note some important consequences.
  Since products and kernels are finite limits, these constructions
  are preserved by any left exact functor.
  Similarly, coproducts and cokernels are finite colimits, and so are
  preserved by right exact functors.
\end{remark}

Next, we introduce analogues of projective and injective modules.
Just as they are powerful in the study of modules, such objects are
indispensable in the study of abelian categories, as will be made
abundantly clear in the two following chapters.

\begin{definition}
  An object $P \in \abcat{A}$ is said to be \emph{projective} if for
  every epic $f: A \epi B$, any morphism $g: P \to B$ factors through $f$:
  \[
    \begin{tikzcd}[cramped]
      & P \\
      A & B.
      \arrow[dashed, from=1-2, to=2-1]
      \arrow["g", from=1-2, to=2-2]
      \arrow["f"', two heads, from=2-1, to=2-2]
    \end{tikzcd}
  \]
  Dually, an object $I \in \abcat{A}$ is said to be \emph{injective}
  if for every monic $f: A \mono B$, any morphism $g: A \to I$
  factors through $f$:
  \[
    \begin{tikzcd}[cramped]
      A & B \\
      I.
      \arrow["f", hook, from=1-1, to=1-2]
      \arrow["g"', from=1-1, to=2-1]
      \arrow[dashed, from=1-2, to=2-1]
    \end{tikzcd}
  \]
\end{definition}

The category $\abcat{A}$ clearly need not be abelian for the
preceding definitions to work, but if it is, the following familiar
characterisations become available.
The proof is similar (though more pedantic) to the argument one gives
for $\Mod{R}$, so we omit it.

\begin{proposition}
  Let $\abcat{A}$ be an abelian category.
  Then $P \in \abcat{A}$ is projective if and only if the functor
  $\Hom_{\abcat{A}}(P, -)$ is exact.
  Dually, $I \in \abcat{A}$ is injective if and only if the functor
  $\Hom_{\abcat{A}}(-, I)$ is exact.
\end{proposition}

\iffalse
\begin{lemma}[Splitting lemma]
  \label{lemma_splitting}
  In an abelian category, the following are equivalent:
  \begin{enumerate}
    \item
      The short exact sequence $0 \to A \stackrel{f}{\to} B
      \stackrel{g}{\to} C \to 0$ splits.

    \item
      There is a morphism $s: B \to A$ such that $sf = \id_{A}$.

    \item
      There is a morphism $t: C \to B$ such that $gu = \id_{C}$.
  \end{enumerate}
\end{lemma}

\begin{proposition}
  \label{prop_proj_inj_iff_ses_splits}
  If $\abcat{A}$ is abelian, then $P \in \abcat{A}$ is projective
  if and only if any short exact sequence ending at $P$ splits:
  \[
    \begin{tikzcd}[cramped]
      0 & A & B & P & 0.
      \arrow[from=1-1, to=1-2]
      \arrow[from=1-2, to=1-3]
      \arrow[from=1-3, to=1-4]
      \arrow[from=1-4, to=1-5]
    \end{tikzcd}
  \]
  Dually, $I \in \abcat{A}$ is injective if and only if any short
  exact sequence starting at $I$ splits:
  \[
    \begin{tikzcd}[cramped]
      0 & I & B & C & 0.
      \arrow[from=1-1, to=1-2]
      \arrow[from=1-2, to=1-3]
      \arrow[from=1-3, to=1-4]
      \arrow[from=1-4, to=1-5]
    \end{tikzcd}
  \]
\end{proposition}
\fi

Recall that in $\Mod{R}$, any $R$-module $M$ admits a surjection $F
\epi M$ with $F$ free (a fortiori, projective) and an injection $M
\mono I$ with $I$ injective (a so-called \emph{injective envelope} of $M$).
These properties end up being quite special in the broader context of
abelian categories.

\begin{definition}
  An abelian category $\abcat{A}$ has \emph{enough projectives} if
  for any object $A \in \abcat{A}$, there exists an epic $P \epi A$
  with $P$ projective.
  Dually, $\abcat{A}$ has \emph{enough injectives} if for any $A$
  there exists a monic $A \mono I$ with $I$ injective.
\end{definition}

We do not even have to leave the realm of module theory to find
abelian categories which fail one or both of these conditions.
This serves also as a preliminary warning that abelian categories can
stray far from the categories of modules we originally used to
motivate their definition, only guaranteeing we have similar
exactness machinery.

\begin{example}
  \label{ex_finab}
  There are no non-zero finite abelian groups which are injective or
  projective, so the category $\FinAb$ certainly has neither enough
  projectives or injectives.
  Broadening our scope to finitely generated abelian groups (i.e. to
  $\mod{\bb{Z}}$) allows us to recover enough projectives, but the
  lack of any injectives persists.
\end{example}

To conclude this chapter, we discuss common ways of proving
statements about an abelian category $\abcat{A}$.
With exact sequences, one is also able to recover an ensemble of
lemmas discussed in
\cite[Section~VIII.4]{category_theory_for_the_working_mathematician},
such as the \emph{five lemma} and the \emph{snake lemma}.
These lemmas go hand in hand with the proof technique of
\emph{diagram chasing} (c.f.
\cite[Section~1.6]{category_theory_in_context}), which is exploited
to maximum effect in homological algebra.

Done purely at the level of categorical abstractions, many of these
proofs, especially those involving diagram chases, are an obsessively
technical sport.
If $\abcat{A}$ is \emph{concrete} (i.e. there is a faithful,
forgetful functor $\abcat{A} \to \Set$), many of these categorical
abstractions can be sidestepped in favour of working directly with
elements of objects.
Since $\Mod{R}$ is concrete, this is often the way one proves things
in the module case.
A deep theorem of abelian categories shows that this method of proof
can still be valid in some instances, since morally such categories
`are' just a category of certain modules.
The proof of this theorem is non-trivial, and we defer the details to
its coverage in \cite[Section~1.6]{weibel}.

\begin{definition}
  A category $\cat{C}$ is said to be \emph{small} if its class of
  objects is a set, and \emph{locally small} if every hom-set of
  $\cat{C}$ is a set.
\end{definition}

\begin{theorem}[Freyd-Mitchell embedding theorem]
  If $\abcat{A}$ is a small abelian category, then there exists a
  ring $R$ and a fully faithful, exact functor $\abcat{A} \to \Mod{R}$.
  In particular, $\abcat{A}$ is equivalent to a full subcategory of $\Mod{R}$.
\end{theorem}

The most interesting abelian categories are rarely small, but some
things one wishes to prove are `local' in the sense that it is enough
to consider a small abelian subcategory amenable to this theorem
(e.g. one that contains all of the objects in some diagram).
One case where this strategy potentially fails is when proving
statements about projectives and injectives, since it may not be true
that the projectives and injectives in a subcategory of $\Mod{R}$ are
the usual projective and injective $R$-modules.

