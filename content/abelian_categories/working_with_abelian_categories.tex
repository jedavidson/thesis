\section{Working with abelian categories}

With the definition at hand, we now look at how an abelian category
can be reasoned about, once again with $\Mod{R}$ serving as inspiration.
Though several concepts that we will introduce here can be defined
just as well for less strictly defined categories, it is often the
case that they are most potent in the abelian setting.
As a first example, we may consider analogues of submodules and
quotient modules in any abelian category, which will be important for
defining \emph{(co)homology} in \cref{sect_complexes}.

\begin{definition}
  Let $\abcat{A}$ be an abelian category and fix an object $A \in \abcat{A}$.
  A \emph{subobject} of $A$ is the data of a monic $s: S \mono A$
  identified up to equivalence with all monics $t: T \mono A$ such
  that $s = t u$ for some isomorphism $u: S \cong T$.
  We use the suggestive notation $S \leq A$.
  \iffalse
  % We may define a partial order on the monics $* \mono A$ as
  % follows: given $s: S \mono A$ and $t: T \mono A$, we say that $s
  % \leq t$ if $s = t \phi$ for some monic $\phi: S \mono T$.
  % This induces an equivalence relation $s \sim t$ by $s \leq t$ and
  % $t \leq s$, and its equivalence classes are called \emph{subobjects} of $A$.
  \fi
  Moreover, we define $A/S := \coker(s)$ to be the \emph{quotient
  object} of $A$ by its subobject $S$.
\end{definition}

Exactly as promised at the beginning of this chapter, the language of
exact sequences can also be directly imported from $\Mod{R}$ to
abelian categories.

\begin{definition}
  An \emph{exact sequence} in an abelian category $\abcat{A}$ is a sequence
  \[
    \begin{tikzcd}[cramped]
      \cdots & {A_{i - 1}} & {A_{i}} & {A_{i + 1}} & \cdots
      \arrow[from=1-1, to=1-2]
      \arrow["{f_{i - 1}}", from=1-2, to=1-3]
      \arrow["{f_i}", from=1-3, to=1-4]
      \arrow[from=1-4, to=1-5]
    \end{tikzcd}
  \]
  such that $\im(f_{i - 1}) = \ker(f_i)$ for each $i$, and a
  \emph{short exact sequence} is one of the form
  \[
    \begin{tikzcd}[cramped]
      0 & A & B & C & 0.
      \arrow[from=1-1, to=1-2]
      \arrow["f", from=1-2, to=1-3]
      \arrow["g", from=1-3, to=1-4]
      \arrow[from=1-4, to=1-5]
    \end{tikzcd}
  \]
  In particular, $f$ is monic and $g$ is epic.
  \iffalse
  We say a short exact sequence \emph{splits} (or is \emph{split}) if
  there is a commutative diagram
  \[
    \begin{tikzcd}[cramped]
      0 & A & B & C & 0 \\
      0 & A & {A \oplus C} & C & 0
      \arrow[from=1-1, to=1-2]
      \arrow["f", from=1-2, to=1-3]
      \arrow["\id"', from=1-2, to=2-2]
      \arrow["g", from=1-3, to=1-4]
      \arrow["\cong"', from=1-3, to=2-3]
      \arrow[from=1-4, to=1-5]
      \arrow["\id"', from=1-4, to=2-4]
      \arrow[from=2-1, to=2-2]
      \arrow[hook, from=2-2, to=2-3]
      \arrow[two heads, from=2-3, to=2-4]
      \arrow[from=2-4, to=2-5]
    \end{tikzcd}
  \]
  in $\abcat{A}$ with exact rows.
  \fi
\end{definition}

\begin{definition}
  \label{def_exact_functors}
  An additive functor $F: \abcat{A} \to \abcat{B}$ between abelian
  categories is \emph{left exact} if it sends any short exact sequence
  \[
    \begin{tikzcd}[cramped]
      0 & A & B & C & 0
      \arrow[from=1-1, to=1-2]
      \arrow["f", from=1-2, to=1-3]
      \arrow["g", from=1-3, to=1-4]
      \arrow[from=1-4, to=1-5]
    \end{tikzcd}
  \]
  in $\abcat{A}$ to the left exact sequence
  \[
    \begin{tikzcd}[cramped]
      0 & F(A) & F(B) & F(C)
      \arrow[from=1-1, to=1-2]
      \arrow["F(f)", from=1-2, to=1-3]
      \arrow["F(g)", from=1-3, to=1-4]
    \end{tikzcd}
  \]
  in $\abcat{B}$.
  Similarly, $F$ is \emph{right exact} if it instead turns it into a
  right exact sequence
  \[
    \begin{tikzcd}[cramped]
      F(A) & F(B) & F(C) & 0.
      \arrow["F(f)", from=1-1, to=1-2]
      \arrow["F(g)", from=1-2, to=1-3]
      \arrow[from=1-3, to=1-4]
    \end{tikzcd}
  \]
  We say that $F$ is \emph{exact} if it is both left and right exact.
\end{definition}

\begin{example}
  \label{exmp_hom_is_left_exact}
  Module theory provides examples of each type of exactness.
  The functors $\Hom_R(M, -)$ and $- \tensor_R M$ are left exact and
  right exact respectively, which in the case where $R$ is
  commutative follows from the tensor-hom adjunction (since left
  adjoint functors are right exact and vice versa).
  Given a multiplicatively closed subset $S \subseteq R$, the
  localisation functor $\locln{(-)}{S}: \Mod{R} \to
  \Mod{\locln{R}{S}}$ is exact.
\end{example}

\begin{remark}
  \label{rem_exact_limit_colimit_pres}
  An equivalent way to define left and right exact functors (as in
    \cite[Section~VIII.3]{category_theory_for_the_working_mathematician},
  say) are as those which preserve all finite \emph{limits} and
  \emph{colimits} respectively.
  We will not define these categorical concepts here, so instead we
  simply note some important consequences.
  Since products and kernels are finite limits, these constructions
  are preserved by any left exact functor.
  Similarly, coproducts and cokernels are finite colimits, and so are
  preserved by right exact functors.
\end{remark}

Next, we introduce analogues of projective and injective modules.
Just as they are powerful in the study of modules, such objects are
indispensable in the study of abelian categories, as will be made
abundantly clear in the two following chapters.

\begin{definition}
  An object $P \in \abcat{A}$ is said to be \emph{projective} if for
  every epic $f: A \epi B$, any morphism $g: P \to B$ factors through $f$:
  \[
    \begin{tikzcd}[cramped]
      & P \\
      A & B.
      \arrow[dashed, from=1-2, to=2-1]
      \arrow["g", from=1-2, to=2-2]
      \arrow["f"', two heads, from=2-1, to=2-2]
    \end{tikzcd}
  \]
  Dually, an object $I \in \abcat{A}$ is said to be \emph{injective}
  if for every monic $f: A \mono B$, any morphism $g: A \to I$
  factors through $f$:
  \[
    \begin{tikzcd}[cramped]
      A & B \\
      I.
      \arrow["f", hook, from=1-1, to=1-2]
      \arrow["g"', from=1-1, to=2-1]
      \arrow[dashed, from=1-2, to=2-1]
    \end{tikzcd}
  \]
\end{definition}

The category $\abcat{A}$ clearly need not be abelian for the
preceding definitions to work, but if it is, the following familiar
characterisations become available.
The proof is similar (though more pedantic) to the argument one gives
for $\Mod{R}$, so we omit it.

\begin{proposition}
  Let $\abcat{A}$ be an abelian category.
  Then $P \in \abcat{A}$ is projective if and only if the functor
  $\Hom_{\abcat{A}}(P, -)$ is exact.
  Dually, $I \in \abcat{A}$ is injective if and only if the functor
  $\Hom_{\abcat{A}}(-, I)$ is exact.
\end{proposition}

\iffalse
\begin{lemma}[Splitting lemma]
  \label{lemma_splitting}
  In an abelian category, the following are equivalent:
  \begin{enumerate}
    \item
      The short exact sequence $0 \to A \stackrel{f}{\to} B
      \stackrel{g}{\to} C \to 0$ splits.

    \item
      There is a morphism $s: B \to A$ such that $sf = \id_{A}$.

    \item
      There is a morphism $t: C \to B$ such that $gu = \id_{C}$.
  \end{enumerate}
\end{lemma}

\begin{proposition}
  \label{prop_proj_inj_iff_ses_splits}
  If $\abcat{A}$ is abelian, then $P \in \abcat{A}$ is projective
  if and only if any short exact sequence ending at $P$ splits:
  \[
    \begin{tikzcd}[cramped]
      0 & A & B & P & 0.
      \arrow[from=1-1, to=1-2]
      \arrow[from=1-2, to=1-3]
      \arrow[from=1-3, to=1-4]
      \arrow[from=1-4, to=1-5]
    \end{tikzcd}
  \]
  Dually, $I \in \abcat{A}$ is injective if and only if any short
  exact sequence starting at $I$ splits:
  \[
    \begin{tikzcd}[cramped]
      0 & I & B & C & 0.
      \arrow[from=1-1, to=1-2]
      \arrow[from=1-2, to=1-3]
      \arrow[from=1-3, to=1-4]
      \arrow[from=1-4, to=1-5]
    \end{tikzcd}
  \]
\end{proposition}
\fi

Recall that in $\Mod{R}$, any $R$-module $M$ admits a surjection $F
\epi M$ with $F$ free (a fortiori, projective) and an injection $M
\mono I$ with $I$ injective (a so-called \emph{injective envelope} of $M$).
These properties end up being quite special in the broader context of
abelian categories.

\begin{definition}
  An abelian category $\abcat{A}$ has \emph{enough projectives} if
  for any object $A \in \abcat{A}$, there exists an epic $P \epi A$
  with $P$ projective.
  Dually, $\abcat{A}$ has \emph{enough injectives} if for any $A$
  there exists a monic $A \mono I$ with $I$ injective.
\end{definition}

We do not even have to leave the realm of module theory to find
abelian categories which fail one or both of these conditions.
This serves also as a preliminary warning that abelian categories can
stray far from the categories of modules we originally used to
motivate their definition, only guaranteeing we have similar
exactness machinery.

\begin{example}
  \label{ex_finab}
  There are no non-zero finite abelian groups which are injective or
  projective, so the category $\FinAb$ certainly has neither enough
  projectives or injectives.
  Broadening our scope to finitely generated abelian groups (i.e. to
  $\mod{\bb{Z}}$) allows us to recover enough projectives, but the
  lack of any injectives persists.
\end{example}

To conclude this chapter, we discuss common ways of proving
statements about an abelian category $\abcat{A}$.
With exact sequences, one is also able to recover an ensemble of
lemmas discussed in
\cite[Section~VIII.4]{category_theory_for_the_working_mathematician},
such as the \emph{five lemma} and the \emph{snake lemma}.
These lemmas go hand in hand with the proof technique of
\emph{diagram chasing} (c.f.
\cite[Section~1.6]{category_theory_in_context}), which is exploited
to maximum effect in homological algebra.

Done purely at the level of categorical abstractions, many of these
proofs, especially those involving diagram chases, are an obsessively
technical sport.
If $\abcat{A}$ is \emph{concrete} (i.e. there is a faithful,
forgetful functor $\abcat{A} \to \Set$), many of these categorical
abstractions can be sidestepped in favour of working directly with
elements of objects.
Since $\Mod{R}$ is concrete, this is often the way one proves things
in the module case.
A deep theorem of abelian categories shows that this method of proof
can still be valid in some instances, since morally such categories
`are' just a category of certain modules.
The proof of this theorem is non-trivial, and we defer the details to
its coverage in \cite[Section~1.6]{weibel}.

\begin{definition}
  A category $\cat{C}$ is said to be \emph{small} if its class of
  objects is a set, and \emph{locally small} if every hom-set of
  $\cat{C}$ is a set.
\end{definition}

\begin{theorem}[Freyd-Mitchell embedding theorem]
  If $\abcat{A}$ is a small abelian category, then there exists a
  ring $R$ and a fully faithful, exact functor $\abcat{A} \to \Mod{R}$.
  In particular, $\abcat{A}$ is equivalent to a full subcategory of $\Mod{R}$.
\end{theorem}

The most interesting abelian categories are rarely small, but some
things one wishes to prove are `local' in the sense that it is enough
to consider a small abelian subcategory amenable to this theorem
(e.g. one that contains all of the objects in some diagram).
One case where this strategy potentially fails is when proving
statements about projectives and injectives, since it may not be true
that the projectives and injectives in a subcategory of $\Mod{R}$ are
the usual projective and injective $R$-modules.
