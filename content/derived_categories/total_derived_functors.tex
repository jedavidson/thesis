\section{Total derived functors}
\label{sect_derfunc}

Aside from allowing one to identify an object with all of its
resolutions, another tangible sense in which derived categories are
`better' for homological algebra is their relationship with the
classical derived functors introduced in \cref{sect_classical_derfunc}.
In this section, we will explore this and show that, in certain
cases, we may consolidate these functors into one \emph{total derived
functor} between derived categories.
\iffalse
For our purposes, it will be sufficient only to go far enough to see
what happens for Ext and Tor.
\fi

To begin, we shall consider the obstructions to an additive functor
$F: \abcat{A} \to \abcat{B}$ between abelian categories inducing a
functor between derived categories.
Any such $F$ extends to an additive functor $\Ch{F}: \Ch{\abcat{A}}
\to \Ch{\abcat{B}}$, and thanks to
\cref{lemma_add_func_preserves_homotopy_and_cones}, it follows that
$\Ch{F}$ extends further to an additive, exact triangulated functor
$\homcat{F}: \homcat{\abcat{A}} \to \homcat{\abcat{B}}$ in the obvious way.
As before, we will commit a slight abuse of notation by conflating
this functor with $F$.
The difficulties begin when we try to extend $F$ to a functor
$\dercat{\abcat{A}} \to \dercat{\abcat{B}}$, that is, to find a
functor $G$ which fits into a commutative diagram
\[
  \begin{tikzcd}
    \homcat{\abcat{A}} & \homcat{\abcat{B}} \\
    \dercat{\abcat{A}} & \dercat{\abcat{B}}.
    \arrow["{\homcat{F}}", from=1-1, to=1-2]
    \arrow["{Q_{\abcat{A}}}"', from=1-1, to=2-1]
    \arrow["{Q_{\abcat{B}}}", from=1-2, to=2-2]
    \arrow["{G}"', dashed, from=2-1, to=2-2]
  \end{tikzcd}
\]
If we are fortunate enough that $F$ preserves quasi-isomorphisms,
then this is possible:
the composite functor $Q_{\abcat{B}} \circ F: \homcat{\abcat{A}} \to
\dercat{\abcat{B}}$ also preserves quasi-isomorphisms, so the
universal property of $Q_{\abcat{A}}$ ensures that such a functor $G$ exists.
By \cref{cor_quasi_isom_has_acyclic_cone} and
\cref{lemma_add_func_preserves_homotopy_and_cones}, we indeed find
ourselves in this situation when $F$ is exact, and moreover it turns
out that $G$ will be an exact triangulated functor.
Under more practical assumptions, we cannot expect that this
na\"{i}ve construction will go through, nor that such a simple
commutative diagram will characterise the resulting functor.

In light of our partial progress, we may as well assume we are given
an exact triangulated functor $F: \homcat{\abcat{A}} \to
\homcat{\abcat{B}}$ as a starting point, which subsumes the typical
case of a left and right exact functor defined at the level of
abelian categories.
Going forward, we will also always need to impose some kind of
boundedness condition on input complexes.

\begin{definition}
  \label{def_total_derfunc}
  Let $F: \homcat[*]{\abcat{A}} \to \homcat{\abcat{B}}$ be an exact
  triangulated functor.
  A \emph{total right derived functor} is an exact triangulated
  functor $\rightderfunc{F}: \dercat[*]{\abcat{A}} \to
  \dercat{\abcat{B}}$ equipped with a natural transformation $\xi:
  Q_{\mathcal{B}} \circ F \Rightarrow \rightderfunc{F} \circ
  Q_{\mathcal{A}}^*$ satisfying the following universal property:
  if $G: \dercat[*]{\abcat{A}} \to \dercat{\abcat{B}}$ is another
  exact triangulated functor, then any natural transformation $\zeta:
  Q_{\mathcal{B}} \circ F \Rightarrow G \circ Q_{\mathcal{A}}^*$
  factors uniquely through $\xi$.
  Explicitly, this means that there is a unique natural
  transformation $\eta: \rightderfunc{F} \Rightarrow G$ such that
  \[
    \zeta_{\cochaincomp{A}}
    = (\eta \circ Q_{\abcat{A}}^*)_{\cochaincomp{A}} \circ \xi_{\cochaincomp{A}}
  \]
  for each $\cochaincomp{A} \in \homcat[*]{\abcat{A}}$, or
  equivalently that the functor diagram
  \[
    \begin{tikzcd}
      {Q_{\abcat{B}} \circ F} & {G \circ Q_{\abcat{A}}^*}. \\
      {\rightderfunc{F} \circ Q_{\abcat{A}}^*} & {}
      \arrow["\zeta", Rightarrow, from=1-1, to=1-2]
      \arrow["\xi"', Rightarrow, from=1-1, to=2-1]
      \arrow["{\eta \, \circ \, Q_{\abcat{A}}^*}"', Rightarrow,
      from=2-1, to=1-2]
    \end{tikzcd}
  \]
  commutes.
  Dually, a \emph{total left derived functor} of $F$ is an exact
  triangulated functor $\leftderfunc{F}: \dercat[*]{\abcat{A}} \to
  \dercat{\abcat{B}}$ and a natural transformation $\xi:
  \leftderfunc{F} \circ Q_{\mathcal{A}}^* \nattrans Q_{\mathcal{B}}
  \circ F$ satisfying the universal property that for any exact
  triangulated functor $G: \dercat[*]{\abcat{A}} \to
  \dercat{\abcat{B}}$, any natural transformation $\zeta: G \circ
  Q_{\mathcal{A}}^* \nattrans Q_{\mathcal{B}} \circ F$ factors through $\xi$.
\end{definition}

In the case where we are initially only given an additive functor $F:
\abcat{A} \to \abcat{B}$, we will also write $\rightderfunc{F}$ and
$\leftderfunc{F}$ for the total derived functors of its induced exact
triangulated functor $\homcat[*]{F}: \homcat[*]{\abcat{A}} \to
\homcat{\abcat{B}}$ as above.

Once again, this is a non-constructive definition.
At the very least, the universal property implies that if a total
left or right derived functor of $F$ exists, it is unique up to
natural isomorphism.
In the presence of stronger assumptions, it turns out that there is a
principled way to go about actually constructing these functors.

\begin{proposition}
  \label{prop_exists_total_derfunc}
  Let $F: \homcat[+]{\abcat{A}} \to \homcat{\abcat{B}}$ be an exact
  triangulated functor, and suppose that $\abcat{A}$ has enough injectives.
  Fix a quasi-inverse $U^+$ of the equivalence $T^+$ in
  \cref{prop_bounded_below_dercat_equiv}.
  Then the composite functor
  \[
    \begin{tikzcd}
      {\rightderfunc[+]{F}: {\dercat[+]{\abcat{A}}}} &
      {\homcat[+]{\abcat{I}_{\abcat{A}}}} & {\homcat[+]{\abcat{B}}} &
      {\dercat[+]{\abcat{B}}}
      \arrow["U^+", from=1-1, to=1-2]
      \arrow["F", from=1-2, to=1-3]
      \arrow["{\dercatlocln[+]{\abcat{B}}}", from=1-3, to=1-4]
    \end{tikzcd}
  \]
  is a total right derived functor of $F$ on ${\dercat[+]{\abcat{A}}}$.
  \iffalse
  Dually, given an exact triangulated functor $F:
  \homcat[-]{\abcat{A}} \to \homcat{\abcat{B}}$ and enough
  projectives in $\abcat{A}$, the composite functor
  \[
    \begin{tikzcd}
      {\leftderfunc[-]{F}: {\dercat[-]{\abcat{A}}}} &
      {\homcat[-]{\abcat{P}_{\abcat{A}}}} & {\homcat[-]{\abcat{B}}} &
      {\dercat[-]{\abcat{B}}}
      \arrow["\equiv", from=1-1, to=1-2]
      \arrow["F", from=1-2, to=1-3]
      \arrow["{\dercatlocln[-]{\abcat{B}}}", from=1-3, to=1-4]
    \end{tikzcd}
  \]
  is a total left derived functor of $F$.
  \fi
\end{proposition}

\begin{proof}
  The most important data for our purposes is the functor
  $\rightderfunc[+]{F}$ itself, but it technically remains to specify
  a natural transformation $\xi: \dercatlocln[+]{\abcat{A}} \circ F
  \nattrans (\dercatlocln[+]{\abcat{B}} \circ F \circ U^+) \circ
  \dercatlocln[+]{\abcat{A}}$.
  We defer the details of this to \cite[Theorem~10.5.6]{weibel}.
\end{proof}

\begin{corollary}
  \label{cor_derfunc_applied_to_proj_inj_complexes}
  Retain the hypotheses of \cref{prop_exists_total_derfunc}.
  If $\cochaincomp{I}$ is a bounded below complex of injectives, then
  $\rightderfunc[+]{F}(\cochaincomp{I}) \cong
  (\dercatlocln[+]{\abcat{B}} \circ F)(\cochaincomp{I})$.
\end{corollary}

The dual statements for total left derived functors naturally hold.
This construction provides two great technical advantages over the
classical approach to derived functors.
The first that we have already mentioned is that in the right
circumstances, the total derived functors capture the ensemble of
classical derived functors from \cref{sect_classical_derfunc}.
In fact, some authors (e.g. \cite{huybrechts}) take the following as
definitions.

\begin{proposition}
  \label{prop_cohomology_of_totderfunc}
  Suppose that $F: \abcat{A} \to \abcat{B}$ is additive.
  Then under suitable conditions on $F$ and $\abcat{A}$ ensuring that
  $\rightderfunc[+]{F}$ or $\leftderfunc[-]{F}$ exist, the $i$th
  classical derived functors of $F$ at an object $A \in \abcat{A}$
  can be recovered as the cohomology
  \[
    R^iF(A) \cong H^i \rightderfunc[+]{F}(A)
    \mathand
    L_iF(A) \cong H^{-i} \leftderfunc[-]{F}(A)
  \]
  when $F$ is left or right exact respectively.
  Naturally, one first embeds objects from $\abcat{A}$ into the
  derived category using the fully faithful functor described in
  \cref{sect_localising_homcat}.
\end{proposition}

\begin{proof}
  Given an injective resolution $A \to \cochaincomp{I}$, we have $A
  \cong \cochaincomp{I}$ as objects in ${\dercat[+]{\abcat{A}}}$.
  Then by \cref{cor_derfunc_applied_to_proj_inj_complexes} and the
  fact that $\dercatlocln[+]{\abcat{B}}$ is the identity on objects, one has
  \[
    H^i \rightderfunc[+]{F}(A)
    \cong H^i \rightderfunc[+]{F}(\cochaincomp{I})
    \cong H^i (F(\cochaincomp{I}))
    = R^i F(A).
  \]
  One argues similarly for the claim in the left derived functor case.
\end{proof}

The other technical advantage is that in suitable situations, derived
functors at the level of derived categories may be composed in a far
simpler manner than derived functors at the level of abelian
categories, which normally requires the use of \emph{spectral sequences}.
The following result shows the necessary hypotheses for composing
total right derived functors, but the dual assertion for total left
derived functors naturally holds.

\begin{proposition}
  \label{prop_total_derfunc_comp}
  Suppose that $F: \homcat[+]{\abcat{A}} \to \homcat[+]{\abcat{B}}$
  and $G: \homcat[+]{\abcat{B}} \to \homcat[+]{\abcat{C}}$ are exact
  triangulated functors for abelian categories $\abcat{A}$,
  $\abcat{B}$ and $\abcat{C}$.
  Let $\cat{I}_{\abcat{A}}$ and $\cat{I}_{\abcat{B}}$ be the
  subcategories of injective objects in $\abcat{A}$ and $\abcat{B}$.
  Then if $\abcat{A}$ and $\abcat{B}$ have enough injectives and $F$
  sends $\abcat{I}_{\abcat{A}}$ into $\cat{I}_{\abcat{B}}$, there is
  a natural isomorphism $\rightderfunc[+](G \circ F) \nattrans
  \rightderfunc[+]{G} \circ \rightderfunc[+]{F}$.
\end{proposition}

\begin{proof}
  Let $\xi_F$, $\xi_G$ and $\xi_{G \circ F}$ be the natural
  transformations associated to the derived functors
  $\rightderfunc[+]{F}$, $\rightderfunc[+]{G}$ and
  $\rightderfunc[+]{(G \circ F)}$ respectively.
  The commutative functor diagram
  \[
    \begin{tikzcd}[cramped]
      {Q_{\abcat{C}}^+ \circ G \circ F} && {\rightderfunc[+]{G} \circ
      Q_{\abcat{B}}^+ \circ F} \\
      \\
      {\rightderfunc[+]{(G \circ F)} \circ Q_{\abcat{A}}^+} &&
      {(\rightderfunc[+]{G}) \circ (\rightderfunc[+]{F}) \circ Q_{\abcat{A}}^+},
      \arrow["{\xi_G \, \circ \, F}", Rightarrow, from=1-1, to=1-3]
      \arrow["{\xi_{G \circ F}}"', Rightarrow, from=1-1, to=3-1]
      \arrow["{\rightderfunc[+]{G} \, \circ \, \xi_F}", Rightarrow,
      from=1-3, to=3-3]
      \arrow["{\eta \, \circ \, Q_{\abcat{A}}^+}"', Rightarrow,
      dashed, from=3-1, to=3-3]
    \end{tikzcd}
  \]
  is completed by a natural transformation $\eta: \rightderfunc[+]{(G
  \circ F)} \nattrans (\rightderfunc[+]{G}) \circ
  (\rightderfunc[+]{F})$ thanks to the universal property of
  $\rightderfunc[+]{(G \circ F)}$.
  Now, take any $\cochaincomp{A} \in \dercat[+]{\abcat{A}}$ and use
  \cref{prop_bounded_below_dercat_equiv} to find a complex of
  injectives $\cochaincomp{I} \in \homcat[+]{\cat{I}_{\abcat{A}}}$
  isomorphic to $\cochaincomp{A}$ in $\dercat[+]{\abcat{A}}$.
  That the component $(\eta \circ Q_{\abcat{A}}^+)_{\cochaincomp{I}}$
  is an isomorphism follows from the calculation
  \begin{align*}
    \rightderfunc[+]{(G \circ F)(\dercatlocln[+]{\abcat{A}}(\cochaincomp{I}))}
    &= (\dercatlocln[+]{\abcat{C}} \circ G \circ F)(\cochaincomp{I}) \\
    &\cong (\rightderfunc[+]{G} \circ \dercatlocln[+]{\abcat{B}}
    \circ F)(\cochaincomp{I}) \\
    &\cong (\rightderfunc[+]{G} \circ
    \rightderfunc[+]{F})(\dercatlocln[+]{\abcat{A}}(\cochaincomp{I}))
  \end{align*}
  by \cref{cor_derfunc_applied_to_proj_inj_complexes}.
  The component $\eta_{\cochaincomp{A}}$ factors through the
  isomorphisms $\cochaincomp{A} \cong \cochaincomp{I}$ and $(\eta
  \circ Q_{\abcat{A}}^+)_{\cochaincomp{I}}$, so we conclude that
  $\eta_{\cochaincomp{A}}$ is also an isomorphism.
\end{proof}

\begin{remark}
  \label{rem_total_derfunc_computing_acyclics}
  Similar to \cref{rem_computing_with_F_acyclics}, the total derived
  functor of an exact triangulated functor $F$ can also be
  constructed instead using complexes of $F$-acyclic objects
  $\cochaincomp{X}$, i.e. such that $H^i(F(\cochaincomp{X})) = 0$ for all $i$.
  As usual, complexes of projectives or injectives are $F$-acyclic
  for every choice of $F$.
  Assuming that every object of $\homcat[*]{\abcat{A}}$ is
  quasi-isomorphic to some $F$-acyclic complex, one can modify the
  proofs of the preceding results so that injective assumptions can
  be suitably replaced with $F$-acyclic assumptions.
  In the statement of \cref{prop_total_derfunc_comp} in particular,
  it is sufficient that $F$ sends complexes of $F$-acyclics to $G$-acyclics.
\end{remark}

We now turn our attention to upgrading Ext and Tor to total derived functors.
It is perfectly valid to take derived functors of the usual Hom and
tensor product functors directly, but it would be most sensible to
work with a more general variant of these functors.
To see why, recall that $\Hom_{\abcat{A}}$ is a covariant bifunctor
\[
  \Hom_{\abcat{A}}(-, -): \op{\abcat{A}} \times \abcat{A} \to \Ab
\]
and notice that we may think of the induced triangulated exact
functor of $\Hom_{\abcat{A}}(A, -)$ on the homotopy category
$\homcat[+]{\abcat{A}}$ also as a covariant bifunctor
\[
  \Hom_{\abcat{A}}(-, -): \op{\abcat{A}} \times \homcat[+]{\abcat{A}}
  \to \homcat{\Ab}.
\]
A similar observation holds for the tensor product functor.
We are thus compelled us to define variants of these functors that
instead take complexes in \emph{both} arguments, and which of course
also induces a triangulated exact functor on $\homcat[*]{\abcat{A}}$.
We will do this by means of constructing an appropriate double
complex and then totalising it.

\begin{definition}
  Let $\cochaincomp{A}$ and $\cochaincomp{B}$ be cochain complexes in
  $\abcat{A}$.
  Consider the double complex $\cochainbicomp{C}$ of abelian groups defined by
  \[
    C^{p, q} = \Hom_{\abcat{A}}(A^{-p}, B^q),
  \]
  with horizontal and vertical differentials defined for each $f:
  A^{-p} \to B^q$ by
  \[
    d_h^{p, q}(f) = d_B \circ f
    \mathand
    d_v^{p, q}(f) = (-1)^{p + q + 1} f \circ d_A.
  \]
  The totalisation $\Hom^{\bullet}_{\abcat{A}}(\cochaincomp{A},
  \cochaincomp{B}) = \totprodcomp{\cochainbicomp{C}}$ is the
  \emph{internal Hom} of $\cochaincomp{A}$ and $\cochaincomp{B}$.
\end{definition}

The sign in $d_v$ is convenient, because with it the cohomology of
internal Hom classifies cochain maps up to homotopy equivalence and shifts.
That is,
\[
  H^n(\Hom^{\bullet}_{\abcat{A}}(\cochaincomp{A}, \cochaincomp{B}))
  \cong
  \Hom_{\dercat{\abcat{A}}}(\cochaincomp{A}, \cochaincomp{B}[n]).
\]
This is explained in more detail in \cite[Section~10.7]{weibel}.
Moreover, since the functor $\Hom_{\abcat{A}}(-, B)$ is contravariant,
% and commutes with products of $\abcat{A}$, <-- wtf does this mean?
it makes sense for the internal Hom to mimic the same contravariance
in first argument, which here will turn a cochain complex into a chain complex.
As a consequence, one must invert the degree of $\cochaincomp{A}$.
In any case, we obtain a functor which induces an exact triangulated
functor on the homotopy category, so we may take the total derived functor.

\begin{definition}
  Suppose that $\abcat{A}$ has enough injectives, and fix a complex
  $\cochaincomp{A}$ in $\abcat{A}$.
  The \emph{derived Hom functor} is the total right derived functor
  \[
    \derhom_{\abcat{A}}(\cochaincomp{A}, \cochaincomp{B})
    := \rightderfunc[+]{\Hom^{\bullet}_{\abcat{A}}(\cochaincomp{A},
    -)}(\cochaincomp{B}).
  \]
  \vspace{-24pt}
\end{definition}

We now repeat the above for the tensor product functor.

\begin{definition}
  Let $\cochaincomp{M}$ and $\cochaincomp{N}$ be cochain complexes of
  right and left $R$-modules respectively.
  Consider the double complex $\cochainbicomp{C}$ of abelian groups defined by
  \[
    C^{p, q} = M^{p} \tensor_R N^{q},
  \]
  with horizontal and vertical differentials defined for each $m
  \tensor n \in M^{p} \tensor_R N^{q}$ by
  \[
    d_h^{p, q}(m \tensor n) = d_M(m) \tensor n
    \mathand
    d_v^{p, q}(m \tensor n) = (-1)^p m \tensor d_N(n).
  \]
  The totalisation $\cochaincomp{M} \tensor_R \cochaincomp{N} =
  \totsumcomp{\cochainbicomp{C}}$ is the \emph{internal tensor
  product} of $\cochaincomp{M}$ and $\cochaincomp{N}$.
\end{definition}

\begin{definition}
  Let $\cochaincomp{M}$ be a complex of right $R$-modules.
  The \emph{derived tensor product functor} is the total left derived
  functor $\dertensor{R}{\cochaincomp{M}}{\cochaincomp{N}} :=
  \leftderfunc[-]{(\cochaincomp{M} \tensor_R -)}(\cochaincomp{N})$.
\end{definition}

\begin{remark}
  The cohomology of the derived Hom and tensor product functors as we
  have now defined them obviously do not recover the usual Ext and
  Tor groups, but rather the \emph{hyperext} and \emph{hypertor}
  groups of the \emph{hyper-derived functors} of $\Hom_{\abcat{A}}$
  and $\tensor_R$.
  These are a notion that sits between classical and total derived
  functors that are unnecessary for our purposes, so we defer further
  discussion to \cite[Chapter~5]{weibel}.
\end{remark}

To close out this chapter, we show that the usual tensor-hom
adjunction has a generalisation to the derived category, which is key
to giving the derived equivalence we seek.
As always, we will assume a commutative ring $R$, in which case both
$\Hom_R(M, N)$ and $M \otimes_R N$ have the structure of an $R$-module.
It follows that $\derhom$ and $\dertensorsymb_R$ map into $\dercat{\Mod{R}}$.
For simplicity, we will write $\dercat[*]{R} = \dercat[*]{\Mod{R}}$.

\begin{theorem}[Derived tensor-hom adjunction]
  Fix a complex $\cochaincomp{B} \in \dercat[-]{R}$.
  Then for any $\cochaincomp{A} \in \dercat[-]{R}$ and
  $\cochaincomp{C} \in \dercat[+]{R}$, there is a natural isomorphism
  \[
    \Hom_{\dercat{R}}(\cochaincomp{A}, \derhom(\cochaincomp{B},
    \cochaincomp{C}))
    \cong
    \Hom_{\dercat{R}}(\dertensor{R}{\cochaincomp{A}}{\cochaincomp{B}},
    \cochaincomp{C}).
  \]
  \vspace{-24pt}
\end{theorem}

\begin{proof}
  We claim instead that there is a natural isomorphism
  \[
    \derhom(\cochaincomp{A}, \derhom(\cochaincomp{B}, \cochaincomp{C}))
    \cong
    \derhom(\dertensor{R}{\cochaincomp{A}}{\cochaincomp{B}}, \cochaincomp{C}),
    \tag{$\dagger$}
  \]
  as this yields the original claim upon taking $H^0$.
  For complexes concentrated in degree 0 (i.e. for $R$-modules $L$,
  $M$ and $N$), one can check after unfolding the definitions of
  $\derhom$ and $\dertensorsymb_R$ that this degenerates to the usual
  tensor-hom adjoint isomorphism
  \[
    \Hom(L, \Hom_{R}(M, N)) \cong \Hom(L \tensor_R M, N), \tag{$\star$}
  \]
  so we now seek to use \cref{prop_total_derfunc_comp} to upgrade
  this to the derived category.
  Since we are working in the derived category, it suffices to assume
  that $\cochaincomp{A}$ and $\cochaincomp{C}$ are complexes of
  projectives and injectives respectively.
  Consider the functors
  \[
    \InnerHom(\cochaincomp{A}, \InnerHom(\cochaincomp{B}, \cochaincomp{C}))
    \mathand
    \InnerHom(\cochaincomp{A} \tensor_R \cochaincomp{B}, \cochaincomp{C})
  \]
  in $\cochaincomp{B}$ in the homotopy category.
  The left-hand side is the composite of the functors
  \[
    F_1(\cochaincomp{B}) = \InnerHom(\cochaincomp{B}, \cochaincomp{C})
    \mathand
    G_1(\cochaincomp{X}) = \InnerHom(\cochaincomp{A}, \cochaincomp{X})
  \]
  while the right-hand side is the composite of
  \[
    F_2(\cochaincomp{B}) = \cochaincomp{A} \tensor_R \cochaincomp{B}
    \mathand
    G_2(\cochaincomp{X}) = \InnerHom(\cochaincomp{X}, \cochaincomp{C}).
  \]
  Since each cochain of $\cochaincomp{A}$ is projective, one can
  check that $F_1$ sends complexes of projectives (i.e. $F_1$-acyclic
  complexes) to complexes of injectives (i.e. $G_1$-acyclic complexes).
  Considering \cref{rem_total_derfunc_computing_acyclics}, we may
  apply \cref{prop_total_derfunc_comp} to obtain the natural
  isomorphism $(\rightderfunc{G_1} \circ
  \rightderfunc{F_1})(\cochaincomp{B}) \cong \rightderfunc{(G_1 \circ
  F_1)}(\cochaincomp{B})$, which is to say that
  \[
    \derhom(\cochaincomp{A}, \derhom(\cochaincomp{B}, \cochaincomp{C}))
    \cong
    \rightderfunc{(\InnerHom(\cochaincomp{A}, \InnerHom(-,
    \cochaincomp{C})))}(\cochaincomp{B}).
  \]
  Similarly, one can check that $F_2$ sends complexes of projectives
  (i.e. $F_2$-acyclic complexes) to complexes of projectives (i.e.
  $G_2$-acyclic complexes).
  Examining the proof of \cref{rem_total_derfunc_computing_acyclics}
  more closely, we see that the conclusion still holds even for the
  mixed total derived functors $(\rightderfunc{G_2} \circ
  \leftderfunc{F_2})(\cochaincomp{B}) \cong \rightderfunc{(G_2 \circ
  F_2)}(\cochaincomp{B})$, which is to say that
  \[
    \derhom(\dertensor{R}{\cochaincomp{A}}{\cochaincomp{B}}, \cochaincomp{C})
    \cong
    \rightderfunc{(\InnerHom(\cochaincomp{A} \tensor_R -,
    \cochaincomp{C}))}(\cochaincomp{B}).
  \]
  By taking appropriate injective and projective resolutions,
  ($\star$) therefore shows that both sides of ($\dagger$) are
  naturally isomorphic to the same total right derived functor.
\end{proof}
