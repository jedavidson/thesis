\section{Triangulated categories}
\label{sect_triangcat}

This particular structure of triangles, including the fact that exact
triangles give rise to long exact cohomology sequences, can be found
in categories other than the homotopy category.
Since this will apply also to derived categories, it is worth taking
an intermission to discuss the general case, which is that of
Verdier's \emph{triangulated categories}.
% These categories in fact have an intimately connected history with
% derived categories, having been formulated by Verdier in 1963
% during the same period as Grothendieck's seminal work on derived categories.

We start with a general definition of triangles and morphisms of triangles.
These concepts are similar to what we have already seen for
$\homcat{\abcat{A}}$, but will make sense in any category with a
functor similar to the shift of complexes.

\begin{definition}
  A \emph{translation functor} on a category $\cat{C}$ is an
  autoequivalence (i.e. an equivalence of categories $T: \cat{C} \to \cat{C}$).
  For notational ease, we shall write $[n] = T^n$ and $[-n] = T^{-n}$
  for each $n > 0$, where $T^n$ and $T^{-n}$ are the $n$-fold
  applications of $T$ and $T^{-1}$ respectively.
\end{definition}

\begin{definition}
  Let $\cat{C}$ be a category equipped with a \emph{translation functor} $T$.
  Given an ordered triple of objects $(A, B, C)$ in $\cat{C}$, a
  \emph{triangle} over $(A, B, C)$ is an ordered triple $(u, v, w)$
  corresponding to a sequence
  \[
    \begin{tikzcd}
      A \arrow[r, "u"]
      & B \arrow[r, "v"]
      & C \arrow[r, "w"]
      & A[1],
    \end{tikzcd}
  \]
  so called as this sequence may suggestively be `folded' into a
  triangular diagram
  \[
    \begin{tikzcd}
      & C \\
      A && B.
      \arrow["w"', from=1-2, to=2-1]
      \arrow["u"', from=2-1, to=2-3]
      \arrow["v"', from=2-3, to=1-2]
    \end{tikzcd}
  \]
\end{definition}

\begin{definition}
  Given two triangles $(u, v, w)$ and $(u', v', w')$ over $(A, B, C)$
  and $(A', B', C')$ respectively, a \emph{morphism of triangles} is
  an ordered triple of morphisms $(f, g, h)$ such that the diagram
  \[
    \begin{tikzcd}
      A & B & C & {A[1]} \\
      {A'} & B' & C' & {A'[1]}
      \arrow["u", from=1-1, to=1-2]
      \arrow["f"', from=1-1, to=2-1]
      \arrow["v", from=1-2, to=1-3]
      \arrow["g"', from=1-2, to=2-2]
      \arrow["w", from=1-3, to=1-4]
      \arrow["h", from=1-3, to=2-3]
      \arrow["{f[1]}", from=1-4, to=2-4]
      \arrow["{u'}"', from=2-1, to=2-2]
      \arrow["{v'}"', from=2-2, to=2-3]
      \arrow["{w'}"', from=2-3, to=2-4]
    \end{tikzcd}
  \]
  commutes in $\cat{C}$, and an \emph{isomorphism of triangles} if
  $f$, $g$ and $h$ are isomorphisms.
\end{definition}

In a manner of speaking, a triangulated category is one in which all
triangles `play nicely' with a certain family of triangles analogous
to the exact triangles in $\homcat{\abcat{A}}$.
This is made more precise by the following definition, rather
infamous for its complexity.

\begin{definition}
  A \emph{triangulated category} is an additive category
  $\triangcat{D}$ equipped with a translation functor $T$ and a
  family of triangles, called the \emph{exact} (or
  \emph{distinguished}) triangles in $\triangcat{D}$, subject to the
  following four axioms:

  \begin{enumerate}[leftmargin=4.4em]
    \item[(TR1)]
      For each object $A$, the triangle $(\id_A, 0, 0)$ is exact over
      $(A, A, 0)$.
      For each morphism $u: A \to B$, there exists an object $C$
      together with morphisms $v$ and $w$ so that the triangle $(u,
      v, w)$ is exact over $(A, B, C)$.
      Any triangle isomorphic to an exact triangle is also exact.

    \item[(TR2)]
      \emph{(The rotation axiom.)}
      If the triangle
      \[
        \begin{tikzcd}
          A \arrow[r, "u"]
          & B \arrow[r, "v"]
          & C \arrow[r, "w"]
          & A[1]
        \end{tikzcd}
      \]
      is exact, then so are the rotated triangles
      \[
        \begin{tikzcd}[cramped]
          B & C & {A[1]} & {B[1]}, \\
          {C[-1]} & A & B & C.
          \arrow["v", from=1-1, to=1-2]
          \arrow["w", from=1-2, to=1-3]
          \arrow["{-u[1]}", from=1-3, to=1-4]
          \arrow["{-w[-1]}", from=2-1, to=2-2]
          \arrow["u", from=2-2, to=2-3]
          \arrow["v", from=2-3, to=2-4]
        \end{tikzcd}
      \]
      % If $(u, v, w)$ is exact over $(A, B, C)$, then the triangles
      % \[
      %     \begin{tikzcd}
      %       & B &&& {T(A)} \\
      %       {T^{-1}(C)} && A, & B && C
      %       \arrow["v"', from=1-2, to=2-1]
      %       \arrow["{-T(u)}"', from=1-5, to=2-4]
      %       \arrow["{-T^{-1}(w)}"', from=2-1, to=2-3]
      %       \arrow["u"', from=2-3, to=1-2]
      %       \arrow["v"', from=2-4, to=2-6]
      %       \arrow["w"', from=2-6, to=1-5]
      %     \end{tikzcd}
      % \]
      % are also exact.
      % \todo{Think about whether changing shift from $[-1]$ to $[1]$
      % means $T^{-1}$ should be $T$.}

    \item[(TR3)]
      \emph{(The morphism axiom.)}
      Given exact triangles $(u, v, w)$ and $(u', v', w')$ over $(A,
      B, C)$ and $(A', B', C')$ respectively, as well as morphisms
      $f: A \to A'$ and $g: B \to B'$ such that $gu = u' f$, there
      exists a morphism $h: C \to C'$ so that $(f, g, h)$ is a
      morphism of triangles, i.e.
      \[
        \begin{tikzcd}
          A & B & C & {A[1]} \\
          {A'} & B' & C' & {A'[1]}
          \arrow["u", from=1-1, to=1-2]
          \arrow["f"', from=1-1, to=2-1]
          \arrow["v", from=1-2, to=1-3]
          \arrow["g"', from=1-2, to=2-2]
          \arrow["w", from=1-3, to=1-4]
          \arrow[dashed, "h", from=1-3, to=2-3]
          \arrow["{f[1]}", from=1-4, to=2-4]
          \arrow["{u'}"', from=2-1, to=2-2]
          \arrow["{v'}"', from=2-2, to=2-3]
          \arrow["{w'}"', from=2-3, to=2-4]
        \end{tikzcd}
      \]
      may be completed to a commutative diagram in $\triangcat{D}$.

    \item[(TR4)]
      \emph{(The octahedral axiom.)}
      Omitted, but the interested reader may refer to
      \cite[Definition~10.2.1]{weibel} for a full statement.
  \end{enumerate}
\end{definition}

Our reasons for omitting (TR4) are twofold.
It is not needed in this thesis, since we will not prove that any
categories are triangulated, and the axiom itself is rather
complicated to print due to the octahedral diagram from which it gets its name.
% Separately, there is also some debate amongst experts as to whether
% these axioms are actually the `correct' definition of triangulated categories.
% Higher-minded replacements for triangulated categories have been
% proposed, such as \emph{differential graded categories} and
% \emph{stable $\infty$-categories}.

\begin{example}
  We claimed that $\homcat{\abcat{A}}$ is triangulated, although so
  far we have only checked that (TR1) holds.
  For the remaining details, see \cite[Proposition~10.2.4]{weibel}.
  Naturally, the same is true for the categories
  $\homcat[*]{\abcat{A}}$, and it will even be true for
  $\homcat[*]{\cat{C}}$ when $\cat{C}$ is some full additive
  subcategory of $\abcat{A}$.
\end{example}

\begin{remark}
  A triangulated category $\triangcat{D}$ is usually not abelian, and
  it turns out that a necessary and sufficient condition for this to
  occur is that $\triangcat{D}$ is a \emph{semisimple} category.
  For details, see
  \cite[Section~III.2.3~and~Exercise~IV.1.1]{gelfand_and_manin}.
  In the particular case of $\homcat{\abcat{A}}$, one notable
  obstruction is that homotopy equivalent maps most likely do not
  share the same kernels or cokernels.
\end{remark}

Of course, what we would really like in order to make the
substitution most effective is a generalisation the result of
\cref{prop_homcat_long_exact_coh_seq} to arbitrary triangulated categories.

\begin{definition}
  Let $\triangcat{D}$ be a triangulated category and $\abcat{A}$ an
  abelian category.
  An additive functor $H: \triangcat{D} \to \abcat{A}$ is said to be
  a \emph{cohomological functor} if, for any exact triangle $(u, v,
  w)$ on $(A, B, C)$, there is a long exact sequence
  \[
    \begin{tikzcd}
      \cdots \arrow[r]
      & H(A[i]) \arrow[r, "u_i^*"]
      & H(B[i]) \arrow[r, "v_i^*"]
      & H(C[i]) \arrow[r, "w_i^*"]
      & H(A[i + 1]) \arrow[r]
      & \cdots
    \end{tikzcd}
  \]
  in $\cat{A}$, where the induced maps are given by $f_i^* = H(f[i])$.
  In this long exact sequence, we often write $H^i = H \circ [i]$ for
  $i > 0$ and $H^0 = H$ for notational ease.
\end{definition}

% \begin{example}
%     Another important example is $\Hom(A, -): \triangcat{K} \to
% \Ab$ for any object $A$ of a triangulated category $\triangcat{K}$,
% which is explored in \cite[Example~10.2.8]{weibel}.
% \end{example}

In other words, cohomological functors send exact triangles to long
exact sequences, but we may also consider functors that instead send
them to exact triangles.
This fills the remaining hole left by the loss of exact sequences,
which is a substitute for the notion of exact functors between
triangulated categories, and will prove useful in \cref{sect_derfunc}.
Though some authors also use the term `exact' to refer to such a
functor, we will opt to make the terminological distinction between
usual exactness and triangulated exactness clear in the following definition.
% After one convinces themselves that the opposite category of a
% triangulated category is also triangulated, we may dualise this
% definition to obtain contravariant cohomological functors.

\begin{definition}
  Let $\triangcat{D}$ and $\triangcat{D}'$ be triangulated
  categories, and denote by $T_{\triangcat{D}}$ and
  $T_{\triangcat{D}'}$ their respective translation functors.
  We say that an additive functor $F: \triangcat{D} \to
  \triangcat{D}'$ is an \emph{exact triangulated functor} if it
  satisfies the following two conditions:
  \begin{enumerate}
    \item
      $F$ commutes with translation up to natural isomorphism,
      i.e. there exists a natural isomorphism of composite functors
      $\eta: F \circ T_{\triangcat{D}} \Rightarrow T_{\triangcat{D}'} \circ F$.
      % That is, for each object $A$ in $\triangcat{K}$ there is an
      % isomorphism $\eta_A: F(T_{\triangcat{K}}(A)) \to
      % T_\triangcat{L}(F(A))$, and for each morphism $f: A \to B$ in
      % $\triangcat{K}$ the associated naturality square commutes in
      % $\triangcat{L}$:
      % \[
      %     \begin{tikzcd}[cramped]
      %       {F(T_{\triangcat{K}}(A))} && {T_{\triangcat{L}}(F(A))} \\
      %       \\
      %       {F(T_{\triangcat{K}}(B))} && {T_{\triangcat{L}}(F(B)).}
      %       \arrow["{\eta_A}", from=1-1, to=1-3]
      %       \arrow["{F(T_{\triangcat{K}}(f))}"', from=1-1, to=3-1]
      %       \arrow["{T_{\triangcat{L}}(F(f))}", from=1-3, to=3-3]
      %       \arrow["{\eta_B}"', from=3-1, to=3-3]
      %     \end{tikzcd}
      % \]

    \item
      $F$ sends exact triangles to exact triangles, in the sense that
      for each exact triangle
      \[
        \begin{tikzcd}
          A \arrow[r, "u"]
          & B \arrow[r, "v"]
          & C \arrow[r, "w"]
          %   & T_{\triangcat{K}}(A)
          & A[1]
        \end{tikzcd}
      \]
      in $\triangcat{D}$, the following is an exact triangle in
      $\triangcat{D}'$:
      \[
        \begin{tikzcd}
          {F(A)} & {F(B)} & {F(C)} && {F(A)[1]}.
          \arrow["{F(u)}", from=1-1, to=1-2]
          \arrow["{F(v)}", from=1-2, to=1-3]
          \arrow["{\eta_A \circ F(w)}", from=1-3, to=1-5]
        \end{tikzcd}
      \]
  \end{enumerate}
\end{definition}

\iffalse
\begin{remark}
  This definition may seem stronger than initially stated, but the
  first condition is essential to impose, as applying $F$ to an exact
  triangle only gives a sequence
  \[
    \begin{tikzcd}
      F(A) \arrow[r, "F(u)"]
      & F(B) \arrow[r, "F(v)"]
      & F(C) \arrow[r, "F(w)"]
      & F(A[1])
    \end{tikzcd}
  \]
  in $\triangcat{D}'$, which is not a well-formed triangle without
  the isomorphism
  % $F(T_{\triangcat{K}}(A)) \cong T_{\triangcat{L}}(F(A))$.
  $F(A[1]) \cong F(A)[1]$.
\end{remark}
\fi
