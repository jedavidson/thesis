\section{Localising the homotopy category}
\label{sect_localising_homcat}

We now have a triangulated category $\homcat{\abcat{A}}$ that turns
homotopy equivalences into isomorphisms.
While it is true that homotopy equivalences are quasi-isomorphisms,
we have noted previously that the converse is generally false.
If we are yet to fully achieve our goal of identifying complexes with
the same cohomology, this observation suggests that we need a way to
make the quasi-isomorphisms actual isomorphisms too.
Taking cues from the method in commutative algebra of the same name,
the second and final phase of constructing the derived category is to
take a \emph{localisation} of $\homcat{\abcat{A}}$.

\begin{definition}
  \label{def_univ_prop_of_locln}
  Let $S$ be a collection of morphisms in a category $\cat{C}$.
  A \emph{localisation of $\cat{C}$ with respect to $S$} is a
  category $\locln{\cat{C}}{S}$ and a \emph{localisation functor}
  $Q_S: \abcat{C} \to \locln{\abcat{C}}{S}$ such that $Q_S(s)$ is an
  isomorphism for every $s \in S$, and which is universal among all
  functors with this property.
  That is, any functor $F: \cat{C} \to \cat{D}$ such that $F(s)$ is
  an isomorphism in $\cat{D}$ for every $s \in S$ factors uniquely
  through $Q_S$:
  \[
    \begin{tikzcd}
      {\cat{C}} & {\cat{D}.} \\
      {\locln{\cat{C}}{S}}
      \arrow["F", from=1-1, to=1-2]
      \arrow["Q_S"', from=1-1, to=2-1]
      \arrow["\exists!"', dashed, from=2-1, to=1-2]
    \end{tikzcd}
  \]
  \vspace{-12pt}
\end{definition}

\iffalse
\begin{example}
  Because of \cref{prop_univ_prop_of_homcat},
  $\homcat{\abcat{A}}$ is in fact a localisation of $\Ch{\abcat{A}}$
  with respect to homotopy equivalences.
\end{example}
\fi

The universal property shows that if a localisation exists, it is
unique up to an equivalence of categories.
We will follow the explicit construction given by the \emph{calculus
of fractions} proposed by Gabriel and Zisman in 1967.
The mild hypothesis that this calculus requires is that $S$ forms a
\emph{multiplicative system}.
There are some more foundational concerns (i.e. hom-sets potentially
no longer being sets) that are technically relevant in what follows,
but which we will ignore.

\begin{definition}
  A \emph{multiplicative system} in a category $\cat{C}$ is a
  collection of morphisms $S$ satisfying the following three axioms:
  \begin{enumerate}
    \item
      \emph{(Closure.)}
      $S$ is closed under composition and contains all identity morphisms.

    \item
      \emph{(The Ore condition.)}
      Given $t: D \to B$ in $S$ and $g: A \to B$ in $\cat{C}$, there
      exist $s: C \to A$ in $S$ and $f: C \to D$ in $\cat{C}$ such
      that the diagram
      \[
        \begin{tikzcd}
          C & D \\
          A & B
          \arrow["{f}", dashed, from=1-1, to=1-2]
          \arrow["{s}"', dashed, from=1-1, to=2-1]
          \arrow["t", from=1-2, to=2-2]
          \arrow["g"', from=2-1, to=2-2]
        \end{tikzcd}
      \]
      commutes.
      Symmetrically, given the morphisms $f$ and $s$ as in the above
      diagram, one may find such $t$ and $g$.
      This is expressed informally as $t^{-1} g = f s^{-1}$.

    \item
      \emph{(Cancellation.)}
      If $f,g: A \to B$ are morphisms in $\cat{C}$, then $sf = sg$
      for some $s: B \to C$ in $S$ if and only if $ft = gt$ for some
      $t: D \to A$ in $S$.
  \end{enumerate}
\end{definition}

\begin{proposition}
  Let $S$ be the collection of quasi-isomorphisms in $\homcat{\abcat{A}}$.
  Then $S$ is a multiplicative system in $\homcat{\abcat{A}}$.
\end{proposition}

\begin{proof}
  By definition, $S$ is the collection of morphisms such that
  $H^i(s)$ is an isomorphism for all $i \in \bb{Z}$ and $s \in S$, so
  we say that $S$ \emph{arises from} the functor $H^0:
  \homcat{\abcat{A}} \to \abcat{A}$.
  By the first claim of \cite[Proposition~10.4.1]{weibel}, any
  collection arising from a cohomological functor must be a
  multiplicative system.
\end{proof}

Proceeding according to the calculus of fractions, localising
$\homcat[*]{\abcat{A}}$ does not require changing the set of objects.
It therefore remains to say what the morphisms should be and how to
compose them, which is the actual namesake of this construction.

\begin{definition}
  An arrangement of cochain maps
  $\leftfrac{\cochaincomp{A}}{s}{\cochaincomp{X}}{f}{\cochaincomp{B}}$
  in $\homcat{\abcat{A}}$ is called a \emph{fraction} (or
  \emph{roof}) if $s$ a quasi-isomorphism.
  We use the more compact and suggestive notation $fs^{-1}$ whenever
  the underlying complexes are understood.
\end{definition}

Given compatible fractions
$\leftfrac{\cochaincomp{A}}{s}{\cochaincomp{X}}{f}{\cochaincomp{B}}$
and
$\leftfrac{\cochaincomp{B}}{t}{\cochaincomp{Y}}{g}{\cochaincomp{C}}$,
let us attempt to compose them.
Using the Ore condition, choose a quasi-isomorphism $r:
\cochaincomp{Z} \to \cochaincomp{X}$ and a cochain map $h:
\cochaincomp{Z} \to \cochaincomp{Y}$ fitting into a commutative diagram
\[
  \begin{tikzcd}
    && \cochaincomp{B} \\
    & \cochaincomp{X} && \cochaincomp{Y} \\
    \cochaincomp{A} && \cochaincomp{Z} && \cochaincomp{C}.
    \arrow["f", from=2-2, to=1-3]
    \arrow["s"', from=2-2, to=3-1]
    \arrow["t"', from=2-4, to=1-3]
    \arrow["g", from=2-4, to=3-5]
    \arrow["r", dashed, from=3-3, to=2-2]
    \arrow["h"', dashed, from=3-3, to=2-4]
  \end{tikzcd}
\]
We declare the fraction
\[
  \leftfrac{\cochaincomp{A}}{sr}{\cochaincomp{Z}}{gh}{\cochaincomp{C}}
\]
to be `the' composite of $fs^{-1}$ and $gt^{-1}$.
It is clear that this composition operation is associative, and that
the fraction
\[
  \leftfrac{\cochaincomp{A}}{\id}{\cochaincomp{A}}{\id}{\cochaincomp{A}}
\]
acts as a two-sided identity.
However, one issue is that since $r$ and $h$ are not guaranteed to be
unique, neither in general is the composition of two fractions.
To remedy this, we need a notion of fraction equivalence.

\begin{definition}
  Two fractions
  \[
    \leftfrac{\cochaincomp{A}}{s_1}{\cochaincomp{X_1}}{f_1}{\cochaincomp{B}}
    \mathand
    \leftfrac{\cochaincomp{A}}{s_2}{\cochaincomp{X_2}}{f_2}{\cochaincomp{B}}
  \]
  are \emph{equivalent} if they are \emph{dominated} by a third
  fraction
  $\leftfrac{\cochaincomp{A}}{s_3}{\cochaincomp{X_3}}{f_3}{\cochaincomp{B}}$,
  i.e. there are cochain maps $\cochaincomp{X_3} \to
  \cochaincomp{X_1}$ and $\cochaincomp{X_3} \to \cochaincomp{X_2}$
  completing a commutative diagram
  \[
    \begin{tikzcd}
      && \cochaincomp{X_3} \\
      & \cochaincomp{X_1} && \cochaincomp{X_2} \\
      \cochaincomp{A} &&&& \cochaincomp{B}
      \arrow["s_1"{description}, from=2-2, to=3-1]
      \arrow["f_1"{description}, from=2-2, to=3-5]
      \arrow["s_2"{description}, from=2-4, to=3-1]
      \arrow[dashed, from=1-3, to=2-2]
      \arrow[dashed, from=1-3, to=2-4]
      \arrow["f_2"{description}, from=2-4, to=3-5]
      \arrow[dashed, "s_3"{description}, curve={height=30pt}, from=1-3, to=3-1]
      \arrow[dashed, "f_3"{description}, curve={height=-30pt}, from=1-3, to=3-5]
    \end{tikzcd}
  \]
  in $\homcat{\abcat{A}}$.
  The collection of equivalence classes of fractions is denoted
  $\Hom_S(\cochaincomp{A}, \cochaincomp{B})$.
\end{definition}

To amend the composition procedure, we now choose a representative
fraction for both classes, composing them as before and then declare
the class of the result to be the composite.
We omit the verification that this is independent of the choice of
representatives and thus a well-defined operation, since it is just
an exercise in setting up the appropriate diagram and checking commutativity.

In view of our earlier observations, the upshot is that the
collections $\Hom_S(\cochaincomp{A}, \cochaincomp{B})$ fulfil the
properties needed to be the hom-sets of a category.
Thus at long last, everything is now in place to define the derived category.
By the Gabriel-Zisman theorem, the definition below will precisely be
a localisation of the homotopy category with respect to the
quasi-isomorphisms, and so has the property we have sought after.
Moreover, thanks to \cite[Proposition~10.4.1]{weibel}, we do not
sacrifice the triangulated structure.

\begin{definition}
  Let $\abcat{A}$ be an abelian category, and $S$ the multiplicative
  system of quasi-isomorphisms in $\homcat{\abcat{A}}$.
  The \emph{derived category} of $\abcat{A}$ is the triangulated
  category $\dercat{\abcat{A}}$ with the same objects as
  $\homcat{\abcat{A}}$ and hom-sets
  \[
    \Hom_{\dercat{\abcat{A}}}(\cochaincomp{A}, \cochaincomp{B}) :=
    \Hom_S(\cochaincomp{A}, \cochaincomp{B}).
  \]
  The localisation functor $\dercatlocln{\abcat{A}}:
  \homcat{\abcat{A}} \to \dercat{\abcat{A}}$ is the identity on
  objects and sends a cochain map $f: \cochaincomp{A} \to
  \cochaincomp{B}$ to the fraction
  \[
    \leftfrac{\cochaincomp{A}}{\id}{\cochaincomp{A}}{f}{\cochaincomp{B}}.
  \]
  The \emph{bounded} derived categories $\dercat[*]{\abcat{A}}$ and
  localisation functors $\dercatlocln[*]{\abcat{A}}:
  \homcat[*]{\abcat{A}} \to \dercat[*]{\abcat{A}}$ are defined
  similarly, and are all full subcategories of $\dercat{\abcat{A}}$.
\end{definition}

\begin{proposition}[{\cite[Proposition~III.5.2]{gelfand_and_manin}}]
  The composite functor
  \[
    \begin{tikzcd}[cramped]
      \iota_{\abcat{A}}: {\abcat{A}} & {\Ch[b]{\abcat{A}}} &
      {\homcat[b]{\abcat{A}}} & {\dercat[b]{\abcat{A}}}
      \arrow[hook, from=1-1, to=1-2]
      \arrow["\pi", from=1-2, to=1-3]
      \arrow["{\dercatlocln[*]{\abcat{A}}}", from=1-3, to=1-4]
    \end{tikzcd}
  \]
  which concentrates objects and morphisms in degree 0 is fully faithful.
\end{proposition}

\iffalse
\begin{proposition}[{\cite[Corollaries~10.3.9--10]{weibel}}]
  \label{prop_locln_properties_of_dercat}
  \begin{enumerate}
    \item
      Two complexes $\cochaincomp{A}$ and $\cochaincomp{B}$ in
      $\Ch{\abcat{A}}$ are isomorphic in $\dercat{\abcat{A}}$ if and only
      if they are quasi-isomorphic in $\homcat{\abcat{A}}$.
      Thus $\cochaincomp{A}$ is zero if and only if it is exact/acyclic.

    \item
      Two cochain maps $f,\, g: \cochaincomp{A} \to
      \cochaincomp{B}$ are the same in $\dercat{\abcat{A}}$ if and only
      if $sf = sg$ for some quasi-isomorphism $s: \cochaincomp{B} \to
      \cochaincomp{C}$.
      Thus $f$ is zero in $\dercat{\abcat{A}}$ if and only if
      $sf$ is null-homotopic.
  \end{enumerate}
\end{proposition}
\fi

With the derived category now at hand, it is worth pausing to reflect
on the goals we established at the start of this chapter.
By \cref{rem_res_is_a_qis}, any resolution of an object in
$\abcat{A}$ is quasi-isomorphic to the image of that object under the
above embedding functor, and by the universal property of
$\dercatlocln{\abcat{A}}$ this becomes an isomorphism of objects in
$\dercat{\abcat{A}}$.
Thus in the derived category, the passage from an object to any of
its resolutions is completely natural, which was precisely the main
motivation for trying to invert quasi-isomorphisms in the first place.
Another benefit of $\dercat{\abcat{A}}$ over $\homcat{\abcat{A}}$ is
that the triangulated structure of the derived category is better behaved.

\begin{proposition}[{\cite[Example~10.4.9]{weibel}}]
  \label{prop_dercat_ses_exact_triang}
  Every short exact sequence
  \[
    \begin{tikzcd}
      \cochaincomp{0} \arrow[r]
      & \cochaincomp{A} \arrow[r, "u"]
      & \cochaincomp{B} \arrow[r, "v"]
      & \cochaincomp{C} \arrow[r]
      & \cochaincomp{0}
    \end{tikzcd}
  \]
  in $\Ch{\abcat{A}}$ fits into an exact triangle $(u, v, w)$ on
  $(\cochaincomp{A}, \cochaincomp{B}, \cochaincomp{C})$ in
  $\dercat{\abcat{A}}$ isomorphic to the strict triangle on $u$.
\end{proposition}

The price to pay for all of this on-paper computational convenience
is that compared to $\homcat{\abcat{A}}$, the morphisms in
$\dercat{\abcat{A}}$ are more elaborate and can be difficult to
actually work with in general, especially in unbounded derived categories.
This may seem like a perverse outcome, but in practice one is most
interested in the bounded derived categories anyway, and it is often
the case that $\abcat{A}$ has certain properties which allow for
simplifications to be made.
We now discuss what can be done in such cases.

The need to localise the homotopy category arose from the observation
that not all quasi-isomorphisms are homotopy equivalences.
However, we have also seen that a partial converse for bounded below
complexes of injectives holds by \cref{prop_qis_is_homotopy_equiv_partial_conv}.
On the face of it, this suggests that whenever $\abcat{A}$ has enough
injectives and $\abcat{I}$ is the full additive subcategory of
$\abcat{A}$ consisting of injectives, the triangulated subcategory
$\homcat[+]{\abcat{I}}$ of $\homcat{\abcat{A}}$ already suffices in
order to invert quasi-isomorphisms.
% Both this statement and its dual turn out to be true in the following way.

\begin{proposition}[{\cite[Theorem~10.4.8]{weibel}}]
  \label{prop_bounded_below_dercat_equiv}
  Let $\abcat{A}$ be an abelian category with enough injectives and
  $\abcat{I}$ the full additive subcategory of injectives in $\abcat{A}$.
  Then
  \[
    \begin{tikzcd}
      {T^+:\homcat[+]{\abcat{I}}} & {\homcat[+]{\abcat{A}}} &
      {\dercat[+]{\abcat{A}}}
      \arrow[hook, from=1-1, to=1-2]
      \arrow["{\dercatlocln[+]{\abcat{A}}}", from=1-2, to=1-3]
    \end{tikzcd}
  \]
  is an equivalence of categories.
  % Dually, if $\abcat{A}$ has enough projectives and $\abcat{P}$ is
  % the full additive subcategory of $\abcat{A}$ consisting of projectives, then
  % \[
  %     \begin{tikzcd}
  %       {\homcat[-]{\abcat{P}}} & {\homcat[-]{\abcat{A}}} &
  % {\dercat[-]{\abcat{A}}}
  %       \arrow[hook, from=1-1, to=1-2]
  %       \arrow["{\dercatlocln[-]{\abcat{A}}}", from=1-2, to=1-3]
  %     \end{tikzcd}
  % \]
  % is an equivalence of categories.
\end{proposition}

We will not prove this.
The argument in \cite{weibel} uses the machinery of
\emph{Cartan-Eilenberg resolutions} (i.e. double complexes acting as
`resolutions' for complexes) and formal properties of localisation.
Alternatively, the (longer) proof of
\cite[Proposition~2.40]{huybrechts} shows more directly that the
composite is fully faithful and essentially surjective, i.e.
\begin{equation}
  \label{eq_bij_dercat_homcat_injs}
  \Hom_{\homcat[+]{\abcat{I}}}(\cochaincomp{I}, \cochaincomp{J})
  \longrightarrow
  \Hom_{\dercat[+]{\abcat{A}}}(\cochaincomp{I}, \cochaincomp{J})
\end{equation}
is bijective and every complex $\cochaincomp{A} \in
\dercat[+]{\abcat{A}}$ is isomorphic in $\dercat[+]{\abcat{A}}$ to
some complex $\cochaincomp{I} \in \homcat[+]{\abcat{I}}$.
Unfortunately, this equivalence of categories does not immediately
descend to $\dercat[b]{\abcat{A}}$ without the inclusion of an extra hypothesis.

\begin{corollary}
  \label{cor_bounded_dercat_equiv}
  The functor of \cref{prop_bounded_below_dercat_equiv} is an
  equivalence of categories
  \[
    \begin{tikzcd}
      {T^b: \homcat[b]{\abcat{I}}} & {\homcat[b]{\abcat{A}}} &
      {\dercat[b]{\abcat{A}}}
      \arrow[hook, from=1-1, to=1-2]
      \arrow["{\dercatlocln[b]{\abcat{A}}}", from=1-2, to=1-3]
    \end{tikzcd}
  \]
  if all objects of $\abcat{A}$ have injective resolutions of finite length.
\end{corollary}

The dual statements of \cref{prop_bounded_below_dercat_equiv} and
\cref{cor_bounded_dercat_equiv} for projective objects and
$\dercat[-]{\abcat{A}}$ also hold, but this case will be of less interest to us.
