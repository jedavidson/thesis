\chapter{Derived categories}
\label{chap_derived_categories}

We are now sufficiently prepared to establish the notion of the
\emph{derived category}, following the texts \cite{weibel,
huybrechts, gelfand_and_manin}.
At its core, this technical apparatus provides a more natural
computational setting for the study of homological algebra than the
category of cochain complexes.
In comparison to the level of abstractions we have encountered thus
far, the material in this chapter will at times become rather
technical, so it will be beneficial to first outline the main
motivations for derived categories and collect some desiderata to hold onto.

The central conceit is that from a computational standpoint, it is
convenient to be able to treat complexes that have the same
cohomology (i.e.\ those which are quasi-isomorphic, homotopy
equivalent or both) as if they are isomorphic.
Indeed, we have seen a concrete instance of this already in
\cref{sect_classical_derfunc}, which suggests that we should be able
to freely interchange the objects of an abelian category with any of
its resolutions when computing derived functors.
At the same time, a complex is more than just its cohomology, and so
on the face of it, associating complexes to a mathematical object of
interest rather than just its cohomology leads to a more fruitful
source of algebraic invariants.
In much the same way, the historical viewpoint of Grothendieck was
that the suite of classical derived functors should be the
information extracted from a single derived functor existing at the
level of complexes in the derived category.

In all that follows, $\abcat{A}$ will be an abelian category.

\section{The homotopy category}

Recall that in \cref{sect_complexes}, we introduced $\Ch{\abcat{A}}$,
the category of cochain complexes with objects in $\abcat{A}$, and
noted that it itself was abelian.
In the first phase of constructing the derived category of
$\abcat{A}$, we wish to alter $\Ch{\abcat{A}}$ to identify cochain
maps up to homotopy equivalence.
The justification for this should be clear, as we have already seen
that homotopy equivalent maps preserve cohomology.

\begin{definition}
  Let $\cat{C}$ be a category.
  A \emph{congruence relation} $\sim$ on $\cat{C}$ consists of an
  equivalence relation $\sim_{A, B}$ on each $\Hom_{\cat{C}}(A, B)$
  which is compatible with composition, that is, if $f \sim_{A, B}
  f'$ and $g \sim_{B, C} g'$, then $gf \sim_{A, C} g'f'$.
  We define the \emph{quotient category} $\quot{\cat{C}}{\sim}$ to
  have the same objects as $\cat{C}$, but whose hom-sets are
  equivalence classes of morphisms in hom-sets of $\cat{C}$, i.e.
  \[
    \Hom_{\quot{\cat{C}}{\sim}}(A, B) := \quot{\Hom_{\cat{C}}(A, B)}{\sim}
  \]
  for each pair of objects $A,\, B \in \cat{C}$.
  The full \emph{quotient functor} $\pi: \cat{C} \to
  \quot{\cat{C}}{\sim}$ is the identity on objects and sends
  morphisms to their equivalence class under $\sim$.
\end{definition}

\begin{definition}
  The \emph{homotopy category} of an abelian category $\cat{A}$ is
  the quotient category $\homcat{\abcat{A}} :=
  \quot{\Ch{\abcat{A}}}{\sim}$ under homotopy equivalence.
\end{definition}

Eventually we will consider the \emph{bounded} homotopy categories
$\homcat[*]{\abcat{A}}$, which are the full subcategories of
$\homcat{\abcat{A}}$ corresponding to the categories
$\Ch[*]{\abcat{A}}$ for each $* \in \{b, +, -\}$.
Under certain hypotheses on $\abcat{A}$, considering the full
subcategories of $\homcat[*]{\abcat{A}}$ consisting of complexes of
injectives and projectives is sufficient to describe its bounded
derived category, which are the particular flavour of derived
category we will concern ourselves with.
These details will be made precise at the end of \cref{sect_localising_homcat}.

By construction, the image of any homotopy equivalence in
$\Ch{\abcat{A}}$ under $\pi$ is an isomorphism in $\homcat{\abcat{A}}$.
It is possible to formulate a universal property for
$\homcat{\abcat{A}}$ along these lines, which provides a useful way
of thinking about the homotopy category.

\begin{proposition}[{\cite[Proposition~10.1.2]{weibel}}]
  \label{prop_univ_prop_of_homcat}
  Any functor $F: \Ch{\abcat{A}} \to \cat{D}$ with the property that
  it sends homotopy equivalences in $\Ch{\abcat{A}}$ to isomorphisms
  in $\cat{D}$ factors uniquely through $\homcat{\abcat{A}}$.
  \[
    \begin{tikzcd}
      \Ch{\abcat{A}} & {\cat{D}.} \\
      \homcat{\abcat{A}}
      \arrow["F", from=1-1, to=1-2]
      \arrow["\pi"', from=1-1, to=2-1]
      \arrow["\exists!"', dashed, from=2-1, to=1-2]
    \end{tikzcd}
  \]
\end{proposition}

\begin{proposition}
  The category $\homcat{\abcat{A}}$ is additive and has additive
  quotient functor.
\end{proposition}

After identifying homotopy equivalent maps, we unfortunately lose the
abelian structure of $\Ch{\abcat{A}}$ in general.
Derived categories suffer the same fate, and at a glance this seems
rather damning for any hopes of doing homological algebra, as we no
longer have access to exact sequences.
However, the shift functor on complexes gives $\homcat{\abcat{A}}$
some additional structure beyond being additive, and it turns out
that there is a principled way to think of certain sequences of maps
in a way that mimics exact sequences.

\begin{definition}
  Let $u: \cochaincomp{A} \to \cochaincomp{B}$ be a map of complexes
  in $\Ch{\abcat{A}}$.
  The \emph{strict triangle} on $u$ is the ordered triple of maps
  $(u, v, \delta)$ in $\homcat{\abcat{A}}$, where $v$ and $\delta$
  are maps in $\Ch{\abcat{A}}$ arising from the short exact sequence
  \[
    \begin{tikzcd}
      \cochaincomp{0} \arrow[r]
      & \cochaincomp{B} \arrow[r, "v"]
      & \cone{u} \arrow[r, "\delta"]
      & \cochaincomp{A}[1] \arrow[r]
      & \cochaincomp{0}.
    \end{tikzcd}
  \]
  \vspace{-24pt}
\end{definition}

\begin{definition}
  \label{def_exact_triang_homcat}
  Given maps
  \[
    \begin{tikzcd}
      \cochaincomp{A} \arrow[r, "u"]
      & \cochaincomp{B} \arrow[r, "v"]
      & \cochaincomp{C} \arrow[r, "w"]
      & \cochaincomp{A}[1]
    \end{tikzcd}
  \]
  in $\homcat{\abcat{A}}$, the ordered triple $(u, v, w)$ is called
  an \emph{exact triangle} on $(\cochaincomp{A}, \cochaincomp{B},
  \cochaincomp{C})$ if there is a strict triangle $(u', v', w')$ and
  a commutative diagram
  \[
    \begin{tikzcd}
      \cochaincomp{A} & \cochaincomp{B} & \cochaincomp{C} &
      {\cochaincomp{A}[1]} \\
      \cochaincomp{(A')} & \cochaincomp{(B')} & \cone{u'} &
      {\cochaincomp{(A')}[1]}
      \arrow["u", from=1-1, to=1-2]
      \arrow["f"', from=1-1, to=2-1]
      \arrow["v", from=1-2, to=1-3]
      \arrow["g"', from=1-2, to=2-2]
      \arrow["w", from=1-3, to=1-4]
      \arrow["h", from=1-3, to=2-3]
      \arrow["{f[1]}", from=1-4, to=2-4]
      \arrow["{u'}"', from=2-1, to=2-2]
      \arrow["{v'}"', from=2-2, to=2-3]
      \arrow["{\delta}"', from=2-3, to=2-4]
    \end{tikzcd}
  \]
  in $\homcat{\abcat{A}}$ such that the maps $f$, $g$ and $h$ are isomorphisms.
\end{definition}

\iffalse
\begin{remark}
  Since choosing maps in $\homcat{\abcat{A}}$ is choosing
  representatives of homotopy equivalence classes from
  $\Ch{\abcat{A}}$, commutative diagrams in $\homcat{\abcat{A}}$ are
  commutative diagrams in $\Ch{\abcat{A}}$ up to a choice of homotopy
  equivalent maps.
\end{remark}
\fi

\iffalse
\begin{example}
  For any complex $\cochaincomp{A}$, the triangle
  \[
    \begin{tikzcd}
      \cochaincomp{A} \arrow[r, "0"]
      & \cochaincomp{A} \arrow[r]
      & \cochaincomp{A} \oplus \cochaincomp{A}[1] \arrow[r]
      & \cochaincomp{A}[1]
    \end{tikzcd}
  \]
  is strict, since $\cone{0} = \cochaincomp{A} \oplus
  \cochaincomp{A}[1]$. The triangle
  \[
    \begin{tikzcd}
      \cochaincomp{A} \arrow[r, "\id"]
      & \cochaincomp{A} \arrow[r]
      & 0 \arrow[r]
      & \cochaincomp{A}[1]
    \end{tikzcd}
  \]
  is exact.

  To see why, note that $\cone{\id}$ is split exact, and so the
  identity map for $\cone{\id}$ is null-homotopic, which in turn
  implies that $\cone{\id} \to \cochaincomp{0}$ is a chain homotopy equivalence.
  This is precisely an isomorphism in $\homcat{\abcat{A}}$, and by
  replacing $\cone{\id}$ by an arbitrary split exact complex
  $\cochaincomp{B}$, we recover the slightly more general fact that
  $\cochaincomp{B} \cong \cochaincomp{0}$ in $\homcat{\abcat{A}}$.
  % since $\cone{\id}$ is a split exact complex and hence isomorphic
  % to 0 in $\homcat{\abcat{A}}$.
  % This fact holds more generally: any
  % and hence isomorphic to 0 in $\homcat{\abcat{A}}$.
\end{example}
\fi

One should think of exactness for triangles as saying that there is
an isomorphism between $(u, v, w)$ and the strict triangle $(u', v', \delta)$.
The main feature of short exact sequences in $\Ch{\abcat{A}}$ is that
the cohomology functor induces long exact sequences in $\abcat{A}$.
When one instead works in $\homcat{\abcat{A}}$, cohomology remains a
well-defined family of functors since homotopic maps induce the same
maps on cohomology, and an exact triangle still induces a long exact
cohomology sequence in $\abcat{A}$.

\begin{proposition}[{\cite[Corollary~10.1.4]{weibel}}]
  \label{prop_homcat_long_exact_coh_seq}
  Given an exact triangle $(u, v, w)$ in $\homcat{\abcat{A}}$, there
  is a long exact sequence
  \[
    \begin{tikzcd}
      \cdots \arrow[r]
      & H^i(\cochaincomp{A}) \arrow[r, "H^i(u)"]
      & H^i(\cochaincomp{B}) \arrow[r, "H^i(v)"]
      & H^i(\cochaincomp{C}) \arrow[r, "H^i(w)"]
      & H^{i + 1}(\cochaincomp{A}) \arrow[r]
      & \cdots
    \end{tikzcd}
  \]
  in $\abcat{A}$, where we make the identification $H^{i +
  1}(\cochaincomp{A}) \cong H^i(\cochaincomp{A}[1])$.
\end{proposition}

\iffalse
\begin{proof}
  If $(u, v, w)$ is itself a strict triangle, then this long exact
  cohomology sequence is precisely the usual one obtained from the
  underlying short exact sequence.
  In all other cases, we have a commutative diagram
  \[
    \begin{tikzcd}
      \cochaincomp{A} & \cochaincomp{B} & \cochaincomp{C} &
      {\cochaincomp{A}[1]} \\
      \cochaincomp{A'} & \cochaincomp{B'} & \cone{u'} & {\cochaincomp{A'}[1]}
      \arrow["u", from=1-1, to=1-2]
      \arrow["f"', from=1-1, to=2-1]
      \arrow["v", from=1-2, to=1-3]
      \arrow["g"', from=1-2, to=2-2]
      \arrow["w", from=1-3, to=1-4]
      \arrow["h", from=1-3, to=2-3]
      \arrow["{f[1]}", from=1-4, to=2-4]
      \arrow["{u'}"', from=2-1, to=2-2]
      \arrow["{v'}"', from=2-2, to=2-3]
      \arrow["{\delta}"', from=2-3, to=2-4]
    \end{tikzcd}
  \]
  in $\homcat{\abcat{A}}$ and a long exact cohomology sequence
  \[
    \cdots
    \stackrel{\delta_{i - 1}^*}{\longrightarrow}
    H^i(\cochaincomp{A})
    \stackrel{(u_i')^*}{\longrightarrow}
    H^i(\cochaincomp{B})
    \stackrel{(v_i')^*}{\longrightarrow}
    H^i(\cone{u'})
    \stackrel{\delta_i^*}{\longrightarrow}
    H^{i + 1}(\cochaincomp{A})
    \stackrel{(u_{i + 1}')^*}{\longrightarrow}
    \cdots
  \]
  induced by strictness of the bottom row.
  Since the vertical maps are isomorphisms in $\homcat{\abcat{A}}$,
  they induce isomorphisms of cohomology. Functors preserve
  isomorphisms, so commutativity of each square now gives a long
  exact cohomology sequence for $\cochaincomp{A}$, $\cochaincomp{B}$
  and $\cochaincomp{C}$. \todo{I want to make this more precise,
    because it looks like we're not getting exactly a sequence induced
    by the maps $u$, $v$ and $w$ as the statement suggests. But it's
  not really a priority right now.}
\end{proof}
\fi

\begin{remark}
  \label{rem_ses_exact_triangs_in_homcat}
  Exact triangles in $\homcat{\abcat{A}}$ are not a perfect
  substitute for exact sequences in $\Ch{\abcat{A}}$.
  To wit, not every short exact sequence
  \[
    \begin{tikzcd}
      \cochaincomp{0} \arrow[r]
      & \cochaincomp{A} \arrow[r, "u"]
      & \cochaincomp{B} \arrow[r, "v"]
      & \cochaincomp{C} \arrow[r]
      & \cochaincomp{0}
    \end{tikzcd}
  \]
  in $\Ch{\abcat{A}}$ can be completed to an exact triangle $(u, v,
  w)$ in $\homcat{\abcat{A}}$, in the sense that the last map $w:
  \cochaincomp{C} \to \cochaincomp{A}[1]$ may not exist.
  For an example, see \cite[Exercise~10.1.2]{weibel}.
\end{remark}

\begin{remark}
  \label{rem_homcat_of_additive_cat}
  With the exception of \cref{prop_homcat_long_exact_coh_seq} (which
  requires that mapping cones exist), everything we have discussed in
  this section only depends on the fact that $\abcat{A}$ is additive.
  This allows us to make sense of $\homcat{\abcat{C}}$ whenever
  $\cat{C}$ is a full additive subcategory of $\abcat{A}$, which will
  have important implications for the derived category later.
  It is also clear that everything we have discussed works with
  $\homcat[*]{\abcat{A}}$ in place of $\homcat{\abcat{A}}$.
\end{remark}

\section{Triangulated categories}
\label{sect_triangcat}

This particular structure of triangles, including the fact that exact
triangles give rise to long exact cohomology sequences, can be found
in categories other than the homotopy category.
Since this will apply also to derived categories, it is worth taking
an intermission to discuss the general case, which is that of
Verdier's \emph{triangulated categories}.
% These categories in fact have an intimately connected history with
% derived categories, having been formulated by Verdier in 1963
% during the same period as Grothendieck's seminal work on derived categories.

We start with a general definition of triangles and morphisms of triangles.
These concepts are similar to what we have already seen for
$\homcat{\abcat{A}}$, but will make sense in any category with a
functor similar to the shift of complexes.

\begin{definition}
  A \emph{translation functor} on a category $\cat{C}$ is an
  autoequivalence (i.e. an equivalence of categories $T: \cat{C} \to \cat{C}$).
  For notational ease, we shall write $[n] = T^n$ and $[-n] = T^{-n}$
  for each $n > 0$, where $T^n$ and $T^{-n}$ are the $n$-fold
  applications of $T$ and $T^{-1}$ respectively.
\end{definition}

\begin{definition}
  Let $\cat{C}$ be a category equipped with a \emph{translation functor} $T$.
  Given an ordered triple of objects $(A, B, C)$ in $\cat{C}$, a
  \emph{triangle} over $(A, B, C)$ is an ordered triple $(u, v, w)$
  corresponding to a sequence
  \[
    \begin{tikzcd}
      A \arrow[r, "u"]
      & B \arrow[r, "v"]
      & C \arrow[r, "w"]
      & A[1],
    \end{tikzcd}
  \]
  so called as this sequence may suggestively be `folded' into a
  triangular diagram
  \[
    \begin{tikzcd}
      & C \\
      A && B.
      \arrow["w"', from=1-2, to=2-1]
      \arrow["u"', from=2-1, to=2-3]
      \arrow["v"', from=2-3, to=1-2]
    \end{tikzcd}
  \]
\end{definition}

\begin{definition}
  Given two triangles $(u, v, w)$ and $(u', v', w')$ over $(A, B, C)$
  and $(A', B', C')$ respectively, a \emph{morphism of triangles} is
  an ordered triple of morphisms $(f, g, h)$ such that the diagram
  \[
    \begin{tikzcd}
      A & B & C & {A[1]} \\
      {A'} & B' & C' & {A'[1]}
      \arrow["u", from=1-1, to=1-2]
      \arrow["f"', from=1-1, to=2-1]
      \arrow["v", from=1-2, to=1-3]
      \arrow["g"', from=1-2, to=2-2]
      \arrow["w", from=1-3, to=1-4]
      \arrow["h", from=1-3, to=2-3]
      \arrow["{f[1]}", from=1-4, to=2-4]
      \arrow["{u'}"', from=2-1, to=2-2]
      \arrow["{v'}"', from=2-2, to=2-3]
      \arrow["{w'}"', from=2-3, to=2-4]
    \end{tikzcd}
  \]
  commutes in $\cat{C}$, and an \emph{isomorphism of triangles} if
  $f$, $g$ and $h$ are isomorphisms.
\end{definition}

In a manner of speaking, a triangulated category is one in which all
triangles `play nicely' with a certain family of triangles analogous
to the exact triangles in $\homcat{\abcat{A}}$.
This is made more precise by the following definition, rather
infamous for its complexity.

\begin{definition}
  A \emph{triangulated category} is an additive category
  $\triangcat{D}$ equipped with a translation functor $T$ and a
  family of triangles, called the \emph{exact} (or
  \emph{distinguished}) triangles in $\triangcat{D}$, subject to the
  following four axioms:

  \begin{enumerate}[leftmargin=4.4em]
    \item[(TR1)]
      For each object $A$, the triangle $(\id_A, 0, 0)$ is exact over
      $(A, A, 0)$.
      For each morphism $u: A \to B$, there exists an object $C$
      together with morphisms $v$ and $w$ so that the triangle $(u,
      v, w)$ is exact over $(A, B, C)$.
      Any triangle isomorphic to an exact triangle is also exact.

    \item[(TR2)]
      \emph{(The rotation axiom.)}
      If the triangle
      \[
        \begin{tikzcd}
          A \arrow[r, "u"]
          & B \arrow[r, "v"]
          & C \arrow[r, "w"]
          & A[1]
        \end{tikzcd}
      \]
      is exact, then so are the rotated triangles
      \[
        \begin{tikzcd}[cramped]
          B & C & {A[1]} & {B[1]}, \\
          {C[-1]} & A & B & C.
          \arrow["v", from=1-1, to=1-2]
          \arrow["w", from=1-2, to=1-3]
          \arrow["{-u[1]}", from=1-3, to=1-4]
          \arrow["{-w[-1]}", from=2-1, to=2-2]
          \arrow["u", from=2-2, to=2-3]
          \arrow["v", from=2-3, to=2-4]
        \end{tikzcd}
      \]
      % If $(u, v, w)$ is exact over $(A, B, C)$, then the triangles
      % \[
      %     \begin{tikzcd}
      %       & B &&& {T(A)} \\
      %       {T^{-1}(C)} && A, & B && C
      %       \arrow["v"', from=1-2, to=2-1]
      %       \arrow["{-T(u)}"', from=1-5, to=2-4]
      %       \arrow["{-T^{-1}(w)}"', from=2-1, to=2-3]
      %       \arrow["u"', from=2-3, to=1-2]
      %       \arrow["v"', from=2-4, to=2-6]
      %       \arrow["w"', from=2-6, to=1-5]
      %     \end{tikzcd}
      % \]
      % are also exact.
      % \todo{Think about whether changing shift from $[-1]$ to $[1]$
      % means $T^{-1}$ should be $T$.}

    \item[(TR3)]
      \emph{(The morphism axiom.)}
      Given exact triangles $(u, v, w)$ and $(u', v', w')$ over $(A,
      B, C)$ and $(A', B', C')$ respectively, as well as morphisms
      $f: A \to A'$ and $g: B \to B'$ such that $gu = u' f$, there
      exists a morphism $h: C \to C'$ so that $(f, g, h)$ is a
      morphism of triangles, i.e.
      \[
        \begin{tikzcd}
          A & B & C & {A[1]} \\
          {A'} & B' & C' & {A'[1]}
          \arrow["u", from=1-1, to=1-2]
          \arrow["f"', from=1-1, to=2-1]
          \arrow["v", from=1-2, to=1-3]
          \arrow["g"', from=1-2, to=2-2]
          \arrow["w", from=1-3, to=1-4]
          \arrow[dashed, "h", from=1-3, to=2-3]
          \arrow["{f[1]}", from=1-4, to=2-4]
          \arrow["{u'}"', from=2-1, to=2-2]
          \arrow["{v'}"', from=2-2, to=2-3]
          \arrow["{w'}"', from=2-3, to=2-4]
        \end{tikzcd}
      \]
      may be completed to a commutative diagram in $\triangcat{D}$.

    \item[(TR4)]
      \emph{(The octahedral axiom.)}
      Omitted, but the interested reader may refer to
      \cite[Definition~10.2.1]{weibel} for a full statement.
  \end{enumerate}
\end{definition}

Our reasons for omitting (TR4) are twofold.
It is not needed in this thesis, since we will not prove that any
categories are triangulated, and the axiom itself is rather
complicated to print due to the octahedral diagram from which it gets its name.
% Separately, there is also some debate amongst experts as to whether
% these axioms are actually the `correct' definition of triangulated categories.
% Higher-minded replacements for triangulated categories have been
% proposed, such as \emph{differential graded categories} and
% \emph{stable $\infty$-categories}.

\begin{example}
  We claimed that $\homcat{\abcat{A}}$ is triangulated, although so
  far we have only checked that (TR1) holds.
  For the remaining details, see \cite[Proposition~10.2.4]{weibel}.
  Naturally, the same is true for the categories
  $\homcat[*]{\abcat{A}}$, and it will even be true for
  $\homcat[*]{\cat{C}}$ when $\cat{C}$ is some full additive
  subcategory of $\abcat{A}$.
\end{example}

\begin{remark}
  A triangulated category $\triangcat{D}$ is usually not abelian, and
  it turns out that a necessary and sufficient condition for this to
  occur is that $\triangcat{D}$ is a \emph{semisimple} category.
  For details, see
  \cite[Section~III.2.3~and~Exercise~IV.1.1]{gelfand_and_manin}.
  In the particular case of $\homcat{\abcat{A}}$, one notable
  obstruction is that homotopy equivalent maps most likely do not
  share the same kernels or cokernels.
\end{remark}

Of course, what we would really like in order to make the
substitution most effective is a generalisation the result of
\cref{prop_homcat_long_exact_coh_seq} to arbitrary triangulated categories.

\begin{definition}
  Let $\triangcat{D}$ be a triangulated category and $\abcat{A}$ an
  abelian category.
  An additive functor $H: \triangcat{D} \to \abcat{A}$ is said to be
  a \emph{cohomological functor} if, for any exact triangle $(u, v,
  w)$ on $(A, B, C)$, there is a long exact sequence
  \[
    \begin{tikzcd}
      \cdots \arrow[r]
      & H(A[i]) \arrow[r, "u_i^*"]
      & H(B[i]) \arrow[r, "v_i^*"]
      & H(C[i]) \arrow[r, "w_i^*"]
      & H(A[i + 1]) \arrow[r]
      & \cdots
    \end{tikzcd}
  \]
  in $\cat{A}$, where the induced maps are given by $f_i^* = H(f[i])$.
  In this long exact sequence, we often write $H^i = H \circ [i]$ for
  $i > 0$ and $H^0 = H$ for notational ease.
\end{definition}

% \begin{example}
%     Another important example is $\Hom(A, -): \triangcat{K} \to
% \Ab$ for any object $A$ of a triangulated category $\triangcat{K}$,
% which is explored in \cite[Example~10.2.8]{weibel}.
% \end{example}

In other words, cohomological functors send exact triangles to long
exact sequences, but we may also consider functors that instead send
them to exact triangles.
This fills the remaining hole left by the loss of exact sequences,
which is a substitute for the notion of exact functors between
triangulated categories, and will prove useful in \cref{sect_derfunc}.
Though some authors also use the term `exact' to refer to such a
functor, we will opt to make the terminological distinction between
usual exactness and triangulated exactness clear in the following definition.
% After one convinces themselves that the opposite category of a
% triangulated category is also triangulated, we may dualise this
% definition to obtain contravariant cohomological functors.

\begin{definition}
  Let $\triangcat{D}$ and $\triangcat{D}'$ be triangulated
  categories, and denote by $T_{\triangcat{D}}$ and
  $T_{\triangcat{D}'}$ their respective translation functors.
  We say that an additive functor $F: \triangcat{D} \to
  \triangcat{D}'$ is an \emph{exact triangulated functor} if it
  satisfies the following two conditions:
  \begin{enumerate}
    \item
      $F$ commutes with translation up to natural isomorphism,
      i.e. there exists a natural isomorphism of composite functors
      $\eta: F \circ T_{\triangcat{D}} \Rightarrow T_{\triangcat{D}'} \circ F$.
      % That is, for each object $A$ in $\triangcat{K}$ there is an
      % isomorphism $\eta_A: F(T_{\triangcat{K}}(A)) \to
      % T_\triangcat{L}(F(A))$, and for each morphism $f: A \to B$ in
      % $\triangcat{K}$ the associated naturality square commutes in
      % $\triangcat{L}$:
      % \[
      %     \begin{tikzcd}[cramped]
      %       {F(T_{\triangcat{K}}(A))} && {T_{\triangcat{L}}(F(A))} \\
      %       \\
      %       {F(T_{\triangcat{K}}(B))} && {T_{\triangcat{L}}(F(B)).}
      %       \arrow["{\eta_A}", from=1-1, to=1-3]
      %       \arrow["{F(T_{\triangcat{K}}(f))}"', from=1-1, to=3-1]
      %       \arrow["{T_{\triangcat{L}}(F(f))}", from=1-3, to=3-3]
      %       \arrow["{\eta_B}"', from=3-1, to=3-3]
      %     \end{tikzcd}
      % \]

    \item
      $F$ sends exact triangles to exact triangles, in the sense that
      for each exact triangle
      \[
        \begin{tikzcd}
          A \arrow[r, "u"]
          & B \arrow[r, "v"]
          & C \arrow[r, "w"]
          %   & T_{\triangcat{K}}(A)
          & A[1]
        \end{tikzcd}
      \]
      in $\triangcat{D}$, the following is an exact triangle in
      $\triangcat{D}'$:
      \[
        \begin{tikzcd}
          {F(A)} & {F(B)} & {F(C)} && {F(A)[1]}.
          \arrow["{F(u)}", from=1-1, to=1-2]
          \arrow["{F(v)}", from=1-2, to=1-3]
          \arrow["{\eta_A \circ F(w)}", from=1-3, to=1-5]
        \end{tikzcd}
      \]
  \end{enumerate}
\end{definition}

\iffalse
\begin{remark}
  This definition may seem stronger than initially stated, but the
  first condition is essential to impose, as applying $F$ to an exact
  triangle only gives a sequence
  \[
    \begin{tikzcd}
      F(A) \arrow[r, "F(u)"]
      & F(B) \arrow[r, "F(v)"]
      & F(C) \arrow[r, "F(w)"]
      & F(A[1])
    \end{tikzcd}
  \]
  in $\triangcat{D}'$, which is not a well-formed triangle without
  the isomorphism
  % $F(T_{\triangcat{K}}(A)) \cong T_{\triangcat{L}}(F(A))$.
  $F(A[1]) \cong F(A)[1]$.
\end{remark}
\fi

\section{Localising the homotopy category}
\label{sect_localising_homcat}

We now have a triangulated category $\homcat{\abcat{A}}$ that turns
homotopy equivalences into isomorphisms.
While it is true that homotopy equivalences are quasi-isomorphisms,
we have noted previously that the converse is generally false.
If we are yet to fully achieve our goal of identifying complexes with
the same cohomology, this observation suggests that we need a way to
make the quasi-isomorphisms actual isomorphisms too.
Taking cues from the method in commutative algebra of the same name,
the second and final phase of constructing the derived category is to
take a \emph{localisation} of $\homcat{\abcat{A}}$.

\begin{definition}
  \label{def_univ_prop_of_locln}
  Let $S$ be a collection of morphisms in a category $\cat{C}$.
  A \emph{localisation of $\cat{C}$ with respect to $S$} is a
  category $\locln{\cat{C}}{S}$ and a \emph{localisation functor}
  $Q_S: \abcat{C} \to \locln{\abcat{C}}{S}$ such that $Q_S(s)$ is an
  isomorphism for every $s \in S$, and which is universal among all
  functors with this property.
  That is, any functor $F: \cat{C} \to \cat{D}$ such that $F(s)$ is
  an isomorphism in $\cat{D}$ for every $s \in S$ factors uniquely
  through $Q_S$:
  \[
    \begin{tikzcd}
      {\cat{C}} & {\cat{D}.} \\
      {\locln{\cat{C}}{S}}
      \arrow["F", from=1-1, to=1-2]
      \arrow["Q_S"', from=1-1, to=2-1]
      \arrow["\exists!"', dashed, from=2-1, to=1-2]
    \end{tikzcd}
  \]
  \vspace{-12pt}
\end{definition}

\iffalse
\begin{example}
  Because of \cref{prop_univ_prop_of_homcat},
  $\homcat{\abcat{A}}$ is in fact a localisation of $\Ch{\abcat{A}}$
  with respect to homotopy equivalences.
\end{example}
\fi

The universal property shows that if a localisation exists, it is
unique up to an equivalence of categories.
We will follow the explicit construction given by the \emph{calculus
of fractions} proposed by Gabriel and Zisman in 1967.
The mild hypothesis that this calculus requires is that $S$ forms a
\emph{multiplicative system}.
There are some more foundational concerns (i.e. hom-sets potentially
no longer being sets) that are technically relevant in what follows,
but which we will ignore.

\begin{definition}
  A \emph{multiplicative system} in a category $\cat{C}$ is a
  collection of morphisms $S$ satisfying the following three axioms:
  \begin{enumerate}
    \item
      \emph{(Closure.)}
      $S$ is closed under composition and contains all identity morphisms.

    \item
      \emph{(The Ore condition.)}
      Given $t: D \to B$ in $S$ and $g: A \to B$ in $\cat{C}$, there
      exist $s: C \to A$ in $S$ and $f: C \to D$ in $\cat{C}$ such
      that the diagram
      \[
        \begin{tikzcd}
          C & D \\
          A & B
          \arrow["{f}", dashed, from=1-1, to=1-2]
          \arrow["{s}"', dashed, from=1-1, to=2-1]
          \arrow["t", from=1-2, to=2-2]
          \arrow["g"', from=2-1, to=2-2]
        \end{tikzcd}
      \]
      commutes.
      Symmetrically, given the morphisms $f$ and $s$ as in the above
      diagram, one may find such $t$ and $g$.
      This is expressed informally as $t^{-1} g = f s^{-1}$.

    \item
      \emph{(Cancellation.)}
      If $f,g: A \to B$ are morphisms in $\cat{C}$, then $sf = sg$
      for some $s: B \to C$ in $S$ if and only if $ft = gt$ for some
      $t: D \to A$ in $S$.
  \end{enumerate}
\end{definition}

\begin{proposition}
  Let $S$ be the collection of quasi-isomorphisms in $\homcat{\abcat{A}}$.
  Then $S$ is a multiplicative system in $\homcat{\abcat{A}}$.
\end{proposition}

\begin{proof}
  By definition, $S$ is the collection of morphisms such that
  $H^i(s)$ is an isomorphism for all $i \in \bb{Z}$ and $s \in S$, so
  we say that $S$ \emph{arises from} the functor $H^0:
  \homcat{\abcat{A}} \to \abcat{A}$.
  By the first claim of \cite[Proposition~10.4.1]{weibel}, any
  collection arising from a cohomological functor must be a
  multiplicative system.
\end{proof}

Proceeding according to the calculus of fractions, localising
$\homcat[*]{\abcat{A}}$ does not require changing the set of objects.
It therefore remains to say what the morphisms should be and how to
compose them, which is the actual namesake of this construction.

\begin{definition}
  An arrangement of cochain maps
  $\leftfrac{\cochaincomp{A}}{s}{\cochaincomp{X}}{f}{\cochaincomp{B}}$
  in $\homcat{\abcat{A}}$ is called a \emph{fraction} (or
  \emph{roof}) if $s$ a quasi-isomorphism.
  We use the more compact and suggestive notation $fs^{-1}$ whenever
  the underlying complexes are understood.
\end{definition}

Given compatible fractions
$\leftfrac{\cochaincomp{A}}{s}{\cochaincomp{X}}{f}{\cochaincomp{B}}$
and
$\leftfrac{\cochaincomp{B}}{t}{\cochaincomp{Y}}{g}{\cochaincomp{C}}$,
let us attempt to compose them.
Using the Ore condition, choose a quasi-isomorphism $r:
\cochaincomp{Z} \to \cochaincomp{X}$ and a cochain map $h:
\cochaincomp{Z} \to \cochaincomp{Y}$ fitting into a commutative diagram
\[
  \begin{tikzcd}
    && \cochaincomp{B} \\
    & \cochaincomp{X} && \cochaincomp{Y} \\
    \cochaincomp{A} && \cochaincomp{Z} && \cochaincomp{C}.
    \arrow["f", from=2-2, to=1-3]
    \arrow["s"', from=2-2, to=3-1]
    \arrow["t"', from=2-4, to=1-3]
    \arrow["g", from=2-4, to=3-5]
    \arrow["r", dashed, from=3-3, to=2-2]
    \arrow["h"', dashed, from=3-3, to=2-4]
  \end{tikzcd}
\]
We declare the fraction
\[
  \leftfrac{\cochaincomp{A}}{sr}{\cochaincomp{Z}}{gh}{\cochaincomp{C}}
\]
to be `the' composite of $fs^{-1}$ and $gt^{-1}$.
It is clear that this composition operation is associative, and that
the fraction
\[
  \leftfrac{\cochaincomp{A}}{\id}{\cochaincomp{A}}{\id}{\cochaincomp{A}}
\]
acts as a two-sided identity.
However, one issue is that since $r$ and $h$ are not guaranteed to be
unique, neither in general is the composition of two fractions.
To remedy this, we need a notion of fraction equivalence.

\begin{definition}
  Two fractions
  \[
    \leftfrac{\cochaincomp{A}}{s_1}{\cochaincomp{X_1}}{f_1}{\cochaincomp{B}}
    \mathand
    \leftfrac{\cochaincomp{A}}{s_2}{\cochaincomp{X_2}}{f_2}{\cochaincomp{B}}
  \]
  are \emph{equivalent} if they are \emph{dominated} by a third
  fraction
  $\leftfrac{\cochaincomp{A}}{s_3}{\cochaincomp{X_3}}{f_3}{\cochaincomp{B}}$,
  i.e. there are cochain maps $\cochaincomp{X_3} \to
  \cochaincomp{X_1}$ and $\cochaincomp{X_3} \to \cochaincomp{X_2}$
  completing a commutative diagram
  \[
    \begin{tikzcd}
      && \cochaincomp{X_3} \\
      & \cochaincomp{X_1} && \cochaincomp{X_2} \\
      \cochaincomp{A} &&&& \cochaincomp{B}
      \arrow["s_1"{description}, from=2-2, to=3-1]
      \arrow["f_1"{description}, from=2-2, to=3-5]
      \arrow["s_2"{description}, from=2-4, to=3-1]
      \arrow[dashed, from=1-3, to=2-2]
      \arrow[dashed, from=1-3, to=2-4]
      \arrow["f_2"{description}, from=2-4, to=3-5]
      \arrow[dashed, "s_3"{description}, curve={height=30pt}, from=1-3, to=3-1]
      \arrow[dashed, "f_3"{description}, curve={height=-30pt}, from=1-3, to=3-5]
    \end{tikzcd}
  \]
  in $\homcat{\abcat{A}}$.
  The collection of equivalence classes of fractions is denoted
  $\Hom_S(\cochaincomp{A}, \cochaincomp{B})$.
\end{definition}

To amend the composition procedure, we now choose a representative
fraction for both classes, composing them as before and then declare
the class of the result to be the composite.
We omit the verification that this is independent of the choice of
representatives and thus a well-defined operation, since it is just
an exercise in setting up the appropriate diagram and checking commutativity.

In view of our earlier observations, the upshot is that the
collections $\Hom_S(\cochaincomp{A}, \cochaincomp{B})$ fulfil the
properties needed to be the hom-sets of a category.
Thus at long last, everything is now in place to define the derived category.
By the Gabriel-Zisman theorem, the definition below will precisely be
a localisation of the homotopy category with respect to the
quasi-isomorphisms, and so has the property we have sought after.
Moreover, thanks to \cite[Proposition~10.4.1]{weibel}, we do not
sacrifice the triangulated structure.

\begin{definition}
  Let $\abcat{A}$ be an abelian category, and $S$ the multiplicative
  system of quasi-isomorphisms in $\homcat{\abcat{A}}$.
  The \emph{derived category} of $\abcat{A}$ is the triangulated
  category $\dercat{\abcat{A}}$ with the same objects as
  $\homcat{\abcat{A}}$ and hom-sets
  \[
    \Hom_{\dercat{\abcat{A}}}(\cochaincomp{A}, \cochaincomp{B}) :=
    \Hom_S(\cochaincomp{A}, \cochaincomp{B}).
  \]
  The localisation functor $\dercatlocln{\abcat{A}}:
  \homcat{\abcat{A}} \to \dercat{\abcat{A}}$ is the identity on
  objects and sends a cochain map $f: \cochaincomp{A} \to
  \cochaincomp{B}$ to the fraction
  \[
    \leftfrac{\cochaincomp{A}}{\id}{\cochaincomp{A}}{f}{\cochaincomp{B}}.
  \]
  The \emph{bounded} derived categories $\dercat[*]{\abcat{A}}$ and
  localisation functors $\dercatlocln[*]{\abcat{A}}:
  \homcat[*]{\abcat{A}} \to \dercat[*]{\abcat{A}}$ are defined
  similarly, and are all full subcategories of $\dercat{\abcat{A}}$.
\end{definition}

\begin{proposition}[{\cite[Proposition~III.5.2]{gelfand_and_manin}}]
  The composite functor
  \[
    \begin{tikzcd}[cramped]
      \iota_{\abcat{A}}: {\abcat{A}} & {\Ch[b]{\abcat{A}}} &
      {\homcat[b]{\abcat{A}}} & {\dercat[b]{\abcat{A}}}
      \arrow[hook, from=1-1, to=1-2]
      \arrow["\pi", from=1-2, to=1-3]
      \arrow["{\dercatlocln[*]{\abcat{A}}}", from=1-3, to=1-4]
    \end{tikzcd}
  \]
  which concentrates objects and morphisms in degree 0 is fully faithful.
\end{proposition}

\iffalse
\begin{proposition}[{\cite[Corollaries~10.3.9--10]{weibel}}]
  \label{prop_locln_properties_of_dercat}
  \begin{enumerate}
    \item
      Two complexes $\cochaincomp{A}$ and $\cochaincomp{B}$ in
      $\Ch{\abcat{A}}$ are isomorphic in $\dercat{\abcat{A}}$ if and only
      if they are quasi-isomorphic in $\homcat{\abcat{A}}$.
      Thus $\cochaincomp{A}$ is zero if and only if it is exact/acyclic.

    \item
      Two cochain maps $f,\, g: \cochaincomp{A} \to
      \cochaincomp{B}$ are the same in $\dercat{\abcat{A}}$ if and only
      if $sf = sg$ for some quasi-isomorphism $s: \cochaincomp{B} \to
      \cochaincomp{C}$.
      Thus $f$ is zero in $\dercat{\abcat{A}}$ if and only if
      $sf$ is null-homotopic.
  \end{enumerate}
\end{proposition}
\fi

With the derived category now at hand, it is worth pausing to reflect
on the goals we established at the start of this chapter.
By \cref{rem_res_is_a_qis}, any resolution of an object in
$\abcat{A}$ is quasi-isomorphic to the image of that object under the
above embedding functor, and by the universal property of
$\dercatlocln{\abcat{A}}$ this becomes an isomorphism of objects in
$\dercat{\abcat{A}}$.
Thus in the derived category, the passage from an object to any of
its resolutions is completely natural, which was precisely the main
motivation for trying to invert quasi-isomorphisms in the first place.
Another benefit of $\dercat{\abcat{A}}$ over $\homcat{\abcat{A}}$ is
that the triangulated structure of the derived category is better behaved.

\begin{proposition}[{\cite[Example~10.4.9]{weibel}}]
  \label{prop_dercat_ses_exact_triang}
  Every short exact sequence
  \[
    \begin{tikzcd}
      \cochaincomp{0} \arrow[r]
      & \cochaincomp{A} \arrow[r, "u"]
      & \cochaincomp{B} \arrow[r, "v"]
      & \cochaincomp{C} \arrow[r]
      & \cochaincomp{0}
    \end{tikzcd}
  \]
  in $\Ch{\abcat{A}}$ fits into an exact triangle $(u, v, w)$ on
  $(\cochaincomp{A}, \cochaincomp{B}, \cochaincomp{C})$ in
  $\dercat{\abcat{A}}$ isomorphic to the strict triangle on $u$.
\end{proposition}

The price to pay for all of this on-paper computational convenience
is that compared to $\homcat{\abcat{A}}$, the morphisms in
$\dercat{\abcat{A}}$ are more elaborate and can be difficult to
actually work with in general, especially in unbounded derived categories.
This may seem like a perverse outcome, but in practice one is most
interested in the bounded derived categories anyway, and it is often
the case that $\abcat{A}$ has certain properties which allow for
simplifications to be made.
We now discuss what can be done in such cases.

The need to localise the homotopy category arose from the observation
that not all quasi-isomorphisms are homotopy equivalences.
However, we have also seen that a partial converse for bounded below
complexes of injectives holds by \cref{prop_qis_is_homotopy_equiv_partial_conv}.
On the face of it, this suggests that whenever $\abcat{A}$ has enough
injectives and $\abcat{I}$ is the full additive subcategory of
$\abcat{A}$ consisting of injectives, the triangulated subcategory
$\homcat[+]{\abcat{I}}$ of $\homcat{\abcat{A}}$ already suffices in
order to invert quasi-isomorphisms.
% Both this statement and its dual turn out to be true in the following way.

\begin{proposition}[{\cite[Theorem~10.4.8]{weibel}}]
  \label{prop_bounded_below_dercat_equiv}
  Let $\abcat{A}$ be an abelian category with enough injectives and
  $\abcat{I}$ the full additive subcategory of injectives in $\abcat{A}$.
  Then
  \[
    \begin{tikzcd}
      {T^+:\homcat[+]{\abcat{I}}} & {\homcat[+]{\abcat{A}}} &
      {\dercat[+]{\abcat{A}}}
      \arrow[hook, from=1-1, to=1-2]
      \arrow["{\dercatlocln[+]{\abcat{A}}}", from=1-2, to=1-3]
    \end{tikzcd}
  \]
  is an equivalence of categories.
  % Dually, if $\abcat{A}$ has enough projectives and $\abcat{P}$ is
  % the full additive subcategory of $\abcat{A}$ consisting of projectives, then
  % \[
  %     \begin{tikzcd}
  %       {\homcat[-]{\abcat{P}}} & {\homcat[-]{\abcat{A}}} &
  % {\dercat[-]{\abcat{A}}}
  %       \arrow[hook, from=1-1, to=1-2]
  %       \arrow["{\dercatlocln[-]{\abcat{A}}}", from=1-2, to=1-3]
  %     \end{tikzcd}
  % \]
  % is an equivalence of categories.
\end{proposition}

We will not prove this.
The argument in \cite{weibel} uses the machinery of
\emph{Cartan-Eilenberg resolutions} (i.e. double complexes acting as
`resolutions' for complexes) and formal properties of localisation.
Alternatively, the (longer) proof of
\cite[Proposition~2.40]{huybrechts} shows more directly that the
composite is fully faithful and essentially surjective, i.e.
\begin{equation}
  \label{eq_bij_dercat_homcat_injs}
  \Hom_{\homcat[+]{\abcat{I}}}(\cochaincomp{I}, \cochaincomp{J})
  \longrightarrow
  \Hom_{\dercat[+]{\abcat{A}}}(\cochaincomp{I}, \cochaincomp{J})
\end{equation}
is bijective and every complex $\cochaincomp{A} \in
\dercat[+]{\abcat{A}}$ is isomorphic in $\dercat[+]{\abcat{A}}$ to
some complex $\cochaincomp{I} \in \homcat[+]{\abcat{I}}$.
Unfortunately, this equivalence of categories does not immediately
descend to $\dercat[b]{\abcat{A}}$ without the inclusion of an extra hypothesis.

\begin{corollary}
  \label{cor_bounded_dercat_equiv}
  The functor of \cref{prop_bounded_below_dercat_equiv} is an
  equivalence of categories
  \[
    \begin{tikzcd}
      {T^b: \homcat[b]{\abcat{I}}} & {\homcat[b]{\abcat{A}}} &
      {\dercat[b]{\abcat{A}}}
      \arrow[hook, from=1-1, to=1-2]
      \arrow["{\dercatlocln[b]{\abcat{A}}}", from=1-2, to=1-3]
    \end{tikzcd}
  \]
  if all objects of $\abcat{A}$ have injective resolutions of finite length.
\end{corollary}

The dual statements of \cref{prop_bounded_below_dercat_equiv} and
\cref{cor_bounded_dercat_equiv} for projective objects and
$\dercat[-]{\abcat{A}}$ also hold, but this case will be of less interest to us.

\section{Total derived functors}
\label{sect_derfunc}

Aside from allowing one to identify an object with all of its
resolutions, another tangible sense in which derived categories are
`better' for homological algebra is their relationship with the
classical derived functors introduced in \cref{sect_classical_derfunc}.
In this section, we will explore this and show that, in certain
cases, we may consolidate these functors into one \emph{total derived
functor} between derived categories.
\iffalse
For our purposes, it will be sufficient only to go far enough to see
what happens for Ext and Tor.
\fi

To begin, we shall consider the obstructions to an additive functor
$F: \abcat{A} \to \abcat{B}$ between abelian categories inducing a
functor between derived categories.
Any such $F$ extends to an additive functor $\Ch{F}: \Ch{\abcat{A}}
\to \Ch{\abcat{B}}$, and thanks to
\cref{lemma_add_func_preserves_homotopy_and_cones}, it follows that
$\Ch{F}$ extends further to an additive, exact triangulated functor
$\homcat{F}: \homcat{\abcat{A}} \to \homcat{\abcat{B}}$ in the obvious way.
As before, we will commit a slight abuse of notation by conflating
this functor with $F$.
The difficulties begin when we try to extend $F$ to a functor
$\dercat{\abcat{A}} \to \dercat{\abcat{B}}$, that is, to find a
functor $G$ which fits into a commutative diagram
\[
  \begin{tikzcd}
    \homcat{\abcat{A}} & \homcat{\abcat{B}} \\
    \dercat{\abcat{A}} & \dercat{\abcat{B}}.
    \arrow["{\homcat{F}}", from=1-1, to=1-2]
    \arrow["{Q_{\abcat{A}}}"', from=1-1, to=2-1]
    \arrow["{Q_{\abcat{B}}}", from=1-2, to=2-2]
    \arrow["{G}"', dashed, from=2-1, to=2-2]
  \end{tikzcd}
\]
If we are fortunate enough that $F$ preserves quasi-isomorphisms,
then this is possible:
the composite functor $Q_{\abcat{B}} \circ F: \homcat{\abcat{A}} \to
\dercat{\abcat{B}}$ also preserves quasi-isomorphisms, so the
universal property of $Q_{\abcat{A}}$ ensures that such a functor $G$ exists.
By \cref{cor_quasi_isom_has_acyclic_cone} and
\cref{lemma_add_func_preserves_homotopy_and_cones}, we indeed find
ourselves in this situation when $F$ is exact, and moreover it turns
out that $G$ will be an exact triangulated functor.
Under more practical assumptions, we cannot expect that this
na\"{i}ve construction will go through, nor that such a simple
commutative diagram will characterise the resulting functor.

In light of our partial progress, we may as well assume we are given
an exact triangulated functor $F: \homcat{\abcat{A}} \to
\homcat{\abcat{B}}$ as a starting point, which subsumes the typical
case of a left and right exact functor defined at the level of
abelian categories.
Going forward, we will also always need to impose some kind of
boundedness condition on input complexes.

\begin{definition}
  \label{def_total_derfunc}
  Let $F: \homcat[*]{\abcat{A}} \to \homcat{\abcat{B}}$ be an exact
  triangulated functor.
  A \emph{total right derived functor} is an exact triangulated
  functor $\rightderfunc{F}: \dercat[*]{\abcat{A}} \to
  \dercat{\abcat{B}}$ equipped with a natural transformation $\xi:
  Q_{\mathcal{B}} \circ F \Rightarrow \rightderfunc{F} \circ
  Q_{\mathcal{A}}^*$ satisfying the following universal property:
  if $G: \dercat[*]{\abcat{A}} \to \dercat{\abcat{B}}$ is another
  exact triangulated functor, then any natural transformation $\zeta:
  Q_{\mathcal{B}} \circ F \Rightarrow G \circ Q_{\mathcal{A}}^*$
  factors uniquely through $\xi$.
  Explicitly, this means that there is a unique natural
  transformation $\eta: \rightderfunc{F} \Rightarrow G$ such that
  \[
    \zeta_{\cochaincomp{A}}
    = (\eta \circ Q_{\abcat{A}}^*)_{\cochaincomp{A}} \circ \xi_{\cochaincomp{A}}
  \]
  for each $\cochaincomp{A} \in \homcat[*]{\abcat{A}}$, or
  equivalently that the functor diagram
  \[
    \begin{tikzcd}
      {Q_{\abcat{B}} \circ F} & {G \circ Q_{\abcat{A}}^*}. \\
      {\rightderfunc{F} \circ Q_{\abcat{A}}^*} & {}
      \arrow["\zeta", Rightarrow, from=1-1, to=1-2]
      \arrow["\xi"', Rightarrow, from=1-1, to=2-1]
      \arrow["{\eta \, \circ \, Q_{\abcat{A}}^*}"', Rightarrow,
      from=2-1, to=1-2]
    \end{tikzcd}
  \]
  commutes.
  Dually, a \emph{total left derived functor} of $F$ is an exact
  triangulated functor $\leftderfunc{F}: \dercat[*]{\abcat{A}} \to
  \dercat{\abcat{B}}$ and a natural transformation $\xi:
  \leftderfunc{F} \circ Q_{\mathcal{A}}^* \nattrans Q_{\mathcal{B}}
  \circ F$ satisfying the universal property that for any exact
  triangulated functor $G: \dercat[*]{\abcat{A}} \to
  \dercat{\abcat{B}}$, any natural transformation $\zeta: G \circ
  Q_{\mathcal{A}}^* \nattrans Q_{\mathcal{B}} \circ F$ factors through $\xi$.
\end{definition}

In the case where we are initially only given an additive functor $F:
\abcat{A} \to \abcat{B}$, we will also write $\rightderfunc{F}$ and
$\leftderfunc{F}$ for the total derived functors of its induced exact
triangulated functor $\homcat[*]{F}: \homcat[*]{\abcat{A}} \to
\homcat{\abcat{B}}$ as above.

Once again, this is a non-constructive definition.
At the very least, the universal property implies that if a total
left or right derived functor of $F$ exists, it is unique up to
natural isomorphism.
In the presence of stronger assumptions, it turns out that there is a
principled way to go about actually constructing these functors.

\begin{proposition}
  \label{prop_exists_total_derfunc}
  Let $F: \homcat[+]{\abcat{A}} \to \homcat{\abcat{B}}$ be an exact
  triangulated functor, and suppose that $\abcat{A}$ has enough injectives.
  Fix a quasi-inverse $U^+$ of the equivalence $T^+$ in
  \cref{prop_bounded_below_dercat_equiv}.
  Then the composite functor
  \[
    \begin{tikzcd}
      {\rightderfunc[+]{F}: {\dercat[+]{\abcat{A}}}} &
      {\homcat[+]{\abcat{I}_{\abcat{A}}}} & {\homcat[+]{\abcat{B}}} &
      {\dercat[+]{\abcat{B}}}
      \arrow["U^+", from=1-1, to=1-2]
      \arrow["F", from=1-2, to=1-3]
      \arrow["{\dercatlocln[+]{\abcat{B}}}", from=1-3, to=1-4]
    \end{tikzcd}
  \]
  is a total right derived functor of $F$ on ${\dercat[+]{\abcat{A}}}$.
  \iffalse
  Dually, given an exact triangulated functor $F:
  \homcat[-]{\abcat{A}} \to \homcat{\abcat{B}}$ and enough
  projectives in $\abcat{A}$, the composite functor
  \[
    \begin{tikzcd}
      {\leftderfunc[-]{F}: {\dercat[-]{\abcat{A}}}} &
      {\homcat[-]{\abcat{P}_{\abcat{A}}}} & {\homcat[-]{\abcat{B}}} &
      {\dercat[-]{\abcat{B}}}
      \arrow["\equiv", from=1-1, to=1-2]
      \arrow["F", from=1-2, to=1-3]
      \arrow["{\dercatlocln[-]{\abcat{B}}}", from=1-3, to=1-4]
    \end{tikzcd}
  \]
  is a total left derived functor of $F$.
  \fi
\end{proposition}

\begin{proof}
  The most important data for our purposes is the functor
  $\rightderfunc[+]{F}$ itself, but it technically remains to specify
  a natural transformation $\xi: \dercatlocln[+]{\abcat{A}} \circ F
  \nattrans (\dercatlocln[+]{\abcat{B}} \circ F \circ U^+) \circ
  \dercatlocln[+]{\abcat{A}}$.
  We defer the details of this to \cite[Theorem~10.5.6]{weibel}.
\end{proof}

\begin{corollary}
  \label{cor_derfunc_applied_to_proj_inj_complexes}
  Retain the hypotheses of \cref{prop_exists_total_derfunc}.
  If $\cochaincomp{I}$ is a bounded below complex of injectives, then
  $\rightderfunc[+]{F}(\cochaincomp{I}) \cong
  (\dercatlocln[+]{\abcat{B}} \circ F)(\cochaincomp{I})$.
\end{corollary}

The dual statements for total left derived functors naturally hold.
This construction provides two great technical advantages over the
classical approach to derived functors.
The first that we have already mentioned is that in the right
circumstances, the total derived functors capture the ensemble of
classical derived functors from \cref{sect_classical_derfunc}.
In fact, some authors (e.g. \cite{huybrechts}) take the following as
definitions.

\begin{proposition}
  \label{prop_cohomology_of_totderfunc}
  Suppose that $F: \abcat{A} \to \abcat{B}$ is additive.
  Then under suitable conditions on $F$ and $\abcat{A}$ ensuring that
  $\rightderfunc[+]{F}$ or $\leftderfunc[-]{F}$ exist, the $i$th
  classical derived functors of $F$ at an object $A \in \abcat{A}$
  can be recovered as the cohomology
  \[
    R^iF(A) \cong H^i \rightderfunc[+]{F}(A)
    \mathand
    L_iF(A) \cong H^{-i} \leftderfunc[-]{F}(A)
  \]
  when $F$ is left or right exact respectively.
  Naturally, one first embeds objects from $\abcat{A}$ into the
  derived category using the fully faithful functor described in
  \cref{sect_localising_homcat}.
\end{proposition}

\begin{proof}
  Given an injective resolution $A \to \cochaincomp{I}$, we have $A
  \cong \cochaincomp{I}$ as objects in ${\dercat[+]{\abcat{A}}}$.
  Then by \cref{cor_derfunc_applied_to_proj_inj_complexes} and the
  fact that $\dercatlocln[+]{\abcat{B}}$ is the identity on objects, one has
  \[
    H^i \rightderfunc[+]{F}(A)
    \cong H^i \rightderfunc[+]{F}(\cochaincomp{I})
    \cong H^i (F(\cochaincomp{I}))
    = R^i F(A).
  \]
  One argues similarly for the claim in the left derived functor case.
\end{proof}

The other technical advantage is that in suitable situations, derived
functors at the level of derived categories may be composed in a far
simpler manner than derived functors at the level of abelian
categories, which normally requires the use of \emph{spectral sequences}.
The following result shows the necessary hypotheses for composing
total right derived functors, but the dual assertion for total left
derived functors naturally holds.

\begin{proposition}
  \label{prop_total_derfunc_comp}
  Suppose that $F: \homcat[+]{\abcat{A}} \to \homcat[+]{\abcat{B}}$
  and $G: \homcat[+]{\abcat{B}} \to \homcat[+]{\abcat{C}}$ are exact
  triangulated functors for abelian categories $\abcat{A}$,
  $\abcat{B}$ and $\abcat{C}$.
  Let $\cat{I}_{\abcat{A}}$ and $\cat{I}_{\abcat{B}}$ be the
  subcategories of injective objects in $\abcat{A}$ and $\abcat{B}$.
  Then if $\abcat{A}$ and $\abcat{B}$ have enough injectives and $F$
  sends $\abcat{I}_{\abcat{A}}$ into $\cat{I}_{\abcat{B}}$, there is
  a natural isomorphism $\rightderfunc[+](G \circ F) \nattrans
  \rightderfunc[+]{G} \circ \rightderfunc[+]{F}$.
\end{proposition}

\begin{proof}
  Let $\xi_F$, $\xi_G$ and $\xi_{G \circ F}$ be the natural
  transformations associated to the derived functors
  $\rightderfunc[+]{F}$, $\rightderfunc[+]{G}$ and
  $\rightderfunc[+]{(G \circ F)}$ respectively.
  The commutative functor diagram
  \[
    \begin{tikzcd}[cramped]
      {Q_{\abcat{C}}^+ \circ G \circ F} && {\rightderfunc[+]{G} \circ
      Q_{\abcat{B}}^+ \circ F} \\
      \\
      {\rightderfunc[+]{(G \circ F)} \circ Q_{\abcat{A}}^+} &&
      {(\rightderfunc[+]{G}) \circ (\rightderfunc[+]{F}) \circ Q_{\abcat{A}}^+},
      \arrow["{\xi_G \, \circ \, F}", Rightarrow, from=1-1, to=1-3]
      \arrow["{\xi_{G \circ F}}"', Rightarrow, from=1-1, to=3-1]
      \arrow["{\rightderfunc[+]{G} \, \circ \, \xi_F}", Rightarrow,
      from=1-3, to=3-3]
      \arrow["{\eta \, \circ \, Q_{\abcat{A}}^+}"', Rightarrow,
      dashed, from=3-1, to=3-3]
    \end{tikzcd}
  \]
  is completed by a natural transformation $\eta: \rightderfunc[+]{(G
  \circ F)} \nattrans (\rightderfunc[+]{G}) \circ
  (\rightderfunc[+]{F})$ thanks to the universal property of
  $\rightderfunc[+]{(G \circ F)}$.
  Now, take any $\cochaincomp{A} \in \dercat[+]{\abcat{A}}$ and use
  \cref{prop_bounded_below_dercat_equiv} to find a complex of
  injectives $\cochaincomp{I} \in \homcat[+]{\cat{I}_{\abcat{A}}}$
  isomorphic to $\cochaincomp{A}$ in $\dercat[+]{\abcat{A}}$.
  That the component $(\eta \circ Q_{\abcat{A}}^+)_{\cochaincomp{I}}$
  is an isomorphism follows from the calculation
  \begin{align*}
    \rightderfunc[+]{(G \circ F)(\dercatlocln[+]{\abcat{A}}(\cochaincomp{I}))}
    &= (\dercatlocln[+]{\abcat{C}} \circ G \circ F)(\cochaincomp{I}) \\
    &\cong (\rightderfunc[+]{G} \circ \dercatlocln[+]{\abcat{B}}
    \circ F)(\cochaincomp{I}) \\
    &\cong (\rightderfunc[+]{G} \circ
    \rightderfunc[+]{F})(\dercatlocln[+]{\abcat{A}}(\cochaincomp{I}))
  \end{align*}
  by \cref{cor_derfunc_applied_to_proj_inj_complexes}.
  The component $\eta_{\cochaincomp{A}}$ factors through the
  isomorphisms $\cochaincomp{A} \cong \cochaincomp{I}$ and $(\eta
  \circ Q_{\abcat{A}}^+)_{\cochaincomp{I}}$, so we conclude that
  $\eta_{\cochaincomp{A}}$ is also an isomorphism.
\end{proof}

\begin{remark}
  \label{rem_total_derfunc_computing_acyclics}
  Similar to \cref{rem_computing_with_F_acyclics}, the total derived
  functor of an exact triangulated functor $F$ can also be
  constructed instead using complexes of $F$-acyclic objects
  $\cochaincomp{X}$, i.e. such that $H^i(F(\cochaincomp{X})) = 0$ for all $i$.
  As usual, complexes of projectives or injectives are $F$-acyclic
  for every choice of $F$.
  Assuming that every object of $\homcat[*]{\abcat{A}}$ is
  quasi-isomorphic to some $F$-acyclic complex, one can modify the
  proofs of the preceding results so that injective assumptions can
  be suitably replaced with $F$-acyclic assumptions.
  In the statement of \cref{prop_total_derfunc_comp} in particular,
  it is sufficient that $F$ sends complexes of $F$-acyclics to $G$-acyclics.
\end{remark}

We now turn our attention to upgrading Ext and Tor to total derived functors.
It is perfectly valid to take derived functors of the usual Hom and
tensor product functors directly, but it would be most sensible to
work with a more general variant of these functors.
To see why, recall that $\Hom_{\abcat{A}}$ is a covariant bifunctor
\[
  \Hom_{\abcat{A}}(-, -): \op{\abcat{A}} \times \abcat{A} \to \Ab
\]
and notice that we may think of the induced triangulated exact
functor of $\Hom_{\abcat{A}}(A, -)$ on the homotopy category
$\homcat[+]{\abcat{A}}$ also as a covariant bifunctor
\[
  \Hom_{\abcat{A}}(-, -): \op{\abcat{A}} \times \homcat[+]{\abcat{A}}
  \to \homcat{\Ab}.
\]
A similar observation holds for the tensor product functor.
We are thus compelled us to define variants of these functors that
instead take complexes in \emph{both} arguments, and which of course
also induces a triangulated exact functor on $\homcat[*]{\abcat{A}}$.
We will do this by means of constructing an appropriate double
complex and then totalising it.

\begin{definition}
  Let $\cochaincomp{A}$ and $\cochaincomp{B}$ be cochain complexes in
  $\abcat{A}$.
  Consider the double complex $\cochainbicomp{C}$ of abelian groups defined by
  \[
    C^{p, q} = \Hom_{\abcat{A}}(A^{-p}, B^q),
  \]
  with horizontal and vertical differentials defined for each $f:
  A^{-p} \to B^q$ by
  \[
    d_h^{p, q}(f) = d_B \circ f
    \mathand
    d_v^{p, q}(f) = (-1)^{p + q + 1} f \circ d_A.
  \]
  The totalisation $\Hom^{\bullet}_{\abcat{A}}(\cochaincomp{A},
  \cochaincomp{B}) = \totprodcomp{\cochainbicomp{C}}$ is the
  \emph{internal Hom} of $\cochaincomp{A}$ and $\cochaincomp{B}$.
\end{definition}

The sign in $d_v$ is convenient, because with it the cohomology of
internal Hom classifies cochain maps up to homotopy equivalence and shifts.
That is,
\[
  H^n(\Hom^{\bullet}_{\abcat{A}}(\cochaincomp{A}, \cochaincomp{B}))
  \cong
  \Hom_{\dercat{\abcat{A}}}(\cochaincomp{A}, \cochaincomp{B}[n]).
\]
This is explained in more detail in \cite[Section~10.7]{weibel}.
Moreover, since the functor $\Hom_{\abcat{A}}(-, B)$ is contravariant,
% and commutes with products of $\abcat{A}$, <-- wtf does this mean?
it makes sense for the internal Hom to mimic the same contravariance
in first argument, which here will turn a cochain complex into a chain complex.
As a consequence, one must invert the degree of $\cochaincomp{A}$.
In any case, we obtain a functor which induces an exact triangulated
functor on the homotopy category, so we may take the total derived functor.

\begin{definition}
  Suppose that $\abcat{A}$ has enough injectives, and fix a complex
  $\cochaincomp{A}$ in $\abcat{A}$.
  The \emph{derived Hom functor} is the total right derived functor
  \[
    \derhom_{\abcat{A}}(\cochaincomp{A}, \cochaincomp{B})
    := \rightderfunc[+]{\Hom^{\bullet}_{\abcat{A}}(\cochaincomp{A},
    -)}(\cochaincomp{B}).
  \]
  \vspace{-24pt}
\end{definition}

We now repeat the above for the tensor product functor.

\begin{definition}
  Let $\cochaincomp{M}$ and $\cochaincomp{N}$ be cochain complexes of
  right and left $R$-modules respectively.
  Consider the double complex $\cochainbicomp{C}$ of abelian groups defined by
  \[
    C^{p, q} = M^{p} \tensor_R N^{q},
  \]
  with horizontal and vertical differentials defined for each $m
  \tensor n \in M^{p} \tensor_R N^{q}$ by
  \[
    d_h^{p, q}(m \tensor n) = d_M(m) \tensor n
    \mathand
    d_v^{p, q}(m \tensor n) = (-1)^p m \tensor d_N(n).
  \]
  The totalisation $\cochaincomp{M} \tensor_R \cochaincomp{N} =
  \totsumcomp{\cochainbicomp{C}}$ is the \emph{internal tensor
  product} of $\cochaincomp{M}$ and $\cochaincomp{N}$.
\end{definition}

\begin{definition}
  Let $\cochaincomp{M}$ be a complex of right $R$-modules.
  The \emph{derived tensor product functor} is the total left derived
  functor $\dertensor{R}{\cochaincomp{M}}{\cochaincomp{N}} :=
  \leftderfunc[-]{(\cochaincomp{M} \tensor_R -)}(\cochaincomp{N})$.
\end{definition}

\begin{remark}
  The cohomology of the derived Hom and tensor product functors as we
  have now defined them obviously do not recover the usual Ext and
  Tor groups, but rather the \emph{hyperext} and \emph{hypertor}
  groups of the \emph{hyper-derived functors} of $\Hom_{\abcat{A}}$
  and $\tensor_R$.
  These are a notion that sits between classical and total derived
  functors that are unnecessary for our purposes, so we defer further
  discussion to \cite[Chapter~5]{weibel}.
\end{remark}

To close out this chapter, we show that the usual tensor-hom
adjunction has a generalisation to the derived category, which is key
to giving the derived equivalence we seek.
As always, we will assume a commutative ring $R$, in which case both
$\Hom_R(M, N)$ and $M \otimes_R N$ have the structure of an $R$-module.
It follows that $\derhom$ and $\dertensorsymb_R$ map into $\dercat{\Mod{R}}$.
For simplicity, we will write $\dercat[*]{R} = \dercat[*]{\Mod{R}}$.

\begin{theorem}[Derived tensor-hom adjunction]
  Fix a complex $\cochaincomp{B} \in \dercat[-]{R}$.
  Then for any $\cochaincomp{A} \in \dercat[-]{R}$ and
  $\cochaincomp{C} \in \dercat[+]{R}$, there is a natural isomorphism
  \[
    \Hom_{\dercat{R}}(\cochaincomp{A}, \derhom(\cochaincomp{B},
    \cochaincomp{C}))
    \cong
    \Hom_{\dercat{R}}(\dertensor{R}{\cochaincomp{A}}{\cochaincomp{B}},
    \cochaincomp{C}).
  \]
  \vspace{-24pt}
\end{theorem}

\begin{proof}
  We claim instead that there is a natural isomorphism
  \[
    \derhom(\cochaincomp{A}, \derhom(\cochaincomp{B}, \cochaincomp{C}))
    \cong
    \derhom(\dertensor{R}{\cochaincomp{A}}{\cochaincomp{B}}, \cochaincomp{C}),
    \tag{$\dagger$}
  \]
  as this yields the original claim upon taking $H^0$.
  For complexes concentrated in degree 0 (i.e. for $R$-modules $L$,
  $M$ and $N$), one can check after unfolding the definitions of
  $\derhom$ and $\dertensorsymb_R$ that this degenerates to the usual
  tensor-hom adjoint isomorphism
  \[
    \Hom(L, \Hom_{R}(M, N)) \cong \Hom(L \tensor_R M, N), \tag{$\star$}
  \]
  so we now seek to use \cref{prop_total_derfunc_comp} to upgrade
  this to the derived category.
  Since we are working in the derived category, it suffices to assume
  that $\cochaincomp{A}$ and $\cochaincomp{C}$ are complexes of
  projectives and injectives respectively.
  Consider the functors
  \[
    \InnerHom(\cochaincomp{A}, \InnerHom(\cochaincomp{B}, \cochaincomp{C}))
    \mathand
    \InnerHom(\cochaincomp{A} \tensor_R \cochaincomp{B}, \cochaincomp{C})
  \]
  in $\cochaincomp{B}$ in the homotopy category.
  The left-hand side is the composite of the functors
  \[
    F_1(\cochaincomp{B}) = \InnerHom(\cochaincomp{B}, \cochaincomp{C})
    \mathand
    G_1(\cochaincomp{X}) = \InnerHom(\cochaincomp{A}, \cochaincomp{X})
  \]
  while the right-hand side is the composite of
  \[
    F_2(\cochaincomp{B}) = \cochaincomp{A} \tensor_R \cochaincomp{B}
    \mathand
    G_2(\cochaincomp{X}) = \InnerHom(\cochaincomp{X}, \cochaincomp{C}).
  \]
  Since each cochain of $\cochaincomp{A}$ is projective, one can
  check that $F_1$ sends complexes of projectives (i.e. $F_1$-acyclic
  complexes) to complexes of injectives (i.e. $G_1$-acyclic complexes).
  Considering \cref{rem_total_derfunc_computing_acyclics}, we may
  apply \cref{prop_total_derfunc_comp} to obtain the natural
  isomorphism $(\rightderfunc{G_1} \circ
  \rightderfunc{F_1})(\cochaincomp{B}) \cong \rightderfunc{(G_1 \circ
  F_1)}(\cochaincomp{B})$, which is to say that
  \[
    \derhom(\cochaincomp{A}, \derhom(\cochaincomp{B}, \cochaincomp{C}))
    \cong
    \rightderfunc{(\InnerHom(\cochaincomp{A}, \InnerHom(-,
    \cochaincomp{C})))}(\cochaincomp{B}).
  \]
  Similarly, one can check that $F_2$ sends complexes of projectives
  (i.e. $F_2$-acyclic complexes) to complexes of projectives (i.e.
  $G_2$-acyclic complexes).
  Examining the proof of \cref{rem_total_derfunc_computing_acyclics}
  more closely, we see that the conclusion still holds even for the
  mixed total derived functors $(\rightderfunc{G_2} \circ
  \leftderfunc{F_2})(\cochaincomp{B}) \cong \rightderfunc{(G_2 \circ
  F_2)}(\cochaincomp{B})$, which is to say that
  \[
    \derhom(\dertensor{R}{\cochaincomp{A}}{\cochaincomp{B}}, \cochaincomp{C})
    \cong
    \rightderfunc{(\InnerHom(\cochaincomp{A} \tensor_R -,
    \cochaincomp{C}))}(\cochaincomp{B}).
  \]
  By taking appropriate injective and projective resolutions,
  ($\star$) therefore shows that both sides of ($\dagger$) are
  naturally isomorphic to the same total right derived functor.
\end{proof}

