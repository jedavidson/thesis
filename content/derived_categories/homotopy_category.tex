\section{The homotopy category}

Recall that in \cref{sect_complexes}, we introduced $\Ch{\abcat{A}}$,
the category of cochain complexes with objects in $\abcat{A}$, and
noted that it itself was abelian.
In the first phase of constructing the derived category of
$\abcat{A}$, we wish to alter $\Ch{\abcat{A}}$ to identify cochain
maps up to homotopy equivalence.
The justification for this should be clear, as we have already seen
that homotopy equivalent maps preserve cohomology.

\begin{definition}
  Let $\cat{C}$ be a category.
  A \emph{congruence relation} $\sim$ on $\cat{C}$ consists of an
  equivalence relation $\sim_{A, B}$ on each $\Hom_{\cat{C}}(A, B)$
  which is compatible with composition, that is, if $f \sim_{A, B}
  f'$ and $g \sim_{B, C} g'$, then $gf \sim_{A, C} g'f'$.
  We define the \emph{quotient category} $\quot{\cat{C}}{\sim}$ to
  have the same objects as $\cat{C}$, but whose hom-sets are
  equivalence classes of morphisms in hom-sets of $\cat{C}$, i.e.
  \[
    \Hom_{\quot{\cat{C}}{\sim}}(A, B) := \quot{\Hom_{\cat{C}}(A, B)}{\sim}
  \]
  for each pair of objects $A,\, B \in \cat{C}$.
  The full \emph{quotient functor} $\pi: \cat{C} \to
  \quot{\cat{C}}{\sim}$ is the identity on objects and sends
  morphisms to their equivalence class under $\sim$.
\end{definition}

\begin{definition}
  The \emph{homotopy category} of an abelian category $\cat{A}$ is
  the quotient category $\homcat{\abcat{A}} :=
  \quot{\Ch{\abcat{A}}}{\sim}$ under homotopy equivalence.
\end{definition}

Eventually we will consider the \emph{bounded} homotopy categories
$\homcat[*]{\abcat{A}}$, which are the full subcategories of
$\homcat{\abcat{A}}$ corresponding to the categories
$\Ch[*]{\abcat{A}}$ for each $* \in \{b, +, -\}$.
Under certain hypotheses on $\abcat{A}$, considering the full
subcategories of $\homcat[*]{\abcat{A}}$ consisting of complexes of
injectives and projectives is sufficient to describe its bounded
derived category, which are the particular flavour of derived
category we will concern ourselves with.
These details will be made precise at the end of \cref{sect_localising_homcat}.

By construction, the image of any homotopy equivalence in
$\Ch{\abcat{A}}$ under $\pi$ is an isomorphism in $\homcat{\abcat{A}}$.
It is possible to formulate a universal property for
$\homcat{\abcat{A}}$ along these lines, which provides a useful way
of thinking about the homotopy category.

\begin{proposition}[{\cite[Proposition~10.1.2]{weibel}}]
  \label{prop_univ_prop_of_homcat}
  Any functor $F: \Ch{\abcat{A}} \to \cat{D}$ with the property that
  it sends homotopy equivalences in $\Ch{\abcat{A}}$ to isomorphisms
  in $\cat{D}$ factors uniquely through $\homcat{\abcat{A}}$.
  \[
    \begin{tikzcd}
      \Ch{\abcat{A}} & {\cat{D}.} \\
      \homcat{\abcat{A}}
      \arrow["F", from=1-1, to=1-2]
      \arrow["\pi"', from=1-1, to=2-1]
      \arrow["\exists!"', dashed, from=2-1, to=1-2]
    \end{tikzcd}
  \]
\end{proposition}

\begin{proposition}
  The category $\homcat{\abcat{A}}$ is additive and has additive
  quotient functor.
\end{proposition}

After identifying homotopy equivalent maps, we unfortunately lose the
abelian structure of $\Ch{\abcat{A}}$ in general.
Derived categories suffer the same fate, and at a glance this seems
rather damning for any hopes of doing homological algebra, as we no
longer have access to exact sequences.
However, the shift functor on complexes gives $\homcat{\abcat{A}}$
some additional structure beyond being additive, and it turns out
that there is a principled way to think of certain sequences of maps
in a way that mimics exact sequences.

\begin{definition}
  Let $u: \cochaincomp{A} \to \cochaincomp{B}$ be a map of complexes
  in $\Ch{\abcat{A}}$.
  The \emph{strict triangle} on $u$ is the ordered triple of maps
  $(u, v, \delta)$ in $\homcat{\abcat{A}}$, where $v$ and $\delta$
  are maps in $\Ch{\abcat{A}}$ arising from the short exact sequence
  \[
    \begin{tikzcd}
      \cochaincomp{0} \arrow[r]
      & \cochaincomp{B} \arrow[r, "v"]
      & \cone{u} \arrow[r, "\delta"]
      & \cochaincomp{A}[1] \arrow[r]
      & \cochaincomp{0}.
    \end{tikzcd}
  \]
  \vspace{-24pt}
\end{definition}

\begin{definition}
  \label{def_exact_triang_homcat}
  Given maps
  \[
    \begin{tikzcd}
      \cochaincomp{A} \arrow[r, "u"]
      & \cochaincomp{B} \arrow[r, "v"]
      & \cochaincomp{C} \arrow[r, "w"]
      & \cochaincomp{A}[1]
    \end{tikzcd}
  \]
  in $\homcat{\abcat{A}}$, the ordered triple $(u, v, w)$ is called
  an \emph{exact triangle} on $(\cochaincomp{A}, \cochaincomp{B},
  \cochaincomp{C})$ if there is a strict triangle $(u', v', w')$ and
  a commutative diagram
  \[
    \begin{tikzcd}
      \cochaincomp{A} & \cochaincomp{B} & \cochaincomp{C} &
      {\cochaincomp{A}[1]} \\
      \cochaincomp{(A')} & \cochaincomp{(B')} & \cone{u'} &
      {\cochaincomp{(A')}[1]}
      \arrow["u", from=1-1, to=1-2]
      \arrow["f"', from=1-1, to=2-1]
      \arrow["v", from=1-2, to=1-3]
      \arrow["g"', from=1-2, to=2-2]
      \arrow["w", from=1-3, to=1-4]
      \arrow["h", from=1-3, to=2-3]
      \arrow["{f[1]}", from=1-4, to=2-4]
      \arrow["{u'}"', from=2-1, to=2-2]
      \arrow["{v'}"', from=2-2, to=2-3]
      \arrow["{\delta}"', from=2-3, to=2-4]
    \end{tikzcd}
  \]
  in $\homcat{\abcat{A}}$ such that the maps $f$, $g$ and $h$ are isomorphisms.
\end{definition}

\iffalse
\begin{remark}
  Since choosing maps in $\homcat{\abcat{A}}$ is choosing
  representatives of homotopy equivalence classes from
  $\Ch{\abcat{A}}$, commutative diagrams in $\homcat{\abcat{A}}$ are
  commutative diagrams in $\Ch{\abcat{A}}$ up to a choice of homotopy
  equivalent maps.
\end{remark}
\fi

\iffalse
\begin{example}
  For any complex $\cochaincomp{A}$, the triangle
  \[
    \begin{tikzcd}
      \cochaincomp{A} \arrow[r, "0"]
      & \cochaincomp{A} \arrow[r]
      & \cochaincomp{A} \oplus \cochaincomp{A}[1] \arrow[r]
      & \cochaincomp{A}[1]
    \end{tikzcd}
  \]
  is strict, since $\cone{0} = \cochaincomp{A} \oplus
  \cochaincomp{A}[1]$. The triangle
  \[
    \begin{tikzcd}
      \cochaincomp{A} \arrow[r, "\id"]
      & \cochaincomp{A} \arrow[r]
      & 0 \arrow[r]
      & \cochaincomp{A}[1]
    \end{tikzcd}
  \]
  is exact.

  To see why, note that $\cone{\id}$ is split exact, and so the
  identity map for $\cone{\id}$ is null-homotopic, which in turn
  implies that $\cone{\id} \to \cochaincomp{0}$ is a chain homotopy equivalence.
  This is precisely an isomorphism in $\homcat{\abcat{A}}$, and by
  replacing $\cone{\id}$ by an arbitrary split exact complex
  $\cochaincomp{B}$, we recover the slightly more general fact that
  $\cochaincomp{B} \cong \cochaincomp{0}$ in $\homcat{\abcat{A}}$.
  % since $\cone{\id}$ is a split exact complex and hence isomorphic
  % to 0 in $\homcat{\abcat{A}}$.
  % This fact holds more generally: any
  % and hence isomorphic to 0 in $\homcat{\abcat{A}}$.
\end{example}
\fi

One should think of exactness for triangles as saying that there is
an isomorphism between $(u, v, w)$ and the strict triangle $(u', v', \delta)$.
The main feature of short exact sequences in $\Ch{\abcat{A}}$ is that
the cohomology functor induces long exact sequences in $\abcat{A}$.
When one instead works in $\homcat{\abcat{A}}$, cohomology remains a
well-defined family of functors since homotopic maps induce the same
maps on cohomology, and an exact triangle still induces a long exact
cohomology sequence in $\abcat{A}$.

\begin{proposition}[{\cite[Corollary~10.1.4]{weibel}}]
  \label{prop_homcat_long_exact_coh_seq}
  Given an exact triangle $(u, v, w)$ in $\homcat{\abcat{A}}$, there
  is a long exact sequence
  \[
    \begin{tikzcd}
      \cdots \arrow[r]
      & H^i(\cochaincomp{A}) \arrow[r, "H^i(u)"]
      & H^i(\cochaincomp{B}) \arrow[r, "H^i(v)"]
      & H^i(\cochaincomp{C}) \arrow[r, "H^i(w)"]
      & H^{i + 1}(\cochaincomp{A}) \arrow[r]
      & \cdots
    \end{tikzcd}
  \]
  in $\abcat{A}$, where we make the identification $H^{i +
  1}(\cochaincomp{A}) \cong H^i(\cochaincomp{A}[1])$.
\end{proposition}

\iffalse
\begin{proof}
  If $(u, v, w)$ is itself a strict triangle, then this long exact
  cohomology sequence is precisely the usual one obtained from the
  underlying short exact sequence.
  In all other cases, we have a commutative diagram
  \[
    \begin{tikzcd}
      \cochaincomp{A} & \cochaincomp{B} & \cochaincomp{C} &
      {\cochaincomp{A}[1]} \\
      \cochaincomp{A'} & \cochaincomp{B'} & \cone{u'} & {\cochaincomp{A'}[1]}
      \arrow["u", from=1-1, to=1-2]
      \arrow["f"', from=1-1, to=2-1]
      \arrow["v", from=1-2, to=1-3]
      \arrow["g"', from=1-2, to=2-2]
      \arrow["w", from=1-3, to=1-4]
      \arrow["h", from=1-3, to=2-3]
      \arrow["{f[1]}", from=1-4, to=2-4]
      \arrow["{u'}"', from=2-1, to=2-2]
      \arrow["{v'}"', from=2-2, to=2-3]
      \arrow["{\delta}"', from=2-3, to=2-4]
    \end{tikzcd}
  \]
  in $\homcat{\abcat{A}}$ and a long exact cohomology sequence
  \[
    \cdots
    \stackrel{\delta_{i - 1}^*}{\longrightarrow}
    H^i(\cochaincomp{A})
    \stackrel{(u_i')^*}{\longrightarrow}
    H^i(\cochaincomp{B})
    \stackrel{(v_i')^*}{\longrightarrow}
    H^i(\cone{u'})
    \stackrel{\delta_i^*}{\longrightarrow}
    H^{i + 1}(\cochaincomp{A})
    \stackrel{(u_{i + 1}')^*}{\longrightarrow}
    \cdots
  \]
  induced by strictness of the bottom row.
  Since the vertical maps are isomorphisms in $\homcat{\abcat{A}}$,
  they induce isomorphisms of cohomology. Functors preserve
  isomorphisms, so commutativity of each square now gives a long
  exact cohomology sequence for $\cochaincomp{A}$, $\cochaincomp{B}$
  and $\cochaincomp{C}$. \todo{I want to make this more precise,
    because it looks like we're not getting exactly a sequence induced
    by the maps $u$, $v$ and $w$ as the statement suggests. But it's
  not really a priority right now.}
\end{proof}
\fi

\begin{remark}
  \label{rem_ses_exact_triangs_in_homcat}
  Exact triangles in $\homcat{\abcat{A}}$ are not a perfect
  substitute for exact sequences in $\Ch{\abcat{A}}$.
  To wit, not every short exact sequence
  \[
    \begin{tikzcd}
      \cochaincomp{0} \arrow[r]
      & \cochaincomp{A} \arrow[r, "u"]
      & \cochaincomp{B} \arrow[r, "v"]
      & \cochaincomp{C} \arrow[r]
      & \cochaincomp{0}
    \end{tikzcd}
  \]
  in $\Ch{\abcat{A}}$ can be completed to an exact triangle $(u, v,
  w)$ in $\homcat{\abcat{A}}$, in the sense that the last map $w:
  \cochaincomp{C} \to \cochaincomp{A}[1]$ may not exist.
  For an example, see \cite[Exercise~10.1.2]{weibel}.
\end{remark}

\begin{remark}
  \label{rem_homcat_of_additive_cat}
  With the exception of \cref{prop_homcat_long_exact_coh_seq} (which
  requires that mapping cones exist), everything we have discussed in
  this section only depends on the fact that $\abcat{A}$ is additive.
  This allows us to make sense of $\homcat{\abcat{C}}$ whenever
  $\cat{C}$ is a full additive subcategory of $\abcat{A}$, which will
  have important implications for the derived category later.
  It is also clear that everything we have discussed works with
  $\homcat[*]{\abcat{A}}$ in place of $\homcat{\abcat{A}}$.
\end{remark}
