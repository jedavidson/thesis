\chapter{The coherent sheaf theory of $\proj{1}$}
\label{chap_coh_sheaf_theory_of_p1}

With our abstract preliminaries from the previous chapters all set
up, this chapter marks a change of course, as we will now move into
the realm of algebraic geometry and ultimately study the
\emph{coherent sheaves} on the projective line and some rudimentary
coherent sheaf cohomology.
These sheaves are characterised by their connection to modules, which
make them amenable to the techniques of commutative algebra.
Moreover, they are sufficiently restrictive as a type of sheaf to be
tractable to study while also encompassing many examples frequently
arising in practice.

% In essence, the property of a sheaf being coherent assumes a
% similar role to that of a module being finitely generated, and
% indeed we shall see that coherent sheaves on $\aff{1}$ and
% $\proj{1}$ have a convenient and rather direct connection to modules.
% (at least over `well-behaved' rings).

We will present a heavily specialised treatment that is sufficient to
obtain a working definition for $\proj{1}$.
Our main source for this are the set of mini-lectures
\cite{users_guide_to_coh_sheaves, cohom_of_coh_sheaves}.
This treatment is neither novel nor particularly unusual beyond this,
as we will also integrate material from the trio of sources
\cite{chen_and_krause, team_edward, crawley} that view these sheaves
in much the same way.

In all that follows, we fix an algebraically closed field $k$.

\section{A primer on sheaves}
\label{sect_sheaf_primer}

This section is intended to provide a brief overview of
\emph{sheaves} in the abstract with a view toward algebraic geometry.
We will strictly consider generalities here, but our needs for the
more specific and forthcoming study of coherent sheaves can be met
with little more than a few basic and well-motivated definitions.

Recall that one of the main ways to study an algebraic variety is to
look at the rings of regular functions defined locally and globally.
An important consequence of the way the Zariski topology is
constructed is that these regular functions enjoy a certain
local-to-global relationship: two regular functions on the variety
$X$ which agree locally on an open subset $U \subseteq X$ agree globally on $X$.
% (c.f. \cite[Remark~I.3.1.1]{hartshorne}).
Regular functions are far from the only data on $X$ that behaves
similar in spirit to this.
As one of many examples, one can develop a cogent theory of
\emph{(K\"{a}hler) differentials} when $X$ is smooth which, among
other things, can be used to define the \emph{geometric genus} of
$X$, an important numerical invariant.
There is an abundance of examples of this phenomena beyond just
algebraic geometry, such as the continuous functions on any topological space.
The language of sheaves arises as a convenient organisational tool in
all of these contexts.

\begin{definition}
  \label{def_presheaf}
  Let $X$ be a topological space.
  A \emph{presheaf of abelian groups} $\shf{F}$ on $X$ is the data of
  abelian group $\shf{F}(U)$ for each open set $U$, and for each
  inclusion of opens $V \subseteq U$ a group homomorphism
  $\res{U}{V}: \shf{F}(U) \to \shf{F}(V)$.
  These homomorphisms are moreover required to be functorial in the
  sense that $\res{U}{U} = \id_{\shf{F}(U)}$ for all open sets $U$
  and $\res{U}{W} = \res{V}{W} \res{U}{V}$ for all inclusions of open
  sets $W \subseteq V \subseteq U$.
\end{definition}

Though in this definition presheaf data takes values in the category
$\Ab$, one can substitute for other concrete categories, such as
$\Set$ and $\Ring$; more usefully and most commonly, one considers an
abelian category.
The elements of $\shf{F}(U)$ are called the \emph{sections} of
$\shf{F}$ over $U$, and the sections over $X$ itself are called the
\emph{global sections}.
It is also common to denote the space of sections of $\shf{F}$ over
$U$ by $\sections{U}{\shf{F}}$ or $H^0(U, \shf{F})$.
The homomorphisms $\res{U}{V}$ are called \emph{restriction maps} due
to how they are typically defined, and for a section $s \in
\shf{F}(U)$ we introduce the suggestive shorthand $\setres{s}{V} =
\res{U}{V}(s)$.

As we have already alluded to above, the most fundamental example of
a presheaf in algebraic geometry is the following.

\begin{example}
  The regular functions of a quasi-projective variety $X$ define a
  presheaf of rings $\strshf{X}$ called the \emph{structure sheaf} of $X$.
  For each open set $U$, we let $\strshf{X}(U)$ be the ring
  $\shf{O}(U)$ of regular functions
  % $f: U \to k$,
  defined on $U$, and take as restriction maps the usual
  contravariant inclusions $\shf{O}(U) \mono \shf{O}(V)$.
\end{example}

The definition of a presheaf captures the organisational aspects of
the data of $\strshf{X}$, but we have already noted that there is
more going on in the form of local-to-global relationships.
Our next definition aims to make this more formal.

\begin{definition}
  \label{def_sheaf}
  A presheaf $\shf{F}$ on $X$ is called a \emph{sheaf} if for any
  open set $U$ and open cover $U = \bigcup_{i \in I} U_i$, the
  following two conditions are satisfied:
  \begin{enumerate}
    \item
      \emph{(Identity.)}
      Given $s \in \shf{F}(U)$ such that $\setres{s}{U_i} = 0$ for
      all $i \in I$, we have $s = 0$.

    \item
      \emph{(Gluability.)}
      Given sections $s_i \in \shf{F}(U_i)$ such that
      \[
        \setres{s_i}{U_i \cap U_j} = \setres{s_j}{U_i \cap U_j}
      \]
      for all $i, j \in I$, there is a section $s \in \shf{F}(U)$
      such that $s_i = \setres{s}{U_i}$ for all $i \in I$.
  \end{enumerate}
\end{definition}

% In other words, a sheaf is a presheaf where families of local
% sections glue together \emph{uniquely} on open covers.
% The observant reader may feel shortchanged by this, since it is not
% as strong as what happens for regular functions in $\strshf{X}$,
% where local agreement immediately becomes global agreement.
% This is entirely to be expected, however: the Zariski topology is
% rather coarse and any variety $X$ is by definition irreducible,
% which are both topologically restrictive conditions.

\begin{example}
  As the name implies, the structure sheaf $\strshf{X}$ is also a sheaf.
  Aside from the importance of regular functions in general, what
  makes $\strshf{X}$ so fundamental is that many other useful sheaves
  occurring in algebraic geometry are `compatible' with it.
  More precisely, we call a sheaf $\shf{F}$ an
  \emph{$\strshf{X}$-module} if each $\shf{F}(U)$ is a module over
  $\strshf{X}(U)$, and for each inclusion of open sets $V \subseteq U
  \subseteq X$ we have
  \[
    \setres{(s + t)}{V} = \setres{s}{V} + \setres{t}{V}
    \mathand
    \setres{(f \cdot s)}{V} = \setres{f}{V} \cdot \setres{s}{V}
  \]
  for all $s, t \in \shf{F}(U)$ and $f \in \strshf{X}(U)$.
  Naturally, $\strshf{X}$ is itself an $\strshf{X}$-module.
\end{example}

With the following notion of morphisms, one obtains the pair of
categories $\Psh{X}$ and $\Sh{X}$ of all (pre)sheaves of abelian
groups on some fixed space $X$.

\begin{definition}
  \label{def_general_sheaf_morphism}
  Let $\shf{F}$ and $\shf{G}$ be (pre)sheaves of abelian groups on $X$.
  A \emph{morphism of (pre)sheaves} $f: \shf{F} \to \shf{G}$ is the
  data of group homomorphism $f_U: \shf{F}(U) \to \shf{G}(U)$ for
  each open $U \subseteq X$ compatible with restriction, i.e. there
  is a commutative diagram
  \[
    \begin{tikzcd}[cramped]
      {\shf{F}(U)} & {\shf{G}(U)} \\
      {\shf{F}(V)} & {\shf{G}(V)}
      \arrow["{f_U}", from=1-1, to=1-2]
      \arrow["{\setres{-}{V}}"', from=1-1, to=2-1]
      \arrow["{\setres{-}{V}}", from=1-2, to=2-2]
      \arrow["{f_V}"', from=2-1, to=2-2]
    \end{tikzcd}
  \]
  for each inclusion of opens $V \subseteq U \subseteq X$.
  The composition of morphisms is defined in the obvious way.
  We say that $f$ is an \emph{isomorphism} if there is a morphism $g:
  \shf{G} \to \shf{F}$ such that $fg = \id_{\shf{G}}$ and $gf = \id_{\shf{F}}$.
\end{definition}

Finally, we define a basic but useful operation on sheaves that we
will need later.

\begin{definition}
  Let $\shf{F}$ be a sheaf on $X$.
  For each open $U \subseteq X$, the \emph{restricted sheaf}
  $\setres{\shf{F}}{U}$ is the sheaf on $U$ such that
  $(\setres{\shf{F}}{U})(V) = \shf{F}(V)$ for each open $V \subseteq
  U$, and $(\setres{\rho}{U})_{V, W} = \res{V}{W}$ for each inclusion
  of opens $W \subseteq V \subseteq U$.
\end{definition}

\section{Modules as sheaves on $\aff{1}$}

As an important precursor to coherent sheaves, we show in this
section that any module over $\shf{O}(\aff{1}) \cong k[x]$ can be
suitably interpreted as a sheaf on $\aff{1}$.
This also gives an explicit algebro-geometric interpretation of
modules and localisation.

We start with some preliminary results concerning open sets in $\aff{1}$.
Recall that a \emph{principal open subset} of an affine variety $X$
is an open subset of the form
\[
  D(f) := X \setminus V(f) = \{x \in X: f(x) \neq 0\}
\]
for some $f \in k[X]$.
The collection of all principal opens forms a base for the Zariski
topology on $X$, but even stronger statements can be made about the
topology on $\aff{1}$.

\begin{lemma}
  \label{lemma_opens_of_aff_line_are_principal}
  Every open subset of $\aff{1}$ is principal.
\end{lemma}

\begin{proof}
  If $U$ is open, then $U = \aff{1} \setminus V(I)$ for some ideal $I
  \idealof k[x]$.
  But $k[x]$ is a principal ideal domain, so $I = \gen{f}$ for some
  $f \in k[x]$ and hence $U = D(f)$.
\end{proof}

\begin{lemma}
  \label{lemma_principal_open_inclusions_of_aff_line}
  Let $f \in k[x]$, and consider the principal open set $D(f)$ in $\aff{1}$.
  Then every principal open set within $D(f)$ is of the form $D(fg)$
  for some $g \in k[x]$.
  Moreover, the intersection of principal opens $D(f)$ and $D(g)$ is $D(fg)$.
\end{lemma}

\begin{proof}
  If $D(h) \subseteq D(f)$, then $V(f) \subseteq V(h)$ and hence
  $\gen{h} \subseteq \gen{f}$ by standard properties of affine
  varieties, which implies that $f$ divides $h$.
  The second claim is equivalent to the fact that $V(f) \cup V(g) = V(fg)$.
\end{proof}

The moral of these results is that to specify a sheaf on $\aff{1}$,
it is entirely sufficient to consider only principal open sets
determined by some polynomial.
Intuitively, the same conclusions ought to hold mutatis mutandis for
any open $U \subseteq \aff{1}$.
To make this idea more formal, if $U = D(f)$ for some $f \in k[x]$,
then by \cite[Proposition~6.3.5]{fulton} we can take the definition
of the ring of regular functions for $U$ to be the localisation
\[
  \shf{O}(U) := \shf{O}(\aff{1})_f \cong k[x, f^{-1}],
\]
and it is not hard to see now that
\cref{lemma_principal_open_inclusions_of_aff_line} holds with
$\aff{1}$ replaced by $U$ and $k[x]$ replaced by $\shf{O}(U)$.
For reasons that will become clear in due course, we will broaden the
scope of the following definition to any open subset of $\aff{1}$
rather than just $\aff{1}$ itself.
% Next, recall that a \emph{quasi-affine} variety $Y$ is an open
% subset of an affine variety $X$.
% Naturally, we will exclusively consider the case $X = \aff{1}$.
% By \cref{lemma_opens_of_aff_line_are_principal}, any quasi-affine
% variety $Y$ in $\aff{1}$ is principal in $\aff{1}$, so $Y = D(f)$
% for some $f \in k[x]$.
% Moreover, by \cite[Proposition~6.3.5]{fulton}, we can define a ring
% of regular functions for any such $Y$ as
% \[
%     \shf{O}(U) := \shf{O}(\aff{1})[f^{-1}] \cong k[x, f^{-1}].
% \]
% Finally, it is clear that the conclusions of
% \cref{lemma_principal_open_inclusions_of_aff_line} hold mutatis
% mutandis for $Y$.
% The need to broaden our scope to quasi-affine varieties rather than
% solely $\aff{1}$ is so that we may directly verify that the
% following construction gives a sheaf of modules.

\begin{definition}
  \label{def_assmod}
  Let $U$ be an open subset of $\aff{1}$ and $M$ a module over the
  ring $\shf{O}(U)$.
  We define a presheaf of abelian groups $\assmod{M}$ on $U$, called
  the \emph{sheaf associated to $M$}, by declaring that the sections
  over each principal open $D(f) \subseteq U$ are
  $\sections{D(f)}{\assmod{M}} = M_f$, and equipping $\assmod{M}$
  with restriction maps
  \[
    \res{f}{fg}: M_f \to M_{fg},
    \qquad \frac{m}{f^r} \mapsto \frac{mg^r}{(fg)^r}.
  \]
\end{definition}

Some a posteriori justification for this definition is as follows.
Algebraically, $D(f) \subseteq U$ is the region in which $f$ is
non-zero and hence invertible in $\shf{O}(U)$, so to specify local
sections it is natural to consider the \emph{extension of scalars}
\[
  \shf{O}(U)_f \tensor_{\shf{O}(U)} M \cong M_f.
\]
Also, while the sheaf $\assmod{M}$ only considers this space of
sections as an abelian group, each $M_f$ has by definition the
structure of an $\shf{O}(U)_f$-module.

\begin{example}
  The structure sheaf $\strshf{\aff{1}}$ is the associated sheaf of
  the $k[x]$-module $k[x]$.
  By our remarks above, it follows that every associated sheaf
  $\assmod{M}$ on $\aff{1}$ is also an example of an $\strshf{\aff{1}}$-module.
\end{example}

\begin{example}
  \label{exmp_skyscraper_shf}
  Let $a \in k$, and consider the $k[x]$-module $M = \quot{k[x]}{\gen{x - a}}$.
  If $U = D(f)$ is an open set in $\aff{1}$, then it follows from
  polynomial division that $f = f(a)$ in $M$.
  If $a \in U$, then $f(a)$ is non-zero in $k$ and hence invertible
  in $M$, so we have $M_f \cong M \cong k$ as $k[x]$-modules.
  If instead $a \not\in U$, then $f(a) = 0$ in $k$, and so $M_f$ must
  be the zero module.
  We say that the sheaf $\assmod{M}$ is \emph{supported} at the point
  $x = a$ in the sense that one only obtains non-zero sections over
  open neighbourhoods of $a$.
  For this reason, we call $\assmod{M}$ a \emph{skyscraper sheaf} on $\aff{1}$.
\end{example}

Let us now check that the presheaf $\assmod{M}$ is deserving of being
called a sheaf.
We will adapt the proofs of a related result for affine schemes in
\cite[Proposition~I.18]{eisenbud_and_harris} and
\cite[Theorem~4.1.2]{vakil}, originally due to Serre.
By virtue of the fact that we are working within $\aff{1}$, we will
be allowed to sidestep some sheaf-theoretic technicalities that one
encounters in this more general setting.
The following lemma is key.

\begin{lemma}
  \label{lemma_quasiaff_is_quasicompact}
  Every open cover of an open subset of $\aff{1}$ has a finite subcover.
\end{lemma}

\begin{proof}
  In light of our observations above, it suffices to consider a cover
  by principal opens $U = \bigcup_i D(f_i)$, with $f_i \in \shf{O}(U)$.
  It is immediate that the zero locus of the ideal $\ideal{a}$
  generated by the $f_i$ must be empty, i.e. $\mathfrak{a} = \shf{O}(U)$.
  Each element of $\ideal{a}$ is by definition a finite linear
  combination of the $f_i$, so in particular there is a
  \emph{partition of unity} $\sum_{i \in I} a_i f_i = 1$ where all
  but finitely many of the $a_i \in \shf{O}(U)$ are non-zero.
  We may thus take as a finite subcover the sets $D(f_i)$ such that
  $a_i \neq 0$.
\end{proof}

% \begin{remark}
%     In more algebro-geometric language,
% \cref{lemma_quasiaff_is_quasicompact} says that $U$ is \emph{quasicompact}.
%     This is terminologically distinguished from compactness by not
% insisting that the space is also Hausdorff, which is rarely ever
% true of the Zariski topology (and never when $k$ is infinite, i.e.
% when $k$ is algebraically closed).
% \end{remark}

\begin{proposition}
  \label{prop_shf_assoc_to_mod}
  The presheaf $\assmod{M}$ in \cref{def_assmod} is a sheaf on any
  open $U \subseteq \aff{1}$.
\end{proposition}

\begin{proof}
  Let $V \subseteq U$ be open.
  Again, in light of the previous observations, we may write $V =
  D(f)$ for some $f \in \shf{O}(U)$ and consider an arbitrary open
  cover $V = \bigcup_{i \in I} D(f_i)$ for some $f_i \in \shf{O}(U)$.
  We claim that is that it is in fact enough to show that the sheaf
  property holds for global sections, i.e. in the case $V = U$.
  To see why, note that $V$ is open in $\aff{1}$, so by
  \cref{lemma_principal_open_inclusions_of_aff_line} we have
  \[
    V
    = \bigcup_{i \in I} D(f_i)
    = \bigcup_{i \in I} V \cap D(f_i)
    = \bigcup_{i \in I} D(ff_i).
  \]
  Via the inclusion of rings $\shf{O}(U) \mono \shf{O}(V)$, each
  function $f f_i$ may be regarded as a function in $\shf{O}(V)$.
  That is, the open cover of $V$ above can be surreptitiously
  rewritten as an open cover $V = \bigcup_{i \in I} D(f_i')$ for some
  $f_i' \in \shf{O}(V)$, which shows that the general case can indeed
  be reduced to this special case.
  Thus for the remainder of the proof, we may assume that $V = U$ so
  that $\assmod{M}(V) \cong M$.
  Before we begin the proof proper, we will also fix a finite
  subcover $U = \bigcup_{\alpha \in A} D(f_\alpha)$ by
  \cref{lemma_quasiaff_is_quasicompact}.

  To start, let us check that the identity property holds.
  Suppose that the global section $s \in M$ vanishes upon restriction
  in $M_{f_i}$ for all $i \in I$.
  Then $s$ is annihilated in $M$ by some power of each $f_\alpha$ on
  the finite subcover.
  By finiteness, we may choose a sufficiently large power $N$ such
  that every $f_\alpha^N$ annihilates $s$.
  It follows that the ideal $\ideal{b}$ generated by the $f_\alpha^N$
  annihilates $s$.
  But via the proof of \cref{lemma_quasiaff_is_quasicompact}, the
  ideal $\ideal{a}$ generated by the $f_\alpha$ is $\shf{O}(U)$, and
  it is not hard to see that $\shf{O}(U) = \ideal{a}^{2N} \subseteq
  \ideal{b}$, so we conclude that $s$ is annihilated by the entire
  ring $\shf{O}(U)$.
  Hence $s = 0$ in $M$.

  Next, we check gluability.
  Suppose we are given sections $s_i \in M_{f_i}$ such that $s_i =
  s_j$ in $M_{f_i f_j}$ for all $i, j \in I$.
  As before, this certainly descends to our finite subcover, so let
  us first prove it on that.
  We can write each section as $s_\alpha =
  m_\alpha/f_\alpha^{r_\alpha}$ for some $m_\alpha \in M$ and integer
  $r_{\alpha}$.
  Since $D(f_\alpha) = D(f_\alpha^{r_\alpha})$, for ease of notation
  we will let $g_\alpha = f_\alpha^{r_\alpha}$ such that each
  $s_\alpha$ is naturally an element of $M_{g_{\alpha}}$.
  By clearing denominators in $M_{g_\alpha g_\beta}$, we may appeal
  once more to finiteness to choose a sufficiently large power $N$ such that
  \[
    (g_\alpha g_\beta)^N(g_\beta m_\alpha - g_\alpha m_\beta) = 0
  \]
  in $M$ for every $\alpha, \beta \in A$.
  For further ease of notation, write $n_\alpha = g_\alpha^N
  m_\alpha$ and $h_\alpha = g_\alpha^{N + 1}$ so that the above
  condition reads $h_\beta n_\alpha = h_\alpha n_\beta$.
  Moreover, since $D(h_\alpha) = D(g_\alpha)$, it follows as in
  \cref{lemma_quasiaff_is_quasicompact} that there is a partition of
  unity $\sum_{\alpha \in A} a_\alpha h_\alpha = 1$ for some
  $a_\alpha \in \shf{O}(U)$.
  Now, considering the global section $s = \sum_{\alpha \in A}
  a_\alpha n_\alpha$, we see that
  \[
    s h_\alpha
    = \sum_{\beta \in A} a_\beta h_\alpha n_\beta
    = \sum_{\beta \in A} a_\beta h_\beta n_\alpha
    = n_\alpha
  \]
  in $M$ for each $\alpha \in A$.
  It is not difficult to see that $M_{h_\alpha} \cong M_{f_\alpha}$,
  and by construction $n_\alpha/h_\alpha = s_\alpha$ in
  $M_{h_\alpha}$, so we conclude from this that $s = s_\alpha$ in
  $M_{f_\alpha}$ as needed.
  It remains to upgrade gluability to infinite open covers.
  The previous argument holds generically for any finite open cover
  of $U$, so for any choice of $i \in I \setminus A$, we may
  similarly construct a global section $s' \in M$ such that $s' =
  s_\alpha$ in each $M_{f_\alpha}$ and $s' = s_i$ in $M_{f_i}$.
  But by the identity property, we must have $s = s'$, and so we are done.
\end{proof}

We now see that the passage to open subsets of $\aff{1}$ in
\cref{def_assmod} was necessary to ensure that we may reduce the
claim to something easier to prove.

\section{Coherent sheaves on $\proj{1}$}
\label{sect_coh_sheaves_p1}

With our preparations out of the way, we are now in a position to
introduce \emph{coherent sheaves} on $\proj{1}$.
It will be instructive to give the usual definitions upfront in order
to provide some sense of where our hands-on definitions we obtain
later come from.

\begin{definition}
  \label{def_general_qcoh}
  Let $\shf{F}$ be an $\strshf{X}$-module on some quasi-projective
  variety $X$ over $k$.
  Then $\shf{F}$ is \emph{quasi-coherent} if there is an affine open
  cover $X = \bigcup_i U_i$ (i.e. with each $U_i$ isomorphic to an
  affine variety) and an $\shf{O}(U_i)$-module $M_i$ such that
  $\setres{\shf{F}}{U_i} \cong \assmod{M_i}$.
  We say that $\shf{F}$ is \emph{coherent} if furthermore each module
  $M_i$ is finitely generated over $\shf{O}(U_i)$.
\end{definition}

There is another equivalent definition, which we package up in the
following result.

\begin{proposition}[{\cite[Proposition~II.5.4]{hartshorne}}]
  \label{prop_general_qcoh_equiv}
  An $\strshf{X}$-module $\shf{F}$ is quasi-coherent if and only if
  for each affine open $U \subseteq X$, there is an
  $\shf{O}(U)$-module $M$ such that $\setres{\shf{F}}{U} \cong \assmod{M}$.
  Moreover, $\shf{F}$ is coherent if and only if $M$ is finitely
  generated over $\shf{O}(U)$.
\end{proposition}

Though \cref{def_general_qcoh} and \cref{prop_general_qcoh_equiv} are
standard fare in algebraic geometry, the reader should feel some
trepidation here.
The issue is that as we have constructed it, \cref{def_assmod} only
applies to quasi-affine varieties in $\aff{1}$ rather than arbitrary
affine varieties, so the assertion that $\setres{\shf{F}}{U} \cong
\assmod{M}$ for some affine open $U \subseteq X$ is ill-defined.
We will only concern ourselves with the varieties $\aff{1}$ and
$\proj{1}$ for which this issue is not a concern.
The intention of presenting these definitions in spite of this
imprecision is to convey the important idea that quasi-coherent
sheaves on $X$ are `locally just modules'.

The category of quasi-coherent sheaves on $X$ is denoted $\QCoh(X)$,
and the coherent sheaves form a full subcategory $\Coh(X)$.
In the case of $\aff{1}$, these categories end up being as nice as
one could realistically ask for.

\begin{corollary}
  \label{cor_affine_coh_equiv_to_mods}
  % If $\shf{F}$ is a (quasi-)coherent sheaf on $\aff{1}$,
  % then $\shf{F}$ is determined by the $k[x]$-module of global
  % sections $\sections{\aff{1}}{\shf{F}}$.
  The category $\QCoh(\aff{1})$ is equivalent to $\Mod{k[x]}$ via the
  pair of naturally inverse functors $\shf{F} \mapsto
  \sections{\aff{1}}{\shf{F}}$ and $M \mapsto \assmod{M}$.
  Moreover, this restricts to an equivalence of categories between
  $\Coh(\aff{1})$ and $\mod{k[x]}$.
\end{corollary}

\begin{proof}
  This is a specialisation and restatement of
  \cite[Corollary~II.5.5]{hartshorne}, but the main contention
  follows from \cref{prop_general_qcoh_equiv}.
\end{proof}

In other words, the coherent sheaves on $\aff{1}$ are little more
than finitely generated $k[x]$-modules.
Technical concerns notwithstanding, the same claim is true mutatis
mutandis for all affine varieties, which is a significant reduction
in the complexity of studying these sheaves.
However, one can go even further specifically in the case of $\aff{1}$.
In particular, the structure theorem for finitely generated modules
over a principal ideal domain can be appropriated as a classification
theorem for coherent sheaves on $\aff{1}$.
For convenience, we state this as the following theorem in its
primary decomposition form.

\begin{theorem}
  \label{thm_coh_sheaf_classification_of_aff_line}
  If $\shf{F}$ is a coherent sheaf on $\aff{1}$, i.e. a finitely
  generated $k[x]$-module $M$, then there are $r, s \in \zpos$,
  powers $n_1, n_2, \ldots, n_s \in \zpos$ and points $a_1, \ldots,
  a_r \in k$, not necessarily distinct, such that $\shf{F}$ is the
  sheaf associated to the module
  \[
    M
    \cong
    k[x]^{\oplus r} \oplus \frac{k[x]}{\gen{x - a_1}^{n_1}} \oplus
    \cdots \oplus \frac{k[x]}{\gen{x - a_s}^{n_s}}.
  \]
  \vspace{-18pt}
  % Moreover, the points $a_i$ are unique up to associates in $k$.
\end{theorem}

Put another way, this theorem also shows that any coherent sheaf on
$\aff{1}$ splits as a sum of a torsion-free and a torsion summand, an
observation we will return to later.

The next natural question to ask is whether there is also an elegant
theory for coherent sheaves on non-affine quasi-projective varieties.
Immediately, $\proj{1}$ makes for a good case study, since $\proj{1}$
has a standard affine open cover by two copies of $\aff{1}$ (i.e. we
are again free of prior technical concerns).

We first fix some conventions for $\proj{1}$ that we will use throughout.
Recall that the aforementioned affine open cover is given by
$\proj{1} = U_x \cup U_y$, where
\[
  U_x = \{(x : y): x \neq 0\} \mathand U_y = \{(x : y): y \neq 0\}.
\]
Each point $(x : y) \in U_x$ is the same as $(1 : y/x)$, so there is
a clear homeomorphism $\phi_x: U_x \to \aff{1}_z$ identifying $U_x$
with the affine line $\aff{1}_z$ with coordinate $z = y/x$.
Similarly, there is a clear homeomorphism $\phi_y: U_y \to
\aff{1}_{w}$ where $\aff{1}_{w}$ has coordinate $w = x/y$.
% This defines an open cover $\proj{1} = U_x \cup U_y$, and the
% aforementioned bijections $\phi_x: U_x \to \aff{1}_z$ and $\phi_y:
% U_y \to \aff{1}_{w}$ are in fact homeomorphisms (c.f.
% \cite[Proposition~I.1.2.2]{hartshorne}).
We call the maps $\phi_x$ and $\phi_y$ \emph{affine charts} for
$\proj{1}$, and similarly we call the open subsets $U_x$ and $U_y$
\emph{affine patches}.
We can convert between affine charts on $U_{xy} = U_x \cap U_y$ using
the \emph{transition map} $\tau: \aff{1}_z \setminus \{0\} \to
\aff{1}_w \setminus \{0\}$ defined by
\[
  \tau(z)
  := \phi_y \phi_x^{-1}(z)
  = \phi_y(1 : z)
  = \phi_y(z^{-1} : 1)
  = z^{-1}.
\]
Because of this, we can also view $\proj{1}$ as the result of gluing
together the affine lines $\aff{1}_z$ and $\aff{1}_w$ according to
the rule $w = z^{-1}$ whenever $z, w \neq 0$.
For algebraic convenience, we will replace the affine coordinate $w$
with $z^{-1}$ so that, formally, $\proj{1} = \aff{1}_z \sqcup \aff{1}_{z^{-1}}$.

In view of \cref{def_general_qcoh}, a quasi-coherent sheaf $\shf{F}$
on $\proj{1}$ is locally determined on this open cover by
$\shf{F}|_{U_x}$ and $\shf{F}|_{U_y}$, i.e. by a module $M$ over
$\shf{O}(U_x) \cong k[z]$ and a module $M'$ over $\shf{O}(U_y) \cong
k[z^{-1}]$ respectively.
One complication in simply forgetting the sheaf structure and
retaining only the pair $(M, M')$ is that between them, they encode a
significant amount of overlapping sheaf data on $U_{xy}$.
To resolve this problem, it is enough to specify an isomorphism of sheaves
\[
  \vartheta:
  \setres{\left(\setres{\shf{F}}{U_x}\right)}{U_{xy}}
  \to
  \setres{\left(\setres{\shf{F}}{U_y}\right)}{U_{xy}}.
\]
Algebraically, $U_{xy}$ is the region in which both $z$ and $z^{-1}$
are invertible in $k[z]$ and $k[z^{-1}]$ respectively, so it stands
to reason that $\vartheta$ can equivalently be specified as an
isomorphism between $\sections{D(z)}{\assmod{M}} = M_z$ and
$\sections{D(z^{-1})}{\assmod{M'}} = M'_{z^{-1}}$ as modules over
$k[z, z^{-1}]$, the ring of \emph{Laurent polynomials} in $z$.
This is the outline for the alternative definition of quasi-coherent
sheaves for $\proj{1}$ we will use in this thesis.
We will also pair it with a more tailored notion of a morphism than
\cref{def_general_sheaf_morphism}.

% In order to generalise \cref{def_assmod} so that we get a sheaf on
% $\proj{1}$, one plausible strategy is specify each section
% separately on the patches $U_x$ and $U_y$ using a pair of modules
% on the affine lines $\aff{1}_z$ and $\aff{1}_{z^{-1}}$.
% We will also have to specify a map to glue this section data
% together on the compatible overlap $U_x \cap U_y$, which
% algebraically is the region where $z$ amd $z^{-1}$ are invertible
% in $k[z]$ and $k[z^{-1}]$ respectively.
% Just as in the construction of the associated sheaf of a module, we
% take appropriate extensions of scalars so that our gluing map will
% be an isomorphism between $M[z^{-1}]$ and $M[z]$.
% Naturally, these are modules over $k[z, z^{-1}]$, the ring of
% \emph{Laurent polynomials} in $z$.

% This is obviously far from a `canonical' way to approach defining
% such a sheaf, since it depends heavily on the particular choice of
% covering of $\proj{1}$, but it turns out that it is once again
% commensurate with the `real' definition of \emph{quasi-coherent}
% sheaves for $\proj{1}$.
% Since we are at pains to avoid the general theory here, this claim
% is out of our reach, but it does justify taking the following as
% our working definition.

\begin{definition}
  \label{def_coh_sheaf}
  A \emph{quasi-coherent sheaf} on $\proj{1}$ is the data of a triple
  $\shf{F} = (M, M', \vartheta)$ consisting of a $k[z]$-module $M$, a
  $k[z^{-1}]$-module $M'$ and a $k[z^{-1}]$-module isomorphism
  $\vartheta: M_{z} \to M'_{z^{-1}}$, called the \emph{descent datum}
  of $\shf{F}$.
  We say that $\shf{F}$ is \emph{coherent} if $M$ and $M'$ are both
  finitely generated over $k[z]$ and $k[z^{-1}]$ respectively.
\end{definition}

\begin{definition}
  \label{def_coh_sheaf_morphism}
  Let $\shf{F} = (M, M', \vartheta)$ and $\shf{G} = (N, N', \varrho)$
  be quasi-coherent sheaves on $\proj{1}$.
  A \emph{morphism} $\Phi: \shf{F} \to \shf{G}$ is a pair $(\varphi,
  \varphi')$ consisting of a $k[z]$-module homomorphism $\varphi: M
  \to N$ and a $k[z^{-1}]$-module homomorphism $\varphi': M' \to N'$,
  subject to the condition that there is a commutative diagram of
  $k[z, z^{-1}]$-modules
  \[
    \begin{tikzcd}
      {M_z} & {N_z} \\
      {M'_{z^{-1}}} & {N'_{z^{-1}}}.
      \arrow["{\varphi_z}", from=1-1, to=1-2]
      \arrow["\vartheta"', from=1-1, to=2-1]
      \arrow["\varrho", from=1-2, to=2-2]
      \arrow["{\varphi'_{z^{-1}}}"', from=2-1, to=2-2]
    \end{tikzcd}
  \]
  % Here, we introduce a slight abuse of notation by identifying
  % $\varphi$ and $\varphi'$ with the homomorphisms induced by
  % localisation so that, say, $\varphi(m/z^i) = \varphi(m)/z^i$.
  If $\Psi: \shf{G} \to \shf{H}$ is another morphism of
  quasi-coherent sheaves with data $(\psi, \psi')$, then we define
  the \emph{composite} $\Psi \circ \Phi: \shf{F} \to \shf{H}$ to be
  the morphism with data $(\psi \circ \varphi, \psi' \circ \varphi')$.
\end{definition}

Choosing to regard quasi-coherent sheaves in this manner is somewhat
`unnatural', insofar as it is often more advantageous to try and work
in coordinate-free settings.
It is often necessary to fix coordinates for actual computations
though, and this presents no significant obstruction in practice.

\begin{proposition}
  Let $\cat{Q}$ be the category of quasi-coherent sheaves on
  $\proj{1}$ in the sense of \cref{def_coh_sheaf} and
  \cref{def_coh_sheaf_morphism}, and let $\cat{C}$ be the full
  subcategory of $\cat{Q}$ consisting of coherent sheaves.
  Then $\QCoh(\proj{1})$ is equivalent to $\cat{Q}$, and this
  restricts to an equivalence of categories between $\Coh(\proj{1})$
  and $\cat{C}$.
\end{proposition}

\begin{proof}
  We give only a brief idea of the proof.
  The functor $\QCoh(\proj{1}) \to \cat{Q}$ in this equivalence is
  easy to construct, as given $\shf{F} \in \QCoh(\proj{1})$ we have
  \[
    \displaystylesections{\setres{\shf{F}}{U_x}}{U_{xy}} =
    \sections{\shf{F}}{U_{xy}} =
    \displaystylesections{\setres{\shf{F}}{U_y}}{U_{xy}},
  \]
  so we simply take $\shf{F} \mapsto \left(\sections{\shf{F}}{U_x},
  \sections{\shf{F}}{U_y}, \id_{\sections{\shf{F}}{U_{xy}}}\right)$.
  The functor $\cat{Q} \to \QCoh(\proj{1})$ naturally inverse to this
  is harder to construct, but in essence will follow from the
  forthcoming description of the sections and restriction maps of
  some $\shf{F} \in \cat{Q}$.
\end{proof}

From this point forward, we therefore identify $\QCoh(\proj{1})$ and
$\Coh(\proj{1})$ with $\cat{Q}$ and $\cat{C}$ and dispense with the
general definitions from the start of this section.
Though we will stop short of an explicit verification that these
triples are actually sheaves in the technical sense of
\cref{def_sheaf}, it will be worthwhile to at least describe what the
sections and restriction maps of some $\shf{F} \in \QCoh(\proj{1})$
are to get a feel for how to work with these sheaves computationally.
We first recall the following facts about $\proj{1}$.

\begin{lemma}
  \label{lemma_opens_of_proj_line_are_principal}
  Every open subset $U \subseteq \proj{1}$ is of the form $D(F) :=
  \proj{1} \setminus V(F)$ for some homogeneous polynomial $F \in k[x, y]$.
  Moreover, any open subset $V \subseteq U$ is of the form $V =
  D(FG)$ for some other homogeneous polynomial $G \in k[x, y]$.
\end{lemma}

% \begin{proof}
%     If $U \subseteq \proj{1}$ is open, then $U = \proj{1} = X
% \setminus V(I)$ for some homogeneous ideal $I \idealof k[x, y]$,
% i.e. $I$ is generated by homogeneous polynomials.
%     Since $k[x, y]$ is Noetherian, $I$ is generated by finitely
% many such polynomials.
%     Thus $V(I)$ is finite, and we can construct a homogeneous
% polynomial $F$ whose zeroes are precisely the points of $V(I)$.
% \end{proof}

This result is the projective analogue of
\cref{lemma_opens_of_aff_line_are_principal} and
\cref{lemma_principal_open_inclusions_of_aff_line}.
Since the proof is similar, we shall omit it and mention only that
one instead needs to use the fact that $k[x, y]$ is Noetherian for
the first claim.
We may always arrange that $F$ and $G$ have minimal degree here,
since for example we can take
\[
  F(x, y) = (b_1 x - a_1 y) (b_2 x - a_2 y) \cdots (b_d x - a_d y),
\]
where the $(a_i : b_i)$ are the distinct points of the (finite) set
$\proj{1} \setminus U$.
At this juncture, it will be convenient to introduce some notation
and an important fact about dehomogenisation of polynomials.

\begin{definition}
  \label{def_canonical_dehomog}
  Let $F \in k[x, y]$ be a homogeneous polynomial.
  Setting $z = y/x$, the \emph{dehomogenisations} of $F$ separately
  with respect to $x$ and $y$ produces the pair of univariate
  polynomials which, whenever they are defined, are denoted by
  \[
    \dehomog{F}{x}(z) := F(1, z) \in k[z]
    \mathand
    \dehomog{F}{y}(z^{-1}) := F(z^{-1}, 1) \in k[z^{-1}].
  \]
  \vspace{-24pt}
\end{definition}

\begin{lemma}
  The dehomogenisations of a homogeneous polynomial $F \in k[x, y]$
  are associates in the ring of Laurent polynomials $k[z, z^{-1}]$,
  where $z = y/x$.
\end{lemma}

\begin{proof}
  This follows from homogenity, since if $d = \deg{F}$, then
  \[
    \dehomog{F}{x}(z) = F(1, z) = F(z z^{-1}, z) = z^d F(z^{-1}, 1) =
    z^d \dehomog{F}{y}(z^{-1}). \qedhere
  \]
  \vspace{-24pt}
\end{proof}

Now, suppose that $U \subseteq \proj{1}$ is the open set $D(F)$ as in
\cref{lemma_opens_of_proj_line_are_principal}.
The dehomogenisations of $F$ identify $U$ in each affine patch, i.e.
we have homeomorphisms
\[
  U \cap U_x \cong D(\dehomog{F}{x})
  \mathand
  U \cap U_y \cong D(\dehomog{F}{y}),
\]
so a section of $\shf{F}$ over $U$ is a pair of affine sections $(s,
s') \in M_{\dehomog{F}{x}} \times M'_{\dehomog{F}{y}}$, say
\[
  s = \quot{m}{\dehomog{F}{x}^i}
  \mathand
  s' = \quot{m'}{\dehomog{F}{y}^j}.
\]
It remains to describe how to glue $s$ and $s'$ on $U \cap U_{xy}$.
Algebraically, this is the region in which both $\dehomog{F}{x}$ and
$\dehomog{F}{y}$ are invertible in $k[z, z^{-1}]$, so we consider the
modules $M_{\dehomog{F}{x}, z}$ and $M'_{\dehomog{F}{y}, z^{-1}}$
over the ring $k[z, z^{-1}, \dehomog{F}{x}^{-1}, \dehomog{F}{y}^{-1}]$.
Since $\dehomog{F}{x} = z^{\deg{F}} \dehomog{F}{y}$ in this ring, the
fraction $\vartheta(m)/\dehomog{F}{x}^i$ is a well-defined element of
$M'_{\dehomog{F}{y}, z^{-1}}$, and we will insist for the sake of
compatibility that it agrees with $s'$, i.e. we impose the condition
\[
  \quot{\vartheta(m)}{\dehomog{F}{x}^i} = \quot{m'}{\dehomog{F}{y}^j}
  % \frac{\vartheta(m)}{f^i} = \frac{m'}{g^j}
\]
in $M'_{\dehomog{F}{y}, z^{-1}}$.
By definition of equality in a localised module, this holds if and only if
\begin{equation}
  \label{eq_local_section_compatibility}
  \dehomog{F}{y}^n(\dehomog{F}{y}^j \vartheta(m) - \dehomog{F}{x}^i m') = 0
\end{equation}
in $M'_{z^{-1}}$ for some power $n \in \zpos$.
The restriction maps at this point are clear, so we present the
following definition to summarise this discussion.

\begin{definition}
  \label{def_coh_sheaf_sects_and_restr}
  Let $\shf{F} = (M, M', \vartheta)$ be a quasi-coherent sheaf on
  $\proj{1}$ and $U = D(F)$ an open subset of $\proj{1}$.
  A \emph{local section} of $\shf{F}$ over $U$ is a pair
  \[
    (s, s')  = (m/\dehomog{F}{x}^i, m'/\dehomog{F}{y}^j) \in
    M_{\dehomog{F}{x}} \times M'_{\dehomog{F}{y}}
  \]
  satisfying \cref{eq_local_section_compatibility},
  and a \emph{global section} of $\shf{F}$ is a pair $(m, m') \in M
  \times M'$ satisfying $\vartheta(m) = m'$.
  For any open subset $V = D(FG) \subseteq U$, the \emph{restriction
  map} $\res{U}{V}$ of $\shf{F}$ sends the section $(s, s') \in
  \sections{U}{\shf{F}}$ to the section
  \[
    \left(\setres{s}{D(\dehomog{F}{x} \dehomog{G}{x})},\,
    \setres{s'}{D(\dehomog{F}{y} \dehomog{G}{y})}\right) \in
    \sections{V}{\shf{F}}
  \]
  via the affine restriction maps for $\assmod{M}$ and $\assmod{M'}$
  on each component.
\end{definition}

We will study the two most fundamental families of coherent sheaves
on $\proj{1}$ in depth for the remainder of this section, but it will
be convenient to introduce some more specific terminology and
definitions as we go.

\begin{definition}
  A quasi-coherent sheaf $\shf{F} = (M, M', \vartheta)$ on $\proj{1}$
  is \emph{locally free} of rank $r$ if $M \cong k[z]^r$ and $M'
  \cong k[z^{-1}]^r$.
  A locally free sheaf of rank 1 is called a \emph{line bundle}.
  % or an \emph{invertible sheaf} (for reasons that will become clear
  % in the next section).
\end{definition}

\begin{example}
  \label{exmp_twists}
  The structure sheaf $\shf{O} := \strshf{\proj{1}}$ can be described
  as the line bundle $(k[z], k[z^{-1}], \id)$.
  This is a special case of a more general family of line bundles on
  $\proj{1}$, as for each $n \in \bb{Z}$ we define a line bundle
  $\twist{n} = (k[z], k[z^{-1}], \vartheta_n)$, called the $n$th
  \emph{twisting sheaf}, where $\vartheta_n$ is the multiplication
  map given by $z^{-n}$ on $k[z, z^{-1}]$.
  These are all examples of $\strshf{\proj{1}}$-modules. \noparskip

  To see just how much varying the degree $n$ of the twist changes
  $\twist{n}$, we consider its global sections.
  These are pairs $(f, g) \in k[z] \times k[z^{-1}]$ such that $g =
  z^{-n} f$ in $k[z, z^{-1}]$, so each section is determined by the
  choice of $f$.
  This immediately implies that there are no non-zero global sections
  when $n < 0$, since $z^{-n} f$ consists of monomials of
  non-negative degree in $z$ while the opposite is true for $g$.
  By a similar degree argument, we see that the only possible choices
  of $f$ when $n \geq 0$ are of the form $f(z) = \sum_{i = 0}^{n}
  \lambda_i z^i$ for scalars $\lambda_i \in k$, so
  $\sections{\proj{1}}{\twist{n}} \cong k^{n + 1}$ as vector spaces over $k$.
  In the case $n = 0$, this recovers the well-known fact that all
  global regular functions on $\proj{1}$ are constant.
\end{example}

\begin{example}
  \label{exmp_morphisms_of_twists}
  Let us determine all morphisms $\twist{m} \to \twist{n}$, i.e. all
  pairs of endomorphisms $\varphi: k[z] \to k[z]$ and $\varphi':
  k[z^{-1}] \to k[z^{-1}]$ such that
  \[
    \begin{tikzcd}
      {k[z, z^{-1}]} & {k[z, z^{-1}]} \\
      {k[z, z^{-1}]} & {k[z, z^{-1}]}
      \arrow["{\varphi_z}", from=1-1, to=1-2]
      \arrow["\times z^{-m}"', from=1-1, to=2-1]
      \arrow["\times z^{-n}", from=1-2, to=2-2]
      \arrow["{\varphi'_{z^{-1}}}"', from=2-1, to=2-2]
    \end{tikzcd}
  \]
  commutes.
  Since $k[z]$ is a free $k[z]$-module of rank 1, $\varphi$ is
  multiplication by some $f \in k[z]$.
  Similarly, $\varphi'$ is multiplication by some $g \in k[z^{-1}]$,
  so the commutativity of the above diagram requires that $z^{-n} f =
  z^{-m} g$, which in turn holds if and only if $f = z^{n - m} g$.
  Since $f$ is determined for a particular choice of $g$ by this
  equation, it suffices to find all possible $g$.
  By the same logic we used in studying the global sections above,
  the degree (in $z^{-1}$) of $g$ must be at most $n - m$, so the
  only morphism when $m > n$ is zero.
  On the other hand, any $g \in \gen{1, z^{-1}, \ldots, z^{-(n - m)}}
  \idealof k[z^{-1}]$ determines a non-trivial morphism when $m \leq n$.
  Hence as a vector space over $k$,
  \[
    \Hom(\twist{m}, \twist{n})
    \cong
    \begin{cases}
      k^{n - m + 1} & \text{if } m \leq n \\
      0             & \text{if } m > n.
    \end{cases}
  \]
  This also shows that $\twist{m} \cong \twist{n}$ if and only if $m = n$.
\end{example}

In the case $m \leq n$, it follows that $\Hom(\twist{m}, \twist{n})
\cong \sections{\proj{1}}{\twist{n - m}}$.
A result that generalises this observation is the following useful fact.

\begin{lemma}
  \label{prop_glob_sect_hom}
  For any quasi-coherent sheaf $\shf{F}$ on $\proj{1}$, we have an isomorphism
  \[
    \Hom_{\QCoh(\proj{1})}(\shf{O}, \shf{F}) \cong \sections{\proj{1}}{\shf{F}}
  \]
  as vector spaces over $k$.
\end{lemma}

\begin{proof}
  This is the sheaf analogue of the group isomorphism $\Hom_R(R, M)
  \cong M$, so we can expect the proof to mimic it.
  Given a sheaf morphism $(\varphi, \varphi'): \shf{O} \to \shf{F}$,
  it follows from the commutativity of the diagram
  \[
    \begin{tikzcd}[cramped]
      {k[z, z^{-1}]} & {M_z} \\
      {k[z, z^{-1}]} & {M'_{z^{-1}}}
      \arrow["\varphi_z", from=1-1, to=1-2]
      \arrow[Rightarrow, no head, from=1-1, to=2-1]
      \arrow["\vartheta", from=1-2, to=2-2]
      \arrow["{\varphi'_{z^{-1}}}"', from=2-1, to=2-2]
    \end{tikzcd}
  \]
  that $(\varphi(1), \varphi'(1))$ is a global section of $\shf{F}$.
  This defines a $k$-linear map whose inverse is the map sending any
  global section $(m, m')$ of $\shf{F}$ to the pair $(\varphi,
  \varphi')$, where $\varphi: k[z] \to M$ is the unique $k[z]$-module
  homomorphism with $\varphi(1) = m$, and $\varphi': k[z^{-1}] \to
  M'$ is similarly determined over $k[z^{-1}]$ by $\varphi'(1) = m'$.
  Since $\vartheta(m) = m'$, it follows that the above diagram
  commutes, so we indeed have a morphism $(\varphi, \varphi'):
  \shf{O} \to \shf{F}$.
\end{proof}

With the twisting sheaves studied sufficiently for now, we move on to
the next key class of coherent sheaves.
These arise as instances of the following construction.

\begin{definition}
  \label{def_qcoh_sub_and_quot_objs}
  Given quasi-coherent sheaves $\shf{F} = (M, M', \vartheta)$ and
  $\shf{G} = (N, N', \varrho)$ on $\proj{1}$, we say that $\shf{G}$
  is a \emph{subsheaf} of $\shf{F}$ if $N \leq M$, $N' \leq M'$ and
  $\varrho$ is a restriction of $\vartheta$.
  The \emph{quotient} of $\shf{F}$ by some subsheaf $\shf{G}$ is
  defined as the quasi-coherent sheaf
  \[
    \quot{\shf{F}}{\shf{G}} := (\quot{M}{N}, \quot{M'}{N'},
    \overline{\vartheta})
  \]
  whose descent datum is the induced isomorphism on quotient modules
  \[
    \overline{\vartheta}: \quot{M_z}{N_z} \to \quot{M'_{z^{-1}}}{N'_{z^{-1}}}.
  \]
  \vspace{-24pt}
\end{definition}

\begin{example}
  \label{exmp_ideal_sheaf}
  Let $D$ be a closed subset of $\proj{1}$, say $D = V(F)$ for some
  homogeneous polynomial $F \in k[x, y]$ by
  \cref{lemma_opens_of_proj_line_are_principal}.
  Then the dehomogenisations of $F$ precisely give the ideals of
  vanishing for $D \cap U_x$ and $D \cap U_y$, i.e.
  \[
    I(D \cap U_x) = \gen{\dehomog{F}{x}} \idealof k[z]
    \mathand
    I(D \cap U_y) = \gen{\dehomog{F}{x}} \idealof k[z^{-1}].
  \]
  Since $\dehomog{F}{x}$ and $\dehomog{F}{y}$ are associates in $k[z,
  z^{-1}]$, the identity map induces an isomorphism
  $\gen{\dehomog{F}{x}}_z \cong \gen{\dehomog{F}{y}}_{z^{-1}}$, so
  the structure sheaf admits the coherent subsheaf
  \[
    \twist{-D} := (\gen{\dehomog{F}{x}}, \gen{\dehomog{F}{y}}, \id)
    \leq \shf{O},
  \]
  called an \emph{ideal sheaf}.
  Moreover, the quotient $\skyscraper{D} :=
  \quot{\shf{O}}{\twist{-D}}$ can be seen as a generalisation of the
  skyscraper sheaf from \cref{exmp_skyscraper_shf} for $\proj{1}$ in
  the case $D = \{p\}$ for some $p \in \proj{1}$.
  We denote the skyscraper sheaf at $p$ by $\skyscraper{p}$. \noparskip

  Let us study the sections of $\skyscraper{p}$ over $U_x$ for each
  projective point $p = (a : b)$.
  By computing the dehomogenisations of the polynomial $bx - ay$, it
  follows that
  \[
    \twist{-p} = (\gen{b - az}, \gen{bz^{-1} -a}, \id),
  \]
  so the sheaf $\skyscraper{p}$ is determined by the modules
  \[
    M = \frac{k[z]}{\gen{b - az}}
    \mathand
    M' = \frac{k[z^{-1}]}{\gen{bz^{-1} - a}}
  \]
  over $k[z]$ and $k[z^{-1}]$ respectively.
  Since $z^{-1}$ is not torsion in $M'$, the sections over $U_x$ will
  be pairs $(\overline{f}, \overline{g}/z^{-j}) \in M \times
  M'_{z^{-1}}$ such that $z^{-j} \overline{f} = \overline{g}$ in $M'_{z^{-1}}$.
  Each section is therefore determined by the choice of
  $\overline{f}$, which is a global section of the skyscraper sheaf
  $\assmod{M}$ on $\aff{1}_z$.
  It follows from the discussion in \cref{exmp_skyscraper_shf} that
  \[
    \sections{U_x}{\skyscraper{p}} \cong
    \begin{cases}
      k & \text{if } p \in U_x \\
      0 & \text{if } p \not\in U_x.
    \end{cases}
  \]
  More generally, one can show the same holds with $U_x$ replaced by
  any open subset $U$ of $\proj{1}$.
  This agrees with our intuition in the affine case.
\end{example}

What makes these examples so compelling is that any coherent sheaf on
$\proj{1}$ is built from twisting sheaves and skyscraper sheaves at a point.
This is the content of \emph{Grothendieck's splitting theorem} for
$\proj{1}$, and serves as the projective analogue of
\cref{thm_coh_sheaf_classification_of_aff_line}.
Before this, we must extend \cref{exmp_ideal_sheaf} by considering
\emph{thickenings} of skyscraper sheaves.

\begin{definition}
  Fix some $m \in \zpos$ and a projective point $p = (a : b)$.
  Then the \emph{thickened skyscraper sheaf} $\skyscraper{mp} :=
  \quot{\shf{O}}{\twist{-mp}}$, where
  \[
    \twist{-mp} := \twist{-p}^m = (\gen{b - az}^m, \gen{bz^{-1} - a}^m, \id).
  \]
  \vspace{-24pt}
\end{definition}

This is the geometric analogue of raising an ideal of a ring to a power.
Applying the argument in \cref{exmp_ideal_sheaf}, it is not difficult
to see that
\[
  \sections{U}{\skyscraper{mp}} \cong
  \begin{cases}
    k^m & \text{if } p \in U \\
    0 & \text{if } p \not\in U.
  \end{cases}
\]
The sheaf $\skyscraper{mp}$ is a torsion $\strshf{\proj{1}}$-module,
and it will play the same role as the module $\quot{k[x]}{\gen{x -
a}^n}$ does in \cref{thm_coh_sheaf_classification_of_aff_line}.
In particular, it suffices to classify the torsion part of any coherent sheaf.

\begin{theorem}[Grothendieck's splitting theorem]
  Every coherent sheaf $\shf{F}$ on $\proj{1}$ decomposes as the direct sum
  \[
    \shf{F} \cong
    \twist{n_1} \oplus \cdots \oplus \twist{n_r} \oplus
    \skyscraper{m_1 p_1} \oplus \cdots \oplus \skyscraper{m_s p_s}.
  \]
  \vspace{-24pt}
\end{theorem}

The direct sum of quasi-coherent sheaves is defined exactly as one
intuitively expects, though an explicit definition will be given in
the next section.
A proof of the splitting theorem is beyond the scope of this thesis.
For our purposes, the upshot is more important, namely that to
understand any coherent sheaf on $\proj{1}$, one only really needs to
understand the twisting sheaves and thickened skyscraper sheaves.

\section{Operations on coherent sheaves}

Since we have explicitly defined quasi-coherent sheaves in terms of
module data, one should intuitively expect that some of the structure
and operations on the category of modules will carry over to
$\QCoh(\proj{1})$ and $\Coh(\proj{1})$.
This indeed ends up being the case, so the goal of this section is to
explicitly spell out the most important aspects.

A key technical detail that we will implicitly use throughout this
discussion is that localisation is exact, so it preserves subobjects,
quotients, direct sums, kernels and cokernels at the level of modules
(c.f. \cref{rem_exact_limit_colimit_pres}).
In particular, we can use this fact to describe the descent data of
the sheaves in the constructions below in a way that is more suited
to computations.
This will also lead to many of the definitions we give in this
section being quite straightforward, such as the following.

\begin{definition}
  Let $\shf{F} = (M, M', \vartheta)$ and $\shf{G} = (N, N', \varrho)$
  be quasi-coherent sheaves on $\proj{1}$.
  The \emph{direct sum} of $\shf{F}$ and $\shf{G}$ is the quasi-coherent sheaf
  \[
    \shf{F} \biprod \shf{G} := (M \biprod N, M' \biprod N', \vartheta
    \biprod \varrho).
  \]
  \vspace{-24pt}
\end{definition}

Here, we have used the fact that $(M \biprod N)_f \cong M_f \biprod
N_f$ to simplify the descent datum.
It is easy to see that $\QCoh(\proj{1})$ and $\Coh(\proj{1})$ are
additive categories, since there is an obvious way to bilinearly add
morphisms and an even more obvious candidate for the zero object.
We leave the task of checking these small technical details to the reader.

Next, we treat the abelian structure.
Retaining the notation of \cref{def_coh_sheaf_morphism}, a
quasi-coherent sheaf morphism $\Phi: \shf{F} \to \shf{G}$ on
$\proj{1}$ gives rise to a commutative diagram
\[
  \begin{tikzcd}
    {\ker(\varphi)_z} & {M_z} & {N_z} & {\coker(\varphi)_z} \\
    {\ker(\varphi')_{z^{-1}}} & {M'_{z^{-1}}} & {N'_{z^{-1}}} &
    {\coker(\varphi')_{z^{-1}}}.
    \arrow[hook, from=1-1, to=1-2]
    \arrow["{\varphi_z}", from=1-2, to=1-3]
    \arrow["\vartheta"', from=1-2, to=2-2]
    \arrow[two heads, from=1-3, to=1-4]
    \arrow["\varrho", from=1-3, to=2-3]
    \arrow[hook, from=2-1, to=2-2]
    \arrow["{\varphi'_{z^{-1}}}"', from=2-2, to=2-3]
    \arrow[two heads, from=2-3, to=2-4]
  \end{tikzcd}
\]
The commutativity of the middle square implies that the composites
\[
  \varphi'_{z^{-1}} \circ \vartheta \circ \ker(\varphi)_z = \varrho
  \circ \varphi_z \circ \ker(\varphi)_z
\]
% \[
%     \varphi'(\vartheta(m/z^i))
%     = \varrho(\varphi(m/z^i))
%     % = \varrho(\varphi(m)/z^i)
%     = 0
% \]
% for any $m \in \ker(\varphi)$,
are zero, so the restriction of $\vartheta$ to $\ker(\varphi)_z$ is a
map into ${\ker(\varphi')_{z^{-1}}}$.

\begin{definition}
  The \emph{kernel} of a quasi-coherent sheaf morphism $\Phi: \shf{F}
  \to \shf{G}$ on $\proj{1}$ is the quasi-coherent subsheaf
  $\ker(\Phi) := (\ker(\varphi), \ker(\varphi'), \vartheta) \leq \shf{F}$.
  Here, we introduce a slight abuse of notation by identifying
  $\vartheta$ with the restriction $\setres{\vartheta}{\ker(\varphi)_z}$.
  One defines $\coker(\Phi)$ using the above commutative diagram
  similarly, and hence also $\im(\Phi)$.
\end{definition}

Since $k[z]$ and $k[z^{-1}]$ are Noetherian rings, we see that
kernels and cokernels of coherent sheaf morphisms must in turn be coherent.
As before, we leave it to the reader to check the category-theoretic
details necessary for the following claim.

\begin{proposition}
  The categories $\QCoh(\proj{1})$ and $\Coh(\proj{1})$ are abelian.
\end{proposition}

With this, we may consider all of the usual notions for an abelian category.
We gave subobjects and quotient objects more palatable, explicit
definitions in \cref{def_qcoh_sub_and_quot_objs}.
By \cref{lemma_abelian_category_monos_and_epis}, the monics, epics
and isomorphisms in $\QCoh(\proj{1})$ are also explicit: we call the
morphism $\Phi = (\varphi, \varphi)$ \emph{injective} if $\varphi$
and $\varphi'$ are injective module homomorphisms. \emph{Surjective}
and \emph{bijective} morphisms are defined similarly.
To see these tools in action, we consider the following result that
will be useful in the next section.

\begin{lemma}
  \label{prop_ideal_sheaf_twist}
  For any $m \in \zpos$ and $p \in \proj{1}$, we have $\twist{-mp}
  \cong \twist{-m}$.
\end{lemma}

\begin{proof}
  Let $p = (a : b)$, and consider the maps $\varphi: k[z] \to k[z]$
  and $\varphi': k[z^{-1}] \to k[z^{-1}]$ given by multiplication by
  $(b - az)^m$ and $(bz^{-1} - a)^m$ respectively.
  As one easily checks, this determines an injective sheaf morphism
  $\twist{-m} \to \shf{O}$ whose image is $\twist{-mp}$.
\end{proof}

Of course, the most significant tool in an abelian category are short
exact sequences, so we consider two important examples for coherent sheaves.

\begin{example}
  There is a short exact sequence of coherent sheaves
  \[
    \begin{tikzcd}[cramped]
      0 & {\twist{-D}} & {\shf{O}} & {\skyscraper{D}} & 0
      \arrow[from=1-1, to=1-2]
      \arrow[from=1-2, to=1-3]
      \arrow[from=1-3, to=1-4]
      \arrow[from=1-4, to=1-5]
    \end{tikzcd}
  \]
  associated to any closed $D \subseteq \proj{1}$, sometimes called
  the \emph{ideal sheaf sequence}.
  This is a geometric analogue of the short exact sequence of $R$-modules
  \[
    \begin{tikzcd}[cramped]
      0 & I & R & {R/I} & 0
      \arrow[from=1-1, to=1-2]
      \arrow[from=1-2, to=1-3]
      \arrow[from=1-3, to=1-4]
      \arrow[from=1-4, to=1-5]
    \end{tikzcd}
  \]
  associated to any ideal $I \idealof R$.
\end{example}

\begin{proposition}[Euler sequence for $\proj{1}$]
  There is a short exact sequence
  \[
    \begin{tikzcd}[cramped]
      0 & {\twist{-2}} & {\twist{-1}^{\biprod 2}} & {\shf{O}} & 0.
      \arrow[from=1-1, to=1-2]
      \arrow[from=1-2, to=1-3]
      \arrow[from=1-3, to=1-4]
      \arrow[from=1-4, to=1-5]
    \end{tikzcd}
  \]
  \vspace{-24pt}
\end{proposition}

\begin{proof}
  As is easy to check, the pairs $(\times 1, \times z^{-1})$ and
  $(\times z, \times 1)$ both determine morphisms $\twist{-1} \to \shf{O}$.
  This induces maps $\varphi: k[z]^{\biprod 2} \to k[z]$ and
  $\varphi': k[z^{-1}]^{\biprod 2} \to k[z^{-1}]$ given explicitly by
  $\varphi(f, g) = f + zg$ and $\varphi'(f, g) = z^{-1}f + g$.
  % \begin{align*}
  %     \varphi(f, g) &= (1, z) \cdot (f, g) & \varphi'(f, g) &=
  % (z^{-1}, 1) \cdot (f, g) \\
  %     &= f + zg, & &= z^{-1}f + g.
  % \end{align*}
  These assemble to determine a morphism $\Phi: \twist{-1}^{\biprod
  2} \to \shf{O}$, since the diagram
  \[
    \begin{tikzcd}[cramped]
      {k[z, z^{-1}]^{\biprod2}} & {k[z, z^{-1}]} \\
      {k[z, z^{-1}]^{\biprod2}} & {k[z, z^{-1}]}
      \arrow["{\varphi_z}", from=1-1, to=1-2]
      \arrow["{\times z}"', from=1-1, to=2-1]
      \arrow[Rightarrow, no head, from=1-2, to=2-2]
      \arrow["{\varphi'_{z^{-1}}}"', from=2-1, to=2-2]
    \end{tikzcd}
  \]
  commutes.
  Both $\varphi$ and $\varphi'$ are surjective module homomorphisms,
  and it follows that $\Phi$ is surjective.
  To obtain the Euler exact sequence, it therefore suffices to show
  that $\twist{-2} \cong \ker(\Phi)$.
  The map $\psi: 1 \mapsto (-z, 1)$ induces a $k[z]$-module isomorphism
  \[
    k[z] \cong \{(-zg, g): g \in k[z]\} = \ker(\varphi),
  \]
  and similarly the map $\psi': 1 \mapsto -(1, -z^{-1})$ induces a
  $k[z^{-1}]$-module isomorphism
  \[
    k[z^{-1}] \cong \{(f, -z^{-1}f): f \in k[z^{-1}]\} = \ker(\varphi').
  \]
  The introduction of the sign in $\psi'$ ensures that the diagram
  \[
    \begin{tikzcd}[cramped]
      {k[z, z^{-1}]} & {\ker(\varphi)[z^{-1}]} \\
      {k[z, z^{-1}]} & {\ker(\varphi')[z]}
      \arrow["{\psi_z}", tail reversed, from=1-1, to=1-2]
      \arrow["{\times z^2}"', from=1-1, to=2-1]
      \arrow["{\times z}", from=1-2, to=2-2]
      \arrow["{\psi'_{z^{-1}}}"', tail reversed, from=2-1, to=2-2]
    \end{tikzcd}
  \]
  commutes both ways, and so the pair $(\psi, \psi')$ is the desired
  sheaf isomorphism.
\end{proof}

Since the tensor product of two modules produces another module over
commutative rings, we can also reasonably expect to have an
equivalent notion for quasi-coherent sheaves on $\proj{1}$.
We once again base our definition on the $R_f$-module isomorphism $(M
\tensor_R N)_f \cong M_f \tensor_{R_f} N_f$.

\begin{definition}
  Given quasi-coherent sheaves $\shf{F} = (M, M', \vartheta)$ and
  $\shf{G} = (N, N', \varrho)$ on $\proj{1}$, the \emph{tensor
  product} of $\shf{F}$ and $\shf{G}$ is the quasi-coherent sheaf
  \[
    \shf{F} \shftensor \shf{G}
    :=
    (M \tensor_{k[z]} N, M' \tensor_{k[z^{-1}]} N', \vartheta
    \tensor_{k[z, z^{-1}]} \varrho).
  \]
  \vspace{-24pt}
\end{definition}

Just as it does for modules, $\shftensor$ defines a bifunctor on
$\QCoh(\proj{1})$ which is right exact in both arguments, though we
will not explicitly verify this claim.
As is a recurring theme at this point, tensor products involving
twisting sheaves are easy to describe.

\begin{proposition}
  \label{prop_shftensor_of_twists}
  For $m, n \in \bb{Z}$ we have $\twist{m} \shftensor \twist{n} \cong
  \twist{m + n}$.
\end{proposition}

\begin{proof}
  The most crucial step is to understand why the sheaf $\twist{m +
  n}$ appears in this isomorphism.
  But this is just an easy consequence of the algebra of tensors, since
  \[
    z^{-m} f \tensor_{k[z, z^{-1}]} z^{-n} g = f \tensor_{k[z,
    z^{-1}]} z^{-(m + n)} g
  \]
  for any $f, g \in k[z, z^{-1}]$, i.e. the descent datum of
  $\twist{m} \shftensor \twist{n}$ and the map $1 \tensor
  \vartheta_{m + n}$ are the same.
  The proof is now clear, as one simply lifts the $R$-module
  isomorphism $R \tensor_R R \cong R$ given by $r \tensor s \mapsto
  rs$ to an appropriate coherent sheaf isomorphism.
\end{proof}

\iffalse
\begin{corollary}
  When $n \geq 0$, we have $\twist{n} = \twist{1}^{\tensor n}$ and
  $\twist{-n} = \twist{-1}^{\tensor n}$.
\end{corollary}
\fi

Our interest in the interaction between $\shftensor$ and these
twisting sheaves stems from the following important class of
quasi-coherent sheaves motivated by module theory.

\begin{definition}
  A quasi-coherent sheaf $\shf{F} = (M, M', \vartheta)$ on $\proj{1}$
  is called \emph{flat} if $M$ and $M'$ are flat modules over $k[z]$
  and $k[z^{-1}]$ respectively.
\end{definition}

The flat modules are equivalently those for which the tensor product
of modules is an exact functor, and we can check that the analogous
statement naturally holds for a flat quasi-coherent sheaf on
$\proj{1}$ in our setup.

\begin{proposition}
  If $\shf{G} = (N, N', \varrho)$ is a flat quasi-coherent sheaf on
  $\proj{1}$, then the functor $- \shftensor \shf{G}: \QCoh(\proj{1})
  \to \QCoh(\proj{1})$ is exact.
\end{proposition}

\begin{proof}
  Consider a short sequence of sheaves
  \[
    \begin{tikzcd}[cramped]
      0 & {\shf{F}_1} & {\shf{F}_2} & {\shf{F}_3} & 0
      \arrow[from=1-1, to=1-2]
      \arrow["{\Phi_1}", from=1-2, to=1-3]
      \arrow["{\Phi_2}", from=1-3, to=1-4]
      \arrow[from=1-4, to=1-5]
    \end{tikzcd}
  \]
  with $\Phi_1 = (\varphi_1, \varphi_1')$ and $\Phi_2 = (\varphi_2,
  \varphi_2')$.
  It is matter of checking definitions to see that this sequence is
  exact if and only if the short sequences of modules
  \[
    \begin{tikzcd}[cramped]
      0 & {M_1} & {M_2} & {M_3} & 0, \\
      0 & {M_1'} & {M_2'} & {M_3'} & 0
      \arrow[from=1-1, to=1-2]
      \arrow["{\varphi_1}", from=1-2, to=1-3]
      \arrow["{\varphi_2}", from=1-3, to=1-4]
      \arrow[from=1-4, to=1-5]
      \arrow[from=2-1, to=2-2]
      \arrow["{\varphi_1'}"', from=2-2, to=2-3]
      \arrow["{\varphi_1'}"', from=2-3, to=2-4]
      \arrow[from=2-4, to=2-5]
    \end{tikzcd}
  \]
  are exact over $k[z]$ and $k[z^{-1}]$ respectively.
  Since $N$ and $N'$ are flat modules, the claim is now immediate
  upon tensoring the above short exact sequence of sheaves by $\shf{G}$.
\end{proof}

The most basic class of flat sheaves are the twisting sheaves, and in
light of this, we define a new operation on quasi-coherent sheaves.

\begin{definition}
  Let $n \in \bb{Z}$ and define
  \[
    \shf{F}(n) := \shf{F} \shftensor \twist{n}
  \]
  for any quasi-coherent sheaf $\shf{F}$ on $\proj{1}$.
  We call $\shf{F}(n)$ the \emph{degree $n$ twist} of $\shf{F}$, and
  it induces an exact \emph{twisting functor} $(n): \QCoh(\proj{1})
  \to \QCoh(\proj{1})$.
\end{definition}

\begin{proposition}
  \label{prop_twist_is_autoequiv}
  Twisting is an autoequivalence of $\QCoh(\proj{1})$.
\end{proposition}

\begin{proof}
  Let $n \in \bb{Z}$.
  The key observation is that by \cref{prop_shftensor_of_twists}, we have
  \begin{align*}
    \shf{F}(-n)(n)
    = (\shf{F} \shftensor \twist{-n}) \shftensor \twist{n}
    \cong \shf{F} \shftensor \shf{O}
    = \shf{F}(0),
  \end{align*}
  and there is an obvious isomorphism $\shf{F}(0) \cong \shf{F}$.
  Similarly, $\shf{F}(n)(-n) \cong \shf{F}$.
  From these facts, it is not difficult to see that the functors
  $(n)$ and $(-n)$ must be naturally inverse to each other, so the
  degree $n$ twist is an equivalence of categories.
\end{proof}

% \begin{lemma}
%     Let $m \in \zpos$ and $p \in \proj{1}$.
%     Then for any $n \in \bb{Z}$, we have $\skyscraper{mp}(n) \cong
% \skyscraper{mp}$.
% \end{lemma}

% \begin{proof}

% \end{proof}

\iffalse
Defining a sheaf version of the Hom of two quasi-coherent sheaves on
$\proj{1}$ is more complicated, since the natural injective map of $R_f$-modules
\[
  \Hom_R(M, N)_f \mono \Hom_{R_f}(M_f, N_f)
\]
need not be surjective.
This is an isomorphism whenever $M$ is finitely presented, and over
Noetherian rings this is equivalent to $M$ being finitely generated.
To avoid these technical concerns, we will therefore only define the
sheaf $\shfhom(\shf{F}, \shf{G})$ in the case where $\shf{F}$ is
coherent, even though this assumption is a bit stronger than needed.

\begin{definition}
  Given quasi-coherent sheaves $\shf{F} = (M, M', \vartheta)$ and
  $\shf{G} = (N, N', \varrho)$ on $\proj{1}$, the \emph{sheaf Hom} of
  $\shf{F}$ and $\shf{G}$ is the quasi-coherent sheaf
  \[
    \shfhom(\shf{F}, \shf{G})
    :=
    (\Hom_{k[z]}(M, N), \Hom_{k[z^{-1}]}(M', N'), [\vartheta, \varrho]),
  \]
  where $[\vartheta, \varrho]$ is the isomorphism sending $f: M_z \to
  N_z$ to $\varrho \circ f \circ \vartheta^{-1}: M'_{z^{-1}} \to N'_{z^{-1}}$.
\end{definition}

Once again, we note without proof that this is a bifunctor on
$\QCoh(\proj{1})$ which is contravariant left exact in its first
argument and covariant left exact in the other.
By construction, the global sections of sheaf Hom also recover the
usual Hom, i.e.
\[
  \sections{\proj{1}}{\shfhom(\shf{F}, \shf{G})} = \Hom(\shf{F}, \shf{G}).
\]
A convenient use for the sheaf Hom is in giving a notion of duality.

\begin{definition}
  For any quasi-coherent sheaf $\shf{F}$ on $\proj{1}$, we define the
  \emph{dual sheaf} of $\shf{F}$ to be $\dual{\shf{F}} :=
  \shfhom(\shf{F}, \shf{O})$.
\end{definition}

This is a sheaf-theoretic incarnation of the dual of an $R$-module
$\dual{M} = \Hom_R(M, R)$, and indeed it is not difficult to see that
the $R$-module isomorphism
\[
  \dual{M} \tensor_R M \cong R, \qquad f \tensor m \mapsto f(m)
\]
lifts to a quasi-coherent sheaf isomorphism $\dual{\shf{F}}
\shftensor \shf{F} \cong \shf{O}$.

\iffalse
The sheaf Hom and tensor product give additional structure to $\Pic$,
the set of isomorphism classes of line bundles on $\proj{1}$, and is
of relevance to the theory of \emph{divisors} on a variety.
It is clear that the tensor product of line bundles is a line bundle,
and also that $\shf{L} \shftensor \shf{O} \cong \shf{L}$ for any line
bundle $\shf{L}$.
We define the \emph{dual bundle} of $\shf{L}$ by $\dual{\shf{L}} :=
\shfhom(\shf{L}, \shf{O})$, and one can check that the $R$-module isomorphism
\[
  \Hom_R(M, R) \tensor_R M \cong R, \qquad f \tensor m \mapsto f(m)
\]
lifts to an isomorphism $\dual{\shf{L}} \shftensor \shf{L} \cong \shf{O}$.
Thus $\Pic$ is an abelian group under $\shftensor$ with identity
element $\shf{O}$, called the \emph{Picard group} of $\proj{1}$.
The dual of a line bundle is by construction its inverse in this
group, which is why line bundles are also called invertible sheaves.
By \cref{prop_shftensor_of_twists}, we have $\dual{\twist{n}} =
\twist{-n}$, and it turns out that $\Pic$ is generated by the line
bundle $\twist{1}$, i.e. $\Pic \cong \bb{Z}$.
This is a non-trivial statement, and is proven in a more formal
setting in \cite[Proposition~II.6.17]{hartshorne}.
\fi

\fi

\section{Coherent sheaf cohomology for $\proj{1}$}
\label{sect_coh_sheaf_cohom}

To conclude this chapter, we discuss some aspects of \emph{coherent
sheaf cohomology} for $\proj{1}$.
In its full form, this is an indispensable technique in modern
algebraic geometry that applies even more generally than just for
quasi-coherent sheaves.
In keeping with the spirit of this chapter so far, we present only
what is needed for our goals.

As a prelude to this discussion, now seems like a fitting time to
address the issue of injectives and projectives in $\QCoh(\proj{1})$
and $\Coh(\proj{1})$, for which there is mixed news.

\begin{proposition}
  The category $\QCoh(\proj{1})$ has enough injectives, but not
  enough projectives.
  The category $\Coh(\proj{1})$ has neither enough injectives nor
  enough projectives.
\end{proposition}

We are not adequately equipped to treat these claims, so we only make
brief remarks.
The first claim is true because $\proj{1}$ is an example of a
\emph{Noetherian scheme}, and the category of quasi-coherent sheaves
on any such scheme will always have enough injectives (c.f.
\cite[Exercise~III.3.6(a)]{hartshorne}).
This does not descend to $\Coh(\proj{1})$ since coherent sheaves are
intuitively ``too small'' to be injective, in much the same way that
the injective envelope of a finitely generated module is not usually
finitely generated.
As for the lack of projectives in both categories, this can be seen
to fail already for the structure sheaf $\shf{O}_{\proj{1}}$, which
is the upshot of \cite[Exercise~III.6.2]{hartshorne}.
This will be a more significant observation for
\cref{chap_beilinsons_theorem}, but the more immediate task of
setting up coherent sheaf cohomology is contingent only on the first claim.

The motivation for this technology is to study potential problems in
lifting global sections.
It turns out that the \emph{global sections functor}
$\sections{\proj{1}}{-}: \QCoh(\proj{1}) \to \Vect{k}$ is left exact,
so given a short exact sequence of quasi-coherent sheaves
\[
  \begin{tikzcd}[cramped]
    0 & {\shf{F}_1} & {\shf{F}_2} & {\shf{F}_3} & 0
    \arrow[from=1-1, to=1-2]
    \arrow[from=1-2, to=1-3]
    \arrow[from=1-3, to=1-4]
    \arrow[from=1-4, to=1-5]
  \end{tikzcd}
\]
on $\proj{1}$, we have only the left exact sequence of vector spaces
\[
  \begin{tikzcd}[cramped]
    0 & {\sections{\proj{1}}{\shf{F}_1}} &
    {\sections{\proj{1}}{\shf{F}_2}} & {\sections{\proj{1}}{\shf{F}_3}}.
    \arrow[from=1-1, to=1-2]
    \arrow[from=1-2, to=1-3]
    \arrow[from=1-3, to=1-4]
  \end{tikzcd}
\]
The failure of the map $\sections{\proj{1}}{\shf{F}_2} \to
\sections{\proj{1}}{\shf{F}_3}$ to be surjective is exactly the
obstruction to lifting a global section from $\shf{F}_3$ to
$\shf{F}_2$, and there are two main ways to qualitatively measure this.
The more sophisticated view is given by Grothendieck in
\cite{tohoku}, who suggests that we should study the \emph{derived
functor cohomology groups}
\[
  H^i(\shf{F}) = H^i(\proj{1}, \shf{F}) := R^i
  \sections{\proj{1}}{\shf{F}} \cong \Ext{i}{}(\shf{O}, \shf{F}).
\]
The last isomorphism follows by \cref{prop_glob_sect_hom}, and it is
for this reason that one has the additional notation $H^0(X,
\shf{F})$ for global sections.
While this is certainly powerful for answering theoretical questions,
this version of sheaf cohomology is inconvenient to compute in
practice since we must explicitly construct an injective resolution
of $\shf{F}$.
Serre's original version of sheaf cohomology for algebraic geometry
as defined in \cite{fac} instead uses the easier to compute
\emph{\v{C}ech cohomology groups} $\cechalt{i}{\shf{F}}$ defined below.

A priori, these two cohomology theories may not yield isomorphic
cohomology groups.
Fortunately, they are always isomorphic for quasi-coherent sheaves on
\emph{Noetherian separated schemes} (c.f.
\cite[Theorem~III.4.5]{hartshorne}) such as $\aff{1}$ and $\proj{1}$.
In our discussion, we therefore mostly consider sheaf cohomology a la Serre.

\begin{definition}
  Fix an open cover $\mathcal{U} = \{U_i\}_{i \in I}$ of a
  topological space $X$ and a well-order on the index set $I$.
  For ease of notation, we will write the intersection of the subsets
  corresponding to the ordered $(p + 1)$-tuple $\bm{i} = (i_0, i_1,
  \ldots, i_p) \in I^{p + 1}$ as
  \[
    U_{i_0 i_1 \cdots i_p} = U_{i_0} \cap U_{i_1} \cap \cdots \cap U_{i_p},
  \]
  where $i_0 < i_1 < \cdots < i_p$.
  Then for any sheaf of abelian groups $\shf{F}$ on $X$, one defines
  the $p$th \emph{\v{C}ech cochains} to be the abelian group
  \[
    \cechcochain{p}{\mathcal{U}}{\shf{F}}
    :=
    \bigoplus_{\bm{i} \in I^{p + 1}} \sections{U_{i_0 i_1 \cdots i_p}}{\shf{F}}
  \]
  for all $p \in \zpos$.
  We define the $p$th differential map $d:
  \cechcochain{p}{\mathcal{U}}{\shf{F}} \to \cechcochain{p +
  1}{\mathcal{U}}{\shf{F}}$ by
  \begin{align*}
    d(s_{i_0 i_1 \cdots i_p})_{\bm{i} \in I^{p + 1}}
    :=
    \left(\sum_{j = 0}^{p + 1} (-1)^j \setres{s_{i_0 i_1 \cdots
      \widehat{i_j} \cdots i_{p + 1}}}{U_{i_0 i_1 \cdots \widehat{i_j}
    \cdots i_{p + 1}}}\right)_{\bm{i} \in I^{p + 2}},
  \end{align*}
  where $i_0 i_1 \cdots \widehat{i_j} \cdots i_{p + 1}$ is the list
  with the index $i_j$ omitted.
  From this, one obtains the \emph{\v{C}ech cochain complex}
  $\cechcochain{\bullet}{\mathcal{U}}{\shf{F}}$,
  \[
    \begin{tikzcd}[cramped]
      0 & {\cechcochain{0}{\mathcal{U}}{\shf{F}}} &
      {\cechcochain{1}{\mathcal{U}}{\shf{F}}} &
      {\cechcochain{2}{\mathcal{U}}{\shf{F}}} & \cdots,
      \arrow[from=1-1, to=1-2]
      \arrow["d", from=1-2, to=1-3]
      \arrow["d", from=1-3, to=1-4]
      \arrow["d", from=1-4, to=1-5]
    \end{tikzcd}
  \]
  and its cohomology are the \emph{\v{C}ech cohomology groups}
  $\cechalt{p}{\shf{F}} = \cech{p}{\mathcal{U}}{\shf{F}}$ for $p \geq 0$.
\end{definition}

Mercifully, this setup is much more tractable for $\proj{1}$.
The \v{C}ech cohomology groups are generally dependent on the choice
of open cover, but not in the aforementioned case of Noetherian
separated schemes, so we can proceed using our standard affine open
cover $\mathcal{U} = \{U_x, U_y\}$ with impunity.
For any quasi-coherent sheaf $\shf{F}$ on $\proj{1}$, we have
\[
  \cechcochain{0}{\mathcal{U}}{\shf{F}} = \sections{U_x}{\shf{F}}
  \biprod \sections{U_y}{\shf{F}}
  \mathand
  \cechcochain{1}{\mathcal{U}}{\shf{F}} = \sections{U_{xy}}{\shf{F}},
\]
and clearly $\cechcochain{p}{\mathcal{U}}{\shf{F}} = 0$ for all $p \geq 2$.
The only non-trivial differential in the \v{C}ech complex is the map
$d: \sections{U_x}{\shf{F}} \biprod \sections{U_y}{\shf{F}} \to
\sections{U_{xy}}{\shf{F}}$ given by
\[
  d(\bm{s}, \bm{t}) = \setres{\bm{s}}{U_{xy}} - \setres{\bm{t}}{U_{xy}},
\]
where $\bm{s} = (s, s')$ and $\bm{t} = (t, t')$ in the notation of
\cref{def_coh_sheaf_sects_and_restr}.
A technical point that is unique to the particular way we have set up
quasi-coherent sheaves on $\proj{1}$ is that one may need to use the
descent datum for $\shf{F}$ first to ensure that these restrictions
can be subtracted in this manner (i.e. they must be compatible with
each other first).

It follows that the only non-zero cohomology groups for $\proj{1}$ are
\[
  \cechalt{0}{\shf{F}} = \ker(d)
  \mathand
  \cechalt{1}{\shf{F}} = \coker(d),
\]
and it is not hard to see that $\cechalt{0}{\shf{F}}$ indeed
corresponds to $\sections{\proj{1}}{\shf{F}}$ by the sheaf property.
The equivalent statement about the vanishing of the higher derived
functor cohomology groups is known as \emph{Grothendieck vanishing},
and is rather non-trivial in comparison (c.f.
\cite[Proposition~20.20.7]{stacks}).
With the advanced knowledge that \v{C}ech and derived functor
cohomology agree in this situation, we will simply write
$H^0(\shf{F})$ and $H^1(\shf{F})$ for both cohomology theories.
We will also persist in calling these spaces cohomology \emph{groups}
even though they are in fact vector spaces over $k$.

The value in introducing \v{C}ech cohomology in general upfront
rather than giving only the specific setup for $\proj{1}$ is that we
see that the \v{C}ech cohomology groups are the cohomology of a
cochain complex, so all of the usual notions of
\cref{sect_cohomology_homotopy} are available.
In particular, each $H^p: \QCoh(\proj{1}) \to \Vect{k}$ is an
additive functor which commutes with direct sums, so as a consequence
of Grothendieck's splitting theorem, the cohomology groups of any
coherent sheaf on $\proj{1}$ reduces to a direct sum of the
cohomology groups of $\twist{n}$ and $\skyscraper{mp}$.
In this vein, we present two fundamental computations.

\begin{lemma}
  For any $m \in \zpos$ and $p \in \proj{1}$, $H^1(\skyscraper{mp}) = 0$.
\end{lemma}

\begin{proof}
  Put another way, the claim is that the differential map $d$ is surjective.
  The sections of $\skyscraper{mp}$ are supported only in open
  neighbourhoods of $p$, so if $p$ is either of the degenerate points
  $(1 : 0)$ or $(0 : 1)$, then $p \not\in U_{xy}$ and $d = 0$, so the
  claim is trivial in this case.
  In the non-degenerate case, we have the isomorphisms of vector spaces
  \[
    \sections{U_x}{\skyscraper{mp}}
    \cong \sections{U_y}{\skyscraper{mp}}
    \cong \sections{U_{xy}}{\skyscraper{mp}}
    \cong k^m.
  \]
  Since the descent datum of $\skyscraper{mp}$ acts as the identity,
  the restriction maps to $U_{xy}$ are simply the identity maps on
  $k^m$, so we see that $d$ is once again surjective.
\end{proof}

\begin{lemma}
  \label{lemma_h1_twist}
  For any $n \in \bb{Z}$, we have
  \[
    H^1(\twist{n})
    % =
    % \begin{cases}
    %     \gen{z^{-1}, z^{-2}, \ldots, z^{-(|n| - 1)}} & \text{if } n < -1 \\
    %     0 & \text{if } n \geq -1.
    % \end{cases}
    \cong
    \begin{cases}
      k^{|n| - 1} & \text{if } n \leq -2 \\
      0 & \text{if } n \geq -1.
    \end{cases}
  \]
\end{lemma}

\begin{proof}
  As should be familiar at this point, we can characterise the
  sections of $\twist{n}$ over $U_x$ as pairs $(f, g) \in k[z] \times
  k[z, z^{-1}]$ such that $g = z^{-n} f$.
  Each section is determined by the choice of $f$, so there is an
  isomorphism $\sections{U_x}{\twist{n}} \cong k[z]$.
  By similar arguments, we have $\sections{U_y}{\twist{n}} \cong
  k[z^{-1}]$, and similarly
  \[
    \sections{U_{xy}}{\twist{n}} \cong k[z]_z \cong k[z, z^{-1}]
    \cong k[z^{-1}]_{z^{-1}}.
  \]
  The corresponding restriction maps from $U_x$ and $U_y$ to $U_{xy}$
  are the canonical inclusions $k[z] \mono k[z]_z$ and $k[z^{-1}]
  \mono k[z]_{z^{-1}}$ respectively, but to ensure compatibility we
  shall postcompose the second restriction map by $\vartheta_n^{-1}:
  k[z^{-1}]_{z^{-1}} \to k[z]_z$, i.e. by multiplying by $z^n$.
  Thus the \v{C}ech differential is ultimately the map $d(f, g) = f - z^n g$.

  It remains to find $\coker(d)$.
  We see that $\im(d)$ is a vector subspace of $k[z, z^{-1}]$ over
  $k$, and clearly it contains the subspace generated by all
  non-negative powers of $z$.
  Moreover, $\im(d)$ contains $z^i = z^{n} z^{-(n - i)}$ for a given
  $i \leq -1$ if and only if $n - i \geq 0$.
  This is trivially true when $n \geq -1$,
  % since
  % \[
  %     n - i \geq -1 - (-1) = 0,
  % \]
  so in this case $\im(d) = k[z, z^{-1}]$ and $H^1(\twist{n}) = 0$.
  When $n \leq -2$, we have $n - i = -|n| - i < 0$ for all $-(|n| -
  1) \leq i \leq -1$, so in this case $\im(d)$ is the (set-theoretic)
  complement of the subspace
  \[
    \gen{z^{-1}, z^{-2}, \ldots, z^{-(|n| - 1)}} \cong k^{|n| - 1}.
  \]
  It follows that $H^1(\twist{n}) \cong k^{|n| - 1}$.
\end{proof}

The final pair of cohomological facts we will collect relate to the Ext groups.
The second of these will rely on an additional result in coherent
sheaf cohomology that is slightly beyond our reach in this thesis.

\begin{lemma}
  \label{prop_ext_from_twist}
  Let $\shf{G}$ be a coherent sheaf on $\proj{1}$.
  Then for all $n \in \bb{Z}$ and $i \in \zpos$,
  \[
    \Extalt{i}(\twist{n}, \shf{G}) \cong H^i(\shf{G} \shftensor \twist{-n}).
  \]
  In particular, $\Extalt{i}(\twist{n}, \shf{G}) = 0$ for $i \geq 2$.
\end{lemma}

\begin{proof}
  Since the degree $-n$ twist is an autoequivalence, by
  \cref{lemma_ext_preserved_by_equiv} we have
  \begin{align*}
    \Extalt{i}(\twist{n}, \shf{G})
    &\cong \Extalt{i}(\shf{O}, \shf{G} \shftensor \twist{-n}) \\
    &\cong H^i(\shf{G} \shftensor \twist{-n})
  \end{align*}
  by construction of the derived functor cohomology.
  These cohomology groups vanish for $i \geq 2$, so the second part
  of this claim is immediate.
\end{proof}

\begin{proposition}
  \label{prop_CohP1_hereditary}
  The category $\Coh(\proj{1})$ is \emph{hereditary}.
  That is, $\Extalt{i}(\shf{F}, \shf{G}) = 0$ for all $i \geq 2$ and
  coherent sheaves $\shf{F}$, $\shf{G}$ on $\proj{1}$.
\end{proposition}

\begin{proof}
  Since $\Extalt{i}$ is an additive functor, it is enough to prove
  the claim when $\shf{F}$ is either a twisting sheaf or a skyscraper
  sheaf by Grothendieck's splitting theorem.
  We have treated the former case already in
  \cref{prop_ext_from_twist}, so suppose that $\shf{F} = \skyscraper{mp}$.
  Recalling \cref{prop_ideal_sheaf_twist}, we have the ideal sheaf sequence
  \[
    \begin{tikzcd}[cramped]
      0 & {\twist{-m}} & {\shf{O}} & {\skyscraper{mp}} & 0,
      \arrow[from=1-1, to=1-2]
      \arrow[from=1-2, to=1-3]
      \arrow[from=1-3, to=1-4]
      \arrow[from=1-4, to=1-5]
    \end{tikzcd}
  \]
  Applying a degree $n$ twist for some $n \in \bb{Z}$ yields a new
  short exact sequence
  \[
    \begin{tikzcd}[cramped]
      0 & {\twist{n - m}} & {\twist{n}} & {\skyscraper{mp}(n)} & 0.
      \arrow[from=1-1, to=1-2]
      \arrow[from=1-2, to=1-3]
      \arrow[from=1-3, to=1-4]
      \arrow[from=1-4, to=1-5]
    \end{tikzcd}
  \]
  First, we note that $\skyscraper{mp}(n) \cong \skyscraper{mp}$.
  Indeed, the sections are still supported only at the point $p$, and
  the algebraic effect of the twist on the descent datum in
  neighbourhoods of $p$ is scaling by some negative power of a unit
  in $k$, which is an invertible operation.
  Next, since the contravariant functor $\Hom(-, \shf{I})$ is exact
  for any injective quasi-coherent sheaf $\shf{I}$, taking an
  injective resolution $\shf{G} \to \cochaincomp{\shf{I}}$ yields a
  short exact sequence
  \[
    \begin{tikzcd}[cramped]
      0 & {\Hom(\skyscraper{mp}, \cochaincomp{\shf{I}})} &
      {\Hom(\twist{n}, \cochaincomp{\shf{I}})} & {\Hom(\twist{n - m},
      \cochaincomp{\shf{I}})} & 0
      \arrow[from=1-1, to=1-2]
      \arrow[from=1-2, to=1-3]
      \arrow[from=1-3, to=1-4]
      \arrow[from=1-4, to=1-5]
    \end{tikzcd}
  \]
  of complexes with the direction reversed.
  Now, for each $i \geq 2$, we consider the exact portion of the
  corresponding long exact cohomology sequence
  \[
    \begin{tikzcd}[cramped]
      {\Extalt{i - 1}(\twist{n - m}, \shf{G})} &
      {\Extalt{i}(\skyscraper{mp}, \shf{G})} & {\Extalt{i}(\twist{n}, \shf{G})}.
      \arrow[from=1-1, to=1-2]
      \arrow[from=1-2, to=1-3]
    \end{tikzcd}
  \]
  The right term vanishes for all $i \geq 2$ by
  \cref{prop_ext_from_twist}, and so does the left term when $i > 2$,
  so we conclude that $\Extalt{i}(\skyscraper{mp}, \shf{G}) = 0$ in this case.
  Thus it remains to show that
  \[
    \Extalt{1}(\twist{n - m}, \shf{G})
    \cong
    \Extalt{1}(\shf{O}, \shf{G}(m - n))
    \cong H^1(\shf{G}(m - n))
  \]
  also vanishes.
  This may not be true a priori for any choice of $n$, but by a
  vanishing theorem of Serre (c.f.
  \cite[Theorem~III.5.2(b)]{hartshorne}), the cohomology group
  $H^1(\shf{G}(m - n))$ is guaranteed to vanish when $m - n$ is
  sufficiently large, i.e. when $n$ is sufficiently negative.
  Taking such an $n$ therefore completes the proof of the claim.
\end{proof}

