\section{Coherent sheaves on $\proj{1}$}
\label{sect_coh_sheaves_p1}

With our preparations out of the way, we are now in a position to
introduce \emph{coherent sheaves} on $\proj{1}$.
It will be instructive to give the usual definitions upfront in order
to provide some sense of where our hands-on definitions we obtain
later come from.

\begin{definition}
  \label{def_general_qcoh}
  Let $\shf{F}$ be an $\strshf{X}$-module on some quasi-projective
  variety $X$ over $k$.
  Then $\shf{F}$ is \emph{quasi-coherent} if there is an affine open
  cover $X = \bigcup_i U_i$ (i.e. with each $U_i$ isomorphic to an
  affine variety) and an $\shf{O}(U_i)$-module $M_i$ such that
  $\setres{\shf{F}}{U_i} \cong \assmod{M_i}$.
  We say that $\shf{F}$ is \emph{coherent} if furthermore each module
  $M_i$ is finitely generated over $\shf{O}(U_i)$.
\end{definition}

There is another equivalent definition, which we package up in the
following result.

\begin{proposition}[{\cite[Proposition~II.5.4]{hartshorne}}]
  \label{prop_general_qcoh_equiv}
  An $\strshf{X}$-module $\shf{F}$ is quasi-coherent if and only if
  for each affine open $U \subseteq X$, there is an
  $\shf{O}(U)$-module $M$ such that $\setres{\shf{F}}{U} \cong \assmod{M}$.
  Moreover, $\shf{F}$ is coherent if and only if $M$ is finitely
  generated over $\shf{O}(U)$.
\end{proposition}

Though \cref{def_general_qcoh} and \cref{prop_general_qcoh_equiv} are
standard fare in algebraic geometry, the reader should feel some
trepidation here.
The issue is that as we have constructed it, \cref{def_assmod} only
applies to quasi-affine varieties in $\aff{1}$ rather than arbitrary
affine varieties, so the assertion that $\setres{\shf{F}}{U} \cong
\assmod{M}$ for some affine open $U \subseteq X$ is ill-defined.
We will only concern ourselves with the varieties $\aff{1}$ and
$\proj{1}$ for which this issue is not a concern.
The intention of presenting these definitions in spite of this
imprecision is to convey the important idea that quasi-coherent
sheaves on $X$ are `locally just modules'.

The category of quasi-coherent sheaves on $X$ is denoted $\QCoh(X)$,
and the coherent sheaves form a full subcategory $\Coh(X)$.
In the case of $\aff{1}$, these categories end up being as nice as
one could realistically ask for.

\begin{corollary}
  \label{cor_affine_coh_equiv_to_mods}
  % If $\shf{F}$ is a (quasi-)coherent sheaf on $\aff{1}$,
  % then $\shf{F}$ is determined by the $k[x]$-module of global
  % sections $\sections{\aff{1}}{\shf{F}}$.
  The category $\QCoh(\aff{1})$ is equivalent to $\Mod{k[x]}$ via the
  pair of naturally inverse functors $\shf{F} \mapsto
  \sections{\aff{1}}{\shf{F}}$ and $M \mapsto \assmod{M}$.
  Moreover, this restricts to an equivalence of categories between
  $\Coh(\aff{1})$ and $\mod{k[x]}$.
\end{corollary}

\begin{proof}
  This is a specialisation and restatement of
  \cite[Corollary~II.5.5]{hartshorne}, but the main contention
  follows from \cref{prop_general_qcoh_equiv}.
\end{proof}

In other words, the coherent sheaves on $\aff{1}$ are little more
than finitely generated $k[x]$-modules.
Technical concerns notwithstanding, the same claim is true mutatis
mutandis for all affine varieties, which is a significant reduction
in the complexity of studying these sheaves.
However, one can go even further specifically in the case of $\aff{1}$.
In particular, the structure theorem for finitely generated modules
over a principal ideal domain can be appropriated as a classification
theorem for coherent sheaves on $\aff{1}$.
For convenience, we state this as the following theorem in its
primary decomposition form.

\begin{theorem}
  \label{thm_coh_sheaf_classification_of_aff_line}
  If $\shf{F}$ is a coherent sheaf on $\aff{1}$, i.e. a finitely
  generated $k[x]$-module $M$, then there are $r, s \in \zpos$,
  powers $n_1, n_2, \ldots, n_s \in \zpos$ and points $a_1, \ldots,
  a_r \in k$, not necessarily distinct, such that $\shf{F}$ is the
  sheaf associated to the module
  \[
    M
    \cong
    k[x]^{\oplus r} \oplus \frac{k[x]}{\gen{x - a_1}^{n_1}} \oplus
    \cdots \oplus \frac{k[x]}{\gen{x - a_s}^{n_s}}.
  \]
  \vspace{-18pt}
  % Moreover, the points $a_i$ are unique up to associates in $k$.
\end{theorem}

Put another way, this theorem also shows that any coherent sheaf on
$\aff{1}$ splits as a sum of a torsion-free and a torsion summand, an
observation we will return to later.

The next natural question to ask is whether there is also an elegant
theory for coherent sheaves on non-affine quasi-projective varieties.
Immediately, $\proj{1}$ makes for a good case study, since $\proj{1}$
has a standard affine open cover by two copies of $\aff{1}$ (i.e. we
are again free of prior technical concerns).

We first fix some conventions for $\proj{1}$ that we will use throughout.
Recall that the aforementioned affine open cover is given by
$\proj{1} = U_x \cup U_y$, where
\[
  U_x = \{(x : y): x \neq 0\} \mathand U_y = \{(x : y): y \neq 0\}.
\]
Each point $(x : y) \in U_x$ is the same as $(1 : y/x)$, so there is
a clear homeomorphism $\phi_x: U_x \to \aff{1}_z$ identifying $U_x$
with the affine line $\aff{1}_z$ with coordinate $z = y/x$.
Similarly, there is a clear homeomorphism $\phi_y: U_y \to
\aff{1}_{w}$ where $\aff{1}_{w}$ has coordinate $w = x/y$.
% This defines an open cover $\proj{1} = U_x \cup U_y$, and the
% aforementioned bijections $\phi_x: U_x \to \aff{1}_z$ and $\phi_y:
% U_y \to \aff{1}_{w}$ are in fact homeomorphisms (c.f.
% \cite[Proposition~I.1.2.2]{hartshorne}).
We call the maps $\phi_x$ and $\phi_y$ \emph{affine charts} for
$\proj{1}$, and similarly we call the open subsets $U_x$ and $U_y$
\emph{affine patches}.
We can convert between affine charts on $U_{xy} = U_x \cap U_y$ using
the \emph{transition map} $\tau: \aff{1}_z \setminus \{0\} \to
\aff{1}_w \setminus \{0\}$ defined by
\[
  \tau(z)
  := \phi_y \phi_x^{-1}(z)
  = \phi_y(1 : z)
  = \phi_y(z^{-1} : 1)
  = z^{-1}.
\]
Because of this, we can also view $\proj{1}$ as the result of gluing
together the affine lines $\aff{1}_z$ and $\aff{1}_w$ according to
the rule $w = z^{-1}$ whenever $z, w \neq 0$.
For algebraic convenience, we will replace the affine coordinate $w$
with $z^{-1}$ so that, formally, $\proj{1} = \aff{1}_z \sqcup \aff{1}_{z^{-1}}$.

In view of \cref{def_general_qcoh}, a quasi-coherent sheaf $\shf{F}$
on $\proj{1}$ is locally determined on this open cover by
$\shf{F}|_{U_x}$ and $\shf{F}|_{U_y}$, i.e. by a module $M$ over
$\shf{O}(U_x) \cong k[z]$ and a module $M'$ over $\shf{O}(U_y) \cong
k[z^{-1}]$ respectively.
One complication in simply forgetting the sheaf structure and
retaining only the pair $(M, M')$ is that between them, they encode a
significant amount of overlapping sheaf data on $U_{xy}$.
To resolve this problem, it is enough to specify an isomorphism of sheaves
\[
  \vartheta:
  \setres{\left(\setres{\shf{F}}{U_x}\right)}{U_{xy}}
  \to
  \setres{\left(\setres{\shf{F}}{U_y}\right)}{U_{xy}}.
\]
Algebraically, $U_{xy}$ is the region in which both $z$ and $z^{-1}$
are invertible in $k[z]$ and $k[z^{-1}]$ respectively, so it stands
to reason that $\vartheta$ can equivalently be specified as an
isomorphism between $\sections{D(z)}{\assmod{M}} = M_z$ and
$\sections{D(z^{-1})}{\assmod{M'}} = M'_{z^{-1}}$ as modules over
$k[z, z^{-1}]$, the ring of \emph{Laurent polynomials} in $z$.
This is the outline for the alternative definition of quasi-coherent
sheaves for $\proj{1}$ we will use in this thesis.
We will also pair it with a more tailored notion of a morphism than
\cref{def_general_sheaf_morphism}.

% In order to generalise \cref{def_assmod} so that we get a sheaf on
% $\proj{1}$, one plausible strategy is specify each section
% separately on the patches $U_x$ and $U_y$ using a pair of modules
% on the affine lines $\aff{1}_z$ and $\aff{1}_{z^{-1}}$.
% We will also have to specify a map to glue this section data
% together on the compatible overlap $U_x \cap U_y$, which
% algebraically is the region where $z$ amd $z^{-1}$ are invertible
% in $k[z]$ and $k[z^{-1}]$ respectively.
% Just as in the construction of the associated sheaf of a module, we
% take appropriate extensions of scalars so that our gluing map will
% be an isomorphism between $M[z^{-1}]$ and $M[z]$.
% Naturally, these are modules over $k[z, z^{-1}]$, the ring of
% \emph{Laurent polynomials} in $z$.

% This is obviously far from a `canonical' way to approach defining
% such a sheaf, since it depends heavily on the particular choice of
% covering of $\proj{1}$, but it turns out that it is once again
% commensurate with the `real' definition of \emph{quasi-coherent}
% sheaves for $\proj{1}$.
% Since we are at pains to avoid the general theory here, this claim
% is out of our reach, but it does justify taking the following as
% our working definition.

\begin{definition}
  \label{def_coh_sheaf}
  A \emph{quasi-coherent sheaf} on $\proj{1}$ is the data of a triple
  $\shf{F} = (M, M', \vartheta)$ consisting of a $k[z]$-module $M$, a
  $k[z^{-1}]$-module $M'$ and a $k[z^{-1}]$-module isomorphism
  $\vartheta: M_{z} \to M'_{z^{-1}}$, called the \emph{descent datum}
  of $\shf{F}$.
  We say that $\shf{F}$ is \emph{coherent} if $M$ and $M'$ are both
  finitely generated over $k[z]$ and $k[z^{-1}]$ respectively.
\end{definition}

\begin{definition}
  \label{def_coh_sheaf_morphism}
  Let $\shf{F} = (M, M', \vartheta)$ and $\shf{G} = (N, N', \varrho)$
  be quasi-coherent sheaves on $\proj{1}$.
  A \emph{morphism} $\Phi: \shf{F} \to \shf{G}$ is a pair $(\varphi,
  \varphi')$ consisting of a $k[z]$-module homomorphism $\varphi: M
  \to N$ and a $k[z^{-1}]$-module homomorphism $\varphi': M' \to N'$,
  subject to the condition that there is a commutative diagram of
  $k[z, z^{-1}]$-modules
  \[
    \begin{tikzcd}
      {M_z} & {N_z} \\
      {M'_{z^{-1}}} & {N'_{z^{-1}}}.
      \arrow["{\varphi_z}", from=1-1, to=1-2]
      \arrow["\vartheta"', from=1-1, to=2-1]
      \arrow["\varrho", from=1-2, to=2-2]
      \arrow["{\varphi'_{z^{-1}}}"', from=2-1, to=2-2]
    \end{tikzcd}
  \]
  % Here, we introduce a slight abuse of notation by identifying
  % $\varphi$ and $\varphi'$ with the homomorphisms induced by
  % localisation so that, say, $\varphi(m/z^i) = \varphi(m)/z^i$.
  If $\Psi: \shf{G} \to \shf{H}$ is another morphism of
  quasi-coherent sheaves with data $(\psi, \psi')$, then we define
  the \emph{composite} $\Psi \circ \Phi: \shf{F} \to \shf{H}$ to be
  the morphism with data $(\psi \circ \varphi, \psi' \circ \varphi')$.
\end{definition}

Choosing to regard quasi-coherent sheaves in this manner is somewhat
`unnatural', insofar as it is often more advantageous to try and work
in coordinate-free settings.
It is often necessary to fix coordinates for actual computations
though, and this presents no significant obstruction in practice.

\begin{proposition}
  Let $\cat{Q}$ be the category of quasi-coherent sheaves on
  $\proj{1}$ in the sense of \cref{def_coh_sheaf} and
  \cref{def_coh_sheaf_morphism}, and let $\cat{C}$ be the full
  subcategory of $\cat{Q}$ consisting of coherent sheaves.
  Then $\QCoh(\proj{1})$ is equivalent to $\cat{Q}$, and this
  restricts to an equivalence of categories between $\Coh(\proj{1})$
  and $\cat{C}$.
\end{proposition}

\begin{proof}
  We give only a brief idea of the proof.
  The functor $\QCoh(\proj{1}) \to \cat{Q}$ in this equivalence is
  easy to construct, as given $\shf{F} \in \QCoh(\proj{1})$ we have
  \[
    \displaystylesections{\setres{\shf{F}}{U_x}}{U_{xy}} =
    \sections{\shf{F}}{U_{xy}} =
    \displaystylesections{\setres{\shf{F}}{U_y}}{U_{xy}},
  \]
  so we simply take $\shf{F} \mapsto \left(\sections{\shf{F}}{U_x},
  \sections{\shf{F}}{U_y}, \id_{\sections{\shf{F}}{U_{xy}}}\right)$.
  The functor $\cat{Q} \to \QCoh(\proj{1})$ naturally inverse to this
  is harder to construct, but in essence will follow from the
  forthcoming description of the sections and restriction maps of
  some $\shf{F} \in \cat{Q}$.
\end{proof}

From this point forward, we therefore identify $\QCoh(\proj{1})$ and
$\Coh(\proj{1})$ with $\cat{Q}$ and $\cat{C}$ and dispense with the
general definitions from the start of this section.
Though we will stop short of an explicit verification that these
triples are actually sheaves in the technical sense of
\cref{def_sheaf}, it will be worthwhile to at least describe what the
sections and restriction maps of some $\shf{F} \in \QCoh(\proj{1})$
are to get a feel for how to work with these sheaves computationally.
We first recall the following facts about $\proj{1}$.

\begin{lemma}
  \label{lemma_opens_of_proj_line_are_principal}
  Every open subset $U \subseteq \proj{1}$ is of the form $D(F) :=
  \proj{1} \setminus V(F)$ for some homogeneous polynomial $F \in k[x, y]$.
  Moreover, any open subset $V \subseteq U$ is of the form $V =
  D(FG)$ for some other homogeneous polynomial $G \in k[x, y]$.
\end{lemma}

% \begin{proof}
%     If $U \subseteq \proj{1}$ is open, then $U = \proj{1} = X
% \setminus V(I)$ for some homogeneous ideal $I \idealof k[x, y]$,
% i.e. $I$ is generated by homogeneous polynomials.
%     Since $k[x, y]$ is Noetherian, $I$ is generated by finitely
% many such polynomials.
%     Thus $V(I)$ is finite, and we can construct a homogeneous
% polynomial $F$ whose zeroes are precisely the points of $V(I)$.
% \end{proof}

This result is the projective analogue of
\cref{lemma_opens_of_aff_line_are_principal} and
\cref{lemma_principal_open_inclusions_of_aff_line}.
Since the proof is similar, we shall omit it and mention only that
one instead needs to use the fact that $k[x, y]$ is Noetherian for
the first claim.
We may always arrange that $F$ and $G$ have minimal degree here,
since for example we can take
\[
  F(x, y) = (b_1 x - a_1 y) (b_2 x - a_2 y) \cdots (b_d x - a_d y),
\]
where the $(a_i : b_i)$ are the distinct points of the (finite) set
$\proj{1} \setminus U$.
At this juncture, it will be convenient to introduce some notation
and an important fact about dehomogenisation of polynomials.

\begin{definition}
  \label{def_canonical_dehomog}
  Let $F \in k[x, y]$ be a homogeneous polynomial.
  Setting $z = y/x$, the \emph{dehomogenisations} of $F$ separately
  with respect to $x$ and $y$ produces the pair of univariate
  polynomials which, whenever they are defined, are denoted by
  \[
    \dehomog{F}{x}(z) := F(1, z) \in k[z]
    \mathand
    \dehomog{F}{y}(z^{-1}) := F(z^{-1}, 1) \in k[z^{-1}].
  \]
  \vspace{-24pt}
\end{definition}

\begin{lemma}
  The dehomogenisations of a homogeneous polynomial $F \in k[x, y]$
  are associates in the ring of Laurent polynomials $k[z, z^{-1}]$,
  where $z = y/x$.
\end{lemma}

\begin{proof}
  This follows from homogenity, since if $d = \deg{F}$, then
  \[
    \dehomog{F}{x}(z) = F(1, z) = F(z z^{-1}, z) = z^d F(z^{-1}, 1) =
    z^d \dehomog{F}{y}(z^{-1}). \qedhere
  \]
  \vspace{-24pt}
\end{proof}

Now, suppose that $U \subseteq \proj{1}$ is the open set $D(F)$ as in
\cref{lemma_opens_of_proj_line_are_principal}.
The dehomogenisations of $F$ identify $U$ in each affine patch, i.e.
we have homeomorphisms
\[
  U \cap U_x \cong D(\dehomog{F}{x})
  \mathand
  U \cap U_y \cong D(\dehomog{F}{y}),
\]
so a section of $\shf{F}$ over $U$ is a pair of affine sections $(s,
s') \in M_{\dehomog{F}{x}} \times M'_{\dehomog{F}{y}}$, say
\[
  s = \quot{m}{\dehomog{F}{x}^i}
  \mathand
  s' = \quot{m'}{\dehomog{F}{y}^j}.
\]
It remains to describe how to glue $s$ and $s'$ on $U \cap U_{xy}$.
Algebraically, this is the region in which both $\dehomog{F}{x}$ and
$\dehomog{F}{y}$ are invertible in $k[z, z^{-1}]$, so we consider the
modules $M_{\dehomog{F}{x}, z}$ and $M'_{\dehomog{F}{y}, z^{-1}}$
over the ring $k[z, z^{-1}, \dehomog{F}{x}^{-1}, \dehomog{F}{y}^{-1}]$.
Since $\dehomog{F}{x} = z^{\deg{F}} \dehomog{F}{y}$ in this ring, the
fraction $\vartheta(m)/\dehomog{F}{x}^i$ is a well-defined element of
$M'_{\dehomog{F}{y}, z^{-1}}$, and we will insist for the sake of
compatibility that it agrees with $s'$, i.e. we impose the condition
\[
  \quot{\vartheta(m)}{\dehomog{F}{x}^i} = \quot{m'}{\dehomog{F}{y}^j}
  % \frac{\vartheta(m)}{f^i} = \frac{m'}{g^j}
\]
in $M'_{\dehomog{F}{y}, z^{-1}}$.
By definition of equality in a localised module, this holds if and only if
\begin{equation}
  \label{eq_local_section_compatibility}
  \dehomog{F}{y}^n(\dehomog{F}{y}^j \vartheta(m) - \dehomog{F}{x}^i m') = 0
\end{equation}
in $M'_{z^{-1}}$ for some power $n \in \zpos$.
The restriction maps at this point are clear, so we present the
following definition to summarise this discussion.

\begin{definition}
  \label{def_coh_sheaf_sects_and_restr}
  Let $\shf{F} = (M, M', \vartheta)$ be a quasi-coherent sheaf on
  $\proj{1}$ and $U = D(F)$ an open subset of $\proj{1}$.
  A \emph{local section} of $\shf{F}$ over $U$ is a pair
  \[
    (s, s')  = (m/\dehomog{F}{x}^i, m'/\dehomog{F}{y}^j) \in
    M_{\dehomog{F}{x}} \times M'_{\dehomog{F}{y}}
  \]
  satisfying \cref{eq_local_section_compatibility},
  and a \emph{global section} of $\shf{F}$ is a pair $(m, m') \in M
  \times M'$ satisfying $\vartheta(m) = m'$.
  For any open subset $V = D(FG) \subseteq U$, the \emph{restriction
  map} $\res{U}{V}$ of $\shf{F}$ sends the section $(s, s') \in
  \sections{U}{\shf{F}}$ to the section
  \[
    \left(\setres{s}{D(\dehomog{F}{x} \dehomog{G}{x})},\,
    \setres{s'}{D(\dehomog{F}{y} \dehomog{G}{y})}\right) \in
    \sections{V}{\shf{F}}
  \]
  via the affine restriction maps for $\assmod{M}$ and $\assmod{M'}$
  on each component.
\end{definition}

We will study the two most fundamental families of coherent sheaves
on $\proj{1}$ in depth for the remainder of this section, but it will
be convenient to introduce some more specific terminology and
definitions as we go.

\begin{definition}
  A quasi-coherent sheaf $\shf{F} = (M, M', \vartheta)$ on $\proj{1}$
  is \emph{locally free} of rank $r$ if $M \cong k[z]^r$ and $M'
  \cong k[z^{-1}]^r$.
  A locally free sheaf of rank 1 is called a \emph{line bundle}.
  % or an \emph{invertible sheaf} (for reasons that will become clear
  % in the next section).
\end{definition}

\begin{example}
  \label{exmp_twists}
  The structure sheaf $\shf{O} := \strshf{\proj{1}}$ can be described
  as the line bundle $(k[z], k[z^{-1}], \id)$.
  This is a special case of a more general family of line bundles on
  $\proj{1}$, as for each $n \in \bb{Z}$ we define a line bundle
  $\twist{n} = (k[z], k[z^{-1}], \vartheta_n)$, called the $n$th
  \emph{twisting sheaf}, where $\vartheta_n$ is the multiplication
  map given by $z^{-n}$ on $k[z, z^{-1}]$.
  These are all examples of $\strshf{\proj{1}}$-modules. \noparskip

  To see just how much varying the degree $n$ of the twist changes
  $\twist{n}$, we consider its global sections.
  These are pairs $(f, g) \in k[z] \times k[z^{-1}]$ such that $g =
  z^{-n} f$ in $k[z, z^{-1}]$, so each section is determined by the
  choice of $f$.
  This immediately implies that there are no non-zero global sections
  when $n < 0$, since $z^{-n} f$ consists of monomials of
  non-negative degree in $z$ while the opposite is true for $g$.
  By a similar degree argument, we see that the only possible choices
  of $f$ when $n \geq 0$ are of the form $f(z) = \sum_{i = 0}^{n}
  \lambda_i z^i$ for scalars $\lambda_i \in k$, so
  $\sections{\proj{1}}{\twist{n}} \cong k^{n + 1}$ as vector spaces over $k$.
  In the case $n = 0$, this recovers the well-known fact that all
  global regular functions on $\proj{1}$ are constant.
\end{example}

\begin{example}
  \label{exmp_morphisms_of_twists}
  Let us determine all morphisms $\twist{m} \to \twist{n}$, i.e. all
  pairs of endomorphisms $\varphi: k[z] \to k[z]$ and $\varphi':
  k[z^{-1}] \to k[z^{-1}]$ such that
  \[
    \begin{tikzcd}
      {k[z, z^{-1}]} & {k[z, z^{-1}]} \\
      {k[z, z^{-1}]} & {k[z, z^{-1}]}
      \arrow["{\varphi_z}", from=1-1, to=1-2]
      \arrow["\times z^{-m}"', from=1-1, to=2-1]
      \arrow["\times z^{-n}", from=1-2, to=2-2]
      \arrow["{\varphi'_{z^{-1}}}"', from=2-1, to=2-2]
    \end{tikzcd}
  \]
  commutes.
  Since $k[z]$ is a free $k[z]$-module of rank 1, $\varphi$ is
  multiplication by some $f \in k[z]$.
  Similarly, $\varphi'$ is multiplication by some $g \in k[z^{-1}]$,
  so the commutativity of the above diagram requires that $z^{-n} f =
  z^{-m} g$, which in turn holds if and only if $f = z^{n - m} g$.
  Since $f$ is determined for a particular choice of $g$ by this
  equation, it suffices to find all possible $g$.
  By the same logic we used in studying the global sections above,
  the degree (in $z^{-1}$) of $g$ must be at most $n - m$, so the
  only morphism when $m > n$ is zero.
  On the other hand, any $g \in \gen{1, z^{-1}, \ldots, z^{-(n - m)}}
  \idealof k[z^{-1}]$ determines a non-trivial morphism when $m \leq n$.
  Hence as a vector space over $k$,
  \[
    \Hom(\twist{m}, \twist{n})
    \cong
    \begin{cases}
      k^{n - m + 1} & \text{if } m \leq n \\
      0             & \text{if } m > n.
    \end{cases}
  \]
  This also shows that $\twist{m} \cong \twist{n}$ if and only if $m = n$.
\end{example}

In the case $m \leq n$, it follows that $\Hom(\twist{m}, \twist{n})
\cong \sections{\proj{1}}{\twist{n - m}}$.
A result that generalises this observation is the following useful fact.

\begin{lemma}
  \label{prop_glob_sect_hom}
  For any quasi-coherent sheaf $\shf{F}$ on $\proj{1}$, we have an isomorphism
  \[
    \Hom_{\QCoh(\proj{1})}(\shf{O}, \shf{F}) \cong \sections{\proj{1}}{\shf{F}}
  \]
  as vector spaces over $k$.
\end{lemma}

\begin{proof}
  This is the sheaf analogue of the group isomorphism $\Hom_R(R, M)
  \cong M$, so we can expect the proof to mimic it.
  Given a sheaf morphism $(\varphi, \varphi'): \shf{O} \to \shf{F}$,
  it follows from the commutativity of the diagram
  \[
    \begin{tikzcd}[cramped]
      {k[z, z^{-1}]} & {M_z} \\
      {k[z, z^{-1}]} & {M'_{z^{-1}}}
      \arrow["\varphi_z", from=1-1, to=1-2]
      \arrow[Rightarrow, no head, from=1-1, to=2-1]
      \arrow["\vartheta", from=1-2, to=2-2]
      \arrow["{\varphi'_{z^{-1}}}"', from=2-1, to=2-2]
    \end{tikzcd}
  \]
  that $(\varphi(1), \varphi'(1))$ is a global section of $\shf{F}$.
  This defines a $k$-linear map whose inverse is the map sending any
  global section $(m, m')$ of $\shf{F}$ to the pair $(\varphi,
  \varphi')$, where $\varphi: k[z] \to M$ is the unique $k[z]$-module
  homomorphism with $\varphi(1) = m$, and $\varphi': k[z^{-1}] \to
  M'$ is similarly determined over $k[z^{-1}]$ by $\varphi'(1) = m'$.
  Since $\vartheta(m) = m'$, it follows that the above diagram
  commutes, so we indeed have a morphism $(\varphi, \varphi'):
  \shf{O} \to \shf{F}$.
\end{proof}

With the twisting sheaves studied sufficiently for now, we move on to
the next key class of coherent sheaves.
These arise as instances of the following construction.

\begin{definition}
  \label{def_qcoh_sub_and_quot_objs}
  Given quasi-coherent sheaves $\shf{F} = (M, M', \vartheta)$ and
  $\shf{G} = (N, N', \varrho)$ on $\proj{1}$, we say that $\shf{G}$
  is a \emph{subsheaf} of $\shf{F}$ if $N \leq M$, $N' \leq M'$ and
  $\varrho$ is a restriction of $\vartheta$.
  The \emph{quotient} of $\shf{F}$ by some subsheaf $\shf{G}$ is
  defined as the quasi-coherent sheaf
  \[
    \quot{\shf{F}}{\shf{G}} := (\quot{M}{N}, \quot{M'}{N'},
    \overline{\vartheta})
  \]
  whose descent datum is the induced isomorphism on quotient modules
  \[
    \overline{\vartheta}: \quot{M_z}{N_z} \to \quot{M'_{z^{-1}}}{N'_{z^{-1}}}.
  \]
  \vspace{-24pt}
\end{definition}

\begin{example}
  \label{exmp_ideal_sheaf}
  Let $D$ be a closed subset of $\proj{1}$, say $D = V(F)$ for some
  homogeneous polynomial $F \in k[x, y]$ by
  \cref{lemma_opens_of_proj_line_are_principal}.
  Then the dehomogenisations of $F$ precisely give the ideals of
  vanishing for $D \cap U_x$ and $D \cap U_y$, i.e.
  \[
    I(D \cap U_x) = \gen{\dehomog{F}{x}} \idealof k[z]
    \mathand
    I(D \cap U_y) = \gen{\dehomog{F}{x}} \idealof k[z^{-1}].
  \]
  Since $\dehomog{F}{x}$ and $\dehomog{F}{y}$ are associates in $k[z,
  z^{-1}]$, the identity map induces an isomorphism
  $\gen{\dehomog{F}{x}}_z \cong \gen{\dehomog{F}{y}}_{z^{-1}}$, so
  the structure sheaf admits the coherent subsheaf
  \[
    \twist{-D} := (\gen{\dehomog{F}{x}}, \gen{\dehomog{F}{y}}, \id)
    \leq \shf{O},
  \]
  called an \emph{ideal sheaf}.
  Moreover, the quotient $\skyscraper{D} :=
  \quot{\shf{O}}{\twist{-D}}$ can be seen as a generalisation of the
  skyscraper sheaf from \cref{exmp_skyscraper_shf} for $\proj{1}$ in
  the case $D = \{p\}$ for some $p \in \proj{1}$.
  We denote the skyscraper sheaf at $p$ by $\skyscraper{p}$. \noparskip

  Let us study the sections of $\skyscraper{p}$ over $U_x$ for each
  projective point $p = (a : b)$.
  By computing the dehomogenisations of the polynomial $bx - ay$, it
  follows that
  \[
    \twist{-p} = (\gen{b - az}, \gen{bz^{-1} -a}, \id),
  \]
  so the sheaf $\skyscraper{p}$ is determined by the modules
  \[
    M = \frac{k[z]}{\gen{b - az}}
    \mathand
    M' = \frac{k[z^{-1}]}{\gen{bz^{-1} - a}}
  \]
  over $k[z]$ and $k[z^{-1}]$ respectively.
  Since $z^{-1}$ is not torsion in $M'$, the sections over $U_x$ will
  be pairs $(\overline{f}, \overline{g}/z^{-j}) \in M \times
  M'_{z^{-1}}$ such that $z^{-j} \overline{f} = \overline{g}$ in $M'_{z^{-1}}$.
  Each section is therefore determined by the choice of
  $\overline{f}$, which is a global section of the skyscraper sheaf
  $\assmod{M}$ on $\aff{1}_z$.
  It follows from the discussion in \cref{exmp_skyscraper_shf} that
  \[
    \sections{U_x}{\skyscraper{p}} \cong
    \begin{cases}
      k & \text{if } p \in U_x \\
      0 & \text{if } p \not\in U_x.
    \end{cases}
  \]
  More generally, one can show the same holds with $U_x$ replaced by
  any open subset $U$ of $\proj{1}$.
  This agrees with our intuition in the affine case.
\end{example}

What makes these examples so compelling is that any coherent sheaf on
$\proj{1}$ is built from twisting sheaves and skyscraper sheaves at a point.
This is the content of \emph{Grothendieck's splitting theorem} for
$\proj{1}$, and serves as the projective analogue of
\cref{thm_coh_sheaf_classification_of_aff_line}.
Before this, we must extend \cref{exmp_ideal_sheaf} by considering
\emph{thickenings} of skyscraper sheaves.

\begin{definition}
  Fix some $m \in \zpos$ and a projective point $p = (a : b)$.
  Then the \emph{thickened skyscraper sheaf} $\skyscraper{mp} :=
  \quot{\shf{O}}{\twist{-mp}}$, where
  \[
    \twist{-mp} := \twist{-p}^m = (\gen{b - az}^m, \gen{bz^{-1} - a}^m, \id).
  \]
  \vspace{-24pt}
\end{definition}

This is the geometric analogue of raising an ideal of a ring to a power.
Applying the argument in \cref{exmp_ideal_sheaf}, it is not difficult
to see that
\[
  \sections{U}{\skyscraper{mp}} \cong
  \begin{cases}
    k^m & \text{if } p \in U \\
    0 & \text{if } p \not\in U.
  \end{cases}
\]
The sheaf $\skyscraper{mp}$ is a torsion $\strshf{\proj{1}}$-module,
and it will play the same role as the module $\quot{k[x]}{\gen{x -
a}^n}$ does in \cref{thm_coh_sheaf_classification_of_aff_line}.
In particular, it suffices to classify the torsion part of any coherent sheaf.

\begin{theorem}[Grothendieck's splitting theorem]
  Every coherent sheaf $\shf{F}$ on $\proj{1}$ decomposes as the direct sum
  \[
    \shf{F} \cong
    \twist{n_1} \oplus \cdots \oplus \twist{n_r} \oplus
    \skyscraper{m_1 p_1} \oplus \cdots \oplus \skyscraper{m_s p_s}.
  \]
  \vspace{-24pt}
\end{theorem}

The direct sum of quasi-coherent sheaves is defined exactly as one
intuitively expects, though an explicit definition will be given in
the next section.
A proof of the splitting theorem is beyond the scope of this thesis.
For our purposes, the upshot is more important, namely that to
understand any coherent sheaf on $\proj{1}$, one only really needs to
understand the twisting sheaves and thickened skyscraper sheaves.
