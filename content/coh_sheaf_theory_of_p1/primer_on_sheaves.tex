\section{A primer on sheaves}
\label{sect_sheaf_primer}

This section is intended to provide a brief overview of
\emph{sheaves} in the abstract with a view toward algebraic geometry.
We will strictly consider generalities here, but our needs for the
more specific and forthcoming study of coherent sheaves can be met
with little more than a few basic and well-motivated definitions.

Recall that one of the main ways to study an algebraic variety is to
look at the rings of regular functions defined locally and globally.
An important consequence of the way the Zariski topology is
constructed is that these regular functions enjoy a certain
local-to-global relationship: two regular functions on the variety
$X$ which agree locally on an open subset $U \subseteq X$ agree globally on $X$.
% (c.f. \cite[Remark~I.3.1.1]{hartshorne}).
Regular functions are far from the only data on $X$ that behaves
similar in spirit to this.
As one of many examples, one can develop a cogent theory of
\emph{(K\"{a}hler) differentials} when $X$ is smooth which, among
other things, can be used to define the \emph{geometric genus} of
$X$, an important numerical invariant.
There is an abundance of examples of this phenomena beyond just
algebraic geometry, such as the continuous functions on any topological space.
The language of sheaves arises as a convenient organisational tool in
all of these contexts.

\begin{definition}
  \label{def_presheaf}
  Let $X$ be a topological space.
  A \emph{presheaf of abelian groups} $\shf{F}$ on $X$ is the data of
  abelian group $\shf{F}(U)$ for each open set $U$, and for each
  inclusion of opens $V \subseteq U$ a group homomorphism
  $\res{U}{V}: \shf{F}(U) \to \shf{F}(V)$.
  These homomorphisms are moreover required to be functorial in the
  sense that $\res{U}{U} = \id_{\shf{F}(U)}$ for all open sets $U$
  and $\res{U}{W} = \res{V}{W} \res{U}{V}$ for all inclusions of open
  sets $W \subseteq V \subseteq U$.
\end{definition}

Though in this definition presheaf data takes values in the category
$\Ab$, one can substitute for other concrete categories, such as
$\Set$ and $\Ring$; more usefully and most commonly, one considers an
abelian category.
The elements of $\shf{F}(U)$ are called the \emph{sections} of
$\shf{F}$ over $U$, and the sections over $X$ itself are called the
\emph{global sections}.
It is also common to denote the space of sections of $\shf{F}$ over
$U$ by $\sections{U}{\shf{F}}$ or $H^0(U, \shf{F})$.
The homomorphisms $\res{U}{V}$ are called \emph{restriction maps} due
to how they are typically defined, and for a section $s \in
\shf{F}(U)$ we introduce the suggestive shorthand $\setres{s}{V} =
\res{U}{V}(s)$.

As we have already alluded to above, the most fundamental example of
a presheaf in algebraic geometry is the following.

\begin{example}
  The regular functions of a quasi-projective variety $X$ define a
  presheaf of rings $\strshf{X}$ called the \emph{structure sheaf} of $X$.
  For each open set $U$, we let $\strshf{X}(U)$ be the ring
  $\shf{O}(U)$ of regular functions
  % $f: U \to k$,
  defined on $U$, and take as restriction maps the usual
  contravariant inclusions $\shf{O}(U) \mono \shf{O}(V)$.
\end{example}

The definition of a presheaf captures the organisational aspects of
the data of $\strshf{X}$, but we have already noted that there is
more going on in the form of local-to-global relationships.
Our next definition aims to make this more formal.

\begin{definition}
  \label{def_sheaf}
  A presheaf $\shf{F}$ on $X$ is called a \emph{sheaf} if for any
  open set $U$ and open cover $U = \bigcup_{i \in I} U_i$, the
  following two conditions are satisfied:
  \begin{enumerate}
    \item
      \emph{(Identity.)}
      Given $s \in \shf{F}(U)$ such that $\setres{s}{U_i} = 0$ for
      all $i \in I$, we have $s = 0$.

    \item
      \emph{(Gluability.)}
      Given sections $s_i \in \shf{F}(U_i)$ such that
      \[
        \setres{s_i}{U_i \cap U_j} = \setres{s_j}{U_i \cap U_j}
      \]
      for all $i, j \in I$, there is a section $s \in \shf{F}(U)$
      such that $s_i = \setres{s}{U_i}$ for all $i \in I$.
  \end{enumerate}
\end{definition}

% In other words, a sheaf is a presheaf where families of local
% sections glue together \emph{uniquely} on open covers.
% The observant reader may feel shortchanged by this, since it is not
% as strong as what happens for regular functions in $\strshf{X}$,
% where local agreement immediately becomes global agreement.
% This is entirely to be expected, however: the Zariski topology is
% rather coarse and any variety $X$ is by definition irreducible,
% which are both topologically restrictive conditions.

\begin{example}
  As the name implies, the structure sheaf $\strshf{X}$ is also a sheaf.
  Aside from the importance of regular functions in general, what
  makes $\strshf{X}$ so fundamental is that many other useful sheaves
  occurring in algebraic geometry are `compatible' with it.
  More precisely, we call a sheaf $\shf{F}$ an
  \emph{$\strshf{X}$-module} if each $\shf{F}(U)$ is a module over
  $\strshf{X}(U)$, and for each inclusion of open sets $V \subseteq U
  \subseteq X$ we have
  \[
    \setres{(s + t)}{V} = \setres{s}{V} + \setres{t}{V}
    \mathand
    \setres{(f \cdot s)}{V} = \setres{f}{V} \cdot \setres{s}{V}
  \]
  for all $s, t \in \shf{F}(U)$ and $f \in \strshf{X}(U)$.
  Naturally, $\strshf{X}$ is itself an $\strshf{X}$-module.
\end{example}

With the following notion of morphisms, one obtains the pair of
categories $\Psh{X}$ and $\Sh{X}$ of all (pre)sheaves of abelian
groups on some fixed space $X$.

\begin{definition}
  \label{def_general_sheaf_morphism}
  Let $\shf{F}$ and $\shf{G}$ be (pre)sheaves of abelian groups on $X$.
  A \emph{morphism of (pre)sheaves} $f: \shf{F} \to \shf{G}$ is the
  data of group homomorphism $f_U: \shf{F}(U) \to \shf{G}(U)$ for
  each open $U \subseteq X$ compatible with restriction, i.e. there
  is a commutative diagram
  \[
    \begin{tikzcd}[cramped]
      {\shf{F}(U)} & {\shf{G}(U)} \\
      {\shf{F}(V)} & {\shf{G}(V)}
      \arrow["{f_U}", from=1-1, to=1-2]
      \arrow["{\setres{-}{V}}"', from=1-1, to=2-1]
      \arrow["{\setres{-}{V}}", from=1-2, to=2-2]
      \arrow["{f_V}"', from=2-1, to=2-2]
    \end{tikzcd}
  \]
  for each inclusion of opens $V \subseteq U \subseteq X$.
  The composition of morphisms is defined in the obvious way.
  We say that $f$ is an \emph{isomorphism} if there is a morphism $g:
  \shf{G} \to \shf{F}$ such that $fg = \id_{\shf{G}}$ and $gf = \id_{\shf{F}}$.
\end{definition}

Finally, we define a basic but useful operation on sheaves that we
will need later.

\begin{definition}
  Let $\shf{F}$ be a sheaf on $X$.
  For each open $U \subseteq X$, the \emph{restricted sheaf}
  $\setres{\shf{F}}{U}$ is the sheaf on $U$ such that
  $(\setres{\shf{F}}{U})(V) = \shf{F}(V)$ for each open $V \subseteq
  U$, and $(\setres{\rho}{U})_{V, W} = \res{V}{W}$ for each inclusion
  of opens $W \subseteq V \subseteq U$.
\end{definition}
