\section{Operations on coherent sheaves}

Since we have explicitly defined quasi-coherent sheaves in terms of
module data, one should intuitively expect that some of the structure
and operations on the category of modules will carry over to
$\QCoh(\proj{1})$ and $\Coh(\proj{1})$.
This indeed ends up being the case, so the goal of this section is to
explicitly spell out the most important aspects.

A key technical detail that we will implicitly use throughout this
discussion is that localisation is exact, so it preserves subobjects,
quotients, direct sums, kernels and cokernels at the level of modules
(c.f. \cref{rem_exact_limit_colimit_pres}).
In particular, we can use this fact to describe the descent data of
the sheaves in the constructions below in a way that is more suited
to computations.
This will also lead to many of the definitions we give in this
section being quite straightforward, such as the following.

\begin{definition}
  Let $\shf{F} = (M, M', \vartheta)$ and $\shf{G} = (N, N', \varrho)$
  be quasi-coherent sheaves on $\proj{1}$.
  The \emph{direct sum} of $\shf{F}$ and $\shf{G}$ is the quasi-coherent sheaf
  \[
    \shf{F} \biprod \shf{G} := (M \biprod N, M' \biprod N', \vartheta
    \biprod \varrho).
  \]
  \vspace{-24pt}
\end{definition}

Here, we have used the fact that $(M \biprod N)_f \cong M_f \biprod
N_f$ to simplify the descent datum.
It is easy to see that $\QCoh(\proj{1})$ and $\Coh(\proj{1})$ are
additive categories, since there is an obvious way to bilinearly add
morphisms and an even more obvious candidate for the zero object.
We leave the task of checking these small technical details to the reader.

Next, we treat the abelian structure.
Retaining the notation of \cref{def_coh_sheaf_morphism}, a
quasi-coherent sheaf morphism $\Phi: \shf{F} \to \shf{G}$ on
$\proj{1}$ gives rise to a commutative diagram
\[
  \begin{tikzcd}
    {\ker(\varphi)_z} & {M_z} & {N_z} & {\coker(\varphi)_z} \\
    {\ker(\varphi')_{z^{-1}}} & {M'_{z^{-1}}} & {N'_{z^{-1}}} &
    {\coker(\varphi')_{z^{-1}}}.
    \arrow[hook, from=1-1, to=1-2]
    \arrow["{\varphi_z}", from=1-2, to=1-3]
    \arrow["\vartheta"', from=1-2, to=2-2]
    \arrow[two heads, from=1-3, to=1-4]
    \arrow["\varrho", from=1-3, to=2-3]
    \arrow[hook, from=2-1, to=2-2]
    \arrow["{\varphi'_{z^{-1}}}"', from=2-2, to=2-3]
    \arrow[two heads, from=2-3, to=2-4]
  \end{tikzcd}
\]
The commutativity of the middle square implies that the composites
\[
  \varphi'_{z^{-1}} \circ \vartheta \circ \ker(\varphi)_z = \varrho
  \circ \varphi_z \circ \ker(\varphi)_z
\]
% \[
%     \varphi'(\vartheta(m/z^i))
%     = \varrho(\varphi(m/z^i))
%     % = \varrho(\varphi(m)/z^i)
%     = 0
% \]
% for any $m \in \ker(\varphi)$,
are zero, so the restriction of $\vartheta$ to $\ker(\varphi)_z$ is a
map into ${\ker(\varphi')_{z^{-1}}}$.

\begin{definition}
  The \emph{kernel} of a quasi-coherent sheaf morphism $\Phi: \shf{F}
  \to \shf{G}$ on $\proj{1}$ is the quasi-coherent subsheaf
  $\ker(\Phi) := (\ker(\varphi), \ker(\varphi'), \vartheta) \leq \shf{F}$.
  Here, we introduce a slight abuse of notation by identifying
  $\vartheta$ with the restriction $\setres{\vartheta}{\ker(\varphi)_z}$.
  One defines $\coker(\Phi)$ using the above commutative diagram
  similarly, and hence also $\im(\Phi)$.
\end{definition}

Since $k[z]$ and $k[z^{-1}]$ are Noetherian rings, we see that
kernels and cokernels of coherent sheaf morphisms must in turn be coherent.
As before, we leave it to the reader to check the category-theoretic
details necessary for the following claim.

\begin{proposition}
  The categories $\QCoh(\proj{1})$ and $\Coh(\proj{1})$ are abelian.
\end{proposition}

With this, we may consider all of the usual notions for an abelian category.
We gave subobjects and quotient objects more palatable, explicit
definitions in \cref{def_qcoh_sub_and_quot_objs}.
By \cref{lemma_abelian_category_monos_and_epis}, the monics, epics
and isomorphisms in $\QCoh(\proj{1})$ are also explicit: we call the
morphism $\Phi = (\varphi, \varphi)$ \emph{injective} if $\varphi$
and $\varphi'$ are injective module homomorphisms. \emph{Surjective}
and \emph{bijective} morphisms are defined similarly.
To see these tools in action, we consider the following result that
will be useful in the next section.

\begin{lemma}
  \label{prop_ideal_sheaf_twist}
  For any $m \in \zpos$ and $p \in \proj{1}$, we have $\twist{-mp}
  \cong \twist{-m}$.
\end{lemma}

\begin{proof}
  Let $p = (a : b)$, and consider the maps $\varphi: k[z] \to k[z]$
  and $\varphi': k[z^{-1}] \to k[z^{-1}]$ given by multiplication by
  $(b - az)^m$ and $(bz^{-1} - a)^m$ respectively.
  As one easily checks, this determines an injective sheaf morphism
  $\twist{-m} \to \shf{O}$ whose image is $\twist{-mp}$.
\end{proof}

Of course, the most significant tool in an abelian category are short
exact sequences, so we consider two important examples for coherent sheaves.

\begin{example}
  There is a short exact sequence of coherent sheaves
  \[
    \begin{tikzcd}[cramped]
      0 & {\twist{-D}} & {\shf{O}} & {\skyscraper{D}} & 0
      \arrow[from=1-1, to=1-2]
      \arrow[from=1-2, to=1-3]
      \arrow[from=1-3, to=1-4]
      \arrow[from=1-4, to=1-5]
    \end{tikzcd}
  \]
  associated to any closed $D \subseteq \proj{1}$, sometimes called
  the \emph{ideal sheaf sequence}.
  This is a geometric analogue of the short exact sequence of $R$-modules
  \[
    \begin{tikzcd}[cramped]
      0 & I & R & {R/I} & 0
      \arrow[from=1-1, to=1-2]
      \arrow[from=1-2, to=1-3]
      \arrow[from=1-3, to=1-4]
      \arrow[from=1-4, to=1-5]
    \end{tikzcd}
  \]
  associated to any ideal $I \idealof R$.
\end{example}

\begin{proposition}[Euler sequence for $\proj{1}$]
  There is a short exact sequence
  \[
    \begin{tikzcd}[cramped]
      0 & {\twist{-2}} & {\twist{-1}^{\biprod 2}} & {\shf{O}} & 0.
      \arrow[from=1-1, to=1-2]
      \arrow[from=1-2, to=1-3]
      \arrow[from=1-3, to=1-4]
      \arrow[from=1-4, to=1-5]
    \end{tikzcd}
  \]
  \vspace{-24pt}
\end{proposition}

\begin{proof}
  As is easy to check, the pairs $(\times 1, \times z^{-1})$ and
  $(\times z, \times 1)$ both determine morphisms $\twist{-1} \to \shf{O}$.
  This induces maps $\varphi: k[z]^{\biprod 2} \to k[z]$ and
  $\varphi': k[z^{-1}]^{\biprod 2} \to k[z^{-1}]$ given explicitly by
  $\varphi(f, g) = f + zg$ and $\varphi'(f, g) = z^{-1}f + g$.
  % \begin{align*}
  %     \varphi(f, g) &= (1, z) \cdot (f, g) & \varphi'(f, g) &=
  % (z^{-1}, 1) \cdot (f, g) \\
  %     &= f + zg, & &= z^{-1}f + g.
  % \end{align*}
  These assemble to determine a morphism $\Phi: \twist{-1}^{\biprod
  2} \to \shf{O}$, since the diagram
  \[
    \begin{tikzcd}[cramped]
      {k[z, z^{-1}]^{\biprod2}} & {k[z, z^{-1}]} \\
      {k[z, z^{-1}]^{\biprod2}} & {k[z, z^{-1}]}
      \arrow["{\varphi_z}", from=1-1, to=1-2]
      \arrow["{\times z}"', from=1-1, to=2-1]
      \arrow[Rightarrow, no head, from=1-2, to=2-2]
      \arrow["{\varphi'_{z^{-1}}}"', from=2-1, to=2-2]
    \end{tikzcd}
  \]
  commutes.
  Both $\varphi$ and $\varphi'$ are surjective module homomorphisms,
  and it follows that $\Phi$ is surjective.
  To obtain the Euler exact sequence, it therefore suffices to show
  that $\twist{-2} \cong \ker(\Phi)$.
  The map $\psi: 1 \mapsto (-z, 1)$ induces a $k[z]$-module isomorphism
  \[
    k[z] \cong \{(-zg, g): g \in k[z]\} = \ker(\varphi),
  \]
  and similarly the map $\psi': 1 \mapsto -(1, -z^{-1})$ induces a
  $k[z^{-1}]$-module isomorphism
  \[
    k[z^{-1}] \cong \{(f, -z^{-1}f): f \in k[z^{-1}]\} = \ker(\varphi').
  \]
  The introduction of the sign in $\psi'$ ensures that the diagram
  \[
    \begin{tikzcd}[cramped]
      {k[z, z^{-1}]} & {\ker(\varphi)[z^{-1}]} \\
      {k[z, z^{-1}]} & {\ker(\varphi')[z]}
      \arrow["{\psi_z}", tail reversed, from=1-1, to=1-2]
      \arrow["{\times z^2}"', from=1-1, to=2-1]
      \arrow["{\times z}", from=1-2, to=2-2]
      \arrow["{\psi'_{z^{-1}}}"', tail reversed, from=2-1, to=2-2]
    \end{tikzcd}
  \]
  commutes both ways, and so the pair $(\psi, \psi')$ is the desired
  sheaf isomorphism.
\end{proof}

Since the tensor product of two modules produces another module over
commutative rings, we can also reasonably expect to have an
equivalent notion for quasi-coherent sheaves on $\proj{1}$.
We once again base our definition on the $R_f$-module isomorphism $(M
\tensor_R N)_f \cong M_f \tensor_{R_f} N_f$.

\begin{definition}
  Given quasi-coherent sheaves $\shf{F} = (M, M', \vartheta)$ and
  $\shf{G} = (N, N', \varrho)$ on $\proj{1}$, the \emph{tensor
  product} of $\shf{F}$ and $\shf{G}$ is the quasi-coherent sheaf
  \[
    \shf{F} \shftensor \shf{G}
    :=
    (M \tensor_{k[z]} N, M' \tensor_{k[z^{-1}]} N', \vartheta
    \tensor_{k[z, z^{-1}]} \varrho).
  \]
  \vspace{-24pt}
\end{definition}

Just as it does for modules, $\shftensor$ defines a bifunctor on
$\QCoh(\proj{1})$ which is right exact in both arguments, though we
will not explicitly verify this claim.
As is a recurring theme at this point, tensor products involving
twisting sheaves are easy to describe.

\begin{proposition}
  \label{prop_shftensor_of_twists}
  For $m, n \in \bb{Z}$ we have $\twist{m} \shftensor \twist{n} \cong
  \twist{m + n}$.
\end{proposition}

\begin{proof}
  The most crucial step is to understand why the sheaf $\twist{m +
  n}$ appears in this isomorphism.
  But this is just an easy consequence of the algebra of tensors, since
  \[
    z^{-m} f \tensor_{k[z, z^{-1}]} z^{-n} g = f \tensor_{k[z,
    z^{-1}]} z^{-(m + n)} g
  \]
  for any $f, g \in k[z, z^{-1}]$, i.e. the descent datum of
  $\twist{m} \shftensor \twist{n}$ and the map $1 \tensor
  \vartheta_{m + n}$ are the same.
  The proof is now clear, as one simply lifts the $R$-module
  isomorphism $R \tensor_R R \cong R$ given by $r \tensor s \mapsto
  rs$ to an appropriate coherent sheaf isomorphism.
\end{proof}

\iffalse
\begin{corollary}
  When $n \geq 0$, we have $\twist{n} = \twist{1}^{\tensor n}$ and
  $\twist{-n} = \twist{-1}^{\tensor n}$.
\end{corollary}
\fi

Our interest in the interaction between $\shftensor$ and these
twisting sheaves stems from the following important class of
quasi-coherent sheaves motivated by module theory.

\begin{definition}
  A quasi-coherent sheaf $\shf{F} = (M, M', \vartheta)$ on $\proj{1}$
  is called \emph{flat} if $M$ and $M'$ are flat modules over $k[z]$
  and $k[z^{-1}]$ respectively.
\end{definition}

The flat modules are equivalently those for which the tensor product
of modules is an exact functor, and we can check that the analogous
statement naturally holds for a flat quasi-coherent sheaf on
$\proj{1}$ in our setup.

\begin{proposition}
  If $\shf{G} = (N, N', \varrho)$ is a flat quasi-coherent sheaf on
  $\proj{1}$, then the functor $- \shftensor \shf{G}: \QCoh(\proj{1})
  \to \QCoh(\proj{1})$ is exact.
\end{proposition}

\begin{proof}
  Consider a short sequence of sheaves
  \[
    \begin{tikzcd}[cramped]
      0 & {\shf{F}_1} & {\shf{F}_2} & {\shf{F}_3} & 0
      \arrow[from=1-1, to=1-2]
      \arrow["{\Phi_1}", from=1-2, to=1-3]
      \arrow["{\Phi_2}", from=1-3, to=1-4]
      \arrow[from=1-4, to=1-5]
    \end{tikzcd}
  \]
  with $\Phi_1 = (\varphi_1, \varphi_1')$ and $\Phi_2 = (\varphi_2,
  \varphi_2')$.
  It is matter of checking definitions to see that this sequence is
  exact if and only if the short sequences of modules
  \[
    \begin{tikzcd}[cramped]
      0 & {M_1} & {M_2} & {M_3} & 0, \\
      0 & {M_1'} & {M_2'} & {M_3'} & 0
      \arrow[from=1-1, to=1-2]
      \arrow["{\varphi_1}", from=1-2, to=1-3]
      \arrow["{\varphi_2}", from=1-3, to=1-4]
      \arrow[from=1-4, to=1-5]
      \arrow[from=2-1, to=2-2]
      \arrow["{\varphi_1'}"', from=2-2, to=2-3]
      \arrow["{\varphi_1'}"', from=2-3, to=2-4]
      \arrow[from=2-4, to=2-5]
    \end{tikzcd}
  \]
  are exact over $k[z]$ and $k[z^{-1}]$ respectively.
  Since $N$ and $N'$ are flat modules, the claim is now immediate
  upon tensoring the above short exact sequence of sheaves by $\shf{G}$.
\end{proof}

The most basic class of flat sheaves are the twisting sheaves, and in
light of this, we define a new operation on quasi-coherent sheaves.

\begin{definition}
  Let $n \in \bb{Z}$ and define
  \[
    \shf{F}(n) := \shf{F} \shftensor \twist{n}
  \]
  for any quasi-coherent sheaf $\shf{F}$ on $\proj{1}$.
  We call $\shf{F}(n)$ the \emph{degree $n$ twist} of $\shf{F}$, and
  it induces an exact \emph{twisting functor} $(n): \QCoh(\proj{1})
  \to \QCoh(\proj{1})$.
\end{definition}

\begin{proposition}
  \label{prop_twist_is_autoequiv}
  Twisting is an autoequivalence of $\QCoh(\proj{1})$.
\end{proposition}

\begin{proof}
  Let $n \in \bb{Z}$.
  The key observation is that by \cref{prop_shftensor_of_twists}, we have
  \begin{align*}
    \shf{F}(-n)(n)
    = (\shf{F} \shftensor \twist{-n}) \shftensor \twist{n}
    \cong \shf{F} \shftensor \shf{O}
    = \shf{F}(0),
  \end{align*}
  and there is an obvious isomorphism $\shf{F}(0) \cong \shf{F}$.
  Similarly, $\shf{F}(n)(-n) \cong \shf{F}$.
  From these facts, it is not difficult to see that the functors
  $(n)$ and $(-n)$ must be naturally inverse to each other, so the
  degree $n$ twist is an equivalence of categories.
\end{proof}

% \begin{lemma}
%     Let $m \in \zpos$ and $p \in \proj{1}$.
%     Then for any $n \in \bb{Z}$, we have $\skyscraper{mp}(n) \cong
% \skyscraper{mp}$.
% \end{lemma}

% \begin{proof}

% \end{proof}

\iffalse
Defining a sheaf version of the Hom of two quasi-coherent sheaves on
$\proj{1}$ is more complicated, since the natural injective map of $R_f$-modules
\[
  \Hom_R(M, N)_f \mono \Hom_{R_f}(M_f, N_f)
\]
need not be surjective.
This is an isomorphism whenever $M$ is finitely presented, and over
Noetherian rings this is equivalent to $M$ being finitely generated.
To avoid these technical concerns, we will therefore only define the
sheaf $\shfhom(\shf{F}, \shf{G})$ in the case where $\shf{F}$ is
coherent, even though this assumption is a bit stronger than needed.

\begin{definition}
  Given quasi-coherent sheaves $\shf{F} = (M, M', \vartheta)$ and
  $\shf{G} = (N, N', \varrho)$ on $\proj{1}$, the \emph{sheaf Hom} of
  $\shf{F}$ and $\shf{G}$ is the quasi-coherent sheaf
  \[
    \shfhom(\shf{F}, \shf{G})
    :=
    (\Hom_{k[z]}(M, N), \Hom_{k[z^{-1}]}(M', N'), [\vartheta, \varrho]),
  \]
  where $[\vartheta, \varrho]$ is the isomorphism sending $f: M_z \to
  N_z$ to $\varrho \circ f \circ \vartheta^{-1}: M'_{z^{-1}} \to N'_{z^{-1}}$.
\end{definition}

Once again, we note without proof that this is a bifunctor on
$\QCoh(\proj{1})$ which is contravariant left exact in its first
argument and covariant left exact in the other.
By construction, the global sections of sheaf Hom also recover the
usual Hom, i.e.
\[
  \sections{\proj{1}}{\shfhom(\shf{F}, \shf{G})} = \Hom(\shf{F}, \shf{G}).
\]
A convenient use for the sheaf Hom is in giving a notion of duality.

\begin{definition}
  For any quasi-coherent sheaf $\shf{F}$ on $\proj{1}$, we define the
  \emph{dual sheaf} of $\shf{F}$ to be $\dual{\shf{F}} :=
  \shfhom(\shf{F}, \shf{O})$.
\end{definition}

This is a sheaf-theoretic incarnation of the dual of an $R$-module
$\dual{M} = \Hom_R(M, R)$, and indeed it is not difficult to see that
the $R$-module isomorphism
\[
  \dual{M} \tensor_R M \cong R, \qquad f \tensor m \mapsto f(m)
\]
lifts to a quasi-coherent sheaf isomorphism $\dual{\shf{F}}
\shftensor \shf{F} \cong \shf{O}$.

\iffalse
The sheaf Hom and tensor product give additional structure to $\Pic$,
the set of isomorphism classes of line bundles on $\proj{1}$, and is
of relevance to the theory of \emph{divisors} on a variety.
It is clear that the tensor product of line bundles is a line bundle,
and also that $\shf{L} \shftensor \shf{O} \cong \shf{L}$ for any line
bundle $\shf{L}$.
We define the \emph{dual bundle} of $\shf{L}$ by $\dual{\shf{L}} :=
\shfhom(\shf{L}, \shf{O})$, and one can check that the $R$-module isomorphism
\[
  \Hom_R(M, R) \tensor_R M \cong R, \qquad f \tensor m \mapsto f(m)
\]
lifts to an isomorphism $\dual{\shf{L}} \shftensor \shf{L} \cong \shf{O}$.
Thus $\Pic$ is an abelian group under $\shftensor$ with identity
element $\shf{O}$, called the \emph{Picard group} of $\proj{1}$.
The dual of a line bundle is by construction its inverse in this
group, which is why line bundles are also called invertible sheaves.
By \cref{prop_shftensor_of_twists}, we have $\dual{\twist{n}} =
\twist{-n}$, and it turns out that $\Pic$ is generated by the line
bundle $\twist{1}$, i.e. $\Pic \cong \bb{Z}$.
This is a non-trivial statement, and is proven in a more formal
setting in \cite[Proposition~II.6.17]{hartshorne}.
\fi

\fi
