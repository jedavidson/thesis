\section{Modules as sheaves on $\aff{1}$}

As an important precursor to coherent sheaves, we show in this
section that any module over $\shf{O}(\aff{1}) \cong k[x]$ can be
suitably interpreted as a sheaf on $\aff{1}$.
This also gives an explicit algebro-geometric interpretation of
modules and localisation.

We start with some preliminary results concerning open sets in $\aff{1}$.
Recall that a \emph{principal open subset} of an affine variety $X$
is an open subset of the form
\[
  D(f) := X \setminus V(f) = \{x \in X: f(x) \neq 0\}
\]
for some $f \in k[X]$.
The collection of all principal opens forms a base for the Zariski
topology on $X$, but even stronger statements can be made about the
topology on $\aff{1}$.

\begin{lemma}
  \label{lemma_opens_of_aff_line_are_principal}
  Every open subset of $\aff{1}$ is principal.
\end{lemma}

\begin{proof}
  If $U$ is open, then $U = \aff{1} \setminus V(I)$ for some ideal $I
  \idealof k[x]$.
  But $k[x]$ is a principal ideal domain, so $I = \gen{f}$ for some
  $f \in k[x]$ and hence $U = D(f)$.
\end{proof}

\begin{lemma}
  \label{lemma_principal_open_inclusions_of_aff_line}
  Let $f \in k[x]$, and consider the principal open set $D(f)$ in $\aff{1}$.
  Then every principal open set within $D(f)$ is of the form $D(fg)$
  for some $g \in k[x]$.
  Moreover, the intersection of principal opens $D(f)$ and $D(g)$ is $D(fg)$.
\end{lemma}

\begin{proof}
  If $D(h) \subseteq D(f)$, then $V(f) \subseteq V(h)$ and hence
  $\gen{h} \subseteq \gen{f}$ by standard properties of affine
  varieties, which implies that $f$ divides $h$.
  The second claim is equivalent to the fact that $V(f) \cup V(g) = V(fg)$.
\end{proof}

The moral of these results is that to specify a sheaf on $\aff{1}$,
it is entirely sufficient to consider only principal open sets
determined by some polynomial.
Intuitively, the same conclusions ought to hold mutatis mutandis for
any open $U \subseteq \aff{1}$.
To make this idea more formal, if $U = D(f)$ for some $f \in k[x]$,
then by \cite[Proposition~6.3.5]{fulton} we can take the definition
of the ring of regular functions for $U$ to be the localisation
\[
  \shf{O}(U) := \shf{O}(\aff{1})_f \cong k[x, f^{-1}],
\]
and it is not hard to see now that
\cref{lemma_principal_open_inclusions_of_aff_line} holds with
$\aff{1}$ replaced by $U$ and $k[x]$ replaced by $\shf{O}(U)$.
For reasons that will become clear in due course, we will broaden the
scope of the following definition to any open subset of $\aff{1}$
rather than just $\aff{1}$ itself.
% Next, recall that a \emph{quasi-affine} variety $Y$ is an open
% subset of an affine variety $X$.
% Naturally, we will exclusively consider the case $X = \aff{1}$.
% By \cref{lemma_opens_of_aff_line_are_principal}, any quasi-affine
% variety $Y$ in $\aff{1}$ is principal in $\aff{1}$, so $Y = D(f)$
% for some $f \in k[x]$.
% Moreover, by \cite[Proposition~6.3.5]{fulton}, we can define a ring
% of regular functions for any such $Y$ as
% \[
%     \shf{O}(U) := \shf{O}(\aff{1})[f^{-1}] \cong k[x, f^{-1}].
% \]
% Finally, it is clear that the conclusions of
% \cref{lemma_principal_open_inclusions_of_aff_line} hold mutatis
% mutandis for $Y$.
% The need to broaden our scope to quasi-affine varieties rather than
% solely $\aff{1}$ is so that we may directly verify that the
% following construction gives a sheaf of modules.

\begin{definition}
  \label{def_assmod}
  Let $U$ be an open subset of $\aff{1}$ and $M$ a module over the
  ring $\shf{O}(U)$.
  We define a presheaf of abelian groups $\assmod{M}$ on $U$, called
  the \emph{sheaf associated to $M$}, by declaring that the sections
  over each principal open $D(f) \subseteq U$ are
  $\sections{D(f)}{\assmod{M}} = M_f$, and equipping $\assmod{M}$
  with restriction maps
  \[
    \res{f}{fg}: M_f \to M_{fg},
    \qquad \frac{m}{f^r} \mapsto \frac{mg^r}{(fg)^r}.
  \]
\end{definition}

Some a posteriori justification for this definition is as follows.
Algebraically, $D(f) \subseteq U$ is the region in which $f$ is
non-zero and hence invertible in $\shf{O}(U)$, so to specify local
sections it is natural to consider the \emph{extension of scalars}
\[
  \shf{O}(U)_f \tensor_{\shf{O}(U)} M \cong M_f.
\]
Also, while the sheaf $\assmod{M}$ only considers this space of
sections as an abelian group, each $M_f$ has by definition the
structure of an $\shf{O}(U)_f$-module.

\begin{example}
  The structure sheaf $\strshf{\aff{1}}$ is the associated sheaf of
  the $k[x]$-module $k[x]$.
  By our remarks above, it follows that every associated sheaf
  $\assmod{M}$ on $\aff{1}$ is also an example of an $\strshf{\aff{1}}$-module.
\end{example}

\begin{example}
  \label{exmp_skyscraper_shf}
  Let $a \in k$, and consider the $k[x]$-module $M = \quot{k[x]}{\gen{x - a}}$.
  If $U = D(f)$ is an open set in $\aff{1}$, then it follows from
  polynomial division that $f = f(a)$ in $M$.
  If $a \in U$, then $f(a)$ is non-zero in $k$ and hence invertible
  in $M$, so we have $M_f \cong M \cong k$ as $k[x]$-modules.
  If instead $a \not\in U$, then $f(a) = 0$ in $k$, and so $M_f$ must
  be the zero module.
  We say that the sheaf $\assmod{M}$ is \emph{supported} at the point
  $x = a$ in the sense that one only obtains non-zero sections over
  open neighbourhoods of $a$.
  For this reason, we call $\assmod{M}$ a \emph{skyscraper sheaf} on $\aff{1}$.
\end{example}

Let us now check that the presheaf $\assmod{M}$ is deserving of being
called a sheaf.
We will adapt the proofs of a related result for affine schemes in
\cite[Proposition~I.18]{eisenbud_and_harris} and
\cite[Theorem~4.1.2]{vakil}, originally due to Serre.
By virtue of the fact that we are working within $\aff{1}$, we will
be allowed to sidestep some sheaf-theoretic technicalities that one
encounters in this more general setting.
The following lemma is key.

\begin{lemma}
  \label{lemma_quasiaff_is_quasicompact}
  Every open cover of an open subset of $\aff{1}$ has a finite subcover.
\end{lemma}

\begin{proof}
  In light of our observations above, it suffices to consider a cover
  by principal opens $U = \bigcup_i D(f_i)$, with $f_i \in \shf{O}(U)$.
  It is immediate that the zero locus of the ideal $\ideal{a}$
  generated by the $f_i$ must be empty, i.e. $\mathfrak{a} = \shf{O}(U)$.
  Each element of $\ideal{a}$ is by definition a finite linear
  combination of the $f_i$, so in particular there is a
  \emph{partition of unity} $\sum_{i \in I} a_i f_i = 1$ where all
  but finitely many of the $a_i \in \shf{O}(U)$ are non-zero.
  We may thus take as a finite subcover the sets $D(f_i)$ such that
  $a_i \neq 0$.
\end{proof}

% \begin{remark}
%     In more algebro-geometric language,
% \cref{lemma_quasiaff_is_quasicompact} says that $U$ is \emph{quasicompact}.
%     This is terminologically distinguished from compactness by not
% insisting that the space is also Hausdorff, which is rarely ever
% true of the Zariski topology (and never when $k$ is infinite, i.e.
% when $k$ is algebraically closed).
% \end{remark}

\begin{proposition}
  \label{prop_shf_assoc_to_mod}
  The presheaf $\assmod{M}$ in \cref{def_assmod} is a sheaf on any
  open $U \subseteq \aff{1}$.
\end{proposition}

\begin{proof}
  Let $V \subseteq U$ be open.
  Again, in light of the previous observations, we may write $V =
  D(f)$ for some $f \in \shf{O}(U)$ and consider an arbitrary open
  cover $V = \bigcup_{i \in I} D(f_i)$ for some $f_i \in \shf{O}(U)$.
  We claim that is that it is in fact enough to show that the sheaf
  property holds for global sections, i.e. in the case $V = U$.
  To see why, note that $V$ is open in $\aff{1}$, so by
  \cref{lemma_principal_open_inclusions_of_aff_line} we have
  \[
    V
    = \bigcup_{i \in I} D(f_i)
    = \bigcup_{i \in I} V \cap D(f_i)
    = \bigcup_{i \in I} D(ff_i).
  \]
  Via the inclusion of rings $\shf{O}(U) \mono \shf{O}(V)$, each
  function $f f_i$ may be regarded as a function in $\shf{O}(V)$.
  That is, the open cover of $V$ above can be surreptitiously
  rewritten as an open cover $V = \bigcup_{i \in I} D(f_i')$ for some
  $f_i' \in \shf{O}(V)$, which shows that the general case can indeed
  be reduced to this special case.
  Thus for the remainder of the proof, we may assume that $V = U$ so
  that $\assmod{M}(V) \cong M$.
  Before we begin the proof proper, we will also fix a finite
  subcover $U = \bigcup_{\alpha \in A} D(f_\alpha)$ by
  \cref{lemma_quasiaff_is_quasicompact}.

  To start, let us check that the identity property holds.
  Suppose that the global section $s \in M$ vanishes upon restriction
  in $M_{f_i}$ for all $i \in I$.
  Then $s$ is annihilated in $M$ by some power of each $f_\alpha$ on
  the finite subcover.
  By finiteness, we may choose a sufficiently large power $N$ such
  that every $f_\alpha^N$ annihilates $s$.
  It follows that the ideal $\ideal{b}$ generated by the $f_\alpha^N$
  annihilates $s$.
  But via the proof of \cref{lemma_quasiaff_is_quasicompact}, the
  ideal $\ideal{a}$ generated by the $f_\alpha$ is $\shf{O}(U)$, and
  it is not hard to see that $\shf{O}(U) = \ideal{a}^{2N} \subseteq
  \ideal{b}$, so we conclude that $s$ is annihilated by the entire
  ring $\shf{O}(U)$.
  Hence $s = 0$ in $M$.

  Next, we check gluability.
  Suppose we are given sections $s_i \in M_{f_i}$ such that $s_i =
  s_j$ in $M_{f_i f_j}$ for all $i, j \in I$.
  As before, this certainly descends to our finite subcover, so let
  us first prove it on that.
  We can write each section as $s_\alpha =
  m_\alpha/f_\alpha^{r_\alpha}$ for some $m_\alpha \in M$ and integer
  $r_{\alpha}$.
  Since $D(f_\alpha) = D(f_\alpha^{r_\alpha})$, for ease of notation
  we will let $g_\alpha = f_\alpha^{r_\alpha}$ such that each
  $s_\alpha$ is naturally an element of $M_{g_{\alpha}}$.
  By clearing denominators in $M_{g_\alpha g_\beta}$, we may appeal
  once more to finiteness to choose a sufficiently large power $N$ such that
  \[
    (g_\alpha g_\beta)^N(g_\beta m_\alpha - g_\alpha m_\beta) = 0
  \]
  in $M$ for every $\alpha, \beta \in A$.
  For further ease of notation, write $n_\alpha = g_\alpha^N
  m_\alpha$ and $h_\alpha = g_\alpha^{N + 1}$ so that the above
  condition reads $h_\beta n_\alpha = h_\alpha n_\beta$.
  Moreover, since $D(h_\alpha) = D(g_\alpha)$, it follows as in
  \cref{lemma_quasiaff_is_quasicompact} that there is a partition of
  unity $\sum_{\alpha \in A} a_\alpha h_\alpha = 1$ for some
  $a_\alpha \in \shf{O}(U)$.
  Now, considering the global section $s = \sum_{\alpha \in A}
  a_\alpha n_\alpha$, we see that
  \[
    s h_\alpha
    = \sum_{\beta \in A} a_\beta h_\alpha n_\beta
    = \sum_{\beta \in A} a_\beta h_\beta n_\alpha
    = n_\alpha
  \]
  in $M$ for each $\alpha \in A$.
  It is not difficult to see that $M_{h_\alpha} \cong M_{f_\alpha}$,
  and by construction $n_\alpha/h_\alpha = s_\alpha$ in
  $M_{h_\alpha}$, so we conclude from this that $s = s_\alpha$ in
  $M_{f_\alpha}$ as needed.
  It remains to upgrade gluability to infinite open covers.
  The previous argument holds generically for any finite open cover
  of $U$, so for any choice of $i \in I \setminus A$, we may
  similarly construct a global section $s' \in M$ such that $s' =
  s_\alpha$ in each $M_{f_\alpha}$ and $s' = s_i$ in $M_{f_i}$.
  But by the identity property, we must have $s = s'$, and so we are done.
\end{proof}

We now see that the passage to open subsets of $\aff{1}$ in
\cref{def_assmod} was necessary to ensure that we may reduce the
claim to something easier to prove.
