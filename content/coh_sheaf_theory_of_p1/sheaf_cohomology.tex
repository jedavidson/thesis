\section{Coherent sheaf cohomology for $\proj{1}$}
\label{sect_coh_sheaf_cohom}

To conclude this chapter, we discuss some aspects of \emph{coherent
sheaf cohomology} for $\proj{1}$.
In its full form, this is an indispensable technique in modern
algebraic geometry that applies even more generally than just for
quasi-coherent sheaves.
In keeping with the spirit of this chapter so far, we present only
what is needed for our goals.

As a prelude to this discussion, now seems like a fitting time to
address the issue of injectives and projectives in $\QCoh(\proj{1})$
and $\Coh(\proj{1})$, for which there is mixed news.

\begin{proposition}
  The category $\QCoh(\proj{1})$ has enough injectives, but not
  enough projectives.
  The category $\Coh(\proj{1})$ has neither enough injectives nor
  enough projectives.
\end{proposition}

We are not adequately equipped to treat these claims, so we only make
brief remarks.
The first claim is true because $\proj{1}$ is an example of a
\emph{Noetherian scheme}, and the category of quasi-coherent sheaves
on any such scheme will always have enough injectives (c.f.
\cite[Exercise~III.3.6(a)]{hartshorne}).
This does not descend to $\Coh(\proj{1})$ since coherent sheaves are
intuitively ``too small'' to be injective, in much the same way that
the injective envelope of a finitely generated module is not usually
finitely generated.
As for the lack of projectives in both categories, this can be seen
to fail already for the structure sheaf $\shf{O}_{\proj{1}}$, which
is the upshot of \cite[Exercise~III.6.2]{hartshorne}.
This will be a more significant observation for
\cref{chap_beilinsons_theorem}, but the more immediate task of
setting up coherent sheaf cohomology is contingent only on the first claim.

The motivation for this technology is to study potential problems in
lifting global sections.
It turns out that the \emph{global sections functor}
$\sections{\proj{1}}{-}: \QCoh(\proj{1}) \to \Vect{k}$ is left exact,
so given a short exact sequence of quasi-coherent sheaves
\[
  \begin{tikzcd}[cramped]
    0 & {\shf{F}_1} & {\shf{F}_2} & {\shf{F}_3} & 0
    \arrow[from=1-1, to=1-2]
    \arrow[from=1-2, to=1-3]
    \arrow[from=1-3, to=1-4]
    \arrow[from=1-4, to=1-5]
  \end{tikzcd}
\]
on $\proj{1}$, we have only the left exact sequence of vector spaces
\[
  \begin{tikzcd}[cramped]
    0 & {\sections{\proj{1}}{\shf{F}_1}} &
    {\sections{\proj{1}}{\shf{F}_2}} & {\sections{\proj{1}}{\shf{F}_3}}.
    \arrow[from=1-1, to=1-2]
    \arrow[from=1-2, to=1-3]
    \arrow[from=1-3, to=1-4]
  \end{tikzcd}
\]
The failure of the map $\sections{\proj{1}}{\shf{F}_2} \to
\sections{\proj{1}}{\shf{F}_3}$ to be surjective is exactly the
obstruction to lifting a global section from $\shf{F}_3$ to
$\shf{F}_2$, and there are two main ways to qualitatively measure this.
The more sophisticated view is given by Grothendieck in
\cite{tohoku}, who suggests that we should study the \emph{derived
functor cohomology groups}
\[
  H^i(\shf{F}) = H^i(\proj{1}, \shf{F}) := R^i
  \sections{\proj{1}}{\shf{F}} \cong \Ext{i}{}(\shf{O}, \shf{F}).
\]
The last isomorphism follows by \cref{prop_glob_sect_hom}, and it is
for this reason that one has the additional notation $H^0(X,
\shf{F})$ for global sections.
While this is certainly powerful for answering theoretical questions,
this version of sheaf cohomology is inconvenient to compute in
practice since we must explicitly construct an injective resolution
of $\shf{F}$.
Serre's original version of sheaf cohomology for algebraic geometry
as defined in \cite{fac} instead uses the easier to compute
\emph{\v{C}ech cohomology groups} $\cechalt{i}{\shf{F}}$ defined below.

A priori, these two cohomology theories may not yield isomorphic
cohomology groups.
Fortunately, they are always isomorphic for quasi-coherent sheaves on
\emph{Noetherian separated schemes} (c.f.
\cite[Theorem~III.4.5]{hartshorne}) such as $\aff{1}$ and $\proj{1}$.
In our discussion, we therefore mostly consider sheaf cohomology a la Serre.

\begin{definition}
  Fix an open cover $\mathcal{U} = \{U_i\}_{i \in I}$ of a
  topological space $X$ and a well-order on the index set $I$.
  For ease of notation, we will write the intersection of the subsets
  corresponding to the ordered $(p + 1)$-tuple $\bm{i} = (i_0, i_1,
  \ldots, i_p) \in I^{p + 1}$ as
  \[
    U_{i_0 i_1 \cdots i_p} = U_{i_0} \cap U_{i_1} \cap \cdots \cap U_{i_p},
  \]
  where $i_0 < i_1 < \cdots < i_p$.
  Then for any sheaf of abelian groups $\shf{F}$ on $X$, one defines
  the $p$th \emph{\v{C}ech cochains} to be the abelian group
  \[
    \cechcochain{p}{\mathcal{U}}{\shf{F}}
    :=
    \bigoplus_{\bm{i} \in I^{p + 1}} \sections{U_{i_0 i_1 \cdots i_p}}{\shf{F}}
  \]
  for all $p \in \zpos$.
  We define the $p$th differential map $d:
  \cechcochain{p}{\mathcal{U}}{\shf{F}} \to \cechcochain{p +
  1}{\mathcal{U}}{\shf{F}}$ by
  \begin{align*}
    d(s_{i_0 i_1 \cdots i_p})_{\bm{i} \in I^{p + 1}}
    :=
    \left(\sum_{j = 0}^{p + 1} (-1)^j \setres{s_{i_0 i_1 \cdots
      \widehat{i_j} \cdots i_{p + 1}}}{U_{i_0 i_1 \cdots \widehat{i_j}
    \cdots i_{p + 1}}}\right)_{\bm{i} \in I^{p + 2}},
  \end{align*}
  where $i_0 i_1 \cdots \widehat{i_j} \cdots i_{p + 1}$ is the list
  with the index $i_j$ omitted.
  From this, one obtains the \emph{\v{C}ech cochain complex}
  $\cechcochain{\bullet}{\mathcal{U}}{\shf{F}}$,
  \[
    \begin{tikzcd}[cramped]
      0 & {\cechcochain{0}{\mathcal{U}}{\shf{F}}} &
      {\cechcochain{1}{\mathcal{U}}{\shf{F}}} &
      {\cechcochain{2}{\mathcal{U}}{\shf{F}}} & \cdots,
      \arrow[from=1-1, to=1-2]
      \arrow["d", from=1-2, to=1-3]
      \arrow["d", from=1-3, to=1-4]
      \arrow["d", from=1-4, to=1-5]
    \end{tikzcd}
  \]
  and its cohomology are the \emph{\v{C}ech cohomology groups}
  $\cechalt{p}{\shf{F}} = \cech{p}{\mathcal{U}}{\shf{F}}$ for $p \geq 0$.
\end{definition}

Mercifully, this setup is much more tractable for $\proj{1}$.
The \v{C}ech cohomology groups are generally dependent on the choice
of open cover, but not in the aforementioned case of Noetherian
separated schemes, so we can proceed using our standard affine open
cover $\mathcal{U} = \{U_x, U_y\}$ with impunity.
For any quasi-coherent sheaf $\shf{F}$ on $\proj{1}$, we have
\[
  \cechcochain{0}{\mathcal{U}}{\shf{F}} = \sections{U_x}{\shf{F}}
  \biprod \sections{U_y}{\shf{F}}
  \mathand
  \cechcochain{1}{\mathcal{U}}{\shf{F}} = \sections{U_{xy}}{\shf{F}},
\]
and clearly $\cechcochain{p}{\mathcal{U}}{\shf{F}} = 0$ for all $p \geq 2$.
The only non-trivial differential in the \v{C}ech complex is the map
$d: \sections{U_x}{\shf{F}} \biprod \sections{U_y}{\shf{F}} \to
\sections{U_{xy}}{\shf{F}}$ given by
\[
  d(\bm{s}, \bm{t}) = \setres{\bm{s}}{U_{xy}} - \setres{\bm{t}}{U_{xy}},
\]
where $\bm{s} = (s, s')$ and $\bm{t} = (t, t')$ in the notation of
\cref{def_coh_sheaf_sects_and_restr}.
A technical point that is unique to the particular way we have set up
quasi-coherent sheaves on $\proj{1}$ is that one may need to use the
descent datum for $\shf{F}$ first to ensure that these restrictions
can be subtracted in this manner (i.e. they must be compatible with
each other first).

It follows that the only non-zero cohomology groups for $\proj{1}$ are
\[
  \cechalt{0}{\shf{F}} = \ker(d)
  \mathand
  \cechalt{1}{\shf{F}} = \coker(d),
\]
and it is not hard to see that $\cechalt{0}{\shf{F}}$ indeed
corresponds to $\sections{\proj{1}}{\shf{F}}$ by the sheaf property.
The equivalent statement about the vanishing of the higher derived
functor cohomology groups is known as \emph{Grothendieck vanishing},
and is rather non-trivial in comparison (c.f.
\cite[Proposition~20.20.7]{stacks}).
With the advanced knowledge that \v{C}ech and derived functor
cohomology agree in this situation, we will simply write
$H^0(\shf{F})$ and $H^1(\shf{F})$ for both cohomology theories.
We will also persist in calling these spaces cohomology \emph{groups}
even though they are in fact vector spaces over $k$.

The value in introducing \v{C}ech cohomology in general upfront
rather than giving only the specific setup for $\proj{1}$ is that we
see that the \v{C}ech cohomology groups are the cohomology of a
cochain complex, so all of the usual notions of
\cref{sect_cohomology_homotopy} are available.
In particular, each $H^p: \QCoh(\proj{1}) \to \Vect{k}$ is an
additive functor which commutes with direct sums, so as a consequence
of Grothendieck's splitting theorem, the cohomology groups of any
coherent sheaf on $\proj{1}$ reduces to a direct sum of the
cohomology groups of $\twist{n}$ and $\skyscraper{mp}$.
In this vein, we present two fundamental computations.

\begin{lemma}
  For any $m \in \zpos$ and $p \in \proj{1}$, $H^1(\skyscraper{mp}) = 0$.
\end{lemma}

\begin{proof}
  Put another way, the claim is that the differential map $d$ is surjective.
  The sections of $\skyscraper{mp}$ are supported only in open
  neighbourhoods of $p$, so if $p$ is either of the degenerate points
  $(1 : 0)$ or $(0 : 1)$, then $p \not\in U_{xy}$ and $d = 0$, so the
  claim is trivial in this case.
  In the non-degenerate case, we have the isomorphisms of vector spaces
  \[
    \sections{U_x}{\skyscraper{mp}}
    \cong \sections{U_y}{\skyscraper{mp}}
    \cong \sections{U_{xy}}{\skyscraper{mp}}
    \cong k^m.
  \]
  Since the descent datum of $\skyscraper{mp}$ acts as the identity,
  the restriction maps to $U_{xy}$ are simply the identity maps on
  $k^m$, so we see that $d$ is once again surjective.
\end{proof}

\begin{lemma}
  \label{lemma_h1_twist}
  For any $n \in \bb{Z}$, we have
  \[
    H^1(\twist{n})
    % =
    % \begin{cases}
    %     \gen{z^{-1}, z^{-2}, \ldots, z^{-(|n| - 1)}} & \text{if } n < -1 \\
    %     0 & \text{if } n \geq -1.
    % \end{cases}
    \cong
    \begin{cases}
      k^{|n| - 1} & \text{if } n \leq -2 \\
      0 & \text{if } n \geq -1.
    \end{cases}
  \]
\end{lemma}

\begin{proof}
  As should be familiar at this point, we can characterise the
  sections of $\twist{n}$ over $U_x$ as pairs $(f, g) \in k[z] \times
  k[z, z^{-1}]$ such that $g = z^{-n} f$.
  Each section is determined by the choice of $f$, so there is an
  isomorphism $\sections{U_x}{\twist{n}} \cong k[z]$.
  By similar arguments, we have $\sections{U_y}{\twist{n}} \cong
  k[z^{-1}]$, and similarly
  \[
    \sections{U_{xy}}{\twist{n}} \cong k[z]_z \cong k[z, z^{-1}]
    \cong k[z^{-1}]_{z^{-1}}.
  \]
  The corresponding restriction maps from $U_x$ and $U_y$ to $U_{xy}$
  are the canonical inclusions $k[z] \mono k[z]_z$ and $k[z^{-1}]
  \mono k[z]_{z^{-1}}$ respectively, but to ensure compatibility we
  shall postcompose the second restriction map by $\vartheta_n^{-1}:
  k[z^{-1}]_{z^{-1}} \to k[z]_z$, i.e. by multiplying by $z^n$.
  Thus the \v{C}ech differential is ultimately the map $d(f, g) = f - z^n g$.

  It remains to find $\coker(d)$.
  We see that $\im(d)$ is a vector subspace of $k[z, z^{-1}]$ over
  $k$, and clearly it contains the subspace generated by all
  non-negative powers of $z$.
  Moreover, $\im(d)$ contains $z^i = z^{n} z^{-(n - i)}$ for a given
  $i \leq -1$ if and only if $n - i \geq 0$.
  This is trivially true when $n \geq -1$,
  % since
  % \[
  %     n - i \geq -1 - (-1) = 0,
  % \]
  so in this case $\im(d) = k[z, z^{-1}]$ and $H^1(\twist{n}) = 0$.
  When $n \leq -2$, we have $n - i = -|n| - i < 0$ for all $-(|n| -
  1) \leq i \leq -1$, so in this case $\im(d)$ is the (set-theoretic)
  complement of the subspace
  \[
    \gen{z^{-1}, z^{-2}, \ldots, z^{-(|n| - 1)}} \cong k^{|n| - 1}.
  \]
  It follows that $H^1(\twist{n}) \cong k^{|n| - 1}$.
\end{proof}

The final pair of cohomological facts we will collect relate to the Ext groups.
The second of these will rely on an additional result in coherent
sheaf cohomology that is slightly beyond our reach in this thesis.

\begin{lemma}
  \label{prop_ext_from_twist}
  Let $\shf{G}$ be a coherent sheaf on $\proj{1}$.
  Then for all $n \in \bb{Z}$ and $i \in \zpos$,
  \[
    \Extalt{i}(\twist{n}, \shf{G}) \cong H^i(\shf{G} \shftensor \twist{-n}).
  \]
  In particular, $\Extalt{i}(\twist{n}, \shf{G}) = 0$ for $i \geq 2$.
\end{lemma}

\begin{proof}
  Since the degree $-n$ twist is an autoequivalence, by
  \cref{lemma_ext_preserved_by_equiv} we have
  \begin{align*}
    \Extalt{i}(\twist{n}, \shf{G})
    &\cong \Extalt{i}(\shf{O}, \shf{G} \shftensor \twist{-n}) \\
    &\cong H^i(\shf{G} \shftensor \twist{-n})
  \end{align*}
  by construction of the derived functor cohomology.
  These cohomology groups vanish for $i \geq 2$, so the second part
  of this claim is immediate.
\end{proof}

\begin{proposition}
  \label{prop_CohP1_hereditary}
  The category $\Coh(\proj{1})$ is \emph{hereditary}.
  That is, $\Extalt{i}(\shf{F}, \shf{G}) = 0$ for all $i \geq 2$ and
  coherent sheaves $\shf{F}$, $\shf{G}$ on $\proj{1}$.
\end{proposition}

\begin{proof}
  Since $\Extalt{i}$ is an additive functor, it is enough to prove
  the claim when $\shf{F}$ is either a twisting sheaf or a skyscraper
  sheaf by Grothendieck's splitting theorem.
  We have treated the former case already in
  \cref{prop_ext_from_twist}, so suppose that $\shf{F} = \skyscraper{mp}$.
  Recalling \cref{prop_ideal_sheaf_twist}, we have the ideal sheaf sequence
  \[
    \begin{tikzcd}[cramped]
      0 & {\twist{-m}} & {\shf{O}} & {\skyscraper{mp}} & 0,
      \arrow[from=1-1, to=1-2]
      \arrow[from=1-2, to=1-3]
      \arrow[from=1-3, to=1-4]
      \arrow[from=1-4, to=1-5]
    \end{tikzcd}
  \]
  Applying a degree $n$ twist for some $n \in \bb{Z}$ yields a new
  short exact sequence
  \[
    \begin{tikzcd}[cramped]
      0 & {\twist{n - m}} & {\twist{n}} & {\skyscraper{mp}(n)} & 0.
      \arrow[from=1-1, to=1-2]
      \arrow[from=1-2, to=1-3]
      \arrow[from=1-3, to=1-4]
      \arrow[from=1-4, to=1-5]
    \end{tikzcd}
  \]
  First, we note that $\skyscraper{mp}(n) \cong \skyscraper{mp}$.
  Indeed, the sections are still supported only at the point $p$, and
  the algebraic effect of the twist on the descent datum in
  neighbourhoods of $p$ is scaling by some negative power of a unit
  in $k$, which is an invertible operation.
  Next, since the contravariant functor $\Hom(-, \shf{I})$ is exact
  for any injective quasi-coherent sheaf $\shf{I}$, taking an
  injective resolution $\shf{G} \to \cochaincomp{\shf{I}}$ yields a
  short exact sequence
  \[
    \begin{tikzcd}[cramped]
      0 & {\Hom(\skyscraper{mp}, \cochaincomp{\shf{I}})} &
      {\Hom(\twist{n}, \cochaincomp{\shf{I}})} & {\Hom(\twist{n - m},
      \cochaincomp{\shf{I}})} & 0
      \arrow[from=1-1, to=1-2]
      \arrow[from=1-2, to=1-3]
      \arrow[from=1-3, to=1-4]
      \arrow[from=1-4, to=1-5]
    \end{tikzcd}
  \]
  of complexes with the direction reversed.
  Now, for each $i \geq 2$, we consider the exact portion of the
  corresponding long exact cohomology sequence
  \[
    \begin{tikzcd}[cramped]
      {\Extalt{i - 1}(\twist{n - m}, \shf{G})} &
      {\Extalt{i}(\skyscraper{mp}, \shf{G})} & {\Extalt{i}(\twist{n}, \shf{G})}.
      \arrow[from=1-1, to=1-2]
      \arrow[from=1-2, to=1-3]
    \end{tikzcd}
  \]
  The right term vanishes for all $i \geq 2$ by
  \cref{prop_ext_from_twist}, and so does the left term when $i > 2$,
  so we conclude that $\Extalt{i}(\skyscraper{mp}, \shf{G}) = 0$ in this case.
  Thus it remains to show that
  \[
    \Extalt{1}(\twist{n - m}, \shf{G})
    \cong
    \Extalt{1}(\shf{O}, \shf{G}(m - n))
    \cong H^1(\shf{G}(m - n))
  \]
  also vanishes.
  This may not be true a priori for any choice of $n$, but by a
  vanishing theorem of Serre (c.f.
  \cite[Theorem~III.5.2(b)]{hartshorne}), the cohomology group
  $H^1(\shf{G}(m - n))$ is guaranteed to vanish when $m - n$ is
  sufficiently large, i.e. when $n$ is sufficiently negative.
  Taking such an $n$ therefore completes the proof of the claim.
\end{proof}
