\chapter{Introduction}
{
  \setlength{\epigraphwidth}{0.5 \textwidth}
  % \setlength{\epigraphrule}{0pt}
  \epigraph{
    \itshape
    Algebra is the offer made by the devil to the mathematician.
    The devil says: `I will give you this powerful machine, it will
    answer any question you like.
    All you need to do is give me your soul: give up geometry and you
    will have this marvelous machine.'
  }{-- Michael F. Atiyah \cite{atiyah_algebra_quote}}
}

\section{Motivation and outline}

The \emph{coherent sheaves} on an algebraic variety $X$ are of
central significance in algebraic geometry, being closely linked to
many geometric properties of $X$ itself.
In a manner of speaking, these are a better behaved generalisation of
\emph{vector bundles}.
Originally the preserve of analytic geometry, the genesis of coherent
sheaves and \emph{coherent sheaf cohomology} in the algebro-geometric
setting was Serre's revolutionary paper \cite{fac}.

It had already been deduced by Serre that the cohomology of
\emph{algebraic} vector bundles on a smooth projective variety
enjoyed a certain duality property, appropriately named \emph{Serre duality}.
Grothendieck sought to generalise this duality to coherent sheaves
and for a wider class of algebraic spaces, ultimately using the
nascent theory of \emph{schemes} introduced by Grothendieck himself
in \cite{ega1}.
In so doing, he realised that whereas the \emph{canonical line
bundle} sufficed to give the usual Serre duality, what was needed in
the general case was instead a \emph{dualising complex} of sheaves to
fulfil the same purpose.
As a response to the shortcomings of homological algebra in
accommodating this shift in viewpoint, Grothendieck and his student
Verdier developed the notion of \emph{derived categories} and a more
unified approach to \emph{derived functors}.

In the years since their development, derived categories have become
a tool of vital importance in algebraic geometry, but are also of
relevance further afield in the study of partial differential
equations (e.g. \emph{$D$-modules}) and string theory (e.g. the
\emph{homological mirror symmetry} program of Kontsevich).
The fusion between derived categories and coherent sheaves in
particular has led to some powerful results.
For example, the (bounded) derived category $\dercat[b]{\Coh(X)}$ of
coherent sheaves on a variety $X$ sometimes captures all there is to
know about said variety:
Bondal and Orlov showed in \cite{bondal_orlov} that under certain
algebro-geometric hypotheses, the data of the category
$\dercat[b]{\Coh(X)}$ is enough to uniquely reconstruct $X$.
This makes it a useful category to try and understand.

An early example of work in this direction is due to Beilinson, who
studied the coherent sheaves of $\proj{n}$ over an algebraically
closed field $k$ in \cite{beilinson}.
The main result, hereafter referred to as \emph{Beilinson's theorem},
is that there is an equivalence of derived categories between
$\dercat[b]{\Coh(\proj{n})}$ and $\dercat[b]{\mod{\op{\Lambda}}}$,
where $\Lambda$ is a certain non-commutative finite dimensional $k$-algebra.
Since one may identify $\Lambda$ with the \emph{path algebra} of a
certain \emph{quiver}, this theorem provides not only a link from
algebraic geometry to the representation theory of finite dimensional
algebras, but also to that of quivers.

There are two well-known proofs of Beilinson's theorem.
The original method of proof in \cite{beilinson} is by constructing a
\emph{resolution of the diagonal} of $\proj{n}$, a technique
belonging to algebraic geometry.
Eventually, the techniques of \emph{tilting theory} emerged as giving
a more satisfying proof instead using the methods of representation theory.
An outcome of the seminal work of Brenner and Butler in 1979, tilting
theory is a systematic means of constructing equivalences between the
module categories of two finite dimensional algebras.
The language of derived categories, which also proved to be useful in
this context, was infused into tilting theory by Happel and Rickard
in the 1980s.

The goal of this thesis is to give an accessible exposition of how
one may prove Beilinson's theorem using tilting theory.
To make this more tractable, we will restrict our attention to the
case of $\proj{1}$ only.
Actually understanding the statement of the theorem even in this
simpler case is already a non-trivial task, so a necessary secondary
goal, and one which occupies the majority of this thesis, is to give
a solid and gentle introduction to all of the required background.
Though this topic fits within the intersection of algebraic geometry,
homological algebra and representation theory, the dominant themes of
this thesis are the former two.

A brief synopsis of the flow of the material per chapter is as follows.
We start with \cref{chap_abelian_categories}, which is a discussion
of abelian categories.
This turns out to be just the right level of categorical abstraction
for many areas of modern mathematics due to it unlocking the
machinery of homological algebra, which is in turn the focus of
\cref{chap_homological}.
The most fundamental tool in this subject is cohomology, and in
considering its utility as a rich and very general method of
producing algebraic invariants, we will see that an `improvement' to
the notion of abelian categories that is more inclined to the study
of cohomology is in order.
Derived categories end up fulfilling this purpose, so we discuss them
at length in \cref{chap_derived_categories}.
Following this, we change course completely in
\cref{chap_coh_sheaf_theory_of_p1} to study the theory of coherent
sheaves on $\proj{1}$ and various aspects of its sheaf cohomology.
The fact that we are working only in this simple case allows us to
present this in a quite hands-on way than the more generic treatment
one finds in the literature.
The preceding chapters clear the path finally for
\cref{chap_beilinsons_theorem}, in which we set up the notion of a
\emph{tilting bundle} which yields Beilinson's equivalence.
For the reader's convenience, \cref{conventions} records all of the
explicit conventions and various pieces of notation we use throughout
the thesis.

There is a substantial amount of technical background that we must
cover to set the appropriate scene in the first three chapters.
We will emphasise breadth rather than depth when it comes to these
topics, so we will be unapologetically light at times on details,
deferring extensively to the relevant literature.
The material in these chapters is to be treated principally as a tool
for what is to come, and it is often the case that for many stated
results, their proofs are not particularly insightful and are little
more than \emph{abstract nonsense} (i.e. category-theoretic formalities).

\section{Assumed knowledge}

The most serious prerequisite is a moderate background in category theory.
In addition to being well-acquainted with its core ideas (categories,
functors and natural transformations), we assume the reader is
familiar with basic categorical relationships (duality, equivalences
of categories and adjoint functors) and universal constructions
(products, coproducts, kernels and cokernels).
The standard reference for this material is
\cite{category_theory_for_the_working_mathematician}.
Some exposure to limits and colimits, which provide a unifying
language for these universal constructions, is nice to have but will
not be explicitly used.

A background in ring and module theory at the level of
\cite[Chapters~8~and~10]{dummit_and_foote} will also be assumed.
We will bundle this with a single fairly light assumption from
commutative algebra that one may encounter in the course of studying
the aforementioned topics: a working knowledge of localisations of
rings and modules.
For details of this process, one may consult
\cite[Chapter~3]{atiyah_macdonald} or \cite[Section~15.4]{dummit_and_foote}.

Finally, we assume familiarity with the basics of classical algebraic geometry.
It is enough to be aware of the main aims of the subject, the
definitions and main results concerning (quasi-)affine and
(quasi-)projective varieties, and the basic study of regular
functions and regular maps of varieties.
For a reference, see \cite[Sections~I.1--4]{hartshorne}.
A very basic acquaintance with affine schemes would be nice to have
but not essential.

\section{Author's note}

When I embarked on this project, my intended goal was to give a
comprehensive treatment of Beilinson's theorem in the simplest case
of $\proj{1}$, including the full details of its proof.
This is a result which has received much attention among algebraic
geometers and representation theorists alike, but as far as I can
tell, there has been no formal literature aimed at a wider audience.
A story like this, to me, is too good not to be told, and as is the
hope of any honours student, I wanted my work to bridge that gap.
This was certainly an ambitious goal, but as is my nature, I threw
myself at it anyway.

There are at least two reasons why this thesis does not strictly meet this goal.
It brings me no joy to admit this whatsoever, but the first is that I
was simply not able to completely bridge that gap for myself this year.
There is a tremendous amount of abstract background material that one
has to learn to even begin to make sense of such a result, and the
scale of this only becomes apparent once one starts to pour over the details.
This ultimately consumed my attention for the entire year.
This being the case, I also realise separately there is still immense
value in having all of the necessary details leading up to some
greater result being explicitly spelled out in one place.
That is a worthwhile goal in and of itself, and the length of this
document is a firm testament to the fact that one can make a whole
thesis project out of it.

I therefore present this work as a complete coverage and background
in the abstract machinery one needs \emph{before} approaching this
theorem, and an exposition of the mathematical beauty present within
these background details themselves.
After reading this work, one would be well prepared to venture into
the more professional literature to see the finer details that I
unfortunately cannot give you.
