\chapter{Conventions}
\label{conventions}

For the avoidance of doubt, the following is a reference for the
conscious conventions and general notational choices we have made in
this thesis:

\begin{itemize}
  \item
    The set of non-negative integers is denoted $\zpos$.
    Similarly, the set of non-positive integers is denoted $\zneg$.

  \item
    Unless otherwise stated, all rings should be assumed to be commutative.
    Most algebras will not be, but will always be associative and unital.

  \item
    If $A$ is a ring or an algebra, the category of left $A$-modules
    is denoted by $\Mod{A}$.
    The full subcategory of finitely generated left $A$-modules is
    denoted $\mod{A}$.
    We do not consider right $A$-modules at any point in this thesis.

  \item
    If $R$ is a ring and $f \in R$, the ring $R_f$ denotes the
    localisation of $R$ at the multiplicatively closed subset $S =
    \{1, f, f^2, \ldots\} \subseteq R$.
    More generally, if $M$ is an $R$-module, the $R_f$-module $M_f$
    denotes the localisation of $M$ at $S$.
    We will often abuse notation by conflating an element $m \in M$
    with the fraction $m/1 \in M_f$.
    Via the canonical ring monomorphism $R \mono R_f$, one may also
    think of $M_f$ as an $R$-module, and it will be convenient to
    implicitly apply this principle at times.

  \item
    Given $f, g \in R$, we denote the \emph{iterated} localisation of
    an $R$-module $M$ at $f$ and then at $g$ by $M_{f, g}$, which is
    an $R[f^{-1}, g^{-1}]$-module.
    Manifestly, the order in which one does these localisations does not matter.
    (Note that this is \emph{not} the same thing as the localisation
      of $M$ at $fg$, as the elements of $M_{f, g}$ are fractions
      $m/(f^i g^j)$ for some $m \in M$ and usually different powers $i,
    j \in \zpos$.)

  \item
    Given a category $\cat{C}$, we write $A \in \cat{C}$ and $f \in
    \cat{C}$ to denote that an object $A$ and a morphism $f$ belong
    to $\cat{C}$, rather than the common Ob and Hom notation.

  \item
    We denote both the object $K$ and the morphism $k: K \mono A$
    constituting a kernel of a morphism $f: A \to B$ in some category
    by $\ker(f)$, with it being clear from context the way in which a
    particular usage should be read.
    We treat all cokernels, images and coimages similarly.

  \item
    Given functors $F,\, G: \cat{C} \to \cat{D}$ and $H: \cat{D} \to
    \cat{E}$ as well as natural transformations $\eta: F \nattrans G$
    and $\varepsilon: G \nattrans H$, we denote by $\varepsilon \circ
    \eta: F \nattrans H$ the \emph{vertical composition} of
    $\varepsilon$ with $\eta$ having components $(\varepsilon \circ
    \eta)_X = \varepsilon_X \circ \eta_X$ for each $X \in \cat{C}$.
    This is the implied composition operation for all commutative
    diagrams of functors.

  \item
    There are two distinct ways of composing a natural transformation
    with a functor.
    Fix functors $G,\, H: \cat{C} \to \cat{D}$ and a natural
    transformation $\eta: G \nattrans H$.
    Given functors $F: \cat{B} \to \cat{C}$ and $K: \cat{D} \to
    \cat{E}$, we define the natural transformations
    \[
      F \circ \eta: F \circ G \nattrans F \circ H
      \mathand
      \eta \circ K: G \circ K \nattrans H \circ K
    \]
    to have components
    \[
      (F \circ \eta)_X = F \circ \eta_X
      \mathand
      (\eta \circ K)_Y = \eta_{K(Y)}
    \]
    for each $X \in \cat{B}$ and $Y \in \cat{D}$.
\end{itemize}
