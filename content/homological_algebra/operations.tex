\section{Operations on complexes}

Aside from taking cohomology, there are other useful things which one
can do with complexes.
We first introduce some more functors on $\Ch{\abcat{A}}$ which we
will need later.
The fact that these define additive functors is immediate by construction.

\begin{definition}
  Let $n \in \bb{Z}$.
  The degreewise (left) shift of complexes and cochain maps by $n$
  defines an additive endofunctor $[n]: \Ch{\abcat{A}} \to
  \Ch{\abcat{A}}$, called the \emph{$n$th shift} (or \emph{$n$th
  translation}) functor.
  Explicitly, $[n]$ sends $\cochaincomp{C}$ to the complex
  $\cochaincomp{C}[n]$ with
  \[
    {C}[n]^i = C^{i + n}
    \mathand
    d[n]^i = (-1)^n d^{i + n},
  \]
  and sends $f: \cochaincomp{C} \to \cochaincomp{D}$ to the cochain
  map $f[n]: \cochaincomp{C} \to \cochaincomp{D}$ with $f[n]^i = f^{i + n}$.
\end{definition}

For technical reasons that we shall not explain, it turns out that
introducing the sign $(-1)^n$ in the differential like this is
generally convenient when setting up the theory.
Though it seems like a small observation now, the shift complex has a
convenient property in the context of \cref{sect_triangcat}.

\begin{proposition}
  The shift functor $[n]$ is autoequivalence for all $n \in \bb{Z}$.
\end{proposition}

This shift functor is related to another common construction.
Associated to any cochain map is its \emph{mapping cone}, once again
directly inspired by topology.

\begin{definition}
  The \emph{mapping cone} of a cochain map $f: \cochaincomp{C} \to
  \cochaincomp{D}$ is the cochain complex $\cone{f}$ whose cochains
  and differentials are given by
  \[
    \cone{f}^i = C^{i + 1} \oplus D^i
    \mathand
    d^n =
    \begin{bmatrix}
      -d_C^{i + 1} & 0 \\
      -f^{i + 1} & d^i_D
    \end{bmatrix},
  \]
  with the differential acting as if by componentwise matrix
  multiplication on $C^{i + 1} \oplus D^i$.
\end{definition}

The mapping cone has two uses.
The first is that it allows one to insert any cochain map $f:
\cochaincomp{C} \to \cochaincomp{D}$ into a short exact sequence in
$\Ch{\abcat{A}}$.
In particular, we have that
\[
  \begin{tikzcd}
    \cochaincomp{0} \arrow[r]
    & \cochaincomp{D} \arrow[r]
    & \cone{f} \arrow[r]
    & \cochaincomp{C}[1] \arrow[r]
    & \cochaincomp{0}
  \end{tikzcd}
\]
is exact, where the two middle cochain maps are the canonical
injection and projection of direct sums respectively.
This leads to its second use, detecting quasi-isomorphisms.

\begin{corollary}
  \label{cor_quasi_isom_has_acyclic_cone}
  A cochain map is a quasi-isomorphism if and only if its mapping
  cone is an exact (equivalently, acyclic) complex.
\end{corollary}

\begin{proof}
  Using the identification $H^i(\cochaincomp{C}[1]) \cong H^{i +
  1}(\cochaincomp{C})$, the long exact cohomology sequence for the
  above short exact sequence is of the form
  \[
    \begin{tikzcd}
      \cdots \arrow[r]
      & H^i(\cochaincomp{D}) \arrow[r]
      & H^i(\cone{f}) \arrow[r]
      & H^{i + 1}(\cochaincomp{C}) \arrow[r, "\partial^i"]
      & H^{i + 1}(\cochaincomp{D}) \arrow[r]
      & \cdots,
    \end{tikzcd}
  \]
  where $\partial$ is the connecting map.
  It is therefore enough to show that $\partial^i = H^{i + 1}(f)$,
  which is the content of \cite[Lemma~1.5.3]{weibel}.
\end{proof}

An even more simple-minded but crucial operation on complexes is
applying an additive functor $F: \abcat{A} \to \abcat{B}$ degreewise.
This induces a functor $\Ch{F}: \Ch{\abcat{A}} \to \Ch{\abcat{B}}$,
though for notational ease we will typically refer to $\Ch{F}$ simply
by $F$ itself.

\begin{lemma}
  \label{lemma_add_func_preserves_homotopy_and_cones}
  The functor $\Ch{F}$ is additive and preserves homotopies, homotopy
  equivalences and mapping cones.
  Moreover, it commutes with shift functors.
\end{lemma}

\begin{proof}
  These properties follow from functoriality and
  \cref{lemma_additive_func_commutes_with_dirsum}.
\end{proof}

In the case of double complexes, the only operation we will concern
ourselves with is \emph{totalisation}, which is a construction that
collapses a double complex down into a normal complex in one of two ways.

\begin{definition}
  \label{def_totalisation}
  Given a double complex $\cochainbicomp{C}$, the \emph{total
  complexes} $\totprodcomp{\cochainbicomp{C}}$ and
  $\totsumcomp{\cochainbicomp{C}}$ are those obtained by taking
  products and direct sums along each lattice diagonal (c.f.
  \cref{def_double_complex}), i.e.
  \[
    \totprodcomp{\cochainbicomp{C}}^i = \prod_{i = p + q} C^{p, q}
    \mathand
    \totsumcomp{\cochainbicomp{C}}^i = \bigoplus_{i = p + q} C^{p, q}.
  \]
  In both cases, the differential is defined componentwise by $d =
  d_h^{p, q} + d_v^{p, q}$.
\end{definition}

\begin{remark}
  A priori, totalising a double complex involves taking infinite
  products or direct sums of objects in $\abcat{A}$, which may not be
  possible if the base category $\abcat{A}$ is not \emph{(co)complete}.
  However, many double complexes arising in practice, including any
  we will encounter in this thesis, have only finitely many objects
  along each lattice diagonal.
  A sufficient condition for this to occur is that the double complex
  to be totalised is constructed out of bounded complexes.
\end{remark}
