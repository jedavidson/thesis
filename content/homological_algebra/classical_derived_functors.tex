\section{Classical derived functors}
\label{sect_classical_derfunc}

Now that we have a systematic way to study the failure of a complex
to be exact, we finish by discussing the same problem for a covariant
functor $F: \abcat{A} \to \abcat{B}$ between abelian categories.
Specifically, we will see that when $F$ is left (resp. right) exact,
there exist \emph{classical right} (resp. \emph{left}) \emph{derived
functors} that measure the failure of $F$ to be exact.
The reason for the term `classical' will become clear in \cref{sect_derfunc}.

Our experience in the previous sections show that cohomology can be a
useful tool in these situations.
Replacing any object $A \in \abcat{A}$ with a right resolution $A \to
\cochaincomp{C}$ and applying a left exact functor $F: \abcat{A} \to
\abcat{B}$ gives a cochain complex
\[
  \begin{tikzcd}[cramped]
    \cdots & 0 & {F(C^0)} & {F(C^1)} & {F(C^2)} & \cdots
    \arrow[from=1-1, to=1-2]
    \arrow[from=1-2, to=1-3]
    \arrow[from=1-3, to=1-4]
    \arrow[from=1-4, to=1-5]
    \arrow[from=1-5, to=1-6]
  \end{tikzcd}
\]
in $\abcat{B}$.
This will typically have non-trivial cohomology, but it is always
acyclic when $F$ is exact.
While this does seem to have the measuring capabilities we seek, a
problem is that the cohomology objects we obtain are sensitive to our
choice of resolution.
If we instead work with injective resolutions, then any two will be
homotopy equivalent by \cref{thm_fund_thm_of_hom_alg}, and so we will
always recover the same cohomology objects.
The intuitive reason for considering right resolutions here to begin
with is that left exact functors have issues extending left exact
sequences to the right (c.f. \cref{def_exact_functors}).

\begin{definition}
  Let $F: \abcat{A} \to \abcat{B}$ be a left exact functor and
  suppose that $\abcat{A}$ has enough injectives.
  For each $A \in \abcat{A}$, fix an injective resolution $A \to
  \cochaincomp{I}$ and define
  \[
    R^iF(A) := H^i(F(\cochaincomp{I}))
  \]
  for $i \in \zpos$.
  The functor $R^iF: \abcat{A} \to \abcat{B}$ is called the
  \emph{$i$th right derived functor} of $F$.
\end{definition}

We have already seen that this is a well-defined map on objects, but
we have yet to specify what $R^iF$ does to morphisms.
Given a morphism $f: A \to B$, fix a pair of injective resolutions $A
\to \cochaincomp{I}$ and $B \to \cochaincomp{J}$ and then lift $f$
using \cref{thm_fund_thm_of_hom_alg} to a cochain map $\widetilde{f}:
\cochaincomp{I} \to \cochaincomp{J}$.
This defines a unique map
\[
  R^iF(f) := H^n(F(\widetilde{f})): H^i(F(\cochaincomp{I})) \to
  H^i(F(\cochaincomp{J})),
\]
since any lift of $f$ is unique up to homotopy equivalence.
It is now clear that $R^iF$ is actually a functor to begin with and
is moreover additive.
% \[
%     R^iF(gf) = R^iF(g) \circ R^iF(f) \mathand R^iF(f + g) = R^iF(f) + R^iF(g)
% \]
% whenever the composite $gf$ and sum $f + g$ are defined.
% Given a compatible morphism $g$, the composite
% $\widetilde{\vphantom{f}g} \widetilde{f}$ is a lift for $gf$, so
% $R^iF(gf) = R^iF(g) \circ R^iF(f)$.
% Finally, if $g$ is now a morphism parallel to $f$, we see that
% $\widetilde{f} + \widetilde{\vphantom{f}g}$ is a lift of $f + g$,
% and so $R^iF(f + g) = R^iF(f) + R^iF(g)$.

The key properties that the functors $R^iF$ have are best expressed
by the following framework due to Grothendieck.
As the name implies, the model for this will be the collection of
cohomology functors of \cref{prop_cochain_map_induced_cohomology},
with the key property being that there is an analogue of the long
exact cohomology sequence.

\begin{definition}
  A \emph{cohomological $\delta$-functor} $T: \abcat{A} \to
  \abcat{B}$ consists of a collection of additive functors $\{T^i:
  \abcat{A} \to \abcat{B}\}_{i \in \zpos}$ subject to the following
  two conditions:
  \begin{enumerate}
    \item
      For each short exact sequence
      \[
        \begin{tikzcd}
          0 & A & B & C & 0
          \arrow[from=1-1, to=1-2]
          \arrow[from=1-2, to=1-3]
          \arrow[from=1-3, to=1-4]
          \arrow[from=1-4, to=1-5]
        \end{tikzcd}
      \]
      in $\abcat{A}$, there exist morphisms $\delta^i: T^i(C) \to
      T^{i + 1}(A)$ and a long exact sequence
      \[
        \begin{tikzcd}[cramped]
          0 & {T^0(A)} & {T^0(B)} & {T^0(C)} \\
          & {T^1(A)} & {T^1(B)} & {T^1(C)} \\
          & {T^2(A)} & {T^2(B)} & {T^2(C)} & \cdots
          \arrow[from=1-1, to=1-2]
          \arrow[from=1-2, to=1-3]
          \arrow[from=1-3, to=1-4]
          \arrow["{\delta^0}"{description}, dashed, from=1-4, to=2-2]
          \arrow[from=2-2, to=2-3]
          \arrow[from=2-3, to=2-4]
          \arrow["{\delta^1}"{description}, dashed, from=2-4, to=3-2]
          \arrow[from=3-2, to=3-3]
          \arrow[from=3-3, to=3-4]
          \arrow[from=3-4, to=3-5, dashed]
        \end{tikzcd}
      \]
      in $\abcat{B}$.
      (In particular, $T^0$ is left exact.)

    \item
      Given a morphism of short exact sequences
      \[
        \begin{tikzcd}[cramped]
          0 & A & B & C & 0 \\
          0 & {A'} & {B'} & {C'} & 0
          \arrow[from=1-1, to=1-2]
          \arrow[from=1-2, to=1-3]
          \arrow[from=1-2, to=2-2]
          \arrow[from=1-3, to=1-4]
          \arrow[from=1-3, to=2-3]
          \arrow[from=1-4, to=1-5]
          \arrow[from=1-4, to=2-4]
          \arrow[from=2-1, to=2-2]
          \arrow[from=2-2, to=2-3]
          \arrow[from=2-3, to=2-4]
          \arrow[from=2-4, to=2-5]
        \end{tikzcd}
      \]
      in $\abcat{A}$, each of the below squares in $\abcat{B}$ commute:
      \[
        \begin{tikzcd}[cramped]
          {T^i(C)} & {T^{i + 1}(A)} \\
          {T^i(C')} & {T^{i + 1}(A')}.
          \arrow["{\delta^i}", from=1-1, to=1-2]
          \arrow[from=1-1, to=2-1]
          \arrow[from=1-2, to=2-2]
          \arrow["{(\delta')^i}"', from=2-1, to=2-2]
        \end{tikzcd}
      \]
  \end{enumerate}
\end{definition}

\begin{definition}
  Let $S, \, T: \abcat{A} \to \abcat{B}$ be cohomological $\delta$-functors.
  A \emph{morphism of $\delta$-functors} $\mu: S \to T$ is a
  collection of natural transformations $\{\mu^i: S^i \Rightarrow
  T^i\}_{i \in \zpos}$ which commute with $\delta_S$ and $\delta_T$.
  That is, for any short exact sequence
  \[
    \begin{tikzcd}
      0 & A & B & C & 0
      \arrow[from=1-1, to=1-2]
      \arrow[from=1-2, to=1-3]
      \arrow[from=1-3, to=1-4]
      \arrow[from=1-4, to=1-5]
    \end{tikzcd}
  \]
  in $\abcat{A}$, each of the below squares in $\abcat{B}$ commute:
  \[
    \begin{tikzcd}[cramped]
      {S^i(C)} & {S^{i + 1}(A)} \\
      {T^i(C)} & {T^{i + 1}(A)}.
      \arrow["{\delta_S^i}", from=1-1, to=1-2]
      \arrow["{\mu_C^i}"', from=1-1, to=2-1]
      \arrow["{\mu_A^{i + 1}}", from=1-2, to=2-2]
      \arrow["{\delta_T^i}"', from=2-1, to=2-2]
    \end{tikzcd}
  \]
  Moreover, $T$ is \emph{universal} if for any choice of $S$ and
  natural transformation $\mu^0: S^0 \Rightarrow T^0$ there exists a
  unique extension of $\mu^0$ to a morphism of $\delta$-functors $\mu: S \to T$.
\end{definition}

\begin{proposition}
  The collection of all right derived functors of a left exact
  functor $F: \abcat{A} \to \abcat{B}$ form a cohomological universal
  $\delta$-functor.
\end{proposition}

\begin{proof}
  Both parts of this claim are non-trivial.
  The proofs in the dual case of the \emph{left} derived functors
  $L_iF$ of a \emph{right} exact $F$ are the content of
  \cite[Theorems~2.4.6 and 2.4.7]{weibel}, but with our initial
  hypotheses it will turn out that
  \[
    R^iF(A) = \op{(L_i \op{F})}(A)
  \]
  for all $A \in \abcat{A}$, so this is indeed enough.
\end{proof}

There are several other benefits to this construction.
First, the information of the original functor $F$ is captured by $R^0F$.
Indeed, since $F$ is left exact, the sequence
\[
  \begin{tikzcd}
    0 & {F(A)} & {F(I^0)} & {F(I^1)}
    \arrow[from=1-1, to=1-2]
    \arrow[from=1-2, to=1-3]
    \arrow[from=1-3, to=1-4]
  \end{tikzcd}
\]
is exact for any injective resolution $A \to \cochaincomp{I}$, so
\begin{align*}
  R^0F(A)
  = H^0(F(\cochaincomp{I}))
  \cong \ker(F(I^0) \to F(I^1))
  = \im(F(A) \to F(I^0))
  = F(A),
\end{align*}
and moreover $R^0F(f) = F(f)$ for any morphism $f$.
If $I$ is an injective object, then computing $R^iF(I)$ using the
injective resolution
\[
  \begin{tikzcd}
    0 & I & I & 0 & 0 & \cdots
    \arrow[from=1-1, to=1-2]
    \arrow["\id", from=1-2, to=1-3]
    \arrow[from=1-3, to=1-4]
    \arrow[from=1-4, to=1-5]
    \arrow[from=1-5, to=1-6]
  \end{tikzcd} \qedhere
\]
shows that $R^iF(I) = 0$ for $i > 0$.
It is also clear that if $F$ is exact, then $R^iF = 0$ for $i > 0$.
The upshot is therefore that the right derived functors of $F$
provide a convenient way of measuring how far $F$ is from being exact
as intended, and the fact that they are a cohomological universal
$\delta$-functor shows that they are the most natural way to do so.

\begin{remark}
  \label{rem_computing_with_F_acyclics}
  Even with enough injectives, actually obtaining a injective
  resolution for an object can be difficult.
  With reference to a specific functor $F$, an object $X \in
  \abcat{A}$ is said to be \emph{$F$-acyclic} if $R^iF(X) = 0$ for $i
  > 0$, and it turns out that it is sufficient to use $F$-acyclic
  resolutions to compute $R^iF(A)$, since $R^iF(A) \cong
  H^i(F(\cochaincomp{X}))$ for any $F$-acyclic resolution $A \to
  \cochaincomp{X}$.
  This can be seen by the \emph{dimension shifting} argument outlined
  (albeit for the left derived case again) in \cite[Exercise~2.4.3]{weibel}.
\end{remark}

We now summarise the analogous construction of left derived functors
when $F$ is right exact.
What we will end up with is a \emph{homological} $\delta$-functor,
which is defined in a dual way to cohomological $\delta$-functors
(c.f. \cite[Section~2.1]{weibel}).
The strategy is the same as last time, except we now use projective
resolutions and take negative cohomology.

\begin{proposition}
  \label{prop_classical_right_derfunc}
  Let $F: \abcat{A} \to \abcat{B}$ be a right exact functor and
  suppose that $\abcat{A}$ has enough projectives.
  For each $A \in \abcat{A}$, fix a projective resolution
  $\cochaincomp{P} \to A$ and define
  \[
    L_i F(A) := H^{-i}(F(\cochaincomp{P}))
  \]
  for each $i \in \zpos$.
  The corresponding functor $L_iF: \abcat{A} \to \abcat{B}$ is called
  the \emph{$n$th classical left derived functor} of $F$, and it
  satisfies the following properties:
  \begin{enumerate}
    \item
      Each functor $L_iF$ is well-defined up to natural isomorphism
      and additive.

    \item
      $L_0F(A) \cong F(A)$ and $L_0F(f) = F(f)$ for each object $A
      \in \abcat{A}$ and morphism $f \in \abcat{A}$.

    \item
      If $P$ is projective, then $L_iF(P) = 0$ for $i > 0$.

    \item
      The left derived functors of $F$ are a homological universal
      $\delta$-functor.
  \end{enumerate}
\end{proposition}

\begin{remark}
  The above discussion assumes covariant functors, but the
  construction of left and right derived functors can easily be
  carried out for a contravariant functor $F: \abcat{A} \to \abcat{B}$.
  Since $F: \op{\abcat{A}} \to \abcat{B}$ is covariant and an
  injective resolution in $\op{\abcat{A}}$ becomes a projective
  resolution in $\abcat{A}$, one can compute right derived functors
  when $F$ is left exact and $\abcat{A}$ has enough projectives, and
  dually for left derived functors.
\end{remark}

We close out this chapter by introducing the two most important
examples of these derived functors, followed by some final remarks.

\begin{definition}
  Let $\abcat{A}$ be an abelian category with enough injectives, and
  fix an object $A \in \abcat{A}$.
  Then the functor $\Hom_{\abcat{A}}(A, -)$ is left exact, and its
  suite of classical right derived functors are the associated
  \emph{Ext functors}
  \[
    \Ext{i}{\abcat{A}}(A, -) := R^i\Hom_{\abcat{A}}(A, -): \abcat{A} \to \Ab.
  \]
  \vspace{-24pt}
\end{definition}

\begin{definition}
  Fix an $R$-module $N$.
  Then the functor $- \tensor_R N$ is right exact, and its suite of
  classical left derived functors are the associated \emph{Tor functors}
  \[
    \Tor{i}{R}(-, N) := L_i (- \tensor_R N): \Mod{R} \to \Ab.
  \]
  \vspace{-24pt}
\end{definition}

\iftrue
The names given to these derived functors are suggestive of some of
their applications.
Given an $R$-module $M$, the Ext groups $\Ext{1}{R}(M, N)$ are
related to the \emph{module extensions} of $M$ by another $R$-module $N$.
If $r \in R$ is not a zero divisor, the Tor group
$\Tor{1}{R}(\quot{R}{(r)}, M)$ is the \emph{$r$-torsion subgroup} of $M$.
We will not discuss these applications any further, but these are the
subject of \cite[Chapter~3]{weibel}.
\else
The names given to these derived functors are suggestive of some of
their applications.
The first Ext group $\Ext{1}{\abcat{A}}(A, B)$ coincides with the
first \emph{Yoneda extension group} of $A$ and $B$.
This is the abelian group of equivalence classes of \emph{extensions}
of the object $B$ by $A$, i.e. short exact sequences
\[
  \begin{tikzcd}
    0 & A & E & B & 0
    \arrow[from=1-1, to=1-2]
    \arrow[from=1-2, to=1-3]
    \arrow[from=1-3, to=1-4]
    \arrow[from=1-4, to=1-5]
  \end{tikzcd}
\]
in $\abcat{A}$, with two extensions deemed equivalent if there is a
commutative diagram
\[
  \begin{tikzcd}
    0 & A & E & B & 0 \\
    0 & A & E & B & 0.
    \arrow[from=1-1, to=1-2]
    \arrow[from=1-2, to=1-3]
    \arrow[Rightarrow, no head, from=1-2, to=2-2]
    \arrow[from=1-3, to=1-4]
    \arrow["\cong"', from=1-3, to=2-3]
    \arrow[from=1-4, to=1-5]
    \arrow[Rightarrow, no head, from=1-4, to=2-4]
    \arrow[from=2-1, to=2-2]
    \arrow[from=2-2, to=2-3]
    \arrow[from=2-3, to=2-4]
    \arrow[from=2-4, to=2-5]
  \end{tikzcd}
\]
The group operation is the \emph{Baer sum} of extensions, and the
split exact sequence
\[
  \begin{tikzcd}[cramped]
    0 & A & {A \biprod B} & B & 0
    \arrow[from=1-1, to=1-2]
    \arrow[from=1-2, to=1-3]
    \arrow[from=1-3, to=1-4]
    \arrow[from=1-4, to=1-5]
  \end{tikzcd}
\]
is its identity element.
Since the details of this group structure are rather technical and
require the categorical notions of \emph{pullbacks} and
\emph{pushouts}, we defer to \cite[Section~3.4]{weibel}.
The interpretation of Tor is less relevant to us, but the relation
between the first Tor group and \emph{torsion subgroups} is
elaborated on in \cite[Section~3.1]{weibel}.
\fi

A useful fact about Ext that we will use later is the following.

\begin{lemma}
  \label{lemma_ext_preserved_by_equiv}
  Let $\abcat{A} \to \abcat{B}$ be an equivalence of abelian
  categories both with enough projectives.
  Then for all $A, B \in \abcat{A}$ and $i \in \zpos$, there is an isomorphism
  \[
    \Ext{i}{\abcat{A}}(A, B) \cong \Ext{i}{\abcat{B}}(F(A), F(B)).
  \]
  \vspace{-24pt}
\end{lemma}

\begin{proof}
  Fix an injective resolution $B \to \cochaincomp{I}$.
  Since $F$ is fully faithful, we have
  \[
    \Hom_{\abcat{A}}(A, I^j) \cong \Hom_{\abcat{B}}(F(A), F(I^j))
  \]
  for each $j \in \zpos$.
  As in \cref{rem_exact_limit_colimit_pres}, it turns out that an
  equivalences of categories also preserves limits and colimits, and
  from this it follows that $F$ is an exact functor that preserves injectives.
  This shows that $F(B) \to F(\cochaincomp{I})$ is a projective resolution, so
  \[
    \Ext{i}{\abcat{A}}(A, B)
    = H^i \Hom_{\abcat{A}}(A, \cochaincomp{I})
    \cong H^i \Hom_{\abcat{B}}(F(A), F(\cochaincomp{I}))
    = \Ext{i}{\abcat{A}}(F(A), F(B))
  \]
  for $i \in \zpos$ as claimed.
\end{proof}

\begin{remark}
  Since $\Hom_R$ and $\tensor_R$ are \emph{bifunctors}, one also
  could have decided to construct Ext and Tor by instead taking the
  derived functors $R^i \Hom_R(-, N)$ and $L_i (M \tensor_R -)$.
  Through a careful process of \emph{balancing}, one sees that in
  fact these functors will be naturally isomorphic to Ext and Tor as
  we have defined them above.
  The details are lengthy and immaterial to the goals of this thesis,
  so we refer the interested reader to \cite[Section~2.7]{weibel}.
\end{remark}

\begin{remark}
  \label{rem_tor_computed_by_flat_res}
  The Tor functors provide the most straightforward example of a
  functor which benefits from being computed by acyclic resolutions
  as described in \cref{rem_computing_with_F_acyclics}.
  Retaining the hypotheses of \cref{prop_classical_right_derfunc}, an
  object $X \in \abcat{A}$ is $F$-acyclic if $L_iF(X) = 0$ for $i >
  0$, and similarly we have $L_iF(A) \cong
  H^{-i}(F(\cochaincomp{X}))$ for an $F$-acyclic resolution
  $\cochaincomp{X} \to A$.
  In particular, $\Tor{i}{R}(A, N) = 0$ for $i > 0$ whenever $A$ is a
  flat $R$-module, so Tor can be computed using flat resolutions.
\end{remark}
