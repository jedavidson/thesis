\section{Cohomology and homotopy}
\label{sect_cohomology_homotopy}

Cohomology is perhaps the most valuable data associated to a cochain
complex, so we now introduce a circle of related ideas.

\begin{proposition}
  \label{prop_cochain_map_induced_cohomology}
  A cochain map $f: \cochaincomp{C} \to \cochaincomp{D}$ induces
  morphisms $Z^i(\cochaincomp{C}) \to Z^i(\cochaincomp{D})$ and
  $B^i(\cochaincomp{C}) \to B^i(\cochaincomp{D})$.
  Thus, cohomology is an additive functor $H^i: \Ch{\abcat{A}} \to \abcat{A}$.
\end{proposition}

\begin{proof}
  The argument for abelian categories is a technical use of universal
  properties, but it is instructive to at least see the explicit
  construction for $\Mod{R}$.
  Using the commutative diagram for $f$, it is not hard to see that
  each map $f^i$ sends cocycles to cocycles and coboundaries to
  coboundaries in degree $i$.
  Thus the map
  \[
    z + B^i(\cochaincomp{C}) \mapsto f^i(z) + B^i(\cochaincomp{D})
  \]
  is the desired module homomorphism $H^i(f): H^i(\cochaincomp{C})
  \to H^i(\cochaincomp{D})$, and it follows that
  \[
    H^i(gf) = H^i(g) H^i(f) \mathand H^i(f + g) = H^i(f) + H^i(g)
  \]
  for all compatible cochain maps $f$ and $g$, so $H^i$ is an
  additive functor as claimed.
\end{proof}

% \begin{proof}
%     The argument for arbitrary abelian categories is a technical
% exercise in \emph{diagram chasing}, so we will consider $\abcat{A}
% = \Mod{R}$ instead for simplicity.
%     Using the commutative diagram for $f$, we see that $f$ must
% send cocycles to cocycles and coboundaries to coboundaries in degree $n$.
%     In particular, $H^n(f)$ is the map $z + B^n(\cochaincomp{C})
% \mapsto f^n(z) + B^n(\cochaincomp{D})$, and clearly we have $H^n(f
% + g) = H^n(f) + H^n(g)$ in $\abcat{A}$.
%     Thus $H^n$ is additive.
% \end{proof}

% For the more general argument, there is this thread:
% https://math.stackexchange.com/questions/4163943/functoriality-of-homology

Since $\Ch{\abcat{A}}$ is abelian, it makes sense to consider short
exact sequences of complexes.
This leads to one of the cornerstone theorems of homological algebra.

\begin{theorem}[Long exact cohomology sequence]
  \label{thm_long_exact_cohomology_seq}
  Consider a short exact sequence
  \[
    \begin{tikzcd}
      \cochaincomp{0} \arrow[r]
      & \cochaincomp{A} \arrow[r, "f"]
      & \cochaincomp{B} \arrow[r, "g"]
      & \cochaincomp{C} \arrow[r]
      & \cochaincomp{0}
    \end{tikzcd}
  \]
  in $\Ch{\abcat{A}}$.
  Then there exist \emph{connecting morphisms} $\partial^i:
  H^i(\cochaincomp{C}) \to H^{i + 1}(\cochaincomp{A})$ such that
  \[
    \begin{tikzcd}
      \cdots & {H^i(\cochaincomp{A})} && {H^i(\cochaincomp{B})} &&
      {H^i(\cochaincomp{C})} \\
      \\
      & {H^{i + 1}(\cochaincomp{A})} && {H^{i + 1}(\cochaincomp{A})}
      && {H^{i + 1}(\cochaincomp{A})} & \cdots
      \arrow[dashed, from=1-1, to=1-2]
      \arrow["{H^i(f)}", from=1-2, to=1-4]
      \arrow["{H^i(g)}", from=1-4, to=1-6]
      \arrow["{\partial^i}"{description}, dashed, from=1-6, to=3-2]
      \arrow["{H^{i + 1}(f)}"', from=3-2, to=3-4]
      \arrow["{H^{i + 1}(g)}"', from=3-4, to=3-6]
      \arrow[dashed, from=3-6, to=3-7]
    \end{tikzcd}
  \]
  is an exact sequence in $\abcat{A}$.
  Moreover, this construction is natural in the sense that given a
  morphism of short exact sequences of complexes
  \[
    \begin{tikzcd}
      \cochaincomp{0} & \cochaincomp{A} & \cochaincomp{B} &
      \cochaincomp{C} & \cochaincomp{0} \\
      \cochaincomp{0} & \cochaincomp{(A')} & \cochaincomp{(B')} &
      \cochaincomp{(C')} & \cochaincomp{0},
      \arrow[from=1-1, to=1-2]
      \arrow[from=1-2, to=1-3]
      \arrow[from=1-2, to=2-2]
      \arrow[from=1-3, to=1-4]
      \arrow[from=1-3, to=2-3]
      \arrow[from=1-4, to=1-5]
      \arrow[from=1-4, to=2-4]
      \arrow[from=2-1, to=2-2]
      \arrow[from=2-2, to=2-3]
      \arrow[from=2-3, to=2-4]
      \arrow[from=2-4, to=2-5]
    \end{tikzcd}
  \]
  in $\Ch{\abcat{A}}$, there are commutative squares
  \[
    \begin{tikzcd}[cramped]
      {H^i(\cochaincomp{C})} & {H^{i + 1}(\cochaincomp{A})} \\
      {H^i(\cochaincomp{(C')})} & {H^{i + 1}(\cochaincomp{(A')})}.
      \arrow["{\partial^i}", from=1-1, to=1-2]
      \arrow[from=1-1, to=2-1]
      \arrow[from=1-2, to=2-2]
      \arrow["{(\partial')^i}"', from=2-1, to=2-2]
    \end{tikzcd}
  \]
  \vspace{-18pt}
\end{theorem}

\begin{proof}
  The construction of the connecting morphisms requires the snake lemma.
  We defer the technical details to the discussion in
  \cite[Section~1.3]{weibel}.
\end{proof}

For computational purposes, it is often useful to be able to detect
when two complexes have the same cohomology.
There are two related ways to do this, the latter of which is
inspired by the concept in topology of the same name.

\begin{definition}
  A cochain map $f: \cochaincomp{C} \to \cochaincomp{D}$ is called a
  \emph{quasi-isomorphism} (or \emph{qis}) if each of the maps
  $H^i(f): H^i(\cochaincomp{C}) \to H^i(\cochaincomp{D})$ induced by
  cohomology is an isomorphism.
  In case such a map $f$ exists, $\cochaincomp{C}$ and
  $\cochaincomp{D}$ are said to be \emph{quasi-isomorphic} complexes.
\end{definition}

% Drawing from the concept of the same name in topology and the
% eponymous area of \emph{homotopy theory}, there is another way in
% which cohomology can be preserved, albeit in a more elaborate and
% less direct manner.
%  two cochain maps which is notable for its interaction with cohomology.
% The inspiration draws again from the concept of the same name in
% topology and the eponymous area of \emph{homotopy theory}.
% A short but potentially useful prelude is the introduction of
% \cite[Section~IX.4.3]{algebra_chapter_0}.

\begin{definition}
  A cochain map $f: \cochaincomp{C} \to \cochaincomp{D}$ is said to
  be \emph{null-homotopic} if there is a collection of morphisms
  $\{s^i: C^i \to D^{i - 1}\}_{i \in \bb{Z}}$, called a
  \emph{contraction} of $f$, such that $f = d_D s + s d_C$.
  We say that $f$ is \emph{homotopic} to another cochain map $g:
  \cochaincomp{C} \to \cochaincomp{D}$ if their difference $f - g$ is
  null-homotopic.
  We write $f \sim g$ and call a contraction of $f - g$ a
  \emph{homotopy} from $f$ to $g$:
  \[
    \begin{tikzcd}[cramped]
      \cdots && {C^{i - 1}} && {C^i} && {C^{i + 1}} && \cdots \\
      \\
      \cdots && {D^{i - 1}} && {D^i} && {D^{i + 1}} && \cdots.
      \arrow[from=1-1, to=1-3]
      \arrow["{d_C}", from=1-3, to=1-5]
      \arrow[from=1-3, to=3-1]
      \arrow["{f}"', shift right, from=1-3, to=3-3]
      \arrow["{g}", shift left, from=1-3, to=3-3]
      \arrow["{d_C}", from=1-5, to=1-7]
      \arrow["{s}"{description}, from=1-5, to=3-3]
      \arrow["{f}"', shift right, from=1-5, to=3-5]
      \arrow["{g}", shift left, from=1-5, to=3-5]
      \arrow[from=1-7, to=1-9]
      \arrow["{s}"{description}, from=1-7, to=3-5]
      \arrow["{f}"', shift right, from=1-7, to=3-7]
      \arrow["{g}", shift left, from=1-7, to=3-7]
      \arrow[from=1-9, to=3-7]
      \arrow[from=3-1, to=3-3]
      \arrow["{d_D}"', from=3-3, to=3-5]
      \arrow["{d_D}"', from=3-5, to=3-7]
      \arrow[from=3-7, to=3-9]
    \end{tikzcd}
  \]
  Moreover, we say that $f$ is a \emph{homotopy equivalence} if $f$
  and $g$ are \emph{homotopy inverses} of each other, i.e. $gf \sim
  \id_{\cochaincomp{C}}$ and $fg \sim \id_{\cochaincomp{D}}$.
\end{definition}

\begin{corollary}
  Let $f,\, g: \cochaincomp{C} \to \cochaincomp{D}$ be cochain maps.
  If $f$ is null-homotopic, then each morphism $H^i(f)$ is zero, and
  if $f \sim g$, then $H^i(f) = H^i(g)$.
\end{corollary}

In other words, homotopic cochain maps induce the same morphisms at
the level of cohomology.
This implies that any homotopy equivalence is also a quasi-isomorphism.
Though the converse is generally false, we have the following
important partial result in this direction.
This takes considerable effort to prove, but one can find a thorough
account of the details in \cite[Section~IX.5.4]{algebra_chapter_0}.

\begin{proposition}[{\cite[Theorem~IX.5.9]{algebra_chapter_0}}]
  \label{prop_qis_is_homotopy_equiv_partial_conv}
  If $\cochaincomp{C}$ and $\cochaincomp{D}$ are both bounded below
  complexes of injectives or both bounded above complexes of
  projectives, then any quasi-isomorphism $f: \cochaincomp{C} \to
  \cochaincomp{D}$ is a homotopy equivalence.
\end{proposition}
