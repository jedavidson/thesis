\section{The category of complexes}
\label{sect_complexes}

Homological algebra can be interpreted as the study of a certain
category, so our first task of this chapter is to define it.
We start with its objects.

\begin{definition}
  \label{def_complexes}
  A \emph{chain complex} in $\abcat{A}$ is a collection of objects
  $\cochaincomp{C} = \{C_i\}_{i \in \bb{Z}}$, called the
  \emph{chains}, and morphisms $\chaincomp{d} = \{d_i: C_i \to C_{i -
  1}\}_{i \in \bb{Z}}$, called the \emph{differentials}, such that
  $d_i d_{i + 1} = 0$ for each $i \in \bb{Z}$:
  \[
    \begin{tikzcd}
      \cdots \arrow[r]
      & C_{i + 1} \arrow[r, "d_{i + 1}"]
      & C_i \arrow[r, "d_i"]
      & C_{i - 1} \arrow[r]
      & \cdots.
    \end{tikzcd}
  \]
  Dually, a \emph{cochain complex} in $\abcat{A}$ is a collection of
  objects $\cochaincomp{C} = \{C^i\}_{i \in \bb{Z}}$, called the
  \emph{cochains}, and morphisms $\cochaincomp{d} = \{d^i: C^i \to
  C^{i + 1}\}_{i \in \bb{Z}}$, also called the \emph{differentials},
  such that $d^{i} d^{i - 1} = 0$ for each $i \in \bb{Z}$:
  \[
    \begin{tikzcd}
      \cdots \arrow[r]
      & C^{i - 1} \arrow[r, "d^{i - 1}"]
      & C^i \arrow[r, "d^i"]
      & C^{i + 1} \arrow[r]
      & \cdots.
    \end{tikzcd}
  \]
  \vspace{-24pt}
\end{definition}

A common topological slogan for the conditions imposed on
differentials is that ``the boundary of the boundary of a sphere is trivial''.
The subscripts and superscripts are often suppressed so that in
either case the condition can be interpreted informally as saying $d^2 = 0$.
Similar informalities within homological algebra abound.

The index $i$ of an individual chain $C_i$ or cochain $C^i$ is called
its \emph{degree}.
With this, we see that the key difference between chain and cochain
complexes is that chain differentials lower degree while cochain
differentials raise it.
For the most part, we will emphasise cochain complexes in this thesis
in order to reflect the historical origins of derived categories and
the preferences that have developed in the literature as a result.
There is no loss of generality in this choice, since to a certain
extent the decision to work with either type of complex over the
other is an aesthetic one.
The following proposition makes this more formal.

\begin{proposition}[Chain-cochain duality]
  Any chain complex $\chaincomp{C}$ may be regarded as a cochain
  complex $\cochaincomp{C}$ by inverting degrees, i.e. by setting
  $C^i = C_{-i}$ and $d^i = d_{-i}$.
\end{proposition}

In practice, the easiest complexes to work with are those which
satisfy some boundedness condition.
This will be a key theme in this thesis, since the theory of derived
categories without such assumptions in place becomes much more involved.

\begin{definition}
  \label{def_bounded_cochain_complexes}
  A complex $\cochaincomp{C}$ is said to be
  \begin{itemize}
    \item
      \emph{bounded above} (resp. \emph{bounded below}) if $C^i = 0$
      for all sufficiently large (resp. sufficiently small) degrees $i$;

    \item
      \emph{bounded} if it is both bounded above and bounded below;

    \item
      \emph{positively} (resp. \emph{negatively}) \emph{graded} if
      $C^i = 0$ for all $i < 0$ (resp. $i > 0$).
  \end{itemize}
\end{definition}

For a cochain complex $\cochaincomp{C}$, we call
$Z^i(\cochaincomp{C}) := \ker(d^i)$ and $B^i(\cochaincomp{C}) :=
\im(d^{i - 1})$ the $i$th \emph{cocycle} and \emph{coboundary} objects.
It is not difficult to show that the condition $d^2 = 0$ implies that
$B^i(\cochaincomp{C}) \leq Z^i(\cochaincomp{C})$, and so we may form
the $i$th \emph{cohomology} object
\[
  H^i(\cochaincomp{C})
  := \quot{Z^i(\cochaincomp{C})}{B^i(\cochaincomp{C})}.
\]
The same considerations naturally apply to any chain complex
$\chaincomp{C}$, and we call
\[
  Z_i(\chaincomp{C}) := \ker(d_i),
  \quad
  B_i(\chaincomp{C}) := \im(d_{i + 1})
  \mathand
  H_i(\chaincomp{C}) := \quot{Z_i(\chaincomp{C})}{B_i(\chaincomp{C})}
\]
the $i$th \emph{cycle}, \emph{boundary} and \emph{homology} objects
of $\chaincomp{C}$ respectively.

The most interesting examples of chain and cochain complexes arise
from applications, and we will see one in \cref{sect_coh_sheaf_cohom}.
For now, we collect some more remedial examples in order to introduce
some useful terminology.

\begin{example}
  Each $A \in \abcat{A}$ may be regarded as a bounded complex
  $\cochaincomp{A}$ \emph{concentrated in degree $n$}, i.e. such that
  $A^n = A$, $A^i = 0$ for all $i \neq n$ and $d = 0$.
  In this case, we have $H^n(\cochaincomp{A}) \cong A$ and
  $H^i(\cochaincomp{A}) = 0$ whenever $i \neq n$.
  Sometimes, we will write $\cochaincomp{A} = A[n]$.
\end{example}

\begin{example}
  Since $\im(f) = \ker(g)$ implies that $g f = 0$, any exact sequence
  in $\abcat{A}$ can be defined as a complex.
  More generally, a complex $\cochaincomp{C}$ is \emph{exact} if
  $Z^i(\cochaincomp{C}) = B^i(\cochaincomp{C})$ or \emph{acyclic} if
  $H^i(\cochaincomp{C}) = 0$ for all $i \in \bb{Z}$, and it is
  immediate that these two notions are in fact equivalent.
  Studying the cohomology objects of $\cochaincomp{C}$ thus provides
  insight into the extent to which $\cochaincomp{C}$ fails to be
  exact degreewise.
\end{example}

We now define a suitable notion of a morphism of complexes.

\begin{definition}
  \label{def_cochain_map}
  Let $\cochaincomp{C}$ and $\cochaincomp{D}$ be cochain complexes in
  $\abcat{A}$ with associated differentials $d_C$ and $d_D$.
  A \emph{morphism of cochain complexes} (or \emph{cochain map}) $f:
  \cochaincomp{C} \to \cochaincomp{D}$ is a collection of morphisms
  $\{f^i: C^i \to D^i\}_{i \in \bb{Z}}$ such that
  $d_D f = f d_C$, i.e. there is a commutative diagram
  \[
    \begin{tikzcd}
      \cdots \arrow[r]
      & C^{i - 1} \arrow[d, "f"'] \arrow[r, "d_C"]
      & C^i \arrow[d, "f"] \arrow[r, "d_C"]
      & C^{i + 1} \arrow[d, "f"] \arrow[r]
      & \cdots
      \\
      \cdots \arrow[r]
      & D^{i - 1} \arrow[r, "d_D"']
      & D^i \arrow[r, "d_D"']
      & D^{i + 1} \ar[r]
      & \cdots.
    \end{tikzcd}
  \]
  We say that $f$ is an \emph{isomorphism} if each $f^i$ is an isomorphism.
\end{definition}

\begin{definition}
  The \emph{category of cochain complexes} in $\abcat{A}$ is the
  category $\Ch{\abcat{A}}$ whose objects and morphisms are as
  defined in \cref{def_complexes} and \cref{def_cochain_map}, with
  composition defined by the degreewise composition of cochain maps.
  Its full subcategories $\Ch[-]{\abcat{A}}$, $\Ch[+]{\abcat{A}}$ and
  $\Ch[b]{\abcat{A}}$ are the categories of \emph{bounded above},
  \emph{bounded below} and \emph{bounded} cochain complexes in
  $\abcat{A}$ respectively.
\end{definition}

It turns out that $\Ch{\abcat{A}}$ is also an abelian category, and
it inherits much of this structure from the base category $\abcat{A}$.
A partial account of the details can be found in
\cite[Section~1.2]{weibel}, but we shall state the main ideas.
The preadditive structure comes from the fact that each hom-set is an
abelian group under degreewise addition of cochain maps, and the
composition of cochain maps is clearly bilinear with respect to this addition.
This upgrades into an additive structure upon noticing that the zero
object of $\abcat{A}$ concentrated in degree 0 plays the role of the
zero object (denoted $\cochaincomp{0}$) of $\Ch{\abcat{A}}$, and that
the degreewise direct sum of any finite collection of complexes
$\{\cochaincomp{C_i}\}$ is a suitable biproduct for complexes.
Given a cochain map $f: \cochaincomp{C} \to \cochaincomp{D}$, the complex
\[
  \begin{tikzcd}
    \cdots & {\ker(f^{i - 1})} && {\ker(f^i)} && {\ker(f^{i + 1})} & \cdots
    \arrow[from=1-1, to=1-2]
    \arrow["{d_C^i \, \circ \, \ker(f^{i - 1})}", from=1-2, to=1-4]
    \arrow["{d_C^{i + 1} \, \circ \, \ker(f^i)}", from=1-4, to=1-6]
    \arrow[from=1-6, to=1-7]
  \end{tikzcd}
\]
is a suitable kernel of $f$, with cokernels constructed similarly.
This shows that $\Ch{\abcat{A}}$ is abelian.
Due to this fact, it is possible to consider complexes of complexes
in $\abcat{A}$, which will play a role in \cref{chap_derived_categories}.
We give only the cochain definition.

\begin{definition}
  \label{def_double_complex}
  A \emph{double cochain complex} $\cochainbicomp{C}$ in $\abcat{A}$
  consists of a collection of objects $\{C^{p, q}\}_{p, q \in
  \bb{Z}}$ together with \emph{horizontal} and \emph{vertical} differentials
  \[
    d_h^{p, q}: C^{p, q} \to C^{p + 1, q}
    \mathand
    d_v^{p, q}: C^{p, q} \to C^{p, q + 1}
  \]
  subject to the anticommutativity relation $d_v d_h + d_h d_v = 0$,
  and such that each row and column in the following lattice diagram
  is a chain complex:
  \[
    \begin{tikzcd}[cramped]
      & \vdots & \vdots & \vdots \\
      \cdots & {C^{p - 1, q - 1}} & {C^{p, q - 1}} & {C^{p + 1, q -
      1}} & \cdots \\
      \cdots & {C^{p - 1, q}} & {C^{p, q}} & {C^{p + 1, q}} & \cdots \\
      \cdots & {C^{p - 1, q + 1}} & {C^{p, q + 1}} & {C^{p + 1, q +
      1}} & \cdots \\
      & \vdots & \vdots & \vdots
      \arrow[from=1-2, to=2-2]
      \arrow[from=1-3, to=2-3]
      \arrow[from=1-4, to=2-4]
      \arrow[from=2-1, to=2-2]
      \arrow["{d_h}", from=2-2, to=2-3]
      \arrow["{d_v}"', from=2-2, to=3-2]
      \arrow["{d_h}", from=2-3, to=2-4]
      \arrow["{d_v}"', from=2-3, to=3-3]
      \arrow[from=2-4, to=2-5]
      \arrow["{d_v}"', from=2-4, to=3-4]
      \arrow[from=3-1, to=3-2]
      \arrow["{d_h}"', from=3-2, to=3-3]
      \arrow["{d_v}"', from=3-2, to=4-2]
      \arrow["{d_h}"', from=3-3, to=3-4]
      \arrow["{d_v}"', from=3-3, to=4-3]
      \arrow[from=3-4, to=3-5]
      \arrow["{d_v}"', from=3-4, to=4-4]
      \arrow[from=4-1, to=4-2]
      \arrow["{d_h}"', from=4-2, to=4-3]
      \arrow[from=4-2, to=5-2]
      \arrow["{d_h}"', from=4-3, to=4-4]
      \arrow[from=4-3, to=5-3]
      \arrow[from=4-4, to=4-5]
      \arrow[from=4-4, to=5-4]
    \end{tikzcd}
  \]
\end{definition}

\begin{remark}[The sign trick]
  For a fixed row $q$, the vertical differentials $d_v^{\bullet, q}$
  in a double complex do not form a morphism of the complexes
  $C^{\bullet, q}$ to $C^{\bullet, q + 1}$.
  This can be remedied by introducing a sign, i.e. by replacing
  $d_v^{p, q}$ with $(-1)^p d_v^{p, q}$.
  This has the effect of making squares in the above lattice
  alternate between commutativity and anticommutativity.
\end{remark}
