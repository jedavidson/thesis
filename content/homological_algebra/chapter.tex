\chapter{Homological algebra}
\label{chap_homological}
% {
%     \setlength{\epigraphwidth}{0.52 \textwidth}
%     \epigraph{
%         \itshape
%         It is less than four years since cohomological methods
% (i.e. methods of homological algebra) were introduced into
% algebraic geometry in Serre's fundamental paper, and it seems
% certain that they are to overflow the part of mathematics in the
% coming years, from the foundations up to the most advanced parts.
%     }{Alexander Grothendieck \cite{grothendeick_ha_quote}}
% }

As we have attempted to communicate in
\cref{chap_abelian_categories}, abelian categories are such a
bountiful abstraction because they grant a category the power of
exact sequences.
At the same time, exactness is not a foregone conclusion for any
sequence of morphisms arising in practice, and a mathematically
interesting problem in its own right is to attempt to understand `how
far' from exactness a sequence is.
This is the fundamental problem studied in \emph{homological
algebra}, which has proved to be an irreplaceable technical tool in
the modern practice of algebraic geometry, algebraic topology and a
whole host of branches of mathematics.

The goal of this chapter is to develop the required background in
this area needed to delve into the derived category of an abelian
category in \cref{chap_derived_categories}, following \cite{weibel}.
Homological algebra is a topic best served with a more tangible
motivating application in mind, and this is traditionally sourced
from algebraic topology.
Since such material is outside of the scope of this thesis, we will
continue to tether our intuitions to $\Mod{R}$.

In all that follows, $\abcat{A}$ and $\abcat{B}$ are always abelian categories.

\section{The category of complexes}
\label{sect_complexes}

Homological algebra can be interpreted as the study of a certain
category, so our first task of this chapter is to define it.
We start with its objects.

\begin{definition}
  \label{def_complexes}
  A \emph{chain complex} in $\abcat{A}$ is a collection of objects
  $\cochaincomp{C} = \{C_i\}_{i \in \bb{Z}}$, called the
  \emph{chains}, and morphisms $\chaincomp{d} = \{d_i: C_i \to C_{i -
  1}\}_{i \in \bb{Z}}$, called the \emph{differentials}, such that
  $d_i d_{i + 1} = 0$ for each $i \in \bb{Z}$:
  \[
    \begin{tikzcd}
      \cdots \arrow[r]
      & C_{i + 1} \arrow[r, "d_{i + 1}"]
      & C_i \arrow[r, "d_i"]
      & C_{i - 1} \arrow[r]
      & \cdots.
    \end{tikzcd}
  \]
  Dually, a \emph{cochain complex} in $\abcat{A}$ is a collection of
  objects $\cochaincomp{C} = \{C^i\}_{i \in \bb{Z}}$, called the
  \emph{cochains}, and morphisms $\cochaincomp{d} = \{d^i: C^i \to
  C^{i + 1}\}_{i \in \bb{Z}}$, also called the \emph{differentials},
  such that $d^{i} d^{i - 1} = 0$ for each $i \in \bb{Z}$:
  \[
    \begin{tikzcd}
      \cdots \arrow[r]
      & C^{i - 1} \arrow[r, "d^{i - 1}"]
      & C^i \arrow[r, "d^i"]
      & C^{i + 1} \arrow[r]
      & \cdots.
    \end{tikzcd}
  \]
  \vspace{-24pt}
\end{definition}

A common topological slogan for the conditions imposed on
differentials is that ``the boundary of the boundary of a sphere is trivial''.
The subscripts and superscripts are often suppressed so that in
either case the condition can be interpreted informally as saying $d^2 = 0$.
Similar informalities within homological algebra abound.

The index $i$ of an individual chain $C_i$ or cochain $C^i$ is called
its \emph{degree}.
With this, we see that the key difference between chain and cochain
complexes is that chain differentials lower degree while cochain
differentials raise it.
For the most part, we will emphasise cochain complexes in this thesis
in order to reflect the historical origins of derived categories and
the preferences that have developed in the literature as a result.
There is no loss of generality in this choice, since to a certain
extent the decision to work with either type of complex over the
other is an aesthetic one.
The following proposition makes this more formal.

\begin{proposition}[Chain-cochain duality]
  Any chain complex $\chaincomp{C}$ may be regarded as a cochain
  complex $\cochaincomp{C}$ by inverting degrees, i.e. by setting
  $C^i = C_{-i}$ and $d^i = d_{-i}$.
\end{proposition}

In practice, the easiest complexes to work with are those which
satisfy some boundedness condition.
This will be a key theme in this thesis, since the theory of derived
categories without such assumptions in place becomes much more involved.

\begin{definition}
  \label{def_bounded_cochain_complexes}
  A complex $\cochaincomp{C}$ is said to be
  \begin{itemize}
    \item
      \emph{bounded above} (resp. \emph{bounded below}) if $C^i = 0$
      for all sufficiently large (resp. sufficiently small) degrees $i$;

    \item
      \emph{bounded} if it is both bounded above and bounded below;

    \item
      \emph{positively} (resp. \emph{negatively}) \emph{graded} if
      $C^i = 0$ for all $i < 0$ (resp. $i > 0$).
  \end{itemize}
\end{definition}

For a cochain complex $\cochaincomp{C}$, we call
$Z^i(\cochaincomp{C}) := \ker(d^i)$ and $B^i(\cochaincomp{C}) :=
\im(d^{i - 1})$ the $i$th \emph{cocycle} and \emph{coboundary} objects.
It is not difficult to show that the condition $d^2 = 0$ implies that
$B^i(\cochaincomp{C}) \leq Z^i(\cochaincomp{C})$, and so we may form
the $i$th \emph{cohomology} object
\[
  H^i(\cochaincomp{C})
  := \quot{Z^i(\cochaincomp{C})}{B^i(\cochaincomp{C})}.
\]
The same considerations naturally apply to any chain complex
$\chaincomp{C}$, and we call
\[
  Z_i(\chaincomp{C}) := \ker(d_i),
  \quad
  B_i(\chaincomp{C}) := \im(d_{i + 1})
  \mathand
  H_i(\chaincomp{C}) := \quot{Z_i(\chaincomp{C})}{B_i(\chaincomp{C})}
\]
the $i$th \emph{cycle}, \emph{boundary} and \emph{homology} objects
of $\chaincomp{C}$ respectively.

The most interesting examples of chain and cochain complexes arise
from applications, and we will see one in \cref{sect_coh_sheaf_cohom}.
For now, we collect some more remedial examples in order to introduce
some useful terminology.

\begin{example}
  Each $A \in \abcat{A}$ may be regarded as a bounded complex
  $\cochaincomp{A}$ \emph{concentrated in degree $n$}, i.e. such that
  $A^n = A$, $A^i = 0$ for all $i \neq n$ and $d = 0$.
  In this case, we have $H^n(\cochaincomp{A}) \cong A$ and
  $H^i(\cochaincomp{A}) = 0$ whenever $i \neq n$.
  Sometimes, we will write $\cochaincomp{A} = A[n]$.
\end{example}

\begin{example}
  Since $\im(f) = \ker(g)$ implies that $g f = 0$, any exact sequence
  in $\abcat{A}$ can be defined as a complex.
  More generally, a complex $\cochaincomp{C}$ is \emph{exact} if
  $Z^i(\cochaincomp{C}) = B^i(\cochaincomp{C})$ or \emph{acyclic} if
  $H^i(\cochaincomp{C}) = 0$ for all $i \in \bb{Z}$, and it is
  immediate that these two notions are in fact equivalent.
  Studying the cohomology objects of $\cochaincomp{C}$ thus provides
  insight into the extent to which $\cochaincomp{C}$ fails to be
  exact degreewise.
\end{example}

We now define a suitable notion of a morphism of complexes.

\begin{definition}
  \label{def_cochain_map}
  Let $\cochaincomp{C}$ and $\cochaincomp{D}$ be cochain complexes in
  $\abcat{A}$ with associated differentials $d_C$ and $d_D$.
  A \emph{morphism of cochain complexes} (or \emph{cochain map}) $f:
  \cochaincomp{C} \to \cochaincomp{D}$ is a collection of morphisms
  $\{f^i: C^i \to D^i\}_{i \in \bb{Z}}$ such that
  $d_D f = f d_C$, i.e. there is a commutative diagram
  \[
    \begin{tikzcd}
      \cdots \arrow[r]
      & C^{i - 1} \arrow[d, "f"'] \arrow[r, "d_C"]
      & C^i \arrow[d, "f"] \arrow[r, "d_C"]
      & C^{i + 1} \arrow[d, "f"] \arrow[r]
      & \cdots
      \\
      \cdots \arrow[r]
      & D^{i - 1} \arrow[r, "d_D"']
      & D^i \arrow[r, "d_D"']
      & D^{i + 1} \ar[r]
      & \cdots.
    \end{tikzcd}
  \]
  We say that $f$ is an \emph{isomorphism} if each $f^i$ is an isomorphism.
\end{definition}

\begin{definition}
  The \emph{category of cochain complexes} in $\abcat{A}$ is the
  category $\Ch{\abcat{A}}$ whose objects and morphisms are as
  defined in \cref{def_complexes} and \cref{def_cochain_map}, with
  composition defined by the degreewise composition of cochain maps.
  Its full subcategories $\Ch[-]{\abcat{A}}$, $\Ch[+]{\abcat{A}}$ and
  $\Ch[b]{\abcat{A}}$ are the categories of \emph{bounded above},
  \emph{bounded below} and \emph{bounded} cochain complexes in
  $\abcat{A}$ respectively.
\end{definition}

It turns out that $\Ch{\abcat{A}}$ is also an abelian category, and
it inherits much of this structure from the base category $\abcat{A}$.
A partial account of the details can be found in
\cite[Section~1.2]{weibel}, but we shall state the main ideas.
The preadditive structure comes from the fact that each hom-set is an
abelian group under degreewise addition of cochain maps, and the
composition of cochain maps is clearly bilinear with respect to this addition.
This upgrades into an additive structure upon noticing that the zero
object of $\abcat{A}$ concentrated in degree 0 plays the role of the
zero object (denoted $\cochaincomp{0}$) of $\Ch{\abcat{A}}$, and that
the degreewise direct sum of any finite collection of complexes
$\{\cochaincomp{C_i}\}$ is a suitable biproduct for complexes.
Given a cochain map $f: \cochaincomp{C} \to \cochaincomp{D}$, the complex
\[
  \begin{tikzcd}
    \cdots & {\ker(f^{i - 1})} && {\ker(f^i)} && {\ker(f^{i + 1})} & \cdots
    \arrow[from=1-1, to=1-2]
    \arrow["{d_C^i \, \circ \, \ker(f^{i - 1})}", from=1-2, to=1-4]
    \arrow["{d_C^{i + 1} \, \circ \, \ker(f^i)}", from=1-4, to=1-6]
    \arrow[from=1-6, to=1-7]
  \end{tikzcd}
\]
is a suitable kernel of $f$, with cokernels constructed similarly.
This shows that $\Ch{\abcat{A}}$ is abelian.
Due to this fact, it is possible to consider complexes of complexes
in $\abcat{A}$, which will play a role in \cref{chap_derived_categories}.
We give only the cochain definition.

\begin{definition}
  \label{def_double_complex}
  A \emph{double cochain complex} $\cochainbicomp{C}$ in $\abcat{A}$
  consists of a collection of objects $\{C^{p, q}\}_{p, q \in
  \bb{Z}}$ together with \emph{horizontal} and \emph{vertical} differentials
  \[
    d_h^{p, q}: C^{p, q} \to C^{p + 1, q}
    \mathand
    d_v^{p, q}: C^{p, q} \to C^{p, q + 1}
  \]
  subject to the anticommutativity relation $d_v d_h + d_h d_v = 0$,
  and such that each row and column in the following lattice diagram
  is a chain complex:
  \[
    \begin{tikzcd}[cramped]
      & \vdots & \vdots & \vdots \\
      \cdots & {C^{p - 1, q - 1}} & {C^{p, q - 1}} & {C^{p + 1, q -
      1}} & \cdots \\
      \cdots & {C^{p - 1, q}} & {C^{p, q}} & {C^{p + 1, q}} & \cdots \\
      \cdots & {C^{p - 1, q + 1}} & {C^{p, q + 1}} & {C^{p + 1, q +
      1}} & \cdots \\
      & \vdots & \vdots & \vdots
      \arrow[from=1-2, to=2-2]
      \arrow[from=1-3, to=2-3]
      \arrow[from=1-4, to=2-4]
      \arrow[from=2-1, to=2-2]
      \arrow["{d_h}", from=2-2, to=2-3]
      \arrow["{d_v}"', from=2-2, to=3-2]
      \arrow["{d_h}", from=2-3, to=2-4]
      \arrow["{d_v}"', from=2-3, to=3-3]
      \arrow[from=2-4, to=2-5]
      \arrow["{d_v}"', from=2-4, to=3-4]
      \arrow[from=3-1, to=3-2]
      \arrow["{d_h}"', from=3-2, to=3-3]
      \arrow["{d_v}"', from=3-2, to=4-2]
      \arrow["{d_h}"', from=3-3, to=3-4]
      \arrow["{d_v}"', from=3-3, to=4-3]
      \arrow[from=3-4, to=3-5]
      \arrow["{d_v}"', from=3-4, to=4-4]
      \arrow[from=4-1, to=4-2]
      \arrow["{d_h}"', from=4-2, to=4-3]
      \arrow[from=4-2, to=5-2]
      \arrow["{d_h}"', from=4-3, to=4-4]
      \arrow[from=4-3, to=5-3]
      \arrow[from=4-4, to=4-5]
      \arrow[from=4-4, to=5-4]
    \end{tikzcd}
  \]
\end{definition}

\begin{remark}[The sign trick]
  For a fixed row $q$, the vertical differentials $d_v^{\bullet, q}$
  in a double complex do not form a morphism of the complexes
  $C^{\bullet, q}$ to $C^{\bullet, q + 1}$.
  This can be remedied by introducing a sign, i.e. by replacing
  $d_v^{p, q}$ with $(-1)^p d_v^{p, q}$.
  This has the effect of making squares in the above lattice
  alternate between commutativity and anticommutativity.
\end{remark}

\section{Cohomology and homotopy}
\label{sect_cohomology_homotopy}

Cohomology is perhaps the most valuable data associated to a cochain
complex, so we now introduce a circle of related ideas.

\begin{proposition}
  \label{prop_cochain_map_induced_cohomology}
  A cochain map $f: \cochaincomp{C} \to \cochaincomp{D}$ induces
  morphisms $Z^i(\cochaincomp{C}) \to Z^i(\cochaincomp{D})$ and
  $B^i(\cochaincomp{C}) \to B^i(\cochaincomp{D})$.
  Thus, cohomology is an additive functor $H^i: \Ch{\abcat{A}} \to \abcat{A}$.
\end{proposition}

\begin{proof}
  The argument for abelian categories is a technical use of universal
  properties, but it is instructive to at least see the explicit
  construction for $\Mod{R}$.
  Using the commutative diagram for $f$, it is not hard to see that
  each map $f^i$ sends cocycles to cocycles and coboundaries to
  coboundaries in degree $i$.
  Thus the map
  \[
    z + B^i(\cochaincomp{C}) \mapsto f^i(z) + B^i(\cochaincomp{D})
  \]
  is the desired module homomorphism $H^i(f): H^i(\cochaincomp{C})
  \to H^i(\cochaincomp{D})$, and it follows that
  \[
    H^i(gf) = H^i(g) H^i(f) \mathand H^i(f + g) = H^i(f) + H^i(g)
  \]
  for all compatible cochain maps $f$ and $g$, so $H^i$ is an
  additive functor as claimed.
\end{proof}

% \begin{proof}
%     The argument for arbitrary abelian categories is a technical
% exercise in \emph{diagram chasing}, so we will consider $\abcat{A}
% = \Mod{R}$ instead for simplicity.
%     Using the commutative diagram for $f$, we see that $f$ must
% send cocycles to cocycles and coboundaries to coboundaries in degree $n$.
%     In particular, $H^n(f)$ is the map $z + B^n(\cochaincomp{C})
% \mapsto f^n(z) + B^n(\cochaincomp{D})$, and clearly we have $H^n(f
% + g) = H^n(f) + H^n(g)$ in $\abcat{A}$.
%     Thus $H^n$ is additive.
% \end{proof}

% For the more general argument, there is this thread:
% https://math.stackexchange.com/questions/4163943/functoriality-of-homology

Since $\Ch{\abcat{A}}$ is abelian, it makes sense to consider short
exact sequences of complexes.
This leads to one of the cornerstone theorems of homological algebra.

\begin{theorem}[Long exact cohomology sequence]
  \label{thm_long_exact_cohomology_seq}
  Consider a short exact sequence
  \[
    \begin{tikzcd}
      \cochaincomp{0} \arrow[r]
      & \cochaincomp{A} \arrow[r, "f"]
      & \cochaincomp{B} \arrow[r, "g"]
      & \cochaincomp{C} \arrow[r]
      & \cochaincomp{0}
    \end{tikzcd}
  \]
  in $\Ch{\abcat{A}}$.
  Then there exist \emph{connecting morphisms} $\partial^i:
  H^i(\cochaincomp{C}) \to H^{i + 1}(\cochaincomp{A})$ such that
  \[
    \begin{tikzcd}
      \cdots & {H^i(\cochaincomp{A})} && {H^i(\cochaincomp{B})} &&
      {H^i(\cochaincomp{C})} \\
      \\
      & {H^{i + 1}(\cochaincomp{A})} && {H^{i + 1}(\cochaincomp{A})}
      && {H^{i + 1}(\cochaincomp{A})} & \cdots
      \arrow[dashed, from=1-1, to=1-2]
      \arrow["{H^i(f)}", from=1-2, to=1-4]
      \arrow["{H^i(g)}", from=1-4, to=1-6]
      \arrow["{\partial^i}"{description}, dashed, from=1-6, to=3-2]
      \arrow["{H^{i + 1}(f)}"', from=3-2, to=3-4]
      \arrow["{H^{i + 1}(g)}"', from=3-4, to=3-6]
      \arrow[dashed, from=3-6, to=3-7]
    \end{tikzcd}
  \]
  is an exact sequence in $\abcat{A}$.
  Moreover, this construction is natural in the sense that given a
  morphism of short exact sequences of complexes
  \[
    \begin{tikzcd}
      \cochaincomp{0} & \cochaincomp{A} & \cochaincomp{B} &
      \cochaincomp{C} & \cochaincomp{0} \\
      \cochaincomp{0} & \cochaincomp{(A')} & \cochaincomp{(B')} &
      \cochaincomp{(C')} & \cochaincomp{0},
      \arrow[from=1-1, to=1-2]
      \arrow[from=1-2, to=1-3]
      \arrow[from=1-2, to=2-2]
      \arrow[from=1-3, to=1-4]
      \arrow[from=1-3, to=2-3]
      \arrow[from=1-4, to=1-5]
      \arrow[from=1-4, to=2-4]
      \arrow[from=2-1, to=2-2]
      \arrow[from=2-2, to=2-3]
      \arrow[from=2-3, to=2-4]
      \arrow[from=2-4, to=2-5]
    \end{tikzcd}
  \]
  in $\Ch{\abcat{A}}$, there are commutative squares
  \[
    \begin{tikzcd}[cramped]
      {H^i(\cochaincomp{C})} & {H^{i + 1}(\cochaincomp{A})} \\
      {H^i(\cochaincomp{(C')})} & {H^{i + 1}(\cochaincomp{(A')})}.
      \arrow["{\partial^i}", from=1-1, to=1-2]
      \arrow[from=1-1, to=2-1]
      \arrow[from=1-2, to=2-2]
      \arrow["{(\partial')^i}"', from=2-1, to=2-2]
    \end{tikzcd}
  \]
  \vspace{-18pt}
\end{theorem}

\begin{proof}
  The construction of the connecting morphisms requires the snake lemma.
  We defer the technical details to the discussion in
  \cite[Section~1.3]{weibel}.
\end{proof}

For computational purposes, it is often useful to be able to detect
when two complexes have the same cohomology.
There are two related ways to do this, the latter of which is
inspired by the concept in topology of the same name.

\begin{definition}
  A cochain map $f: \cochaincomp{C} \to \cochaincomp{D}$ is called a
  \emph{quasi-isomorphism} (or \emph{qis}) if each of the maps
  $H^i(f): H^i(\cochaincomp{C}) \to H^i(\cochaincomp{D})$ induced by
  cohomology is an isomorphism.
  In case such a map $f$ exists, $\cochaincomp{C}$ and
  $\cochaincomp{D}$ are said to be \emph{quasi-isomorphic} complexes.
\end{definition}

% Drawing from the concept of the same name in topology and the
% eponymous area of \emph{homotopy theory}, there is another way in
% which cohomology can be preserved, albeit in a more elaborate and
% less direct manner.
%  two cochain maps which is notable for its interaction with cohomology.
% The inspiration draws again from the concept of the same name in
% topology and the eponymous area of \emph{homotopy theory}.
% A short but potentially useful prelude is the introduction of
% \cite[Section~IX.4.3]{algebra_chapter_0}.

\begin{definition}
  A cochain map $f: \cochaincomp{C} \to \cochaincomp{D}$ is said to
  be \emph{null-homotopic} if there is a collection of morphisms
  $\{s^i: C^i \to D^{i - 1}\}_{i \in \bb{Z}}$, called a
  \emph{contraction} of $f$, such that $f = d_D s + s d_C$.
  We say that $f$ is \emph{homotopic} to another cochain map $g:
  \cochaincomp{C} \to \cochaincomp{D}$ if their difference $f - g$ is
  null-homotopic.
  We write $f \sim g$ and call a contraction of $f - g$ a
  \emph{homotopy} from $f$ to $g$:
  \[
    \begin{tikzcd}[cramped]
      \cdots && {C^{i - 1}} && {C^i} && {C^{i + 1}} && \cdots \\
      \\
      \cdots && {D^{i - 1}} && {D^i} && {D^{i + 1}} && \cdots.
      \arrow[from=1-1, to=1-3]
      \arrow["{d_C}", from=1-3, to=1-5]
      \arrow[from=1-3, to=3-1]
      \arrow["{f}"', shift right, from=1-3, to=3-3]
      \arrow["{g}", shift left, from=1-3, to=3-3]
      \arrow["{d_C}", from=1-5, to=1-7]
      \arrow["{s}"{description}, from=1-5, to=3-3]
      \arrow["{f}"', shift right, from=1-5, to=3-5]
      \arrow["{g}", shift left, from=1-5, to=3-5]
      \arrow[from=1-7, to=1-9]
      \arrow["{s}"{description}, from=1-7, to=3-5]
      \arrow["{f}"', shift right, from=1-7, to=3-7]
      \arrow["{g}", shift left, from=1-7, to=3-7]
      \arrow[from=1-9, to=3-7]
      \arrow[from=3-1, to=3-3]
      \arrow["{d_D}"', from=3-3, to=3-5]
      \arrow["{d_D}"', from=3-5, to=3-7]
      \arrow[from=3-7, to=3-9]
    \end{tikzcd}
  \]
  Moreover, we say that $f$ is a \emph{homotopy equivalence} if $f$
  and $g$ are \emph{homotopy inverses} of each other, i.e. $gf \sim
  \id_{\cochaincomp{C}}$ and $fg \sim \id_{\cochaincomp{D}}$.
\end{definition}

\begin{corollary}
  Let $f,\, g: \cochaincomp{C} \to \cochaincomp{D}$ be cochain maps.
  If $f$ is null-homotopic, then each morphism $H^i(f)$ is zero, and
  if $f \sim g$, then $H^i(f) = H^i(g)$.
\end{corollary}

In other words, homotopic cochain maps induce the same morphisms at
the level of cohomology.
This implies that any homotopy equivalence is also a quasi-isomorphism.
Though the converse is generally false, we have the following
important partial result in this direction.
This takes considerable effort to prove, but one can find a thorough
account of the details in \cite[Section~IX.5.4]{algebra_chapter_0}.

\begin{proposition}[{\cite[Theorem~IX.5.9]{algebra_chapter_0}}]
  \label{prop_qis_is_homotopy_equiv_partial_conv}
  If $\cochaincomp{C}$ and $\cochaincomp{D}$ are both bounded below
  complexes of injectives or both bounded above complexes of
  projectives, then any quasi-isomorphism $f: \cochaincomp{C} \to
  \cochaincomp{D}$ is a homotopy equivalence.
\end{proposition}

\section{Operations on complexes}

Aside from taking cohomology, there are other useful things which one
can do with complexes.
We first introduce some more functors on $\Ch{\abcat{A}}$ which we
will need later.
The fact that these define additive functors is immediate by construction.

\begin{definition}
  Let $n \in \bb{Z}$.
  The degreewise (left) shift of complexes and cochain maps by $n$
  defines an additive endofunctor $[n]: \Ch{\abcat{A}} \to
  \Ch{\abcat{A}}$, called the \emph{$n$th shift} (or \emph{$n$th
  translation}) functor.
  Explicitly, $[n]$ sends $\cochaincomp{C}$ to the complex
  $\cochaincomp{C}[n]$ with
  \[
    {C}[n]^i = C^{i + n}
    \mathand
    d[n]^i = (-1)^n d^{i + n},
  \]
  and sends $f: \cochaincomp{C} \to \cochaincomp{D}$ to the cochain
  map $f[n]: \cochaincomp{C} \to \cochaincomp{D}$ with $f[n]^i = f^{i + n}$.
\end{definition}

For technical reasons that we shall not explain, it turns out that
introducing the sign $(-1)^n$ in the differential like this is
generally convenient when setting up the theory.
Though it seems like a small observation now, the shift complex has a
convenient property in the context of \cref{sect_triangcat}.

\begin{proposition}
  The shift functor $[n]$ is autoequivalence for all $n \in \bb{Z}$.
\end{proposition}

This shift functor is related to another common construction.
Associated to any cochain map is its \emph{mapping cone}, once again
directly inspired by topology.

\begin{definition}
  The \emph{mapping cone} of a cochain map $f: \cochaincomp{C} \to
  \cochaincomp{D}$ is the cochain complex $\cone{f}$ whose cochains
  and differentials are given by
  \[
    \cone{f}^i = C^{i + 1} \oplus D^i
    \mathand
    d^n =
    \begin{bmatrix}
      -d_C^{i + 1} & 0 \\
      -f^{i + 1} & d^i_D
    \end{bmatrix},
  \]
  with the differential acting as if by componentwise matrix
  multiplication on $C^{i + 1} \oplus D^i$.
\end{definition}

The mapping cone has two uses.
The first is that it allows one to insert any cochain map $f:
\cochaincomp{C} \to \cochaincomp{D}$ into a short exact sequence in
$\Ch{\abcat{A}}$.
In particular, we have that
\[
  \begin{tikzcd}
    \cochaincomp{0} \arrow[r]
    & \cochaincomp{D} \arrow[r]
    & \cone{f} \arrow[r]
    & \cochaincomp{C}[1] \arrow[r]
    & \cochaincomp{0}
  \end{tikzcd}
\]
is exact, where the two middle cochain maps are the canonical
injection and projection of direct sums respectively.
This leads to its second use, detecting quasi-isomorphisms.

\begin{corollary}
  \label{cor_quasi_isom_has_acyclic_cone}
  A cochain map is a quasi-isomorphism if and only if its mapping
  cone is an exact (equivalently, acyclic) complex.
\end{corollary}

\begin{proof}
  Using the identification $H^i(\cochaincomp{C}[1]) \cong H^{i +
  1}(\cochaincomp{C})$, the long exact cohomology sequence for the
  above short exact sequence is of the form
  \[
    \begin{tikzcd}
      \cdots \arrow[r]
      & H^i(\cochaincomp{D}) \arrow[r]
      & H^i(\cone{f}) \arrow[r]
      & H^{i + 1}(\cochaincomp{C}) \arrow[r, "\partial^i"]
      & H^{i + 1}(\cochaincomp{D}) \arrow[r]
      & \cdots,
    \end{tikzcd}
  \]
  where $\partial$ is the connecting map.
  It is therefore enough to show that $\partial^i = H^{i + 1}(f)$,
  which is the content of \cite[Lemma~1.5.3]{weibel}.
\end{proof}

An even more simple-minded but crucial operation on complexes is
applying an additive functor $F: \abcat{A} \to \abcat{B}$ degreewise.
This induces a functor $\Ch{F}: \Ch{\abcat{A}} \to \Ch{\abcat{B}}$,
though for notational ease we will typically refer to $\Ch{F}$ simply
by $F$ itself.

\begin{lemma}
  \label{lemma_add_func_preserves_homotopy_and_cones}
  The functor $\Ch{F}$ is additive and preserves homotopies, homotopy
  equivalences and mapping cones.
  Moreover, it commutes with shift functors.
\end{lemma}

\begin{proof}
  These properties follow from functoriality and
  \cref{lemma_additive_func_commutes_with_dirsum}.
\end{proof}

In the case of double complexes, the only operation we will concern
ourselves with is \emph{totalisation}, which is a construction that
collapses a double complex down into a normal complex in one of two ways.

\begin{definition}
  \label{def_totalisation}
  Given a double complex $\cochainbicomp{C}$, the \emph{total
  complexes} $\totprodcomp{\cochainbicomp{C}}$ and
  $\totsumcomp{\cochainbicomp{C}}$ are those obtained by taking
  products and direct sums along each lattice diagonal (c.f.
  \cref{def_double_complex}), i.e.
  \[
    \totprodcomp{\cochainbicomp{C}}^i = \prod_{i = p + q} C^{p, q}
    \mathand
    \totsumcomp{\cochainbicomp{C}}^i = \bigoplus_{i = p + q} C^{p, q}.
  \]
  In both cases, the differential is defined componentwise by $d =
  d_h^{p, q} + d_v^{p, q}$.
\end{definition}

\begin{remark}
  A priori, totalising a double complex involves taking infinite
  products or direct sums of objects in $\abcat{A}$, which may not be
  possible if the base category $\abcat{A}$ is not \emph{(co)complete}.
  However, many double complexes arising in practice, including any
  we will encounter in this thesis, have only finitely many objects
  along each lattice diagonal.
  A sufficient condition for this to occur is that the double complex
  to be totalised is constructed out of bounded complexes.
\end{remark}

\section{Resolutions}
\label{sect_resolutions}

A particularly useful philosophy in homological algebra is to try and
replace a single object by an exact sequence of objects that are
somehow easier to study.
We will see in the next section that this has certain computational
benefits, but a more honest motivation is as follows.
For a given $R$-module $M$, the length of an exact sequence
\[
  \begin{tikzcd}[cramped]
    \cdots & {P_2} & {P_1} & {P_0} & M & 0
    \arrow[from=1-1, to=1-2]
    \arrow[from=1-2, to=1-3]
    \arrow[from=1-3, to=1-4]
    \arrow[from=1-4, to=1-5]
    \arrow[from=1-5, to=1-6]
  \end{tikzcd}
\]
where each $P_i$ is a projective module is the maximal index $i$ such
that $P_i \neq 0$ but $P_j = 0$ for all $j \geq i$.
The minimal length of any such sequence is called the
\emph{projective dimension} of $M$.
This quantity provides a handle on how far $M$ is from being
projective, since any projective module has projective dimension 0.
Moreover, the supremum over the projective dimensions of all
$R$-modules is called the \emph{global dimension} of the ring $R$ and
has important applications in the dimension theory of Noetherian rings.

\begin{definition}
  A \emph{left resolution} of an object $A \in \abcat{A}$ is a
  negatively graded complex $\cochaincomp{C}$ extended by a morphism
  $\varepsilon: C^0 \to A$ to an exact sequence
  \[
    \begin{tikzcd}
      \cdots & {C^{-2}} & {C^{-1}} & {C^0} & A & 0.
      \arrow[from=1-1, to=1-2]
      \arrow["{d}", from=1-2, to=1-3]
      \arrow["{d}", from=1-3, to=1-4]
      \arrow["\varepsilon", from=1-4, to=1-5]
      \arrow[from=1-5, to=1-6]
    \end{tikzcd}
  \]
  We denote this by $\cochaincomp{C} \to A$.
  Similarly, a \emph{right resolution} $A \to \cochaincomp{C}$  is a
  positively graded complex $\cochaincomp{C}$ extended by a morphism
  $\varepsilon: A \to C^0$ to an exact sequence
  \[
    \begin{tikzcd}
      0 & A & {C^0} & {C^1} & {C^2} & \cdots.
      \arrow[from=1-1, to=1-2]
      \arrow["\varepsilon", from=1-2, to=1-3]
      \arrow["{d}", from=1-3, to=1-4]
      \arrow["{d}", from=1-4, to=1-5]
      \arrow[from=1-5, to=1-6]
    \end{tikzcd}
  \]
  \vspace{-24pt}
\end{definition}

\begin{remark}
  \label{rem_left_res_as_chain_complex}
  Left resolutions are typically chain complexes so that the grading
  of both left and right resolutions are positive, but this is just
  an aesthetic choice.
\end{remark}

\begin{remark}
  \label{rem_res_is_a_qis}
  Resolutions of an object are equivalent to quasi-isomorphisms to or
  from that object concentrated in degree 0.
  For example, a right resolution $A \to \cochaincomp{C}$ is a
  quasi-isomorphism to some positively graded cochain complex $\cochaincomp{C}$.
  \[
    \begin{tikzcd}
      \cdots & 0 & A & 0 & 0 & \cdots \\
      \cdots & 0 & {C^0} & {C^1} & {C^2} & \cdots
      \arrow[from=1-1, to=1-2]
      \arrow[from=1-2, to=1-3]
      \arrow[from=1-2, to=2-2]
      \arrow[from=1-3, to=1-4]
      \arrow["\varepsilon"', from=1-3, to=2-3]
      \arrow[from=1-4, to=1-5]
      \arrow[from=1-4, to=2-4]
      \arrow[from=1-5, to=1-6]
      \arrow[from=1-5, to=2-5]
      \arrow[from=2-1, to=2-2]
      \arrow[from=2-2, to=2-3]
      \arrow[from=2-3, to=2-4]
      \arrow[from=2-4, to=2-5]
      \arrow[from=2-5, to=2-6]
    \end{tikzcd}
  \]
  In particular, this implies that $\cochaincomp{C}$ is exact except
  in degree 0.
\end{remark}

% https://math.stackexchange.com/questions/1271086/definition-of-left-resolution

The most significant class of resolutions is the following.

\begin{definition}
  A \emph{projective resolution} of $A \in \abcat{A}$ is a left
  resolution $\cochaincomp{P} \to A$ such that each $P^i$ is
  projective and an \emph{injective resolution} of $A$ is a right
  resolution $A \to \cochaincomp{I}$ such that each $I^i$ is injective.
\end{definition}

In the case of $\Mod{R}$, the existence of injective envelopes shows
that every module has an injective resolution, and separately one can
also find free (a fortiori, projective and flat) resolutions.
The situation for other abelian categories can be drastically
different, as the case of $\FinAb$ from \cref{ex_finab} shows.

\begin{proposition}[{\cite[Lemma~2.2.5]{weibel}}]
  If $\abcat{A}$ has enough projectives (resp. injectives), then
  every object of $\abcat{A}$ has a projective (resp. injective) resolution.
\end{proposition}

Injective and projective resolutions of an object need not be unique
whenever they exist, but any two of the same kind are always homotopy
equivalent.
This is an immediate corollary of the following result, which is so
useful that it is commonly known as the \emph{fundamental
theorem/lemma of homological algebra}.
We present only the statement for projective resolutions, but the
dual statement for injective resolutions also holds.

\begin{theorem}[{\cite[Theorem~2.2.6]{weibel}}]
  \label{thm_fund_thm_of_hom_alg}
  Let $f: A \to B$ be a morphism in $\abcat{A}$.
  Then for any projective resolution $\cochaincomp{P} \to A$ and left
  resolution $\cochaincomp{C} \to B$, there is a lift of $f$ to a
  chain map $\widetilde{f}: \cochaincomp{P} \to \cochaincomp{C}$,
  unique up to homotopy equivalence, making the following diagram commute:
  \[
    \begin{tikzcd}
      \cdots & {P^{-2}} & {P^{-1}} & {P^0} & A & 0 \\
      \cdots & {C^{-2}} & {C^{-1}} & {C^0} & B & 0.
      \arrow[from=1-1, to=1-2]
      \arrow[from=1-2, to=1-3]
      \arrow["{\widetilde{f}}", dashed, from=1-2, to=2-2]
      \arrow[from=1-3, to=1-4]
      \arrow["{\widetilde{f}}", dashed, from=1-3, to=2-3]
      \arrow[from=1-4, to=1-5]
      \arrow["{\widetilde{f}}", dashed, from=1-4, to=2-4]
      \arrow[from=1-5, to=1-6]
      \arrow["f", from=1-5, to=2-5]
      \arrow[from=2-1, to=2-2]
      \arrow[from=2-2, to=2-3]
      \arrow[from=2-3, to=2-4]
      \arrow[from=2-4, to=2-5]
      \arrow[from=2-5, to=2-6]
    \end{tikzcd}
  \]
  % Then
  % \begin{enumerate}
  %     \item
  %     For any projective resolution $\cochaincomp{P} \to A$ and
  % left resolution $\cochaincomp{C} \to B$, there is a lift of $f$
  % to a chain map $\widetilde{f}: \cochaincomp{P} \to
  % \cochaincomp{C}$, unique up to homotopy equivalence, making the
  % following diagram commute:
  %     \[
  %         \begin{tikzcd}
  %           \cdots & {P^{-2}} & {P^{-1}} & {P^0} & A & 0 \\
  %           \cdots & {C^{-2}} & {C^{-1}} & {C^0} & B & 0.
  %           \arrow[from=1-1, to=1-2]
  %           \arrow[from=1-2, to=1-3]
  %           \arrow["{\widetilde{f}}", dashed, from=1-2, to=2-2]
  %           \arrow[from=1-3, to=1-4]
  %           \arrow["{\widetilde{f}}", dashed, from=1-3, to=2-3]
  %           \arrow[from=1-4, to=1-5]
  %           \arrow["{\widetilde{f}}", dashed, from=1-4, to=2-4]
  %           \arrow[from=1-5, to=1-6]
  %           \arrow["f", from=1-5, to=2-5]
  %           \arrow[from=2-1, to=2-2]
  %           \arrow[from=2-2, to=2-3]
  %           \arrow[from=2-3, to=2-4]
  %           \arrow[from=2-4, to=2-5]
  %           \arrow[from=2-5, to=2-6]
  %         \end{tikzcd}
  %     \]

  %     \item
  %     For any right resolution $A \to \cochaincomp{C}$ and
  % injective resolution $B \to \cochaincomp{I}$, there is a lift of
  % $f$ to a cochain map $\widetilde{f}: \cochaincomp{C} \to
  % \cochaincomp{I}$, unique up to homotopy equivalence, making the
  % following diagram commute:
  %     \[
  %         \begin{tikzcd}
  %           0 & A & {C^0} & {C^1} & {C^2} & \cdots \\
  %           0 & B & {I^0} & {I^1} & {I^2} & \cdots.
  %           \arrow[from=1-1, to=1-2]
  %           \arrow[from=1-2, to=1-3]
  %           \arrow["f"', from=1-2, to=2-2]
  %           \arrow[from=1-3, to=1-4]
  %           \arrow["{\widetilde{f}}"', dashed, from=1-3, to=2-3]
  %           \arrow[from=1-4, to=1-5]
  %           \arrow["{\widetilde{f}}"', dashed, from=1-4, to=2-4]
  %           \arrow[from=1-5, to=1-6]
  %           \arrow["{\widetilde{f}}"', dashed, from=1-5, to=2-5]
  %           \arrow[from=2-1, to=2-2]
  %           \arrow[from=2-2, to=2-3]
  %           \arrow[from=2-3, to=2-4]
  %           \arrow[from=2-4, to=2-5]
  %           \arrow[from=2-5, to=2-6]
  %         \end{tikzcd}
  %     \]
  % \end{enumerate}
\end{theorem}

\section{Classical derived functors}
\label{sect_classical_derfunc}

Now that we have a systematic way to study the failure of a complex
to be exact, we finish by discussing the same problem for a covariant
functor $F: \abcat{A} \to \abcat{B}$ between abelian categories.
Specifically, we will see that when $F$ is left (resp. right) exact,
there exist \emph{classical right} (resp. \emph{left}) \emph{derived
functors} that measure the failure of $F$ to be exact.
The reason for the term `classical' will become clear in \cref{sect_derfunc}.

Our experience in the previous sections show that cohomology can be a
useful tool in these situations.
Replacing any object $A \in \abcat{A}$ with a right resolution $A \to
\cochaincomp{C}$ and applying a left exact functor $F: \abcat{A} \to
\abcat{B}$ gives a cochain complex
\[
  \begin{tikzcd}[cramped]
    \cdots & 0 & {F(C^0)} & {F(C^1)} & {F(C^2)} & \cdots
    \arrow[from=1-1, to=1-2]
    \arrow[from=1-2, to=1-3]
    \arrow[from=1-3, to=1-4]
    \arrow[from=1-4, to=1-5]
    \arrow[from=1-5, to=1-6]
  \end{tikzcd}
\]
in $\abcat{B}$.
This will typically have non-trivial cohomology, but it is always
acyclic when $F$ is exact.
While this does seem to have the measuring capabilities we seek, a
problem is that the cohomology objects we obtain are sensitive to our
choice of resolution.
If we instead work with injective resolutions, then any two will be
homotopy equivalent by \cref{thm_fund_thm_of_hom_alg}, and so we will
always recover the same cohomology objects.
The intuitive reason for considering right resolutions here to begin
with is that left exact functors have issues extending left exact
sequences to the right (c.f. \cref{def_exact_functors}).

\begin{definition}
  Let $F: \abcat{A} \to \abcat{B}$ be a left exact functor and
  suppose that $\abcat{A}$ has enough injectives.
  For each $A \in \abcat{A}$, fix an injective resolution $A \to
  \cochaincomp{I}$ and define
  \[
    R^iF(A) := H^i(F(\cochaincomp{I}))
  \]
  for $i \in \zpos$.
  The functor $R^iF: \abcat{A} \to \abcat{B}$ is called the
  \emph{$i$th right derived functor} of $F$.
\end{definition}

We have already seen that this is a well-defined map on objects, but
we have yet to specify what $R^iF$ does to morphisms.
Given a morphism $f: A \to B$, fix a pair of injective resolutions $A
\to \cochaincomp{I}$ and $B \to \cochaincomp{J}$ and then lift $f$
using \cref{thm_fund_thm_of_hom_alg} to a cochain map $\widetilde{f}:
\cochaincomp{I} \to \cochaincomp{J}$.
This defines a unique map
\[
  R^iF(f) := H^n(F(\widetilde{f})): H^i(F(\cochaincomp{I})) \to
  H^i(F(\cochaincomp{J})),
\]
since any lift of $f$ is unique up to homotopy equivalence.
It is now clear that $R^iF$ is actually a functor to begin with and
is moreover additive.
% \[
%     R^iF(gf) = R^iF(g) \circ R^iF(f) \mathand R^iF(f + g) = R^iF(f) + R^iF(g)
% \]
% whenever the composite $gf$ and sum $f + g$ are defined.
% Given a compatible morphism $g$, the composite
% $\widetilde{\vphantom{f}g} \widetilde{f}$ is a lift for $gf$, so
% $R^iF(gf) = R^iF(g) \circ R^iF(f)$.
% Finally, if $g$ is now a morphism parallel to $f$, we see that
% $\widetilde{f} + \widetilde{\vphantom{f}g}$ is a lift of $f + g$,
% and so $R^iF(f + g) = R^iF(f) + R^iF(g)$.

The key properties that the functors $R^iF$ have are best expressed
by the following framework due to Grothendieck.
As the name implies, the model for this will be the collection of
cohomology functors of \cref{prop_cochain_map_induced_cohomology},
with the key property being that there is an analogue of the long
exact cohomology sequence.

\begin{definition}
  A \emph{cohomological $\delta$-functor} $T: \abcat{A} \to
  \abcat{B}$ consists of a collection of additive functors $\{T^i:
  \abcat{A} \to \abcat{B}\}_{i \in \zpos}$ subject to the following
  two conditions:
  \begin{enumerate}
    \item
      For each short exact sequence
      \[
        \begin{tikzcd}
          0 & A & B & C & 0
          \arrow[from=1-1, to=1-2]
          \arrow[from=1-2, to=1-3]
          \arrow[from=1-3, to=1-4]
          \arrow[from=1-4, to=1-5]
        \end{tikzcd}
      \]
      in $\abcat{A}$, there exist morphisms $\delta^i: T^i(C) \to
      T^{i + 1}(A)$ and a long exact sequence
      \[
        \begin{tikzcd}[cramped]
          0 & {T^0(A)} & {T^0(B)} & {T^0(C)} \\
          & {T^1(A)} & {T^1(B)} & {T^1(C)} \\
          & {T^2(A)} & {T^2(B)} & {T^2(C)} & \cdots
          \arrow[from=1-1, to=1-2]
          \arrow[from=1-2, to=1-3]
          \arrow[from=1-3, to=1-4]
          \arrow["{\delta^0}"{description}, dashed, from=1-4, to=2-2]
          \arrow[from=2-2, to=2-3]
          \arrow[from=2-3, to=2-4]
          \arrow["{\delta^1}"{description}, dashed, from=2-4, to=3-2]
          \arrow[from=3-2, to=3-3]
          \arrow[from=3-3, to=3-4]
          \arrow[from=3-4, to=3-5, dashed]
        \end{tikzcd}
      \]
      in $\abcat{B}$.
      (In particular, $T^0$ is left exact.)

    \item
      Given a morphism of short exact sequences
      \[
        \begin{tikzcd}[cramped]
          0 & A & B & C & 0 \\
          0 & {A'} & {B'} & {C'} & 0
          \arrow[from=1-1, to=1-2]
          \arrow[from=1-2, to=1-3]
          \arrow[from=1-2, to=2-2]
          \arrow[from=1-3, to=1-4]
          \arrow[from=1-3, to=2-3]
          \arrow[from=1-4, to=1-5]
          \arrow[from=1-4, to=2-4]
          \arrow[from=2-1, to=2-2]
          \arrow[from=2-2, to=2-3]
          \arrow[from=2-3, to=2-4]
          \arrow[from=2-4, to=2-5]
        \end{tikzcd}
      \]
      in $\abcat{A}$, each of the below squares in $\abcat{B}$ commute:
      \[
        \begin{tikzcd}[cramped]
          {T^i(C)} & {T^{i + 1}(A)} \\
          {T^i(C')} & {T^{i + 1}(A')}.
          \arrow["{\delta^i}", from=1-1, to=1-2]
          \arrow[from=1-1, to=2-1]
          \arrow[from=1-2, to=2-2]
          \arrow["{(\delta')^i}"', from=2-1, to=2-2]
        \end{tikzcd}
      \]
  \end{enumerate}
\end{definition}

\begin{definition}
  Let $S, \, T: \abcat{A} \to \abcat{B}$ be cohomological $\delta$-functors.
  A \emph{morphism of $\delta$-functors} $\mu: S \to T$ is a
  collection of natural transformations $\{\mu^i: S^i \Rightarrow
  T^i\}_{i \in \zpos}$ which commute with $\delta_S$ and $\delta_T$.
  That is, for any short exact sequence
  \[
    \begin{tikzcd}
      0 & A & B & C & 0
      \arrow[from=1-1, to=1-2]
      \arrow[from=1-2, to=1-3]
      \arrow[from=1-3, to=1-4]
      \arrow[from=1-4, to=1-5]
    \end{tikzcd}
  \]
  in $\abcat{A}$, each of the below squares in $\abcat{B}$ commute:
  \[
    \begin{tikzcd}[cramped]
      {S^i(C)} & {S^{i + 1}(A)} \\
      {T^i(C)} & {T^{i + 1}(A)}.
      \arrow["{\delta_S^i}", from=1-1, to=1-2]
      \arrow["{\mu_C^i}"', from=1-1, to=2-1]
      \arrow["{\mu_A^{i + 1}}", from=1-2, to=2-2]
      \arrow["{\delta_T^i}"', from=2-1, to=2-2]
    \end{tikzcd}
  \]
  Moreover, $T$ is \emph{universal} if for any choice of $S$ and
  natural transformation $\mu^0: S^0 \Rightarrow T^0$ there exists a
  unique extension of $\mu^0$ to a morphism of $\delta$-functors $\mu: S \to T$.
\end{definition}

\begin{proposition}
  The collection of all right derived functors of a left exact
  functor $F: \abcat{A} \to \abcat{B}$ form a cohomological universal
  $\delta$-functor.
\end{proposition}

\begin{proof}
  Both parts of this claim are non-trivial.
  The proofs in the dual case of the \emph{left} derived functors
  $L_iF$ of a \emph{right} exact $F$ are the content of
  \cite[Theorems~2.4.6 and 2.4.7]{weibel}, but with our initial
  hypotheses it will turn out that
  \[
    R^iF(A) = \op{(L_i \op{F})}(A)
  \]
  for all $A \in \abcat{A}$, so this is indeed enough.
\end{proof}

There are several other benefits to this construction.
First, the information of the original functor $F$ is captured by $R^0F$.
Indeed, since $F$ is left exact, the sequence
\[
  \begin{tikzcd}
    0 & {F(A)} & {F(I^0)} & {F(I^1)}
    \arrow[from=1-1, to=1-2]
    \arrow[from=1-2, to=1-3]
    \arrow[from=1-3, to=1-4]
  \end{tikzcd}
\]
is exact for any injective resolution $A \to \cochaincomp{I}$, so
\begin{align*}
  R^0F(A)
  = H^0(F(\cochaincomp{I}))
  \cong \ker(F(I^0) \to F(I^1))
  = \im(F(A) \to F(I^0))
  = F(A),
\end{align*}
and moreover $R^0F(f) = F(f)$ for any morphism $f$.
If $I$ is an injective object, then computing $R^iF(I)$ using the
injective resolution
\[
  \begin{tikzcd}
    0 & I & I & 0 & 0 & \cdots
    \arrow[from=1-1, to=1-2]
    \arrow["\id", from=1-2, to=1-3]
    \arrow[from=1-3, to=1-4]
    \arrow[from=1-4, to=1-5]
    \arrow[from=1-5, to=1-6]
  \end{tikzcd} \qedhere
\]
shows that $R^iF(I) = 0$ for $i > 0$.
It is also clear that if $F$ is exact, then $R^iF = 0$ for $i > 0$.
The upshot is therefore that the right derived functors of $F$
provide a convenient way of measuring how far $F$ is from being exact
as intended, and the fact that they are a cohomological universal
$\delta$-functor shows that they are the most natural way to do so.

\begin{remark}
  \label{rem_computing_with_F_acyclics}
  Even with enough injectives, actually obtaining a injective
  resolution for an object can be difficult.
  With reference to a specific functor $F$, an object $X \in
  \abcat{A}$ is said to be \emph{$F$-acyclic} if $R^iF(X) = 0$ for $i
  > 0$, and it turns out that it is sufficient to use $F$-acyclic
  resolutions to compute $R^iF(A)$, since $R^iF(A) \cong
  H^i(F(\cochaincomp{X}))$ for any $F$-acyclic resolution $A \to
  \cochaincomp{X}$.
  This can be seen by the \emph{dimension shifting} argument outlined
  (albeit for the left derived case again) in \cite[Exercise~2.4.3]{weibel}.
\end{remark}

We now summarise the analogous construction of left derived functors
when $F$ is right exact.
What we will end up with is a \emph{homological} $\delta$-functor,
which is defined in a dual way to cohomological $\delta$-functors
(c.f. \cite[Section~2.1]{weibel}).
The strategy is the same as last time, except we now use projective
resolutions and take negative cohomology.

\begin{proposition}
  \label{prop_classical_right_derfunc}
  Let $F: \abcat{A} \to \abcat{B}$ be a right exact functor and
  suppose that $\abcat{A}$ has enough projectives.
  For each $A \in \abcat{A}$, fix a projective resolution
  $\cochaincomp{P} \to A$ and define
  \[
    L_i F(A) := H^{-i}(F(\cochaincomp{P}))
  \]
  for each $i \in \zpos$.
  The corresponding functor $L_iF: \abcat{A} \to \abcat{B}$ is called
  the \emph{$n$th classical left derived functor} of $F$, and it
  satisfies the following properties:
  \begin{enumerate}
    \item
      Each functor $L_iF$ is well-defined up to natural isomorphism
      and additive.

    \item
      $L_0F(A) \cong F(A)$ and $L_0F(f) = F(f)$ for each object $A
      \in \abcat{A}$ and morphism $f \in \abcat{A}$.

    \item
      If $P$ is projective, then $L_iF(P) = 0$ for $i > 0$.

    \item
      The left derived functors of $F$ are a homological universal
      $\delta$-functor.
  \end{enumerate}
\end{proposition}

\begin{remark}
  The above discussion assumes covariant functors, but the
  construction of left and right derived functors can easily be
  carried out for a contravariant functor $F: \abcat{A} \to \abcat{B}$.
  Since $F: \op{\abcat{A}} \to \abcat{B}$ is covariant and an
  injective resolution in $\op{\abcat{A}}$ becomes a projective
  resolution in $\abcat{A}$, one can compute right derived functors
  when $F$ is left exact and $\abcat{A}$ has enough projectives, and
  dually for left derived functors.
\end{remark}

We close out this chapter by introducing the two most important
examples of these derived functors, followed by some final remarks.

\begin{definition}
  Let $\abcat{A}$ be an abelian category with enough injectives, and
  fix an object $A \in \abcat{A}$.
  Then the functor $\Hom_{\abcat{A}}(A, -)$ is left exact, and its
  suite of classical right derived functors are the associated
  \emph{Ext functors}
  \[
    \Ext{i}{\abcat{A}}(A, -) := R^i\Hom_{\abcat{A}}(A, -): \abcat{A} \to \Ab.
  \]
  \vspace{-24pt}
\end{definition}

\begin{definition}
  Fix an $R$-module $N$.
  Then the functor $- \tensor_R N$ is right exact, and its suite of
  classical left derived functors are the associated \emph{Tor functors}
  \[
    \Tor{i}{R}(-, N) := L_i (- \tensor_R N): \Mod{R} \to \Ab.
  \]
  \vspace{-24pt}
\end{definition}

\iftrue
The names given to these derived functors are suggestive of some of
their applications.
Given an $R$-module $M$, the Ext groups $\Ext{1}{R}(M, N)$ are
related to the \emph{module extensions} of $M$ by another $R$-module $N$.
If $r \in R$ is not a zero divisor, the Tor group
$\Tor{1}{R}(\quot{R}{(r)}, M)$ is the \emph{$r$-torsion subgroup} of $M$.
We will not discuss these applications any further, but these are the
subject of \cite[Chapter~3]{weibel}.
\else
The names given to these derived functors are suggestive of some of
their applications.
The first Ext group $\Ext{1}{\abcat{A}}(A, B)$ coincides with the
first \emph{Yoneda extension group} of $A$ and $B$.
This is the abelian group of equivalence classes of \emph{extensions}
of the object $B$ by $A$, i.e. short exact sequences
\[
  \begin{tikzcd}
    0 & A & E & B & 0
    \arrow[from=1-1, to=1-2]
    \arrow[from=1-2, to=1-3]
    \arrow[from=1-3, to=1-4]
    \arrow[from=1-4, to=1-5]
  \end{tikzcd}
\]
in $\abcat{A}$, with two extensions deemed equivalent if there is a
commutative diagram
\[
  \begin{tikzcd}
    0 & A & E & B & 0 \\
    0 & A & E & B & 0.
    \arrow[from=1-1, to=1-2]
    \arrow[from=1-2, to=1-3]
    \arrow[Rightarrow, no head, from=1-2, to=2-2]
    \arrow[from=1-3, to=1-4]
    \arrow["\cong"', from=1-3, to=2-3]
    \arrow[from=1-4, to=1-5]
    \arrow[Rightarrow, no head, from=1-4, to=2-4]
    \arrow[from=2-1, to=2-2]
    \arrow[from=2-2, to=2-3]
    \arrow[from=2-3, to=2-4]
    \arrow[from=2-4, to=2-5]
  \end{tikzcd}
\]
The group operation is the \emph{Baer sum} of extensions, and the
split exact sequence
\[
  \begin{tikzcd}[cramped]
    0 & A & {A \biprod B} & B & 0
    \arrow[from=1-1, to=1-2]
    \arrow[from=1-2, to=1-3]
    \arrow[from=1-3, to=1-4]
    \arrow[from=1-4, to=1-5]
  \end{tikzcd}
\]
is its identity element.
Since the details of this group structure are rather technical and
require the categorical notions of \emph{pullbacks} and
\emph{pushouts}, we defer to \cite[Section~3.4]{weibel}.
The interpretation of Tor is less relevant to us, but the relation
between the first Tor group and \emph{torsion subgroups} is
elaborated on in \cite[Section~3.1]{weibel}.
\fi

A useful fact about Ext that we will use later is the following.

\begin{lemma}
  \label{lemma_ext_preserved_by_equiv}
  Let $\abcat{A} \to \abcat{B}$ be an equivalence of abelian
  categories both with enough projectives.
  Then for all $A, B \in \abcat{A}$ and $i \in \zpos$, there is an isomorphism
  \[
    \Ext{i}{\abcat{A}}(A, B) \cong \Ext{i}{\abcat{B}}(F(A), F(B)).
  \]
  \vspace{-24pt}
\end{lemma}

\begin{proof}
  Fix an injective resolution $B \to \cochaincomp{I}$.
  Since $F$ is fully faithful, we have
  \[
    \Hom_{\abcat{A}}(A, I^j) \cong \Hom_{\abcat{B}}(F(A), F(I^j))
  \]
  for each $j \in \zpos$.
  As in \cref{rem_exact_limit_colimit_pres}, it turns out that an
  equivalences of categories also preserves limits and colimits, and
  from this it follows that $F$ is an exact functor that preserves injectives.
  This shows that $F(B) \to F(\cochaincomp{I})$ is a projective resolution, so
  \[
    \Ext{i}{\abcat{A}}(A, B)
    = H^i \Hom_{\abcat{A}}(A, \cochaincomp{I})
    \cong H^i \Hom_{\abcat{B}}(F(A), F(\cochaincomp{I}))
    = \Ext{i}{\abcat{A}}(F(A), F(B))
  \]
  for $i \in \zpos$ as claimed.
\end{proof}

\begin{remark}
  Since $\Hom_R$ and $\tensor_R$ are \emph{bifunctors}, one also
  could have decided to construct Ext and Tor by instead taking the
  derived functors $R^i \Hom_R(-, N)$ and $L_i (M \tensor_R -)$.
  Through a careful process of \emph{balancing}, one sees that in
  fact these functors will be naturally isomorphic to Ext and Tor as
  we have defined them above.
  The details are lengthy and immaterial to the goals of this thesis,
  so we refer the interested reader to \cite[Section~2.7]{weibel}.
\end{remark}

\begin{remark}
  \label{rem_tor_computed_by_flat_res}
  The Tor functors provide the most straightforward example of a
  functor which benefits from being computed by acyclic resolutions
  as described in \cref{rem_computing_with_F_acyclics}.
  Retaining the hypotheses of \cref{prop_classical_right_derfunc}, an
  object $X \in \abcat{A}$ is $F$-acyclic if $L_iF(X) = 0$ for $i >
  0$, and similarly we have $L_iF(A) \cong
  H^{-i}(F(\cochaincomp{X}))$ for an $F$-acyclic resolution
  $\cochaincomp{X} \to A$.
  In particular, $\Tor{i}{R}(A, N) = 0$ for $i > 0$ whenever $A$ is a
  flat $R$-module, so Tor can be computed using flat resolutions.
\end{remark}

