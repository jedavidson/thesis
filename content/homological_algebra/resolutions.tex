\section{Resolutions}
\label{sect_resolutions}

A particularly useful philosophy in homological algebra is to try and
replace a single object by an exact sequence of objects that are
somehow easier to study.
We will see in the next section that this has certain computational
benefits, but a more honest motivation is as follows.
For a given $R$-module $M$, the length of an exact sequence
\[
  \begin{tikzcd}[cramped]
    \cdots & {P_2} & {P_1} & {P_0} & M & 0
    \arrow[from=1-1, to=1-2]
    \arrow[from=1-2, to=1-3]
    \arrow[from=1-3, to=1-4]
    \arrow[from=1-4, to=1-5]
    \arrow[from=1-5, to=1-6]
  \end{tikzcd}
\]
where each $P_i$ is a projective module is the maximal index $i$ such
that $P_i \neq 0$ but $P_j = 0$ for all $j \geq i$.
The minimal length of any such sequence is called the
\emph{projective dimension} of $M$.
This quantity provides a handle on how far $M$ is from being
projective, since any projective module has projective dimension 0.
Moreover, the supremum over the projective dimensions of all
$R$-modules is called the \emph{global dimension} of the ring $R$ and
has important applications in the dimension theory of Noetherian rings.

\begin{definition}
  A \emph{left resolution} of an object $A \in \abcat{A}$ is a
  negatively graded complex $\cochaincomp{C}$ extended by a morphism
  $\varepsilon: C^0 \to A$ to an exact sequence
  \[
    \begin{tikzcd}
      \cdots & {C^{-2}} & {C^{-1}} & {C^0} & A & 0.
      \arrow[from=1-1, to=1-2]
      \arrow["{d}", from=1-2, to=1-3]
      \arrow["{d}", from=1-3, to=1-4]
      \arrow["\varepsilon", from=1-4, to=1-5]
      \arrow[from=1-5, to=1-6]
    \end{tikzcd}
  \]
  We denote this by $\cochaincomp{C} \to A$.
  Similarly, a \emph{right resolution} $A \to \cochaincomp{C}$  is a
  positively graded complex $\cochaincomp{C}$ extended by a morphism
  $\varepsilon: A \to C^0$ to an exact sequence
  \[
    \begin{tikzcd}
      0 & A & {C^0} & {C^1} & {C^2} & \cdots.
      \arrow[from=1-1, to=1-2]
      \arrow["\varepsilon", from=1-2, to=1-3]
      \arrow["{d}", from=1-3, to=1-4]
      \arrow["{d}", from=1-4, to=1-5]
      \arrow[from=1-5, to=1-6]
    \end{tikzcd}
  \]
  \vspace{-24pt}
\end{definition}

\begin{remark}
  \label{rem_left_res_as_chain_complex}
  Left resolutions are typically chain complexes so that the grading
  of both left and right resolutions are positive, but this is just
  an aesthetic choice.
\end{remark}

\begin{remark}
  \label{rem_res_is_a_qis}
  Resolutions of an object are equivalent to quasi-isomorphisms to or
  from that object concentrated in degree 0.
  For example, a right resolution $A \to \cochaincomp{C}$ is a
  quasi-isomorphism to some positively graded cochain complex $\cochaincomp{C}$.
  \[
    \begin{tikzcd}
      \cdots & 0 & A & 0 & 0 & \cdots \\
      \cdots & 0 & {C^0} & {C^1} & {C^2} & \cdots
      \arrow[from=1-1, to=1-2]
      \arrow[from=1-2, to=1-3]
      \arrow[from=1-2, to=2-2]
      \arrow[from=1-3, to=1-4]
      \arrow["\varepsilon"', from=1-3, to=2-3]
      \arrow[from=1-4, to=1-5]
      \arrow[from=1-4, to=2-4]
      \arrow[from=1-5, to=1-6]
      \arrow[from=1-5, to=2-5]
      \arrow[from=2-1, to=2-2]
      \arrow[from=2-2, to=2-3]
      \arrow[from=2-3, to=2-4]
      \arrow[from=2-4, to=2-5]
      \arrow[from=2-5, to=2-6]
    \end{tikzcd}
  \]
  In particular, this implies that $\cochaincomp{C}$ is exact except
  in degree 0.
\end{remark}

% https://math.stackexchange.com/questions/1271086/definition-of-left-resolution

The most significant class of resolutions is the following.

\begin{definition}
  A \emph{projective resolution} of $A \in \abcat{A}$ is a left
  resolution $\cochaincomp{P} \to A$ such that each $P^i$ is
  projective and an \emph{injective resolution} of $A$ is a right
  resolution $A \to \cochaincomp{I}$ such that each $I^i$ is injective.
\end{definition}

In the case of $\Mod{R}$, the existence of injective envelopes shows
that every module has an injective resolution, and separately one can
also find free (a fortiori, projective and flat) resolutions.
The situation for other abelian categories can be drastically
different, as the case of $\FinAb$ from \cref{ex_finab} shows.

\begin{proposition}[{\cite[Lemma~2.2.5]{weibel}}]
  If $\abcat{A}$ has enough projectives (resp. injectives), then
  every object of $\abcat{A}$ has a projective (resp. injective) resolution.
\end{proposition}

Injective and projective resolutions of an object need not be unique
whenever they exist, but any two of the same kind are always homotopy
equivalent.
This is an immediate corollary of the following result, which is so
useful that it is commonly known as the \emph{fundamental
theorem/lemma of homological algebra}.
We present only the statement for projective resolutions, but the
dual statement for injective resolutions also holds.

\begin{theorem}[{\cite[Theorem~2.2.6]{weibel}}]
  \label{thm_fund_thm_of_hom_alg}
  Let $f: A \to B$ be a morphism in $\abcat{A}$.
  Then for any projective resolution $\cochaincomp{P} \to A$ and left
  resolution $\cochaincomp{C} \to B$, there is a lift of $f$ to a
  chain map $\widetilde{f}: \cochaincomp{P} \to \cochaincomp{C}$,
  unique up to homotopy equivalence, making the following diagram commute:
  \[
    \begin{tikzcd}
      \cdots & {P^{-2}} & {P^{-1}} & {P^0} & A & 0 \\
      \cdots & {C^{-2}} & {C^{-1}} & {C^0} & B & 0.
      \arrow[from=1-1, to=1-2]
      \arrow[from=1-2, to=1-3]
      \arrow["{\widetilde{f}}", dashed, from=1-2, to=2-2]
      \arrow[from=1-3, to=1-4]
      \arrow["{\widetilde{f}}", dashed, from=1-3, to=2-3]
      \arrow[from=1-4, to=1-5]
      \arrow["{\widetilde{f}}", dashed, from=1-4, to=2-4]
      \arrow[from=1-5, to=1-6]
      \arrow["f", from=1-5, to=2-5]
      \arrow[from=2-1, to=2-2]
      \arrow[from=2-2, to=2-3]
      \arrow[from=2-3, to=2-4]
      \arrow[from=2-4, to=2-5]
      \arrow[from=2-5, to=2-6]
    \end{tikzcd}
  \]
  % Then
  % \begin{enumerate}
  %     \item
  %     For any projective resolution $\cochaincomp{P} \to A$ and
  % left resolution $\cochaincomp{C} \to B$, there is a lift of $f$
  % to a chain map $\widetilde{f}: \cochaincomp{P} \to
  % \cochaincomp{C}$, unique up to homotopy equivalence, making the
  % following diagram commute:
  %     \[
  %         \begin{tikzcd}
  %           \cdots & {P^{-2}} & {P^{-1}} & {P^0} & A & 0 \\
  %           \cdots & {C^{-2}} & {C^{-1}} & {C^0} & B & 0.
  %           \arrow[from=1-1, to=1-2]
  %           \arrow[from=1-2, to=1-3]
  %           \arrow["{\widetilde{f}}", dashed, from=1-2, to=2-2]
  %           \arrow[from=1-3, to=1-4]
  %           \arrow["{\widetilde{f}}", dashed, from=1-3, to=2-3]
  %           \arrow[from=1-4, to=1-5]
  %           \arrow["{\widetilde{f}}", dashed, from=1-4, to=2-4]
  %           \arrow[from=1-5, to=1-6]
  %           \arrow["f", from=1-5, to=2-5]
  %           \arrow[from=2-1, to=2-2]
  %           \arrow[from=2-2, to=2-3]
  %           \arrow[from=2-3, to=2-4]
  %           \arrow[from=2-4, to=2-5]
  %           \arrow[from=2-5, to=2-6]
  %         \end{tikzcd}
  %     \]

  %     \item
  %     For any right resolution $A \to \cochaincomp{C}$ and
  % injective resolution $B \to \cochaincomp{I}$, there is a lift of
  % $f$ to a cochain map $\widetilde{f}: \cochaincomp{C} \to
  % \cochaincomp{I}$, unique up to homotopy equivalence, making the
  % following diagram commute:
  %     \[
  %         \begin{tikzcd}
  %           0 & A & {C^0} & {C^1} & {C^2} & \cdots \\
  %           0 & B & {I^0} & {I^1} & {I^2} & \cdots.
  %           \arrow[from=1-1, to=1-2]
  %           \arrow[from=1-2, to=1-3]
  %           \arrow["f"', from=1-2, to=2-2]
  %           \arrow[from=1-3, to=1-4]
  %           \arrow["{\widetilde{f}}"', dashed, from=1-3, to=2-3]
  %           \arrow[from=1-4, to=1-5]
  %           \arrow["{\widetilde{f}}"', dashed, from=1-4, to=2-4]
  %           \arrow[from=1-5, to=1-6]
  %           \arrow["{\widetilde{f}}"', dashed, from=1-5, to=2-5]
  %           \arrow[from=2-1, to=2-2]
  %           \arrow[from=2-2, to=2-3]
  %           \arrow[from=2-3, to=2-4]
  %           \arrow[from=2-4, to=2-5]
  %           \arrow[from=2-5, to=2-6]
  %         \end{tikzcd}
  %     \]
  % \end{enumerate}
\end{theorem}
