\documentclass[xcolor=dvipsnames]{beamer}

\usepackage{tikz-cd}
\usepackage{amsmath}

\renewcommand{\footnotesize}{\scriptsize}

\usetheme{Singapore}
\usecolortheme{rose}

\colorlet{themecolour}{Fuchsia}
\setbeamercolor{structure}{fg=themecolour}
\renewcommand{\thefootnote}{\textcolor{themecolour}{*}}

\title{Opening Presentation}
\author{James Davidson\footnote{Supervisor: Professor Daniel Chan}}
\institute{School of Mathematics and Statistics \\ UNSW Sydney}
\date{February 28, 2024}

\begin{document}

\maketitle

\section{Who}

\begin{frame}{Background}
  I'm doing a \alert{standalone} honours in the pure stream this year, but I finished my double degree in CS and Math in the summer term last year.
  (What did I do in the interim? Industry for a bit, travelling for a bit, then tutoring in the school of CSE.)
\end{frame}

\section{What}

\begin{frame}{My goal}
  The goal of my thesis is to understand the following (and its generalisations, if time allows):

  \begin{block}{Theorem (Beilinson, 1978)}
    Fix some field $k$. Let $\mathsf{Coh} \, \mathbb{P}^1$ be the category of coherent sheaves of the projective line over $k$,
    and $\mathsf{mod} \, {k \Delta}$ the category of modules over $k \Delta$, the path $k$-algebra of the Kronecker quiver.
    Then there is an equivalence of (bounded) derived categories
    \[
      \mathbf{D}^b(\mathsf{Coh} \, \mathbb{P}^1) \longrightarrow \mathbf{D}^b(\mathsf{mod} \, {k \Delta}).
    \]
  \end{block}

  This gives a link between representation theory and algebraic geometry, and hinges on a tool from the former:
  \alert{tilting theory}, which generalises Morita equivalence of rings.
\end{frame}

\begin{frame}[fragile]{A little glimpse}
  Most things in the foregoing result probably make about as much sense to you as they do to me right now, but one bit is easy.
  A quiver is essentially just a graph permitting multiple parallel edges, and the \alert{Kronecker quiver} is quite simple among such objects:

  \begin{figure}
    \centering
    \begin{tikzcd}[ampersand replacement=\&]
      1
      \arrow[r, bend left, "\alpha"]
      \arrow[r, bend right, swap, "\beta"]
      \&
      2
    \end{tikzcd}
  \end{figure}

  Somehow, this little thing has a connection to algebraic geometry, and therein lies my thesis topic.
\end{frame}

\end{document}
