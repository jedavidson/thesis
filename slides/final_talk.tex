\documentclass[
  xcolor=dvipsnames,
  aspectratio=169,
  compress
]{beamer}

\usepackage{../commands}

\usepackage{quiver}

\usefonttheme[onlymath]{serif}

\colorlet{themecolour}{purple}
\setbeamercolor{structure}{fg=themecolour}
\setbeamertemplate{navigation symbols}[default]

\renewcommand{\thefootnote}{\textcolor{purple}{$*$}}
\renewcommand{\footnotesize}{\scriptsize}

\usetheme{Singapore}
\usecolortheme{rose}

\title{
  Tilting theory and the derived category \\ of coherent sheaves on $\proj{1}$
}
\author{James Davidson\footnote{Supervisor: Professor Daniel Chan}}
\institute{
  School of Mathematics and Statistics \\
  UNSW Sydney
}
\date{November 15, 2024}

\begin{document}

\maketitle

\begin{frame}{Leitfaden}
  \begin{exampleblock}{Aims}
    \begin{itemize}
      \item
        What are quivers?

      \item
        What is a coherent sheaf on $\proj{1}$?
        \begin{itemize}
          \item I need to assume a \emph{little} bit of algebraic geometry here. Nothing too serious.
        \end{itemize}

      \item
        What is the relationship between these?
    \end{itemize}
  \end{exampleblock}

  \pause

  \begin{alertblock}{Non-aims}
    \begin{itemize}
      \item
        Drown you with too much \alert{abstract nonsense}

      \item
        Explain what a derived category is
    \end{itemize}
  \end{alertblock}
\end{frame}

\section{Quivers}

\begin{frame}[fragile]
  \frametitle{What is a quiver?}
  Just a directed multigraph.
  This is the \alert{Kronecker quiver} $\Delta$:
  \[
    \begin{tikzcd}
      1 & 2
      \arrow["\alpha", shift left=2, from=1-1, to=1-2]
      \arrow["\beta"', shift right=2, from=1-1, to=1-2]
    \end{tikzcd}
  \]

  \pause

  \begin{block}{A more careful definition}
    A \alert{quiver} $Q$ is a 4-tuple $(Q_0, Q_1, s, t)$, where
    \begin{itemize}
      \item $Q_0$ is its set of \emph{vertices},
      \item $Q_1$ is its set of \emph{arrows},
      \item $s, t: Q_1 \to Q_0$ are functions assigning each arrow its \emph{source} and \emph{target} vertex.
    \end{itemize}
  \end{block}
\end{frame}

\begin{frame}[fragile]{Why $\Delta$?}
  \begin{block}{Kronecker's problem (c. 1890)}
    Classify all pairs $(S, T)$ of parallel linear maps
    \[
      \begin{tikzcd}
        {\bb{C}^m} & {\bb{C}^n}
        \arrow["S", shift left=2, from=1-1, to=1-2]
        \arrow["T"', shift right=2, from=1-1, to=1-2]
      \end{tikzcd}
    \]
    up to \alert{change of basis},
    i.e. $(S, T) \sim (S', T')$ if
    \[
      S' = B^{-1} S A
      \mathand
      T' = B^{-1} T A
    \]
    for some linear isomorphisms $A: \bb{C}^m \to \bb{C}^m$ and $B: \bb{C}^n \to \bb{C}^n$.
  \end{block}
\end{frame}

\begin{frame}[fragile]{Paths and the path algebra}
  \begin{definition}
    Given a quiver $Q$, a \alert{path} $p$ in $Q$ is a (possibly empty) sequence of arrows
    \[
      \begin{tikzcd}[cramped]
        \bullet & \bullet & \cdots & \bullet
        \arrow["{a_1}", from=1-1, to=1-2]
        \arrow["{a_2}", from=1-2, to=1-3]
        \arrow["{a_n}", from=1-3, to=1-4]
      \end{tikzcd}
    \]
    such that $t(a_i) = s(a_{i + 1})$ for all $1 \leq i < n$.
    We say that the \emph{source} and \emph{target} of $p$ are the vertices $s(p) = s(a_1)$ and $t(p) = t(a_n)$.
  \end{definition}

  \pause

  \begin{exampleblock}{Important example}
    There is a \alert{trivial path} $e_i$ for each $i \in Q_0$ with no arrows such that $s(e_i) = t(e_i) = i$.
  \end{exampleblock}
\end{frame}

\begin{frame}
  \begin{definition}
    Given paths $p$ and $q$ such that $t(p) = h(q)$, we may define their \alert{concatenation} $pq$ to be the path obtained by joining the head of $q$ at the tail of $p$.
  \end{definition}

  \pause

  \begin{definition}
    Given a quiver $Q$, let $kQ$ be the vector space over $k$ freely generated by paths in $Q$:
    \[
      kQ = \left\{
        \sum_j \lambda_j p_j: \lambda_j \in k, \, p_j \text{ a path in } Q
      \right\}.
    \]
    This becomes a $k$-algebra under concatenation after setting $pq = 0$ if $t(p) \neq h(q)$, and we call $kQ$ the \alert{path algebra} of $Q$ over $k$.
  \end{definition}
\end{frame}

\begin{frame}{Computing the path algebra of $\Delta$}
  \begin{exampleblock}{Example}
    There are only 4 paths in $\Delta$ (the trivial paths $e_1$ and $e_2$, and the arrows $\alpha$ and $\beta$), so
    \[
      k \Delta = k e_1 \oplus k e_2 \oplus k \alpha \oplus k \beta.
    \]
    \pause
    The only non-zero products of these basis elements in $k \Delta$ are
    \[
      e_1^2 = e_1, \quad
      e_2^2 = e_2, \quad
      e_1 \alpha = \alpha, \quad
      e_1 \beta = \beta, \quad
      \alpha e_2 = \alpha \mathand
      \beta e_2 = \beta.
    \]
    The elements $e_1$ and $e_2$ are \alert{idempotents}.
  \end{exampleblock}
\end{frame}

\section{Coherent sheaves on $\proj{1}$}

\iffalse
\begin{frame}{The affine line}
  \begin{itemize}
    \item
      Recall that $\aff{1} = k$ has a topology by declaring that the \alert{closed} sets are
      \[
        V(I) := \{a \in \aff{1}: f(a) = 0 \text{ for all } f \in I\}
      \]
      for an ideal $I \idealof k[x]$.

    \item
      Since $k[x]$ is a PID, $I = \gen{f}$, so $V(I)$ is actually just the zeros of $f$.
  \end{itemize}

  \pause
  \begin{corollary}
    If $U \subseteq \aff{1}$ is open, then $U = D(f) := \{a \in \aff{1}: f(a) \alert{\neq} 0\}$ for some $f \in k[x]$.
    Every open subset of $U$ is of the form $D(fg)$ for some $g \in k[x]$.
  \end{corollary}
\end{frame}
\fi

\begin{frame}{A particular construction of $\proj{1}$}
  \begin{itemize}
    \item
      Recall that each point of $\proj{1}$ can be given \alert{homogeneous coordinates} $(x : y)$, with at least one of $x, y \in k$ non-zero.
      It has the Zariski topology.

      \pause
    \item
      If $z = y/x$, then $\proj{1}$ is covered by the affine open subsets
      \[
        U_x = \{(x : y): x \neq 0\} \cong \aff{1}_z
        \mathand
        U_y = \{(x : y): y \neq 0\} \cong \aff{1}_{z^{-1}}.
      \]

      \pause
    \item
      Conclusion: $\proj{1}$ is \alert{charted} by two affine lines which we glue together according to the rule $z \leftrightarrow z^{-1}$ whenever $z, z^{-1} \neq 0$.
  \end{itemize}
\end{frame}

\iffalse
\begin{frame}{An algebraic geometer's sales pitch for sheaves}
  \begin{itemize}
    \item
      We can learn a lot about an algebraic variety $X$ (e.g. $\aff{1}$ or $\proj{1}$) by studying its ring $\shf{O}(X)$ of \emph{regular functions} $f: X \to k$.
      \begin{itemize}
        \item Recall that $f: X \to k$ is \emph{regular} if $f = g/h$ for appropriate polynomials $g, h$ in an open neighbourhood of every point of $X$.
      \end{itemize}

      \pause

    \item
      We can study a variety \emph{locally} via regular functions defined on a (Zariski) open subset $U \subseteq X$.
      We even get another ring $\shf{O}(U)$ in the process.

      \pause

    \item
      This class of functions has a very special property: if $f, g \in \shf{O}(X)$ agree over some open $U \subseteq X$, then $f = g$ on $X$.

      \pause

    \item
      This is far from the only example of data that behaves like this!
      \begin{itemize}
        \item
          in the studying the orders of zeros/poles, \emph{divisors} and \emph{rational functions}

        \item
          in the study of smoothness, \emph{K\"{a}hler differentials}
      \end{itemize}
  \end{itemize}
\end{frame}
\fi

\iffalse
\begin{frame}{The motivating example}
  \begin{example}
    The \alert{structure sheaf} $\strshf{X}$ is the sheaf of rings on any variety $X$ over $k$ whose sections are the regular functions over some set.
    \begin{itemize}
      \item
        For each open $U \subseteq X$, we have \alert{sections} $\strshf{X}(U) := \shf{O}(U)$.

      \item
        The usual ring monomorphisms $\shf{O}(U) \mono \shf{O}(V)$ for each inclusion of opens $V \subseteq U \subseteq X$ allow us to \alert{restrict} sections onto smaller open subsets.

      \item
        If $f, g \in \shf{O}(X)$ agree over some open $U \subseteq X$, then $f = g$ on $X$.
        Thus sections on $\strshf{X}$ \alert{glue}.
    \end{itemize}
  \end{example}
\end{frame}
\fi

\begin{frame}{The tenets of coherent sheaves}
  \begin{enumerate}
    \item
      Sheaves are useful, but in general too abstract to study.

      \pause
    \item
      Not every sheaf arising in algebraic geometry is so abstract.
      \begin{itemize}
        \item
          the \alert{structure sheaf} $\strshf{X}$, the \alert{sheaf of K\"{a}hler differentials} $\Omega^1_{X}$, ...
      \end{itemize}

      \pause
    \item
      Restricting our attention to sheaves which have a tight connection to an area we understand well (e.g. \alert{commutative algebra}) is a good idea.

      \pause
    \item
      We also like `finite' objects (e.g. \alert{finitely generated modules}).
  \end{enumerate}
\end{frame}

\iffalse
\begin{frame}{Making things formally invertible}
  \begin{definition}
    The \alert{localisation} of a ring $R$ at some element $f \in R$ is the ring of formal fractions
    \[
      R_f = \left\{
        \frac{r}{f^i}: r \in R,\, i \in \zpos
      \right\}.
    \]
    \pause
    Given an $R$-module $M$, the \alert{localisation} of $M$ at $f$ is the $R_f$-module
    \[
      M_f = \left\{
        \frac{m}{f^i}: m \in M,\, i \in \zpos
      \right\}.
    \]
    \pause
    Equality of fractions is what you expect.\footnote{...if you ignore torsion.}
  \end{definition}
\end{frame}
\fi

\begin{frame}{Coherent sheaves on $\aff{1}$}
  We will say that a \alert{coherent sheaf} on $\aff{1}$ is just a f.g. $k[x]$-module. How?

  \pause

  \begin{definition}
    The \alert{associated sheaf} of $M$ is the sheaf of abelian groups $\assmod{M}$ on $\aff{1}$ with
    \begin{itemize}
      \item
        sections $\assmod{M}(D(f)) := M_f$ for each open $D(f) \subseteq \aff{1}$,

      \item
        obvious\footnote{If you insist, the map is $m/f^i \mapsto (g^i m)/(f^i g^i)$.} restrictions $M_f \to M_{fg}$ for opens $D(fg) \subseteq D(f) \subseteq \aff{1}$.
    \end{itemize}
  \end{definition}

  \begin{example}
    The structure sheaf $\strshf{\aff{1}}$ is the coherent sheaf $\assmod{k[x]}$.
  \end{example}
\end{frame}

\iffalse
\begin{frame}{Examples}
  \begin{example}
    The structure sheaf $\strshf{\aff{1}}$ is the coherent sheaf $\assmod{k[x]}$.
  \end{example}

  \pause

  \begin{example}
    Let $M = \quot{k[x]}{\gen{x - a}} \cong k$.
    \begin{itemize}
      \item
        If $a \in D(f)$, then $f(a) \neq 0$.
        Then $f$ already acts invertibly on $M$, so localising shouldn't do anything.
        We get $\assmod{M}(D(f)) \cong k$.

        \pause

      \item
        If $a \not\in D(f)$, then $f(a) = 0$.
        Thus $f = 0$ in $M$.
        On the other hand, localising $M$ at $f$ makes $f$ invertible too.
        This can only be true if $\assmod{M}(D(f)) = 0$.
    \end{itemize}

    \pause
    As the sections of $\assmod{M}$ are \alert{supported} only at the point $a$, we call it a \alert{skyscraper sheaf}.
  \end{example}
\end{frame}
\fi

\iffalse
\begin{frame}{From $\aff{1}$ to $\proj{1}$}
  \begin{itemize}
    \item
      Since $\proj{1}$ is charted by two copies of $\aff{1}$, why not define \alert{coherent sheaves} to be pairs of f.g. modules $(M, M')$ over $k[z]$ and $k[z^{-1}]$?

      \pause
    \item
      We need to be careful: the intersection $U_{xy} = U_x \cap U_y$ is huge, so $M$ and $M'$ will have lots of overlapping sheaf data.

      \pause
      \iffalse
    \item
      Both $z$ and $z^{-1}$ are invertible on $U_{xy}$, so there we are really working\footnote{Scheme theory makes this quite formal, as $U_{xy} = \Spec{k[z, z^{-1}]}$.} over the ring $k[z, z^{-1}]$ of \alert{Laurent polynomials} in $z$.
      \fi

      \pause
    \item
      The localisations $M_z$ and $M'_{z^{-1}}$ are naturally modules over $k[z, z^{-1}]$, and they both contain sheaf data for $U_{xy}$.
  \end{itemize}
\end{frame}
\fi

\begin{frame}{Coherent sheaves on $\proj{1}$}
  \begin{definition}
    A \alert{coherent sheaf} on $\proj{1}$ is the data of a triple $\shf{F} = (M, M', \vartheta)$ consisting of
    \begin{itemize}
      \item a f.g. $k[z]$-module $M$,
      \item a f.g. $k[z^{-1}]$-module $M'$,
      \item a $k[z, z^{-1}]$-module isomorphism $\vartheta: M_{z} \to M'_{z^{-1}}$ (the \alert{descent datum} of $\shf{F}$).
    \end{itemize}
  \end{definition}

  \pause

  \iffalse
  Describing sections and restriction maps is quite technical, but \alert{global} sections are easy: just pairs $(m, m') \in M \times M'$ such that $\vartheta(m) = m'$.
  \fi
\end{frame}

\begin{frame}[fragile]{Morphisms}
  \begin{definition}
    A \alert{morphism} $\Phi: (M, M', \vartheta) \to (N, N', \varrho)$ is a pair $(\varphi, \varphi')$ consisting of
    \begin{itemize}
      \item a $k[z]$-module homomorphism $\varphi: M \to N$,
      \item a $k[z^{-1}]$-module homomorphism $\varphi': M' \to N'$.
    \end{itemize}

    These maps must commute with the descent data over $k[z, z^{-1}]$:
    \[
      \begin{tikzcd}
        {M_z} & {N_z} \\
        {M'_{z^{-1}}} & {N'_{z^{-1}}}.
        \arrow["{\varphi}", from=1-1, to=1-2]
        \arrow["\vartheta"', from=1-1, to=2-1]
        \arrow["\varrho", from=1-2, to=2-2]
        \arrow["{\varphi'}"', from=2-1, to=2-2]
      \end{tikzcd}
    \]
  \end{definition}
\end{frame}

\begin{frame}{The most imporant coherent sheaves}
  \begin{example}
    For each $n \in \bb{Z}$, the \alert{twisting sheaf}
    \[
      \twist{n} = (k[z], k[z^{-1}], \vartheta_n),
      \quad
      \vartheta_n(f) = z^{-n} f.
    \]
    The structure sheaf $\shf{O} := \strshf{\proj{1}}$ is the same as $\twist{0}$.
  \end{example}

  \pause

  \begin{fact}
    \[
      \Hom(\twist{m}, \twist{n})
      \cong
      \begin{cases}
        k^{n - m + 1} & \text{if } m \leq n \\
        0             & \text{if } m > n.
      \end{cases}
    \]
  \end{fact}
\end{frame}

\iffalse
\begin{frame}{As a category}
  $\Coh(\proj{1})$ has lots of interesting structure.
  It is
  \begin{itemize}
      \pause
    \item $k$-linear (i.e. every hom-set is a vector space over $k$),

      \pause
    \item additive (i.e. there is a \alert{zero sheaf}, can take finite \alert{direct sums}),

      \pause
    \item abelian (i.e. can take \alert{(co)kernels}, \alert{images}, \alert{quotients}, \alert{exact sequences}),

      \pause
    \item symmetric monoidal (i.e. can take \alert{tensor products}).
  \end{itemize}
\end{frame}
\fi

\section{So what?}

\begin{frame}{From coherent sheaves to algebras}
  \begin{itemize}
    \item
      Something curious happens if we consider the \alert{tilting bundle}
      \[
        \shf{T}
        := \tiltshf
        = (k[z]^2,\, k[z^{-1}]^2,\, \vartheta),
      \]
      where $\vartheta(f, g) = (f, z^{-1}g)$.

      \pause
    \item
      Let $\Lambda := \End(\shf{T}) = \Hom(\shf{T}, \shf{T})$ be the \alert{tilted algebra} over $k$.

      \pause
    \item
      Taking the two-sided \alert{Peirce decomposition} of $\Lambda$, we can write
      \[
        \Lambda = e \Lambda e \biprod e \Lambda (1 - e) \biprod (1 - e) \Lambda e \biprod (1 - e) \Lambda (1 - e)
      \]
      for any idempotent endomorphism $e \in \Lambda$.
  \end{itemize}
\end{frame}

\begin{frame}{Idempotents for $\Lambda$}
  \begin{fact}
    The \alert{first projection} $\Pi_1 = (\pi_1, \pi_1'): \shf{T} \to \shf{T}$ given by
    \[
      \pi_1(f, g) = (f, 0)
      \mathand
      \pi_1'(f, g) = (f, 0).
    \]
    is an idempotent of $\Lambda$.
    The \alert{second projection} $\Pi_2 = 1 - \Pi_1$ is also an idempotent.
  \end{fact}
\end{frame}

\begin{frame}{Weighted shifts}
  \begin{definition}
    Let $(\lambda, \lambda') \in k[z] \times k[z^{-1}]$ be a pair of \alert{weights} for $\shf{T}$, i.e. $\lambda' = z^{-1} \lambda$ in $k[z, z^{-1}]$.\footnote{The restriction on weights ensures that the needed diagram commutes.}
    The \alert{shift of $\shf{T}$ with weight $(\lambda, \lambda')$} is the endomorphism $S_{\lambda, \lambda'} := (s_\lambda, s'_{\lambda'})$, where
    \[
      s_\lambda(f, g) = (0, \lambda f) \mathand s'_{\lambda'}(f, g) = (0, \lambda' f).
    \]
  \end{definition}
\end{frame}

\begin{frame}{The structure of $\Lambda$}
  \begin{fact}
    \begin{itemize}
      \item $\Pi_1 \Lambda \Pi_1 = k \Pi_1$,
      \item $\Pi_2 \Lambda \Pi_2 = k \Pi_2$,
      \item $\Pi_1 \Lambda \Pi_2 = 0$,
      \item $\Pi_2 \Lambda \Pi_1 = k S_{1, z^{-1}} \oplus k S_{z, 1}$.
    \end{itemize}
  \end{fact}

  \pause

  In addition to $\Pi_1^2 = \Pi_1^2$, $\Pi_2^2 = \Pi_2$, one has
  \[
    S_{1, z^{-1}} \Pi_1 = S_{1, z^{-1}},
    \quad
    S_{z, 1} \Pi_1 = S_{z, 1},
    \quad
    \Pi_2 S_{1, z^{-1}} = S_{1, z^{-1}}
    \mathand
    \Pi_2 S_{z, 1} = S_{z, 1}.
  \]
  All other compositions are zero.
  \pause
  Seems familiar...
\end{frame}

\begin{frame}{First links}
  \begin{fact}
    The map
    \[
      e_1 \mapsto \Pi_1,
      \quad
      e_2 \mapsto \Pi_2,
      \quad
      \alpha \mapsto S_{1, z^{-1}}
      \mathand
      \beta \mapsto S_{z, 1}
    \]
    determines a $k$-algebra isomorphism $k \Delta \cong \op{\Lambda}$, the \alert{opposite algebra} of $\Lambda$.
  \end{fact}

  \pause
  \begin{fact}
    For any coherent sheaf $\shf{F}$ on $\proj{1}$, $\Hom(\shf{T}, \shf{F})$ has the structure\footnote{Just precompose by an endomorphism.} of a f.g. \alert{right} $\Lambda$-module, i.e. a f.g. \alert{left} $\op{\Lambda}$-module.
    This defines a functor
    \[
      \Hom(\shf{T}, -): \Coh(\proj{1}) \to \mod{\op{\Lambda}}.
    \]
  \end{fact}
\end{frame}

\begin{frame}{Why the derived category?}
  \begin{itemize}
    \item
      Unfortunately, we don't get an equivalence of categories.
      \pause
      \begin{itemize}
        \item
          Any equivalence of categories is fully faithful.

        \item
          The identity is a non-trivial endomorphism of $\twist{-1}$ in $\Coh(\proj{1})$.

        \item
          The only endomorphism of $\Hom(\shf{T}, \twist{-1}) = 0$ in $\mod{\op{\Lambda}}$ is zero.
      \end{itemize}

      \pause
    \item
      However, its first \alert{derived functor} $\Extalt{1}(\shf{T}, -)$ is non-zero.
      This is valuable cohomological information that we can't see in these categories.

      \pause
    \item
      Conventional wisdom suggests that we should look at the \alert{derived categories}.
  \end{itemize}
\end{frame}

\begin{frame}{Beilinson's theorem}
  \begin{block}{Theorem (Beilinson, 1978)}
    The \alert{total right derived functor} of $\Hom(\shf{T}, -)$ induces a triangular equivalence
    \[
      \derhom(\shf{T}, -): \dercat[b]{\Coh(\proj{1})} \to \dercat[b]{\mod{\op{\Lambda}}}
    \]
    on the bounded derived categories.
  \end{block}

  This is quite an escalation in the problem: now we are talking about \alert{cochain complexes} of coherent sheaves and modules.
\end{frame}

\begin{frame}{Final remarks}
  \begin{itemize}
    \item
      $\dercat[b]{\Coh(X)}$ is important in modern algebraic geometry.\footnote{Fourier-Mukai transforms, the Bondal-Orlov reconstruction theorem, ...}

    \item
      Beilinson's original proof is via used algebro-geometric techniques (a \alert{resolution of the diagonal} for $\proj{n}$).

    \item
      Later papers by Baer (1988) and Bondal (1990) show that this equivalence also falls out of \alert{tilting theory}, a technique from representation theory.

    \item
      The appearance of the projective line in this theorem is in fact not so mysterious, as explained by a theorem of Happel (2001).
  \end{itemize}
\end{frame}

\section{Dessert}

\begin{frame}{Dessert: the tilted algebra as matrices}
  It can be nice to describe elements of $\Lambda$ as elements of the \alert{generalised matrix algebra}
  \[
    \mattwo{k}{0}{k^2}{k}
    :=
    \left\{
      \mattwo{a}{0}{(b, c)}{d}: a, b, c, d \in k
    \right\}.
  \]
  Matrix multiplication makes this into a $k$-algebra isomorphic to $\Lambda$:
  \begin{align*}
    \Pi_1 &\mapsto \mattwo{1}{0}{(0, 0)}{0} & \Pi_2 &\mapsto \mattwo{0}{0}{(0, 0)}{1} \\
    S_{1, z^{-1}} &\mapsto \mattwo{0}{0}{(1, 0)}{0} & S_{z, 1} &\mapsto \mattwo{0}{0}{(0, 1)}{0}.
  \end{align*}
\end{frame}

\end{document}
